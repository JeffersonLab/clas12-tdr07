\documentclass[12pt]{article}
\usepackage{epsfig}
\input epsf
%
\def\beqn{\begin{eqnarray}}
\def\eeqn{\end{eqnarray}}

\textwidth  6.5in
\textheight 9.in
\topmargin -0.15 in
\headheight 0.2in
\headsep 0.2in
\oddsidemargin 0in
\evensidemargin 0in
\parsep 0 in
\pagenumbering{arabic}

%
\newcommand{\la}{\langle}
\newcommand{\ra}{\rangle}
\newcommand{\zh}{z}
\newcommand{\xbj}{x_{\scriptscriptstyle B}}
%
\newcommand{\Epgb}{$\vec ep~\rightarrow~ep(p,\Delta, N^*)\gamma~$}
\newcommand{\Epgl}{$\vec e\vec p~\rightarrow~ep(p,\Delta, N^*)\gamma~$}
\newcommand{\Eppiz}{$ep~\rightarrow~ep\pi^0~$}
\newcommand{\Enpip}{$ep~\rightarrow~en\pi^+~$}
\newcommand{\EppiD}{$ep~\rightarrow~e\pi \Delta~$}
\newcommand{\Epeta}{$ep~\rightarrow~ep\eta~$}
\newcommand{\Epr}{$ep~\rightarrow~ep\rho~$}
\newcommand{\EpX}{$ep\rightarrow epX~$}
\newcommand{\EpKY}{$ep~\rightarrow~eKY~$}
\newcommand{\vEpg}{$\vec ep~\rightarrow~ep\gamma~$}
\newcommand{\xidef}{$\xi=x_B\frac{1+\frac{Delta^2}{2Q^2}}{2-x_B+x_B\frac{Delta^2}{2Q^2}}$}
\def\gevc2{(GeV/c)$^2$}
\newcommand{\EpgX}{$ep~\rightarrow~ep\gamma X~$}
%%%%%%%%%%%%%%%%%%%%%%%%%%%%%%%%%%%%%%%%%%%%%%%%%%%

\renewcommand{\topfraction}{0.75}
\renewcommand{\bottomfraction}{0.75}
\renewcommand{\textfraction}{0.2}
\renewcommand{\floatpagefraction}{0.7}
%\includeonly{dvcs_3, targ_asymm, correct}
%
\begin{document}
\pagestyle{plain}

\begin{titlepage}
% title

{\center \large\bf The Longitudinal Spin Structure of the Nucleon\\}
{\center A 12 GeV Research Proposal to Jefferson Lab (PAC 30)\\}

\vspace*{9mm}

% authors

{\center Moskov Amarian, Stephen B\"ultmann, Gail Dodge, Nevzat Guler, 
Henry Juengst, Sebastian Kuhn$^{\dagger *}$,  
Lawrence Weinstein \\}
\vspace{-8pt}
{\center \small  \it  Old Dominion University \\}

{\center Harut Avakian, Peter Bosted, Volker Burkert, Alexandre Deur$^{\dagger}$, 
Vipuli Dharmawardane$^{\dagger}$\\}
\vspace{-8pt}
{\center \small \it  Jefferson Lab \\}

{\center Keith Griffioen$^{\dagger}$ \\}
\vspace{-8pt}
{\center \small \it  The College of William and Mary \\}

%{\center Marco Anghinolfi, Marco Battaglieri, 
%Raffaella De Vita,
%Michail Osipenko, Marco Ripani, Mauro Taiuti \\}
%\vspace{-8pt}
%{\center \small \it  Istituto Nazionale di Fisica Nucleare, Genova, Italy \\}
%
{\center Hovanes Egiyan,  Maurik Holtrop$^{\dagger}$ \\}
\vspace{-8pt}
{\center \small \it  University of New Hampshire \\}

{\center Stanley Kowalski, Yelena Prok$^{\dagger}$ \\}
\vspace{-8pt}
{\center \small \it Massachusetts Institute of Technology \\}

{\center Don Crabb$^{\dagger}$, Karl Slifer \\}
\vspace{-8pt}
{\center \small \it University of Virginia \\}

{\center Tony Forest$^{\dagger}$ \\}
\vspace{-8pt}
{\center \small \it Louisiana Tech \\}

{\center Angela Biselli \\}
\vspace{-8pt}
{\center \small \it Fairfield University \\}

{\center Kyungseon Joo \\}
\vspace{-8pt}
{\center \small \it University of Connecticut \\}

{\center Mahbub Khandaker \\}
\vspace{-8pt}
{\center \small \it Norfolk State University \\}

{\center Elliot Leader \\}
\vspace{-8pt}
{\center \small \it Imperial College, London, England \\}

{\center Aleksander V. Sidorov \\}
\vspace{-8pt}
{\center \small \it Bogoliubov Theoretical Laboratory, JINR Dubna, Russia \\}

{\center Dimiter B. Stamenov \\}
\vspace{-8pt}
{\center \small \it Inst. for Nuclear Research and Nuclear Energy,
Bulgarian Academy of Sciences, Sofia, Bulgaria \\}

{\center \large A CLAS collaboration proposal\\}



\vspace*{12 pt}
{\small $^{\dagger}$ Co-spokesperson}
{\small $^*$ Contact: Sebastian Kuhn, Department of Physics, Old Dominion University,
Norfolk VA 23529. Email: skuhn@odu.edu}

\end{titlepage}
\newpage
\pagebreak
\clearpage

{\center \large \bf  Collaborators' commitment to the 12 GeV upgrade of Jefferson Lab}

\begin{itemize}

\item
The Old Dominion University group (Prof. Amarian, B\"ultmann, Dodge, Kuhn and Weinstein)
is actively involved in
this proposal, as well as two other proposal using CLAS12.  Other  members of
our group are pursuing a proposal for Hall A, but their contributions  are not included here.
Among CLAS12 baseline equipment, the group intends to
take responsibility for the design, prototyping, construction and  testing
of the Region 1 Drift Chamber. Five faculty (including one research  faculty)
and one technician are likely to work at least part time on this
project in the next few years. Funding for the group is from DOE and  from
the university (75\% of research faculty salary + one regular faculty  summer salary
+ 50\% of the technician).
The university has also provided 6000 square feet of high bay  laboratory space with
clean room capabilities for our use.  We will seek other sources of  funding as appropriate.
Gail Dodge is the chair of the CLAS12 Steering Committee and the user  coordinator
for the CLAS12 tracking technical working group.
Beyond the baseline equipment, the group is also
interested in exploring improvements to the BoNuS detector and
a future RICH detector for CLAS12. 

\item
The UNH group is committed to significant contributions in the  development of the CLAS12 software. 
Maurik Holtrop is currently chair  of the CLAS12 GEANT4 simulation group to which our post-doc 
Hovanes  Egiyan is also contributing. Since currently the main software  efforts for CLAS12 are in
 the area of simulation we are also part of  and contributing to the general CLAS12 Software group. 
 Current man  power commitments to this effort are 0.15 FTE of a faculty and 0.4  FTE of one post-doc. 
 We expect to increase this effort as our CLAS  activities wind down and our CLAS12 activities pick up 
 and we expect  to attract some talented undergraduate students to this project.  
 These efforts are funded from our current grant with DOE.
In addition to the software efforts the UNH group is planning to  contribute to the prototyping and 
 construction of the silicon vertex  detector. No formal agreements have been made on this effort yet 
 and  no addition grants have been written yet. However, it is expected  that we will be able to attract 
 additional funding for this project  with which we will fund an additional post-doc and one or two  
 undergraduate students. One of the faculty from the UNH Space Science  Center, Jim Connel, 
 a cosmic ray experimentalist with a background in  nuclear physics, is very interested in joining 
 the vertex detector  project. He has considerable experience with silicon detectors for  space observations. 

\item
The UConn group has made a commitment to help build the CLAS12 high threshold cerenkov counter
 (HTCC) with Youri Sharabian and the RPI group. Early this year our group got funding of \$65,000 
(\$32,500 from UConn and \$32,500 from JLab) to build a HTCC prototype. Also UConn is providing 
1/2 postdoctoral support for Maurizio Ungaro for the next two years and commits to support 1/2 graduate 
student for 1 year for the CLAS12 upgrade efforts.  Our group of one PI, one postdoc and 4 graduate 
students will provide substantial manpower towards the 12 GeV upgrade.
Recently we got a DOE STTR Phase I grant to build a software framework for data archiving and data 
analysis for nuclear physics experiment with one UCon computer science professor and a local software company. 
With this grant, we expect to contribute to software development for the 12 GeV upgrade.

\item
The College of William and Mary group is actively involved in this  proposal, as
well as several other proposals using CLAS12.  Other members of our  group are also
pursuing a proposal for Hall A, but their contributions are not  included here.
Among CLAS12 baseline equipment, the group is committed
to building part of the forward tracking system, but the exact
tasks have not yet been determined.  At least one faculty member, two
graduate students, half a post-doc and several undergraduates are
likely to work at least part time on this project in the next
few years.  Funding for the group is from the DOE and from the NSF.
Additional funding will be sought for building the base equipment.
Facilities at William and Mary include a clean room suitable for  drift-chamber
construction, and, on the time scale of a few years in the future,  ample space
for detector construction and testing.

\item
The University of Virginia Polarized Target Group is actively involved in this
proposal as well as other proposals using CLAS12. Some members of the group
are also involved in proposals for Hall C.
The group's contribution to the CLAS12 baseline equipment will be the
design, construction and testing of the longitudinal polarized target
discussed in this proposal. The target  will use a horizontal  $^{4}$He
evaporation refrigerator with a conventional design and similar to ones built
and operated in the past. The refrigerator will be constructed in the Physics
Department workshop; the workshop staff have experience with building such
devices. Testing will be done in our lab where all the necessary
infrastructure is on hand. 
Two Research Professors (75\% of salary from UVA, 17\% from DOE), two Post-Docs
and two graduate students, all supported by DOE, will spend their time as
needed on this project. Other funding will be pursued as necessary.
Outside the base equipment considerations one member of the group (DGC) has started
working with Oxford Instruments on a design for an optimized transverse target
magnet to be used for transverse polarization measurements with CLAS12.

\end{itemize}

\newpage
\pagebreak
\clearpage

\begin{abstract}
We are proposing a comprehensive program to map out the $x$- and $Q^2$-dependence 
of the helicity structure of the nucleon in the region of moderate to very large $x$ where
presently the experimental uncertainties are still large. The experiment will use the upgraded
CLAS12 detector, 11 GeV highly polarized electron beam, and longitudinally polarized
solid ammonia targets (NH$_3$ and ND$_3$). 
Thanks to the large acceptance of CLAS12, we will cover a large
kinematical region simultaneously. We will detect both the scattered electrons and leading
hadrons from the hadronization of the struck quark, allowing us to gain information on
its flavor. 
Using both inclusive and semi-inclusive data,
we will separate the contribution from up and down valence and sea quarks in the
region $0.1 \leq x \leq 0.8$. These results will unambiguously test various models of
the helicity structure of the nucleon as $x \rightarrow 1$.
A combined Next-to-Leading Order (NLO) pQCD analysis
of our expected data together with
the existing world data will significantly improve our knowledge of all polarized parton
distribution functions, including for the gluons (through $Q^2$--evolution). High
statistics data on the deuteron in the region of moderate $x$ 
and with a fairly large range in $Q^2$ are crucial for this purpose.
Finally, we will be able to improve significantly the precision of various moments
of spin structure functions at moderate $Q^2$, which will allow us to 
study duality and higher-twist contributions.

We request 30 days of running on NH$_3$ and 50 days of running on ND$_3$
(or possibly $^6$LiD), including
about 20\% overhead for target anneals, polarization reversal, and auxiliary measurements.
\end{abstract}
\newpage
\pagebreak
\clearpage

\tableofcontents
\newpage
\pagebreak
\clearpage

\section{Introduction}
\input{Intro.tex} %Sebastian

\section{Physics Motivation and Existing Data}
In the following, we will outline the main topics addressed by the
proposed experiment and explain how new data can significantly 
improve upon the present state of knowledge.

\subsection{Nucleon Helicity Structure at Large $x$}%Vipuli, Sebastian
\input{EG12_A1.tex}

\subsection{Flavor Decomposition of the Proton Helicity Structure} %Tony, Peter, Harut, Kyungseon
\input{SIDIS.tex}

%\subsection{NLO Fits and Parton Distribution Functions} %Sebastian, Vipuli
%\subsubsection{Intro}
%\subsubsection{NLO technique}
%\subsubsection{Existing Data}

\subsection{Sum Rules, Higher Twist and Duality} %Alexandre
\section{Moments of Structure Functions}

Moments of structure functions provide powerful insight into nucleon 
structure.  Inclusive data at JLab have permitted evaluation of the moments 
at low and intermediate $Q^2$~\cite{Fatemi:2003yh,Yun:2002td,Chen:2005td}.  
With a maximum beam energy of 6~GeV, however, the measured strength of the 
moments becomes rather limited for $Q^2$ greater than a few GeV$^2$. The 
12-GeV upgrade removes this problem and allows for measurements to higher 
$Q^2$. 

Moments of structure functions can be related to the nucleon static properties 
by sum rules. At large $Q^2$ the Bjorken sum rule relates $\int g_1^{p-n} dx$ 
to the nucleon axial charge~\cite{Bjorken:1966jh}.  At the other end of the 
spectrum, $Q^2$ = 0, the Gerasimov-Drell-Hearn (GDH) sum rule links the 
difference of spin dependent cross sections, integrated over photon energy, 
to the anomalous magnetic moment of the nucleon~\cite{Drell:1966jv,
Gerasimov:1965et}.  The two sum rules are aspects of a general one derived 
recently by Ji and Osborne~\cite{Ji:1999mr} that is valid at any $Q^2$
and links the first moments of spin structure functions to spin-dependent 
Compton amplitudes. Low $Q^2$ is a testing ground for chiral perturbation 
theory, while large $Q^2$ data can be compared to higher-twist series derived 
within the operator product expansion (OPE) method.  Lattice QCD can calculate 
higher-twist terms, thus extending the validity domain of OPE to lower $Q^2$. 
However OPE is unusable at low $Q^2$.  To bridge the gap, lattice QCD can be 
used to compute Compton amplitudes at any $Q^2$.  Hence, the Ji and Osborne 
sum rule can be computed and compared to experiments at any $Q^2$, making it a 
unique tool to study the transition from partonic to hadronic degrees of 
freedom.

%%%%%%%%%%%%%%%%%%%%%%%%%%%%%%%%%%%%%%%%%%%%%%%%%%%%%%%%%%%%%%%%%%%%%%%%%%
\begin{figure}[htb]
\centerline{\epsfxsize=5in\epsfbox{../strucfunc/expect.eps}}
\caption{\small{Left plot: expected precision on $\Gamma_1^p$ for {\tt CLAS12}
and 30~days of running.  {\tt CLAS} EG1a~\cite{Fatemi:2003yh, Yun:2002td} data 
and preliminary results from EG1b are shown for comparison.  The data and 
systematic uncertainties do not include estimates of the unmeasured DIS 
contribution.  HERMES~\cite{Airapetian:2002wd} data, and E143~\cite{Abe:1998wq} 
and E155 data~\cite{Anthony:2000fn} from SLAC are also shown (including DIS 
contribution estimates).  The model is from Burkert and Ioffe
\cite{Burkert:1992tg,Burkert:1993ya}.  Right plot: same as the left but 
including an estimate of the DIS contribution.}}
\label{expect}
\end{figure}
%%%%%%%%%%%%%%%%%%%%%%%%%%%%%%%%%%%%%%%%%%%%%%%%%%%%%%%%%%%%%%%%%%%%%%%%%%

The left plot in Fig.~\ref{expect} shows the expected precision on the 
measured part of $\Gamma_1^p$.  The inner error bar is statistical while the 
outer one is the statistics and systematics added in quadrature.  Published 
results and preliminary results from EG1b are also displayed for comparison. 
Like the {\tt CLAS12} data, the EG1 data do not include the unmeasured DIS 
contribution.  The hatched blue band corresponds to the systematic uncertainty 
on the EG1b data points.  The red band indicates the estimated systematic 
uncertainty from {\tt CLAS12}.  The right plot in Fig.~\ref{expect} shows the 
results on $\Gamma_1^p$ and $\Gamma_1^d$ including an estimate of the 
unmeasured DIS contribution.  The systematic uncertainties for EG1 and 
{\tt CLAS12} here include the estimated uncertainty on the unmeasured DIS part 
estimated using the model from Bianchi and Thomas~\cite{Thomas:2000pf}.  As 
can be seen, moments can be measured up to $Q^2$=6~GeV$^2$ with a statistical 
accuracy improved several fold over that of the existing world data.

Higher moments are also of interest: generalized spin polarizabilities
are linked to higher moments of spin structure functions by sum rules based 
on similar grounds as the GDH sum rule. Higher moments are less sensitive to 
the unmeasured low-$x$ part, so measurements are possible up to higher $Q^2$ 
compared to first moments.  Just like the GDH/Bjorken sum rules, measurements 
of the $Q^2$-evolution allow us to study the parton-hadron transition since 
theoretical predictions exist at low and large $Q^2$~\cite{Chen:2005td}.  In 
addition, spin polarizabilities are also fundamental observables characterizing 
the nucleon structure and the only practical way known to measure them is 
through measurement of moments and application of the corresponding sum rules.

Finally, moments in the low ($\simeq$ 0.5 GeV$^2$) to moderate 
($\simeq$4~GeV$^2$) $Q^2$ range enable us to extract higher-twist parameters,
which represent correlations between quarks in the nucleon.  This extraction 
can be done by studying the $Q^2$ evolution of first moments~\cite{Chen:2005td}.
Higher twists have been consistently found to have, overall, a surprisingly 
smaller effect than expected.  Going to lower $Q^2$ enhances the higher-twist 
effects but makes it harder to disentangle a high twist from the yet higher 
ones.  Furthermore, the uncertainty on $\alpha _s$ becomes prohibitive at low 
$Q^2$.  Hence, higher twists turn out to be hard to measure, even at the 
present JLab energies.  Adding higher $Q^2$ to the present JLab data set 
removes the issues of disentangling higher twists from each other and of the 
$\alpha _s$ uncertainty.  The smallness of higher twists, however, requires 
statistically precise measurements with small point-to-point correlated 
systematic uncertainties.  Such precision at moderate $Q^2$ has not been 
achieved by the experiments done at high energy accelerators, while JLab at 
12~GeV presents the opportunity to reach it considering the expected 
statistical and systematic uncertainties of E12-06-109.  The total 
point-to-point uncorrelated uncertainty on the twist-4 term for the Bjorken 
sum, $f_2^{p-n}$, decreases by a factor of 5.6 compared to results obtained in
Ref.~\cite{Deur:2004ti}. 

\subsection{The GDH Sum Rule}

Despite its fundamental nature, the GDH sum rule has not yet been fully 
verified experimentally.  Combined results from MAMI and ELSA~\cite{GDH04}
are about 10\% above the expected value.  This is for an upper integration 
limit in photon energy of about 2.8~GeV.  With the cancellation of the fixed 
target program at SLAC and consequently of experiment E159~\cite{SLACGDH} that 
would have investigated the GDH strength at large $\nu$, JLab is now the best 
place to test the convergence of the GDH sum.  Using real photons or 
near real photons, we can measure the contribution to the GDH sum rule up to 
10.5~GeV, about 4 times the maximum energy reached at ELSA (see 
Fig.~\ref{gdhf}).  A non-convergence of the sum rule would be intriguing 
and may signal physics beyond the Standard Model.  In any case it will provide 
important insight on soft Regge physics.

%%%%%%%%%%%%%%%%%%%%%%%%%%%%%%%%%%%%%%%%%%%%%%%%%%%%%%%%%%%%%%%%%%%%%%%%%%
\begin{figure}
\begin{center}
\begin{minipage}[t]{0.6\linewidth}
\centerline{\epsfxsize=4.in\epsfbox{../strucfunc/gdhprev.eps}}
\end{minipage}\hfill
\begin{minipage}[c]{0.35\linewidth}
\vspace*{-7cm}
\caption{\small{Coverage of the GDH sum rule for low-$Q^2$ experiments with 
{\tt CLAS} and {\tt CLAS12}.  The data points are the GDH running sum from 
MAMI and ELSA at $Q^2=0$.}}
\label{gdhf}
\end{minipage}
\end{center}
\end{figure}
%%%%%%%%%%%%%%%%%%%%%%%%%%%%%%%%%%%%%%%%%%%%%%%%%%%%%%%%%%%%%%%%%%%%%%%%%%

\subsection{Moments of $F_2$ and the Precise Determination of $\alpha_s(M_Z)$}

Simulated results for the moments $\int x^n F_2 dx$  with $n \leq 8$ are 
shown in Fig.~\ref{fig:moments}.  It reveals that {\tt CLAS12} will provide 
a unique tool to extract moments of $F_2$ up to $Q^2$ values of 
10 - 14~GeV$^2$.  These can be used to extract the strong coupling constant 
$\alpha_{s}(M_{z})$.  The extraction of $\alpha_s(M_Z)$ from the scaling 
violations of the proton structure function $F_2$ is one of the most precise 
methods available up to now (see Fig.~\ref{fig:alpha}).  Simulation shows that 
a new procedure for the extraction of $\alpha_s(M_Z)$~\cite{Simula:2003vf} 
together with the {\tt CLAS12} data can allow an unprecedentedly accurate 
determination of $\alpha_s(M_Z)$ with a statistical uncertainty of 0.0008 and 
a systematic uncertainty of about 0.0007.

%%%%%%%%%%%%%%%%%%%%%%%%%%%%%%%%%%%%%%%%%%%%%%%%%%%%%%%%%%%%%%%%%%%%%%%%%%
\begin{figure}[htbp]
\vspace{8.0cm}
\special{psfile=../strucfunc/fig_moments.ps hscale=45 vscale=45 hoffset=100 voffset=-70}
\caption{\small{Expected moments of the proton structure function $F_2$ 
obtained with the {\tt CLAS12} detector simulations for a few days of 
running.  The meaning of the markers and the scale factors for each moment 
are indicated in the inset.}}
\label{fig:moments} 
\end{figure}
%%%%%%%%%%%%%%%%%%%%%%%%%%%%%%%%%%%%%%%%%%%%%%%%%%%%%%%%%%%%%%%%%%%%%%%%%%

%%%%%%%%%%%%%%%%%%%%%%%%%%%%%%%%%%%%%%%%%%%%%%%%%%%%%%%%%%%%%%%%%%%%%%%%%%
\begin{figure}[htbp]
\vspace{10.5cm}
\special{psfile=../strucfunc/alpha_s.epsi hscale=65 vscale=62 hoffset=30 voffset=-100}
\caption{\small{Existing determinations of the QCD coupling constant 
$\alpha_S(M_Z)$~\cite{alpha}.}}
\label{fig:alpha} 
\end{figure}
%%%%%%%%%%%%%%%%%%%%%%%%%%%%%%%%%%%%%%%%%%%%%%%%%%%%%%%%%%%%%%%%%%%%%%%%%%



\section{Experimental Details}
\subsection{CLAS12}
\input{CLAS12.tex} %to be taken from similar write up for other proposals

\subsection{Polarized Target}
\input{target_clas12.tex}  %Don Crabb, Yelena Prok

\subsection{Running Conditions}
\input{RunCon.tex}  %Don Crabb, Yelena Prok

\subsection{Analysis}
\input{Analysis.tex} %Sebastian

\section{Expected Results}
\subsection{Simulation} 
\input{Simulation.tex} %Maurik, Hovanes, Harut, Angela, Yelena

\subsection{Statistical and systematic errors}
\input{Errors.tex} %Maurik, Hovanes, Harut, Angela, Yelena

\subsection{Inclusive Spin Structure Functions}
\input{DISresults.tex} %Sebastian, Keith, Vipuli

\subsection{Semi-inclusive Results} 
\input{SIDISresults.tex}%Tony, Peter, Harut, Kyungseon

\subsection{Integrals and Sum Rules} 
\input{Momentresults.tex}%Alexandre

\section{Summary and Request}
\input{Summary.tex}

\begin{thebibliography}{99}
\addtocontents{toc}{\vspace*{12pt}}
%%%%%%%%%%%%%%%%%%%%%%%%%%%%%%%%%%%%%%%%%%%

%Intro begin
%------ DATA FROM----------------------
\bibitem{AAC}
AAC Collaboration, hep-ph/0603213.

\bibitem{E130g1p}
G. Baum {\it et~al.} [E130 Collaboration],
Phys. Rev. Lett. {\bf 51}, 1135 (1983).

\bibitem{EMCfinal}
J. Ashman {\it et~al.} [EMC Collaboration],
Nucl. Phys. {\bf B328}, 1 (1989).

%\cite{Adams:1997hc}
\bibitem{Adams:1997hc}
  D.~Adams {\it et al.}  [Spin Muon Collaboration (SMC)],
  %``The spin-dependent structure function g1(x) of the deuteron from  polarized
  %deep-inelastic muon scattering,''
  Phys.\ Lett.\  {\bf B396}, 338 (1997).
  %%CITATION = PHLTA,B396,338;%%

%\cite{Adeva:1997is}
\bibitem{Adeva:1997is}
  B.~Adeva {\it et al.}  [Spin Muon Collaboration (SMC)],
  %``The spin-dependent structure function g1(x) of the proton from  polarized
  %deep-inelastic muon scattering,''
  Phys.\ Lett.\  {\bf B412}, 414 (1997).
  %%CITATION = PHLTA,B412,414;%%

%\cite{Abe:1998wq}
\bibitem{Abe:1998wq}
  K.~Abe {\it et al.}  [E143 collaboration],
  %``Measurements of the proton and deuteron spin structure functions g1  and
  %g2,''
  Phys.\ Rev.\ D {\bf 58}, 112003 (1998)
  [arXiv:hep-ph/9802357].
  %%CITATION = HEP-PH 9802357;%%

%\cite{Abe:1997cx}
\bibitem{Abe:1997cx}
  K.~Abe {\it et al.}  [E154 Collaboration],
  %``Precision determination of the neutron spin structure function g1(n),''
  Phys.\ Rev.\ Lett.\  {\bf 79}, 26 (1997)
  [arXiv:hep-ex/9705012].
  %%CITATION = HEP-EX 9705012;%%

%\cite{Anthony:2000fn}
\bibitem{Anthony:2000fn}
  P.~L.~Anthony {\it et al.}  [E155 Collaboration],
  %``Measurements of the Q**2 dependence of the proton and neutron spin
  %structure functions g1(p) and g1(n),''
  Phys.\ Lett.\  {\bf B493}, 19 (2000)
  [arXiv:hep-ph/0007248].
  %%CITATION = HEP-PH 0007248;%%
  
%\cite{Airapetian:2004zf} - HERMES
\bibitem{Airapetian:2004zf}
  A.~Airapetian {\it et al.}  [HERMES Collaboration],
  %``Quark helicity distributions in the nucleon for up, down, and strange
  %quarks from semi-inclusive deep-inelastic scattering,''
  Phys.\ Rev.\ D {\bf 71}, 012003 (2005)
  [arXiv:hep-ex/0407032].
  %%CITATION = HEP-EX 0407032;%%      

 \bibitem{EG1a_p}
R. Fatemi {\it et~al.} [CLAS Collaboration],
Phys. Rev. Lett. {\bf 91}, 222002 (2003).

\bibitem{EG1a_d}
J. Yun  {\it et~al.} [CLAS Collaboration],
Phys. Rev. C {\bf 67}, 055204 (2003).

%\cite{Zheng:2004ce}
\bibitem{Zheng:2004ce}
  X.~Zheng {\it et al.}  [Jefferson Lab Hall A Collaboration],
  %``Precision measurement of the neutron spin asymmetries and  spin-dependent
  %structure functions in the valence quark region,''
  Phys.\ Rev.\ C {\bf 70}, 065207 (2004)
  [arXiv:nucl-ex/0405006].
  %%CITATION = NUCL-EX 0405006;%%
%

 %\cite{Ageev:2005gh}
\bibitem{Ageev:2005gh}
  E.~S.~Ageev {\it et al.}  [COMPASS Collaboration],
  %``Measurement of the spin structure of the deuteron in the DIS region,''
  Phys.\ Lett.\  {\bf B612}, 154 (2005)
  [arXiv:hep-ex/0501073].
  %%CITATION = HEP-EX 0501073;%%    
  
\bibitem{RHIC}
S.~S. Adler {\it et al.} [Phenix Collaboration],
Phys.Rev.Lett. {\bf 93}, 202002 (2004).

%\cite{Dharmawardane:2006zd}
\bibitem{Dharmawardane:2006zd}
  K.~V.~Dharmawardane, S.~E.~Kuhn, P.~Bosted and Y.~Prok  [the CLAS
                  Collaboration],
  %``Measurement of the $x$- and $Q^2$-Dependence of the Asymmetry $A_1$ on the
  %Nucleon,''
  to be published in Phys. Lett., arXiv:nucl-ex/0605028.
  %%CITATION = NUCL-EX 0605028;%%

%Intro end

%A1motivation begin

\bibitem{Isgur} N. Isgur, Phys. Rev. D {\bf 59}, 34013 (1999).
%

\bibitem{HFpert} N. Isgur, G. Karl and R. Koniuk, Phys. Rev. Lett. {\bf 41},
1269 (1978); N. Isgur, G. Karl and R. Koniuk, Phys. Rev. Lett. {\bf 45}, 1738
(1980).
%

\bibitem{CloseWally} F. E. Close and  W. Melnitchouk, Phys. Rev. C {\bf 68},
  035210 (2003).
% 

%\cite{Close:1988br}
\bibitem{Close:1988br}
  F.~E.~Close and A.~W.~Thomas,
  %``The Spin And Flavor Dependence Of Parton Distribution Functions,''
  Phys.\ Lett.\  {\bf B212}, 227 (1988).
  %%CITATION = PHLTA,B212,227;%%
%\cite{Gluck:2000dy}

\bibitem{Farrar} G. R. Farrar and D. R. Jackson, Phys. Rev. Lett. {\bf 35}, 
1416 (1975); G. R. Farrar, Phys. Lett.  {\bf B70}, 346 (1977).
%\cite{Brodsky:1994kg}

\bibitem{Brodsky:1994kg}
  S.~J.~Brodsky, M.~Burkardt and I.~Schmidt,
  %``Perturbative QCD constraints on the shape of polarized quark and gluon
  %distributions,''
  Nucl.\ Phys.\  {\bf B441}, 197 (1995)
  [arXiv:hep-ph/9401328].
  %%CITATION = HEP-PH 9401328;%%%

\bibitem{Gluck:2000dy}
  M.~Gluck, E.~Reya, M.~Stratmann and W.~Vogelsang,
  %``Models for the polarized parton distributions of the nucleon,''
  Phys.\ Rev.\ D {\bf 63}, 094005 (2001)
  [arXiv:hep-ph/0011215].
  %%CITATION = HEP-PH 0011215;%
%\cite{Leader:2005ci}

\bibitem{Leader:2005ci}
  E.~Leader, A.~V.~Sidorov and D.~B.~Stamenov,
  %``Longitudinal polarized parton densities updated,''
  Phys.\ Rev.\ D {\bf 73}, 034023 (2006)
  [arXiv:hep-ph/0512114].
  %%CITATION = HEP-PH 0512114;%%

\bibitem{Bluemlein}
 J.~Bluemlein and H.~Boettcher, arXiv:hep-ph/0203155.
%\cite{Hirai:2003pm}

\bibitem{Hirai:2003pm}
  M.~Hirai, S.~Kumano and N.~Saito  [Asymmetry Analysis Collaboration],
  %``Determination of polarized parton distribution functions and their
  %uncertainties,''
  Phys.\ Rev.\ D {\bf 69}, 054021 (2004)
  [arXiv:hep-ph/0312112].
  %%CITATION = HEP-PH 0312112;%
%

\bibitem{DGLAP} V. N. Gribov and L. N. Lipatov, Sov. J. Nucl. Phys. {\bf 15},
138 (1972); Y. L. Dokahitzer, Sov. Phys. JETP. {\bf 16}, 161 (1977); G.
Altarelli and G. Parisi, Nucl. Phys.  {\bf B126}, 298 (1977).
%

%\cite{Hirai:2006sr}
%\bibitem{Hirai:2006sr}
%  M.~Hirai, S.~Kumano and N.~Saito,
  %``Determination of polarized parton distribution functions with recent data
  %on polarization asymmetries,''
%  arXiv:hep-ph/0603213.
  %%CITATION = HEP-PH 0603213;%%
  
%\cite{Leader:2005ci}
%\cite{Bourrely:2001du}
\bibitem{Bourrely:2001du}
  C.~Bourrely, J.~Soffer and F.~Buccella,
  %``A statistical approach for polarized parton distributions,''
  Eur.\ Phys.\ J.\  {\bf C23}, 487 (2002)
  [arXiv:hep-ph/0109160].
  %%CITATION = HEP-PH 0109160;%%
%

%\bibitem{privatecomm_LSS} D.~B.~Stamenov, Private communications.

%\cite{Close:1977qx}
%\bibitem{Close:1977qx}
%  F.~E.~Close and D.~W.~Sivers,
  %``Whirlpools In The Sea: Polarization Of Anti-Quarks In A Spinning Proton,''
%  Phys.\ Rev.\ Lett.\  {\bf 39}, 1116  (1977).
  %%CITATION = PRLTA,39,1116;%%
  
%A1motivation end

%SIDIS begin

\bibitem{LOI} H. Avakian {\it et al.}, ``Semi-Inclusive Pion Production with a Longitudinally
Polarized Target at 12 GeV'',
Letter of Intent to Jefferson Lab PAC 30, (2006).

\bibitem{Frankfurt1989} L. L. Frankfurt {\it et al.}, Phys. Lett. {\bf B230}, 141 (1989).

\bibitem{SMC98} B. Adeva {\it et al}, Phys. Lett. {\bf B420}, 180-190 (1998).

\bibitem{HermesDeltaD} A. Airapetian, Phys. Rev. D {\bf 71}, 012003 (2005) (hep-ex/0407032).

\bibitem{PEPSI}
L. Mankiewicz, A. Schafer, and M. Veltri, 
Comput. Phys. Commun. {\bf 71}, 305 (1992).

\bibitem{Ackerstaff98}K. Ackerstaff {\it et al.} Phys. Rev. Lett. {\bf 81}, 5519 (1998).    

\bibitem{ChristovaLeader99} E. Christova {\it et al.}, Phys. Lett. {\bf B468},  299 (1999).

\bibitem{ChristovaLeader00} E. Christova and E. Leader, Nucl. Phys. {\bf B607}, 369 (2001).

\bibitem{Florian96} D. de. Florian, C. A. Garcia Canal,
R. Sassot, Nucl. Phys. {\bf B470},195 (1996).

\bibitem{Graudenz94} D. Graudenz, Nucl. Phys. {\bf B432}, 351 (1994).

\bibitem{COMPASS05}V. Alexhakin {\it et al.} Phys. Rev. Lett. {\bf 94}, 202002
(2005).

 %hep-ph/9907265} 

%SIDIS end

%Moments begin
\bibitem{rev_mom}
J.-P. Chen, A. Deur, Z.-E. Meziani, Mod. Phys. Lett. \textbf{A20}, 2745 (2005);
M. Osipenko {\it et al.}, Phys. Rev. D {\bf 71}, 054007 (2005).

\bibitem{Bj}
J. D. Bjorken, Phys. Rev. \textbf{148}, 1467 (1966).

\bibitem{GDH}
S. D. Drell and
A. C. Hearn, Phys. Rev. Lett. \textbf{16}, 908 (1966);
S. Gerasimov, Sov. J. Nucl. Phys. \textbf{2}, 430 (1966).

\bibitem{JiSR}
X. Ji and J. Osborne, J. Phys. {\bf G27} 127 (2001).

%Moments end

\bibitem {mckee} P. McKee, Nucl. Instrum. Meth. {\bf A526}, 60 (2004).

\bibitem{dulya}
C. Dulya {\it et al.,}
Nucl. Instr. and Meth. {\bf  354}, 249 (1995).

\bibitem{cdk} C. Keith, Nucl. Instrum. Meth. {\bf A501}, 327 (2003).

%\bibitem{pdg}
%Particle Data Group; D. E. Groom {\it et al.,} Eur. Phys. J. {\bf C15}, 1  (2000).

%Analysis begin
\bibitem{Kukhto}
T.~V. Kukhto and N.~M. Shumeiko,
Nucl. Phys. {\bf B219},
412 (1983).

\bibitem{Tsai}
Y.-S. Tsai,
Rev. Mod. Phys. {\bf 46},
815 (1974).
%Analysis end

%Results begin
%\bibitem{HallADeltaD} X. Zheng, Phys. Rev. C {\bf 70}, 065207 (2004) (nucl-ex/0405006).

\bibitem{WallyduRatio} W. Melnitchouk and A.W. Thomas, Acta Phys. Polon. {\bf B27},
1407 (1996) (nucl-th/9603021).

\bibitem{christy} M.E. Christy http://www.jlab.org/\textasciitilde{}christy/cs\_fits/cs\_fits.html.

\bibitem{NMC} R. Arneodo {\it et al.} Phys. Lett. {\bf B364}, 107 (1995).%AD


\bibitem{R1998} K. Abe {\it et al.} Phys. Lett. {\bf B452}, 194 (1999).%AD

\bibitem{QFS} J. W. Lightbody, Jr. and J. S. O'Connell, Computers in Physics {\bf 2}, 57 (1988).

\bibitem{g1p} S. Simula\emph{ et al.} Phys. Rev. D \textbf{65}, 034017 (2002).

\bibitem{BT}
N. Bianchi and E. Thomas, Nucl. Phys. Proc. Suppl. \textbf{82}, 256 (2000).

\bibitem{HERMES}
HERMES collaboration: A. Airapetian \emph{et al.}, 
Eur. Phys. J. {\bf C26}, 527 (2003).

\bibitem{E143}
E143 collaboration: K. Abe \emph{et al.},
Phys. Rev. Lett. \textbf{78}, 815 (1997); 
K. Abe \emph{et al.}, 
Phys. Rev. D {\bf58}, 112003 (1998).

\bibitem{E155}
P.L. Anthony {\it et al.,} Phys. Lett. {\bf B493}, 19 (2000).

\bibitem{AO}
V. D. Burkert and B. L. Ioffe,
Phys. Lett. \textbf{B296}, 223 (1992);
J. Exp. Theor. Phys. {\bf 78}, 619 (1994).

\bibitem{RSS} 
JLab E01-006, O. Rondon and M. Jones spokespersons.
See also, e.g., K. Slifer,  ``The Hall C Spin Program at JLab'',
to be published in  Czech. J. Phys., (2006).

\bibitem{E155x} Anthony {\it et al.} Phys. Lett. {\bf B553}, 18 (2003);
Anthony {\it et al.} Phys. Lett. {\bf B458}, 529 (1999).%AD

%
\bibitem{bjht} A. Deur {\it et al.} Phys. Rev. Lett.%AD
{\bf 93}, 212001 (2004).

% Results end

\end{thebibliography}

\end{document}
