The proposed experimental program requires use of a polarized solid state 
target.  The target will be polarized via the method of Dynamic Nuclear 
Polarization (DNP), which is a well established technique that has been 
used extensively in nuclear and particle physics experiments, including 
the ones performed in Hall B.  Dynamically polarized target systems 
consist of a hydrogenated (polarized protons) or deuterated (polarized 
neutrons) compound containing paramagnetic centers, such as unpaired 
electrons, placed in a high magnetic field and cooled to low temperatures, 
with a $B/T$ ratio of the order of 5~T/K.  In these conditions, the free 
electron spins can approach polarization of 100\%. The high polarization 
of unpaired electrons is then transferred dynamically to the nucleons by 
irradiating the target material at frequency near that of electron spin 
resonance. This technique typically achieves a proton polarization of 
80-90\%, and a deuteron polarization of 30-40\%.  The nucleons in the 
target will be polarized either parallel or anti-parallel to the electron 
beam direction.  The main systems required to realize DNP are the 
superconducting magnet to provide a strong (5~T) field, a $^4$He 
evaporation refrigerator to maintain the target material at $\sim$1~K, 
a target insert that will house the target material and some additional 
instrumentation, a microwave system to transfer the polarization to the 
nucleon spins and a Nuclear Magnetic Resonance (NMR) system to determine 
the state of polarization.
