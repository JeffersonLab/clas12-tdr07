\section{Other Beamline Devices}

The other beamline devices concerned by the upgrade are:

\begin{itemize}

\item Raster. The raster is used to evenly spread the heat load of the beam
on the surface of the target. It consists of two pairs of kicking magnets
for vertical and horizontal rastering. The water-cooled magnet coils can be 
used to raster an 11-GeV beam. The present power supplies are not powerful 
enough to provide a suitable magnetic field, but they will be upgraded for 
the {\tt CLAS} 6~GeV DVCS experiment. The new power supplies will be 
powerful enough for a 11~GeV beam. Consequently, no upgrade will be 
necessary for the raster.

\item Faraday Cup. The Faraday cup (FCup) is the device presently used for 
monitoring of the absolute beam current in Hall B. The higher luminosity and
beam energy will make standard operations of the FCup impossible. However,
the FCup can be used at low luminosity for calibration of other devices
measuring beam current. The beam stopper will then be inserted at high
luminosity and the current will be provided by nanoAmps BPMs (see next
item). Consequently, no hardware or software upgrade is necessary for the 
Faraday cup. 

\item Beam Position Monitors (BPMs). The BPMs measure both the position and
the current of the electron beam. The BPM can work with an 11-GeV beam 
as they are. They are currently read out by the slow controls system. 
However, upgrading the data acquisition to 60~Hz is necessary in order to 
monitor the beam current and beam charge asymmetry at a large enough speed
in the absence of Faraday cup monitoring. This necessitate a fast intensity
60~Hz signal. The lock-in amplifiers will have to be reconfigured. It is
necessary to provide VtoA for the 6 channels (phase/amp) per nA and scalers
for the 18 channels, as well as writing the corresponding software.
%The cost of the upgrade of the nanoAmps BPM is estimated to 10K. The other 
%Hall B beam line monitors, such as SLM, do not require any upgrade to operate 
%at 12 GeV.
\end{itemize}

