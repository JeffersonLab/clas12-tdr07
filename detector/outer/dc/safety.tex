\section{Drift Chamber Safety and Quality Assurance Issues}

\subsection{Safety}
There are safety issues in the construction, installation and operation
phases of the CLAS12 drift chamber project which we address in this section.
Table~\ref{dc-safety} briefly summarizes the issues and the mitigation
strategies employed.
Safety will be addressed from the start as an integral part of all activities and plans. All 
of our workers will receive training prior to being permitted to perform any activities 
requiring safety training.

%%%%%%%%%%%%%%%%%%%%%%%%%%%%%%%%%%%%%%%%%%%%%%%%%%%%%%%%%%%%%%%%%%%%%%%%%
\begin{table}[htbp]
\begin{center}
\begin{tabular} {||c|c|c||} \hline \hline
{\bf Operation} & {\bf Issue}     &  {\bf Mitigation} \\ \hline
cleaning parts   & solvents & use of non-volatiles; masks \\ \hline
box assembly & material handling, welding & standard procedures \\ \hline
stringing & material handling & design overview, reviewed procedures \\ \hline
installation & material handling & design overview, reviewed procedures \\ \hline
gas delivery & flammability; ODH & non-flammable mixture, small lines \\ \hline
gas delivery & over/under pressure & active controls, passive bubblers \\ \hline
low-volatage power & over-heating & each line individually fused \\ \hline
high-voltage power & shock & current-limited supplies, reviewed procedures \\ \hline
routine operation & sparking & fast-trip supplies \\ \hline \hline
repair $ material handling & design overview, reviewed procedures \\ \hline
\end{tabular}
\caption{\small{Safety issues in the CLAS12 drift chamber project.}}
\label{dc-safety}
\end{center}
\end{table}
%%%%%%%%%%%%%%%%%%%%%%%%%%%%%%%%%%%%%%%%%%%%%%%%%%%%%%%%%%%%%%%%%%%%%%%%%

The only significant safety issues are in the construction, installation and
operation of the chambers.  During the construction phase, we receive a number of
machined parts from industry which must be cleaned thoroughly before assembly into
a chamber.  Our anticipated cleaning strategy is to use safe, non-volatile cleaning
agents which are friendly to the environment and to human health.  All of our
activities are coordinated by written procedures which are reviewed by our safety
staff at the lab.

During construction of the chamber boxes we handle heavy metallic items and bolt
or weld certain items.  All of these activities are handled by personnel with
adequate safety training wearing suitable protective gear.  The design of all
mechanical assemblies are reviewed by trained mechanical engineers and experiened
technicians.

During stringing of the chamber, the technicians and stringers work on elevated
platforms.  The platform designs are reviewed in a similar fashion to the 
chamber mechanical designs and all work is closely supervised by experienced 
technicians.

Operation of the chambers involves the use of a gas-handling system.  We use non-flammable
gas mixtures delivered to the chambers at low pressure.  There are intermediate 
storage tanks filled with gas at moderate (a few atmospheres) of pressure.  The 
gas-mixing and delivery system will be the same as the present CLAS system with minor
modifications, so we will continue to follow our present safety procedures which
include some engineering controls in the gas-mixing shed to mitigate possible
ODH conditions and extensive active and passive controls to mitigate over- and
under-pressure conditions which could harm the drift chambers.


The normal operation also involves delivery of low-voltage power to the on-chamber
amplifiers.  Although the low voltage (less than 8 Volts) means that there is no
electrocution hazard, there is a potential for over-heating and fire ignition because
of the relatively high currents involved.  Each supply is capable of supplying up
to 50 Amps of current.  Our mitigation strategy is to mot allow more than 3 Amps
of current on any output line, enforced by the placement of individual fuses on
every wire that leaves the supply. 

The Hall B Drift Chamber Low Voltage (DCLV) supply and distribution system consists of 18 
HP 6651A power supplies. Each power supply has a max current output of 8VDC and 50A of 
current. The power supplies are connected to Hall Bs 120VAC clean power located in the 
individual racks of the space frame which are protected by an electrical breaker panel. The 
supplies have a 10A current limiting fuse for its own A.C. line protection. The DCLV system 
supply DC power to the Drift Chamber Signal Translator Boards (STBs). The output voltage 
can be set at the supply as well as a current limit. In Hall B the current limits are 
individually set for each region/sector. The power is then supplied to the STBs via 
distribution chassis. Each chassis is safety interlocked and contains 3 bus bars (Positive, 
Negative, and Earth ground), of which the positive and negative are connected to the 
individual STBs through current limiting fuses. The individual fuses provide the nominal 
current draw plus 20% for transient spikes. The distribution system for region 3 was upgraded 
in 2003 to provide segmentation of axial and stereo boards, thus reducing the damage of a 
permanent short by half. The upgrade consisted of adding new power wires from the 
distribution to the STBs and adding an additional break out chassis. The breakout chassis 
provided individual fuses for the each side of the STB in front of the distribution fuse. 
Each breakout fuse is rated for the nominal current draw of its respective channel on the STB 
plus 40% for transient spikes.   In the event of a short on the chamber there is a written 
procedure for changing out fuses located in the Hall B counting house and the Drift Chamber 
expert manual. Monitoring of the supplies is done using a PC based LabView program from the 
Hall B counting house.

Operation also involves the delivery of high-voltage to the drift chamber wires.
The operating voltages are between 1000 and 2000 Volts, depending on the section
in question.  In all cases, the high voltage supplies are current-limited.  No
channel can deliver more than 40 micro-Amperes of current.  Administrative procedures
are in place to prevent accidental shocks from taking place.

The Hall B Drift Chamber High Voltage (DCHV) system consists of 3 CAEN 527 supplies and each 
containing up to 10 A934 (P/N) modules, distribution chassis, and software based monitoring 
and control system. Each module is has a safety interlocked which controls the output of 
power. The entire system has programmable firmware to set the current limit, voltage, and 
ramp rate as well as system monitoring. The crate is connected to Hall Bs clean power system 
and A.C. input power is fused at the chassis. The modules output high voltage to a 
distribution system which consists of summation and distribution chassis for the areas of 
the drift chamber. The current is limited to 20uA output from the modules. Monitoring and 
control is done via the DCHV TCL/TK program which can be accessed using the CLON computer 
system in the counting house and Hall B. The Drift Chamber Expert manual has operating 
procedures for the software and procedures for troubleshooting and repair can be found in the 
DC On-call manual.
al.

\subsection{Conformance with Jlab ES&H Manual}
	
	All activities will be analyzed in accordance with (IAW) JLAB ES&H Manual 
Chapter 3210 - Hazard ID and Characterization http://www.jlab.org/ehs/manual/PDF/3210HazID.pdf  
to identify any and all hazards associated with the work and any and all safety requirements 
mandated.
	All persons involved and all persons in the vicinity will be briefed on all potential 
hazards involved in all activities IAW JLAB ES&H Manual Chapter 3220 - Hazard Communication - 
http://www.jlab.org/ehs/manual/PDF/3220HazardCommunication.pdf
	Standard operating procedures will be developed IAW JLAB ES&H Manual Chapter 3310 - 
Standard Operating Procedures and Operational Safety Procedures - 
http://www.jlab.org/ehs/manual/PDF/3310SOPsOSPs.pdf as all activities are considered low risk, 
common, and routine in nature and are all fully covered by the  ES&H Manual. For example, 
JLAB ES&H Manual Chapter 6120 - Hand and Power Tools -  
http://www.jlab.org/ehs/manual/PDF/6120HandTools.pdf covers the use of tools, drills, etc 
which will be used to fabricate the detectors.
	Many operations will be performed around and in the vicinity of equipment and other 
activities in close vicinities. All required safety practices as detailed in JLAB ES&H 
Manual Chapter 6131 - Trip and Fall Protection -  
http://www.jlab.org/ehs/manual/PDF/6131TripFall.pdf will be followed.
	Much of the work will involve using ladders, work platforms, and scaffolds. 
JLAB ES&H Manual Chapter 6132 - Ladders and Scaffolds - 
http://www.jlab.org/ehs/manual/PDF/6132LaddersScaffolds.pdf will be followed at all times.
	Many operations will involve material handling using cranes and hoists of varying 
types. All such operations will be performed IAW JLAB ES&H Manual Chapter 6140 - 
Cranes and Hoists - http://www.jlab.org/ehs/manual/PDF/6140Cranes.pdf . In addition, all 
current Hall B techs have completed the Master Rigger course and each has a minimum of 
10+ experience. 
	Much of the installation will require frequent use of several types of aerial 
work platforms. All such operations will be performed IAW JLAB ES&H Manual Chapter 6147  
Aerial Work Platforms -  http://www.jlab.org/ehs/manual/PDF/6147AerialWorkPlatforms.pdf
	Instrumenting, testing, and troubleshooting the detectors may require Class 1 Mode 1 
and Class 1 Mode 2 limited testing and diagnostics and will be performed IAW
ES&H Manual Chapter 6230 - Electronic Equipment Safety -  
http://www.jlab.org/ehs/manual/PDF/6230ElectronicEquipment.pdf
Work control documents and required approvals will be completed prior to performing any 
CLAS 1 MODE 2 work.





\subsection{Quality Assurance}
	Quality Assurance begins before the procurement of components and detector 
construction 
and continues beyond detector installation and commissioning. Aside from the drift chambers, 
it includes the associated high voltage system, low voltage system, gas system, and the actual 
environment the detector lives in. There will be six identical units of each of three detector 
regions for a total of 18 drift chambers. The challenge here will be to tightly control detector 
fabrication, stringing, instrumentation, and testing while staying within the project's budget 
and schedule. The following summary outlines some of the QA procedures that we will be 
implementing.
	The electronics components will be accepted from the vendor by the JLAB Fast Electronics 
Group (FEG). The FEG will inspect, test, and clean all components in accordance with written 
procedures that they will develop. All components will be sealed in anti-static bags and stored 
in a proper environment until needed. Written certifications will be kept for all components.
	The detector frame or box will be a mechanical assembly. This assembly will be built 
on an alignment fixture. Precision positioning pins to locate reference points to the actual 
wire positions will be one of the features of the fixture. The JLAB survey group will be 
involved in this process from its inception. Each completed assembly will be surveyed to verify 
it is within tolerance. The detector frame or box will then be cleaned before it enters the 
clean room for final component assembly and stringing.
	All of the components will be inspected and cleaned prior to use as it is critical for 
maximizing detector lifetime. Written procedures will be followed and records will be kept to 
identify the cleaning and inspection status of all components. Components will be stored in 
sealed bags in the clean room where feasible.
	The CLAS12 Drift Chambers will be strung in a clean room environment. Technicians will 
wear clean room coats, gloves, shoe covers, and hair nets. Only clean gloves and clean tools 
will come into contact with the detector and it's components. All technicians who work in the 
clean room will be trained in proper clean room practices. Written cleaning procedures and 
schedules will be developed to insure proper clean room maintenance. Supervisors will frequently 
inspect all areas to verify proper practices are followed.
	Written procedures will be developed from the prototype experience for all detector 
assembly and stringing steps. Fabricators and stringers will be trained by experienced persons 
in all required steps and procedures. They will be closely supervised to ensure compliance with 
written procedures and to prevent any deviation from accepted practice. 
	At the end of each stringing shift, the wires will be tested for proper tension using a 
magnetic field and variable frequency oscillator. The frequency measured at the wires first 
fundamental for that length wire will be used to determine the wires tension. Wires with out of 
specification tension will be removed and new wires will be strung in their place. This test 
also verifies proper continuity of the wire from pin to pin. It will also identify crossed or 
twisted wires because those will show a decrease in signal amplitude.
	The next tests are a redundant check for shorted or twisted wires missed during visual 
inspection and tension testing. Prior to signal and high voltage board installation the Field 
and Guard wire layers are wire wrapped, one side at a time, high voltage side first. Once the 
high voltage side is wrapped, a multi-meter will be used to determine if there are any shorts 
between layers or between and sense wires and the wire wrapped layers. A multi-meter will also 
be used to determine shorts and twisted wires in the same layer by measuring the resistance of 
each wire between the wire wrap side and the pin on the opposite side. Since wire lengths are 
all known, the resistance of each wire is also known. Twisted wires will have much lower 
resistance than would be expected from that length wire. The high voltage boards are installed 
and connections are made. A multi meter is then used to measure the resistance between sense 
layers. There should be infinite resistance between layers, any other result indicates a short 
or twisted pair. The multi-meter is then used to measure the resistance from the signal side of 
the wire to the high voltage bus. This should read ~ 1 mega ohm, which is the value of the 
current-limiting resistor in the circuit. A reading of 0.5 mega ohms indicates a shorted or 
twisted pair.
	Final testing includes high a voltage test and cosmic ray tests before and then after 
final cabling and patch panel installation. The detector will then be stored in a temperature 
and humidity controlled area until installation. 
	 


