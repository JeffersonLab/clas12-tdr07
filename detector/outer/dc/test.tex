
\section{CLAS12 Physics Requirements}

The {\tt CLAS} detector in Hall B is being upgraded to take advantage 
of the increase of the CEBAF beam energy from 6 to 12~GeV, thus the 
new name, {\tt CLAS12}.  There are several broad areas of physics 
enquiry that drive these changes: spectroscopic studies of excited 
baryons, investigations of the influence of nuclear matter on propagating 
quarks, studies of polarized and unpolarized quark distributions, and a 
comprehensive measurement of generalized parton distributions (GPDs).  
Many of the reactions of interest are electroproduction of exclusive and 
semi-inclusive final states.  The cross sections for these processes are 
small, necessitating high-luminosity experiments.  A variety of proposed 
experiments rely on luminosities of \1035 to achieve the desired statistical 
accuracy in runs of a few months duration.  The deep exclusive reactions in 
which an electron scattering at moderate to high values of $Q^2$ results in 
a meson-baryon final state, provides the most stringent requirements for 
the {\tt CLAS12} tracking system.  A final state of a few high-momentum, 
forward-going particles (the electron as well as one or more mesons), 
combined with a moderate-momentum baryon emitted at large angles, is the 
typical event type on which the specifications of the tracking system are 
based.  

In broad strokes, the tracking system must measure forward-going particles 
down to laboratory angles as small as 5$^{\circ}$ and as large as 
40$^{\circ}$ in order to cover as much of the hadronic center-of-mass 
region as possible.  We require very good momentum and angular resolution 
for the scattered electron (on the order of $\Delta p/p = 1\%$ and 
$\Delta \theta$ = 1~mrad) in order to determine the virtual photon flux 
factor $\Gamma_v$, and hence the cross sections, to an accuracy of a few 
percent.  In addition, a momentum resolution of about 20 to 50~MeV is 
necessary in order to positively identify a missing hadron in these 
exclusive reactions.  Finally, good vertex resolution will allow detection 
of secondary decay vertices and serve as a good marker for strangeness 
production.

A tracking system capable of achieving these standards was described
in the PCDR~\cite{pcdr} and quantitatively parameterized in a ``fast'' 
Monte Carlo program~\cite{fastmc}.  A number of {\tt CLAS} collaborators 
used the existing model of the detector as described in the {\tt FASTMC} 
in proposals presented to JLab PAC-30~\cite{pac} in August 2006, the 
first PAC to consider 12-GeV proposals.

As discussed above, the {\tt CLAS12} detector is designed to detect 
semi-inclusive and exclusive events with a modest number (up to 4 or 5) 
of outgoing hadrons.  In addition to the electron and hadrons associated 
with the event of interest, the detected event will contain both 
electromagnetic and hadronic ``accidentals''.  Because the physics goal is 
to run at beam-target luminosities of \1035 or higher, we have simulated 
the expected accidental particle flux associated with these luminosities.
Using a modified version of the EGS program~\cite{egs}, we simulated the 
total hadronic and electromagnetic particle fluxes generated when an 
electron beam is incident upon a liquid-hydrogen target with a luminosity 
of \1035.  We calculated the total flux through a measurement layer during 
its active (measurement) time and multiplied by the probability of the 
particle interacting in that layer and divided by the number of cells in 
the layer to get an estimate of the fractional occupancy of that layer due 
to background. The expected layer occupancy under these conditions is
on the order of 3-4\%.  Our experience with the present {\tt CLAS} detector 
is that track-finding is highly efficient if the accidental occupancy is 
less than 4\%~\cite{dcnim}.  

\section{Overview of the CLAS12 Detector}

In Fig.~\ref{clas12} we show a CAD model picture of the proposed 
{\tt CLAS12} detector highlighting each of the detector subsystems, 
including the three forward-angle drift chambers, called Regions 1, 2, 
and 3, as well as the central silicon vertex tracker.  
Table~\ref{tracker-specs} provides a listing of the {\tt CLAS12} tracking 
system design requirements.  As noted in the previous section, the new 
detector is being designed to operate at a luminosity of \1035.  This 
higher luminosity goal (the present {\tt CLAS} detector operates at a 
maximum luminosity of 10$^{34}$~cm$^{-2}$s$^{-1}$) necessitates the use 
of a solenoidal magnet and conical absorber to shield the detector from 
M{\o}ller electrons.  To reduce interactions between this solenoidal field 
and the main {\tt CLAS} toroidal field, and to facilitate construction and 
installation of new detector elements, the torus has been re-designed.  It 
is more compact than the present torus while providing equivalent bending 
power for charged particles between 5$^{\circ}$ and 40$^{\circ}$.

%%%%%%%%%%%%%%%%%%%%%%%%%%%%%%%%%%%%%%%%%%%%%%%%%%%%%%%%%%%%%%%%%%%%%%%%%
\begin{figure}[htbp]
\vspace{11.0cm}
\special{psfile=clas12_system.eps hscale=70 vscale=70 hoffset=0 voffset=0}
\caption{\small{A three-dimensional view of the proposed {\tt CLAS12} 
detector highlighting the various subsystems.  The small inset on the left
shows the same area highlighted by the dashed box on the right but with
the solenoid removed to show the SVT system.}}
\label{clas12}
\end{figure}
%%%%%%%%%%%%%%%%%%%%%%%%%%%%%%%%%%%%%%%%%%%%%%%%%%%%%%%%%%%%%%%%%%%%%%%%%

%%%%%%%%%%%%%%%%%%%%%%%%%%%%%%%%%%%%%%%%%%%%%%%%%%%%%%%%%%%%%%%%%%%%%%%%%
\begin{table}[htbp]
\begin{center}
\begin{tabular} {||c|c||} \hline \hline
{\bf Category}      & {\bf Requirement} \\ \hline
Angular coverage    & 5$^{\circ}$ - 135$^{\circ}$ \\ \hline
Momentum resolution & 0.020 to 0.050~GeV          \\ \hline
Angular resolution  & 1~mrad (electrons), 10- 20~mrad (hadrons) \\ \hline
Luminosity     &  10$^{35}$~cm$^{-2}$s$^{-1}$ \\ \hline
\end{tabular}
\caption{\small{General specifications for {\tt CLAS12} tracking.}}
\label{tracker-specs}
\end{center}
\end{table}
%%%%%%%%%%%%%%%%%%%%%%%%%%%%%%%%%%%%%%%%%%%%%%%%%%%%%%%%%%%%%%%%%%%%%%%%%

We have designed the tracking detectors with these external constraints: 
a central solenoid of 5~T central field value and a radius available for 
tracking detectors of 25~cm, a new torus with a different aspect ratio 
but with the same number of amp-turns as the present {\tt CLAS} torus, 
an expected background rate consistent with a luminosity of \1035, and a 
separation between the ``forward'' and ``central'' regions defined to be 
at about 40$^{\circ}$; specifically the forward tracking chambers are 
designed to cover scattering angles between 5$^{\circ}$ and 40$^{\circ}$ 
and the central tracker will cover 40$^{\circ}$ to 135$^{\circ}$.  Although 
the torus cryostat will limit the azimuthal coverage to about 50\% at 
5$^{\circ}$, our goal is that the inactive portion of the drift chambers 
not further intrude into the active volume, i.e. the dead areas of the drift 
chambers (endplates, electronics, etc.) will be located in the ``shadow'' of 
the coil as viewed from the target.  For Region 2, this is not possible, but 
we shall try to minimize this dead area.

The higher beam energies available to {\tt CLAS12} mean that tracks will 
go more forward and have higher momentum than for the present {\tt CLAS} 
experiments.  We thus require better resolution from the forward drift 
chambers.  Our design should give better spatial resolution than the
present {\tt CLAS} chambers for several reasons: smaller cells and use of 
thicker (30~$\mu$m diameter) sense wires will result in a more linear drift
velocity (see Fig.~\ref{garfield}), all cells in a superlayer will be 
identical, easing the calibration, and the simpler mechanical structure 
should make these chambers easier to survey.  The other feature of higher 
energy and associated smaller cross sections, requires the use of higher 
intensity beams.  The resultant higher backgrounds is the primary motivation 
for the central solenoidal magnet and M{\o}ller absorber.  The higher 
background can also be mitigated by cells which cover a smaller angular 
range and have a smaller active time window.  

%%%%%%%%%%%%%%%%%%%%%%%%%%%%%%%%%%%%%%%%%%%%%%%%%%%%%%%%%%%%%%%%%%%%%%%%%%%
\begin{figure}[htbp]
\vspace{7.2cm}
\special{psfile=btoro1.ps hscale=50 vscale=40 hoffset=90 voffset=-5}
\caption{\small{The integral of the B field times path length along
rays from the target at various angles.}}
\label{bdl}
\end{figure}
%%%%%%%%%%%%%%%%%%%%%%%%%%%%%%%%%%%%%%%%%%%%%%%%%%%%%%%%%%%%%%%%%%%%%%%%%%%

Forward tracks (angles between 5$^{\circ}$ and 40$^{\circ}$) will be 
momentum-analyzed by passing through the magnetic field of the torus.
The magnet provides an integral Bdl of almost 3 T-m at 10$^{\circ}$, 
falling to about 1 T-m at 30$^{\circ}$ (see Fig.~\ref{bdl}).  Such forward 
tracks will first pass through six layers of a forward silicon vertex tracker 
(FVT); a silicon strip tracker with a strip pitch of 300~$\mu$m arranged with 
alternating $U$-$V$ stereo layers with a stereo angle of $\pm$9$^{\circ}$ 
located about 20~cm from the target.  These tracks will then traverse the 
high-threshold {\v C}erenkov counter (HTCC) before entering the Region 1 
drift chamber at a distance of 2.1~m from the target.  The track continues 
through the magnetic field region and its trajectory is measured in two 
more drift chambers, denoted Regions 2 and 3, respectively.  The Region 2 
and 3 chambers are located at 3.3 and 4.5~m from the target, respectively.  
The FVT should localize hits with an estimated accuracy of about 50~$\mu$m 
perpendicular to the strip direction, while the three regions of drift 
chambers are expected to have spatial resolutions of about 200~$\mu$m per 
layer.  The expected momentum resolution from such an assembly is a function 
of angle, ranging from about 0.3\% at 5$^{\circ}$ to about 1.0\% at 
30$^{\circ}$ and nearly constant as a function of momentum.  The angular 
resolution falls rapidly with increasing momentum, but should be better than 
2~mrad at a momentum of 1~GeV.

The momentum and angular resolution goals for the central tracker are set 
by the requirement that we be able to positively identify a single missing 
pion; roughly a 50~MeV resolution is required to achieve a fractional momentum 
resolution of 1\% or better at a track momentum of 5~GeV. Because the events 
of interest have about 10\% of the total charged particle momenta measured 
in the central region, the design value for the fractional momentum 
resolution of the central tracker is 5\% at a momentum of 1~GeV.  

We have studied two options for central tracks: 8 layers of silicon strips 
with alternating plus and minus stereo angle strips being one, and 4 
layers of silicon followed by an 8-layer micromegas detector being the 
other.  For forward tracks we have a conceptual design based upon 6 layers 
of silicon with alternating plus/minus stereo angle strips followed by 
three regions of drift chambers with the wires of each chamber arranged in 
two 6-layer ``superlayers'' with alternating plus/minus stereo angles.

A side-view of the proposed {\tt CLAS12} detector (cut through the beamline) 
is shown in Fig.~\ref{clas12side}.  A solenoidal magnet contains the 
target, the central silicon vertex tracker (SVT), and the central 
time-of-flight system (CTOF), as well as the M{\o}ller absorber.  Charged 
particles with emission angles greater than 40$^{\circ}$ follow helical 
paths through the 8 layers of the SVT, which are arranged into four $U$-$V$ 
modules with ``$U$'' and ``$V$'' referring to strip orientations of 
$\pm$1.5$^{\circ}$, respectively.  The time resolution of the CTOF 
($\sim$200~ps) will enable particle identification of the charged tracks, 
as well as allowing a very efficient rejection of out-of-time accidentals.

%%%%%%%%%%%%%%%%%%%%%%%%%%%%%%%%%%%%%%%%%%%%%%%%%%%%%%%%%%%%%%%%%%%%%%%%%%%
\begin{figure}[htbp]
\vspace{12.0cm}
\special{psfile=clas12_sideview.eps hscale=60 vscale=60 hoffset=-15 voffset=350 angle=-90}
\caption{\small{A side-view of the {\tt CLAS12} detector showing the
different detector subsystems, and highlighting the central and forward
vertex trackers and drift chambers.}}
\label{clas12side}
\end{figure}
%%%%%%%%%%%%%%%%%%%%%%%%%%%%%%%%%%%%%%%%%%%%%%%%%%%%%%%%%%%%%%%%%%%%%%%%%%%

Forward-going tracks (with angles less than 40$^{\circ}$) pass through
the forward vertex tracker (FVT), which is arranged into three $U$-$V$ 
modules with the ``$U$'' and ``$V$'' strips oriented at $\pm$9$^{\circ}$ 
angles.  Following the FVT is a high-threshold {\v C}erenkov counter (HTCC), 
a threshold {\v C}erenkov counter filled with atmospheric CO$_2$ gas, 
designed for electron identification.  It will effectively reject other 
charged particles, up to the momentum threshold for pions ($\sim$6~GeV).  
Following the HTCC is the toroidal magnet which, besides providing the 
magnetic field for momentum analysis, supports three regions of drift 
chambers (denoted Regions 1, 2, and 3) for charged track detection.  
Regions 1 and 3 are attached to the front and rear faces of the toroid, 
while Region 2 is located between the torus coils.

Following the torus-drift chamber assembly is the forward detector,
consisting of a low-threshold {\v C}erenkov counter (LTCC) for charged 
hadron identification, the main time-of-flight (TOF) system, and the 
pre-shower calorimeter (PCAL) and main electromagnetic calorimeter (EC).  
The TOF is used to define the main event start time and to enhance charged 
hadron identification, while the PCAL and EC are used for electron and
photon detection.

\section{Central Vertex Tracker}

There are two options for a central tracker: an 8-layer silicon strip 
detector or a 4-layer silicon strip detector surrounded by an 8-layer 
micromegas chamber.  The 8-layer silicon strip detector is the standard 
option. The silicon strips are laid out with a $\pm$1.5$^{\circ}$ stereo 
angle as are alternate layers in the micromegas tracker.  Each design 
consists of concentric shells of measurement layers at successively greater 
radius. The $r-\phi$ coordinate is measured by the strip coordinate in 
either case. The $r-z$ coordinate is measured by a stereo angle projection. 
Table~\ref{central-silicon-specs} lists the specifications for the SVT-only
design.  Fig.~\ref{centralsvt1} highlights the layout of the central SVT and 
Fig.~\ref{centralsvt2} provides an end view of the device to show the
four individual layers.

%%%%%%%%%%%%%%%%%%%%%%%%%%%%%%%%%%%%%%%%%%%%%%%%%%%%%%%%%%%%%%%%%%%%%%%%%
\begin{table}[htbp]
\begin{center}
\begin{tabular} {||c|c|c|c|c||} \hline \hline
{\bf layer} & {\bf no. sides} & {\bf no. readout} & {\bf radius} & {\bf length}\\ \hline
1 - 2 & 7 & 3580 & 4.4 cm & 11.1 cm    \\ \hline
3 - 4 & 12 & 6144 & 8.0 cm & 22.2 cm    \\ \hline
5 - 6 & 17 & 8704 & 11.4 cm & 33.3 cm    \\ \hline
7 - 8 & 22 & 11264 & 14.7 cm & 44.4 cm    \\ \hline
\end{tabular}
\caption{\small{Specifications for the central SVT.}}
\label{central-silicon-specs}
\end{center}
\end{table}
%%%%%%%%%%%%%%%%%%%%%%%%%%%%%%%%%%%%%%%%%%%%%%%%%%%%%%%%%%%%%%%%%%%%%%%%%

%%%%%%%%%%%%%%%%%%%%%%%%%%%%%%%%%%%%%%%%%%%%%%%%%%%%%%%%%%%%%%%%%%%%%%%%%%%
\begin{figure}[htbp]
\vspace{7.0cm}
\special{psfile=svt.eps hscale=40 vscale=40 hoffset=90 voffset=-10}
\caption{\small{A CAD model of the central SVT looking along the beam
direction.  The individual sensors overlap in eight layers forming four
superlayers of adjacent $\pm$1.5$^{\circ}$ $U$-$V$ strip orientations.}}
\label{centralsvt1}
\end{figure}
%%%%%%%%%%%%%%%%%%%%%%%%%%%%%%%%%%%%%%%%%%%%%%%%%%%%%%%%%%%%%%%%%%%%%%%%%%%

%%%%%%%%%%%%%%%%%%%%%%%%%%%%%%%%%%%%%%%%%%%%%%%%%%%%%%%%%%%%%%%%%%%%%%%%%%%
\begin{figure}[htbp]
\vspace{7.5cm}
\special{psfile=svt_cutview.eps hscale=30 vscale=30 hoffset=120 voffset=-5}
\caption{\small{An end view of the central SVT looking along the beam
direction highlighting the layout of the staves into polygons for each
of the four layers.}}
\label{centralsvt2}
\end{figure}
%%%%%%%%%%%%%%%%%%%%%%%%%%%%%%%%%%%%%%%%%%%%%%%%%%%%%%%%%%%%%%%%%%%%%%%%%%%

The SVT is designed to operate in the 5~T field of the solenoid.  The
current design, shown in Fig.~\ref{centralsvt2}, employs four 2-layer
superlayers with a butt-joint design between the staves that make up
each face of the polygons for each layer.  As the sensors on each stave
are oriented at a $\pm$1.5$^{\circ}$ stereo angle, each layer has a
small dead area where the sensors do not overlap.  To combat the impact
of this feature on the resolution, successive layers of the SVT are
clocked with respect to each other.

The active strips on each layer of the SVT are laid out with a 75~$\mu$m,
with neighboring strips capacitively coupled together, effectively
forming a 150~$\mu$m readout pitch.  The momentum resolution of the
SVT is given roughly by $\delta p/p \sim (r_{out} - r_{in})^{-2}$.  The
strips will be read out using the SVX4 custom 128-channel analog to
digital converter chips used by the D0 and CDF experiments at Fermilab
to read out their silicon strip detectors.  Fig.~\ref{strip_layout}
provides a picture of two wire-bonded silicon sensors, the $U$ and $V$
strip orientations, the SVX4 chips, and the associated local readout
electronics.

\vskip 3.0cm

%%%%%%%%%%%%%%%%%%%%%%%%%%%%%%%%%%%%%%%%%%%%%%%%%%%%%%%%%%%%%%%%%%%%%%%%%%%
\begin{figure}[hb]
\vspace{7.0cm}
\special{psfile=strip_layout.eps hscale=40 vscale=40 hoffset=15 voffset=-5}
\caption{\small{A photograph of a prototype of a portion of the SVT,
including two wire-bonded silicon sensors, the SVX4 readout chips, and the
associated local readout electronics.}}
\label{strip_layout}
\end{figure}
%%%%%%%%%%%%%%%%%%%%%%%%%%%%%%%%%%%%%%%%%%%%%%%%%%%%%%%%%%%%%%%%%%%%%%%%%%%

\vfil
\eject

\section{Forward Vertex Tracker}

The forward silicon tracker consists of six single-sided silicon strip 
detector planes (which form three double $U$-$V$ layers.  Each plane is 
formed from twenty pieces, each roughly trapezoidal in shape with strips 
on one plane running parallel to one of the sides of the triangle with 
readout on the outer-radius side and strips from the other plane being 
parallel to the other side of the triangle. In this way, the strips cover 
the entire area of the triangular surface and always run to the ``outside'' 
where the readout electronics is located.  The strips vary in length, from 
the longest which is adjacent to its parallel side down to a short strip 
opposite. See Fig.~\ref{fwdsvt} for a layout of the detector and 
Table~\ref{fwd-silicon-specs} for the detector specifications.  The readout 
will employ the SVX4 chips discussed above for the SVT.

%%%%%%%%%%%%%%%%%%%%%%%%%%%%%%%%%%%%%%%%%%%%%%%%%%%%%%%%%%%%%%%%%%%%%%%%%
\begin{table}[htbp]
\begin{center}
\begin{tabular} {||c|c|c|c||} \hline \hline
&{\bf Region 1}    &  {\bf Region 2} & {\bf Region 3}\\ \hline
dist. from target  & 18 cm & 19 cm   & 20 cm \\ \hline
num. $U/V$ strips  & 5000 & 5000  & 5000 \\ \hline
stereo angle       & +/- 9$^{\circ}$ & +/- 9$^{\circ}$   & +/- 9$^{\circ}$ \\ \hline
strip pitch        & 0.0075 cm & 0.0075 cm & 0.0075 cm \\ \hline
\end{tabular}
\caption{\small{Specifications for the forward SVT.}}
\label{fwd-silicon-specs}
\end{center}
\end{table}
%%%%%%%%%%%%%%%%%%%%%%%%%%%%%%%%%%%%%%%%%%%%%%%%%%%%%%%%%%%%%%%%%%%%%%%%%

%%%%%%%%%%%%%%%%%%%%%%%%%%%%%%%%%%%%%%%%%%%%%%%%%%%%%%%%%%%%%%%%%%%%%%%%%%%
\begin{figure}[htbp]
\vspace{7.0cm}
\special{psfile=fsvt.eps hscale=55 vscale=55 hoffset=60 voffset=0}
\caption{\small{A view of the forward SVT showing its arrangement into 
three superlayers of two $\pm$9$^{\circ}$ $U$-$V$ stereo strips.}}
\label{fwdsvt}
\end{figure}
%%%%%%%%%%%%%%%%%%%%%%%%%%%%%%%%%%%%%%%%%%%%%%%%%%%%%%%%%%%%%%%%%%%%%%%%%%%

\section{Drift Chambers}

The overall tracking requirements (0.5 -  1\% fractional momentum resolution 
at 5~GeV and efficient tracking at a luminosity of \1035) are the main
drivers for drift chamber design.  Because the {\tt CLAS} drift chamber
system~\cite{dcnim} has operated successfully for 8 years, we plan to 
re-use many of the design concepts and most of the utility infrastructure.  
In particular, we plan to re-use the present gas mixing and handling system, 
the high-voltage and low-voltage systems, the FASTBUS TDC system and the 
post-amplifier/multiplexer systems.  The construction project thus consists 
of new chambers, on-board electronics, and on-board jumper cables.

The required better resolution of the tracking system compared to the 
present {\tt CLAS} detector will be achieved by chambers with a smaller 
cell size and thus inherently better spatial resolution, and the smaller 
physical size and thus more accurate placement of the chambers with respect 
to the magnet coils.  The design of the forward chambers is very similar to 
the present {\tt CLAS} chambers.  The cell design is hexagonal and the sense 
wire layers are arranged in 6-layer superlayers as in the present chambers. 
The major difference is that the cells are approximately two thirds as big 
as the present chambers allowing efficient tracking at higher luminosities 
because the accidental occupancy from particles not associated with the 
event is smaller. Table~\ref{fwd-dc-design-parms} lists the design 
parameters and Fig.~\ref{fwddc} shows a single sector of the new Region 1
chamber.  For the purposes of simulating track resolutions, we assumed that 
the position resolution of the individual drift cells would be 250~$\mu$m.  
For reference, the present {\tt CLAS} chambers have resolutions of 310, 315,
and 380 $\mu m$ for R1, R2, and R3, respectively.

%%%%%%%%%%%%%%%%%%%%%%%%%%%%%%%%%%%%%%%%%%%%%%%%%%%%%%%%%%%%%%%%%%%%%%%%%%%
\begin{figure}[htbp]
\vspace{9.0cm}
\special{psfile=r1.eps hscale=47 vscale=47 hoffset=120 voffset=0}
\caption{\small{A view of one sector of the Region 1 drift chamber 
showing the projection of the wires onto the chamber endplate.}}
\label{fwddc}
\end{figure}
%%%%%%%%%%%%%%%%%%%%%%%%%%%%%%%%%%%%%%%%%%%%%%%%%%%%%%%%%%%%%%%%%%%%%%%%%%%

%%%%%%%%%%%%%%%%%%%%%%%%%%%%%%%%%%%%%%%%%%%%%%%%%%%%%%%%%%%%%%%%%%%%%%%%%
\begin{table}[htbp]
\begin{center}
\begin{tabular} {||c|c|c|c||} \hline \hline
&{\bf Region 1}      &  {\bf Region 2} & {\bf Region 3}\\ \hline
dist. from target    & 2.1 m & 3.3 m   & 4.5 m \\ \hline
num. of superlayers  & 2 & 2   & 2 \\ \hline
layers/superlayer    & 6 & 6   & 6 \\ \hline
wires/layer          & 112 & 112   & 112 \\ \hline
cell size            & 0.86 cm & 1.36 cm   & 1.88 cm \\ \hline
assumed resolution per wire  & 0.025 cm & 0.025 cm   & 0.025 cm \\ \hline
\end{tabular}
\caption{\small{Design parameters for the {\tt CLAS12} drift chambers.}}
\label{fwd-dc-design-parms}
\end{center}
\end{table}
%%%%%%%%%%%%%%%%%%%%%%%%%%%%%%%%%%%%%%%%%%%%%%%%%%%%%%%%%%%%%%%%%%%%%%%%%

The chambers differ from the present {\tt CLAS} chambers in a number of 
ways.  Successive superlayers have their wires arranged with a plus or 
minus 6$^{\circ}$ stereo angle; the present arrangement has an axial layer and 
a 6$^{\circ}$ stereo layer.  For the present {\tt CLAS} detector, the $\phi$ 
resolution is about four times larger than the $\theta$ resolution.  To 
have more equal resolution in the two angles, we decided that we needed to 
double our effective stereo angle in order to improve the  $\phi$ resolution.
Unlike the present chambers, all of the wires in one of the superlayers are
strictly parallel, and in a plane perpendicular to the wire direction form
perfect hexagons.  This should allow a more accurate drift velocity 
calibration than the current design with its layer-to-layer increase in 
cell size.  The choice of gas; a 92:08 Argon:CO$_2$ mixture is a small 
departure from our present 90:10 mixture and should result in a higher and 
more constant drift velocity.  We plan to run with a gas gain of $5\cdot10^4$.

Another departure from the present design is to design every chamber (in 
all three regions) to be self-supporting in order to ensure that they are 
easy to install and remove for maintenance.  In the present {\tt CLAS}, the 
Region 1 chambers are all bound together into a single unit in order to 
maintain the wire tension without excessively thick endplates, and the 
Region 2 chambers are actually mounted onto the magnet cryostat with the 
cryostat itself maintaining the internal wire tension.  None of the present 
Region 1 or 2 chambers can be accessed individually without a lengthy 
``tension-transfer'' process.  To avoid this, we are designing all chambers 
to be self-supporting like our present Region 3 chambers.  The key will be 
ultra-stiff endplates which obtain their stiffness by a flanged design.  

A third design change is to use 30~$\mu$m diameter sense wire rather than 
the more common 20~$\mu$m wire. Our choice of wire is 30~$\mu$m diameter, 
gold-plated tungsten for the sense wires, 140~$\mu$m diameter, gold-plated 
aluminum for the field wires and 140~$\mu$m diameter, stainless steel for
the guard wires.   This should make the chamber more robust to wire 
breakages.  Higher voltages will be required to achieve the same gas gain, 
and the resulting higher electric field in the drift cells will result in 
a more nearly constant drift velocity which should be easier to calibrate.
Prototypes are being built to study possible negative side-effects of the 
higher voltage operation such as leakage currents on the circuit boards 
and/or higher rates of cathode emission from the field wire surfaces.
Fig.~\ref{garfield} shows GARFIELD calculations for a Region 3 drift cell
with both a 20~$\mu$m and a 30~$\mu$m diameter sense wire.  Here the
cells with the thicker sense wire will have a significantly higher drift 
velocity which is desirable to reduce the time window, and hence the chamber 
occupancy.

%%%%%%%%%%%%%%%%%%%%%%%%%%%%%%%%%%%%%%%%%%%%%%%%%%%%%%%%%%%%%%%%%%%%%%%%%%%
\begin{figure}[htbp]
\vspace{12.0cm}
\special{psfile=garfield1.eps hscale=30 vscale=27 hoffset=50 voffset=165}
\special{psfile=garfield2.eps hscale=30 vscale=27 hoffset=50 voffset=-5}
\special{psfile=garfield3.eps hscale=30 vscale=27 hoffset=240 voffset=165}
\special{psfile=garfield4.eps hscale=30 vscale=27 hoffset=240 voffset=-5}
\caption{\small{GARFIELD calculations of the electric field lines (top)
and drift time vs. drift distance (bottom) for a Region 3 drift cell.  The 
left plots show the configuration with a 20~$\mu$m diameter sense wire and 
the right plots show the configuration with a 30~$\mu$m diameter sense wire.
The high voltages were set to provide the same gas gain for each
configuration.}}
\label{garfield}
\end{figure}
%%%%%%%%%%%%%%%%%%%%%%%%%%%%%%%%%%%%%%%%%%%%%%%%%%%%%%%%%%%%%%%%%%%%%%%%%%%

\section{Expected Detector Performance}

We studied the position, angle, and momentum resolution for a number of 
possible detector options using MOMRES~\cite{momres}.  Our procedure was 
to produce a MOMRES input file that characterized the detector position, 
material thickness, and estimated hit resolution for a particular track 
angle.  We also produced a B-field file which was a tabulation of the 
B-field strength vs. path length for a particular track angle.  These, 
and the desired range of momenta, were the inputs to MOMRES. MOMRES 
calculated the expected components of the resolutions due to multiple 
scattering and measurement resolution, respectively.  We fit these outputs 
to the expected kinematic form. From these fits we extracted two parameters 
(sig1 and sig2) for each of the three terms ($dp/p$, $d\theta$, $dx$). 
These six parameters summarize the output of MOMRES. In addition, we 
calculated the angle resolution in the non-bend plane in a manner 
analogous to that of MOMRES, using estimates for the effects of multiple 
scattering and measurement error, fitting the resulting smeared trajectory 
by a straight line, and extracting the $\sigma$1 and $\sigma$2 parameters 
that characterize the resolution in this out-of-bend-plane angle.  Thus, 
eight parameters for each value of track angle fully characterize the 
tracking resolution for any one detector option.  Fig.~\ref{30degres} 
shows the momentum, $\theta$, $\phi$, and vertex resolutions for three 
detector options for tracks emitted at 30$^{\circ}$.  Fig.~\ref{90degres} 
shows the momentum, $\theta$, $\phi$, and vertex resolutions for three 
detector options for tracks emitted at 90$^{\circ}$, i.e. into the central 
detector. 

%%%%%%%%%%%%%%%%%%%%%% Figure : 30deg resolution %%%%%%%%%%%%%%%%%%%%
\begin{figure}[htpb]
\vspace{9.0cm}
\special{psfile=30deg_resln.eps hscale=50 vscale=50 hoffset=90 voffset=0}
\caption{\small{Resolution simulated using MOMRES plotted vs. particle 
momentum for 30$^{\circ}$ tracks.  Sub-figures a, b, c, and d show the 
momentum, $\theta$, $\phi$, and vertex resolutions, respectively.  Three 
options are shown: solid- 3 SVT planes + CC + 3 DC planes; dashed- CC + 
3 DC planes; dotted - 3 DC planes.}}
\label{30degres}
\end{figure}
%%%%%%%%%%%%%%%%%%%%%%%%%%%%%%%%%%%%%%%%%%%%%%%%%%%%%%%%%%%%%%%%%%%%%%%%%

Fig.~\ref{resol} provides a few more results of these studies to
parameterize and understand the resolution expected for the {\tt CLAS12}
tracking system.  The plots show the expected angular resolution, the 
expected position resolution, and the expected momentum resolution
for different angle bins, each as a function of momentum.  These plots
were made using the current design of the tracking system.

Fig.~\ref{resplot} shows the reconstructed momentum for 1000 events
each containing a 1~GeV track emitted at a 45$^{\circ}$ polar angle
but at random values of azimuthal angle and with random multiple
scattering.  The dark histogram shows the expected reconstructed
spectrum at a luminosity of 10$^{35}$~cm$^{-2}$s$^{-1}$.  The gray
histogram and the solid line show the resolution with 40 times the
nominal background (consisting of random SVT hits).  The gray histogram
was constructed using the nominal $\pm$1.5$^{\circ}$ $U$-$V$ strip
pitch, and the solid line was made increasing the strip pitch by a
factor of two above the nominal.  The studies serve to validate the
basic SVT design parameters.

%%%%%%%%%%%%%%%%%%%%%% Figure : 90deg resolution %%%%%%%%%%%%%%%%%%%%
\begin{figure}[htbp]
\vspace{8.0cm}
\special{psfile=90deg_resln.eps hscale=50 vscale=50 hoffset=90 voffset=-5}
\caption{\small{Resolution simulated using MOMRES plotted vs. particle 
momentum for 90$^{\circ}$ tracks.  Sub-figures a, b, c, and d show the 
momentum, $\theta$, $\phi$, and vertex resolutions, respectively.  Three 
options are shown: solid- 3 SVT planes; dashed- 16 DC-stereo planes; 
dotted - 8 DC-cathode pad planes.}}
\label{90degres}
\end{figure}
%%%%%%%%%%%%%%%%%%%%%%%%%%%%%%%%%%%%%%%%%%%%%%%%%%%%%%%%%%%%%%%%%%%%%%%%%

%%%%%%%%%%%%%%%%%%%%%%%%%%%%%%%%%%%%%%%%%%%%%%%%%%%%%%%%%%%%%%%%%%%%%%%%%
\begin{figure}[htbp]
\vspace{12.0cm}
\special{psfile=resol_1.eps hscale=35 vscale=30 hoffset=15 voffset=160}
\special{psfile=resol_2.eps hscale=35 vscale=30 hoffset=225 voffset=160}
\special{psfile=resol_3.eps hscale=35 vscale=30 hoffset=115 voffset=-10}
\caption{\small{Simulation results for the {\tt CLAS12} tracking system
showing the expected angular resolution, position resolution, and
momentum resolution for different angle bins as a function of momentum.}
\label{resol}}
\end{figure}
%%%%%%%%%%%%%%%%%%%%%%%%%%%%%%%%%%%%%%%%%%%%%%%%%%%%%%%%%%%%%%%%%%%%%%%%%

%%%%%%%%%%%%%%%%%%%%%%%%%%%%%%%%%%%%%%%%%%%%%%%%%%%%%%%%%%%%%%%%%%%%%%%%%
\begin{figure}[htpb]
\vspace{6.5cm}
\special{psfile=centralrates1.eps hscale=120 vscale=85 hoffset=75 voffset=0}
\caption{\small{A plot of reconstructed momentum (GeV) for 1000 events,
each containing a 1~GeV track emitted at a 45$^{\circ}$ polar angle
but at random values of azimuthal angle and with random multiple
scattering.  The solid histogram is for events with the background 
expected at a luminosity of \1035.  The gray histogram ($\pm$1.5$^{\circ}$
strip pitch) and solid line ($\pm$3$^{\circ}$ strip pitch) were
simulations run increasing the background by a factor of 40.}}
\label{resplot}
\end{figure}
%%%%%%%%%%%%%%%%%%%%%%%%%%%%%%%%%%%%%%%%%%%%%%%%%%%%%%%%%%%%%%%%%%%%%%%%%

\section{Expected Physics Performance}

We used a series of programs to calculate the acceptance and reconstructed 
physics parameters for event types of interest.  The program {\it clasev}
\cite{clasev} served as an event generator and analysis program. 
Depending on the value of input flags, it generates certain types of 
events; that is, it produces a set of 4-momenta for the primary hadrons 
in the hadronic center-of-mass and allows some of them to decay into the 
final-state hadrons and transforms their momenta to the lab system.  For 
each final-state track, it calls {\tt FASTMC} to determine if the track 
falls within a fiducial acceptance window and to determine its final, 
smeared lab momentum.  It then produces selected physics analysis variables 
such as missing mass from calculations involving the smeared momenta of 
those tracks that were accepted. 

Fig.~\ref{massplot} shows the expected missing mass resolution 
expected for {\tt CLAS12} from simulation studies based on the current
design specifications for a number of different reactions.  For all
cases studied the results are quite encouraging in terms of identifying
the missing particle cleanly for each reaction.

%%%%%%%%%%%%%%%%%%%%%%%%%%%%%%%%%%%%%%%%%%%%%%%%%%%%%%%%%%%%%%%%%%%%%%%%%%%
\begin{figure}[htbp]
\vspace{14.0cm}
\special{psfile=mm1.ps hscale=39 vscale=36 hoffset=-10 voffset=160}
\special{psfile=mm2.ps hscale=39 vscale=36 hoffset=225 voffset=160}
\special{psfile=mm3.ps hscale=40 vscale=37 hoffset=-10 voffset=-55}
\special{psfile=mm4.ps hscale=40 vscale=37 hoffset=230 voffset=-55}
\caption{\small{Simulation results highlighting the expected missing
mass resolution of {\tt CLAS12} with the nominal design specifications for
the drift chambers.  Shown are the spectra for the reactions
$ep \to e'\pi+X$ (UL), $ep \to e'K^+X$ (UR), $ep \to e'p\pi^+X$ (LL),
and $ep \to e\rho^+X$ (LR).}}
\label{massplot}
\end{figure}
%%%%%%%%%%%%%%%%%%%%%%%%%%%%%%%%%%%%%%%%%%%%%%%%%%%%%%%%%%%%%%%%%%%%%%%%%%%

\vfil
\eject

\begin{thebibliography}{99}

\bibitem{pcdr} 
Pre-Conceptual Design Report the Science and Experimental Equipment
for the 12~GeV Upgrade of CEBAF, see www.jlab.org/12GeV/collaboration.html.

\bibitem{fastmc}
FASTMC is a fast parametric Monte Carlo of {\tt CLAS12}; the code and 
documentation are contained in CVS under 12GeV/fastmc.

\bibitem{pac} 
JLab PAC30 meeting, Aug. 2006; see www.jlab.org/exp\_prog/PACpage/pac.html.

\bibitem{egs}
R.L. Ford and W.R. Nelson, The EGS code system: Computer program for the 
Monte Carlo simulation of electromagnetic cascade showers, SLAC-210 (1978).
 
\bibitem{dcnim} 
M.D. Mestayer {\it et al.}, Nucl. Inst. and Meth. A {\bf 449}, 81 (2000).

\bibitem{momres} 
B.A. Mecking, The MOMRES package is used as an input to FASTMC and is
a program to estimate the momentum and angle resolution given a table of 
material thicknesses, position resolution, and integral magnetic field.
The progam is in CVS under 12GeV/momres.

\bibitem{clasev} 
H. Avakian, General event generator.  The program is in CVS under
12GeV/fastmc/clasev.

\end{thebibliography}

\end{document}

