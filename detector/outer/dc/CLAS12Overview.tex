\section{Overview of the CLAS12 Detector}

In Fig.~\ref{clas12} we show a CAD model picture of the proposed 
{\tt CLAS12} detector highlighting each of the detector subsystems, 
including the three forward-angle drift chambers, called Regions 1, 2, 
and 3, as well as the central silicon vertex tracker.  
Table~\ref{tracker-specs} provides a listing of the {\tt CLAS12} tracking 
system design requirements.  {\tt CLAS12} is being designed to operate at 
a luminosity of \1035.  This higher luminosity goal (the present {\tt CLAS} 
detector operates at a maximum luminosity of 10$^{34}$~cm$^{-2}$s$^{-1}$) 
necessitates the use of a solenoidal magnet and conical absorber to shield 
the detector from M{\o}ller electrons.  To reduce interactions between this 
solenoidal field and the main {\tt CLAS} toroidal field, and to facilitate 
construction and installation of new detector elements, the torus has been 
re-designed.  It is more compact than the present torus while providing 
equivalent bending power for charged particles between 5$^{\circ}$ and 
40$^{\circ}$.

\vfil
\eject

%%%%%%%%%%%%%%%%%%%%%%%%%%%%%%%%%%%%%%%%%%%%%%%%%%%%%%%%%%%%%%%%%%%%%%%%%
\begin{figure}[htbp]
\vspace{11.0cm}
\special{psfile=clas12_system.eps hscale=70 vscale=70 hoffset=0 voffset=0}
\caption{\small{A three-dimensional view of the proposed {\tt CLAS12} 
detector highlighting the various subsystems.  The small inset on the left
shows the same area highlighted by the dashed box on the right but with
the solenoid removed to show the SVT system.}}
\label{clas12}
\end{figure}
%%%%%%%%%%%%%%%%%%%%%%%%%%%%%%%%%%%%%%%%%%%%%%%%%%%%%%%%%%%%%%%%%%%%%%%%%

%%%%%%%%%%%%%%%%%%%%%%%%%%%%%%%%%%%%%%%%%%%%%%%%%%%%%%%%%%%%%%%%%%%%%%%%%
\begin{table}[htbp]
\begin{center}
\begin{tabular} {||c|c||} \hline \hline
{\bf Category}      & {\bf Requirement} \\ \hline
Angular coverage    & 5$^{\circ}$ - 135$^{\circ}$ \\ \hline
Momentum resolution & 0.020 to 0.050~GeV          \\ \hline
Angular resolution  & 1~mrad (electrons), few~mrads (hadrons) \\ \hline
Luminosity     &  10$^{35}$~cm$^{-2}$s$^{-1}$ \\ \hline
\end{tabular}
\caption{\small{General specifications for {\tt CLAS12} tracking.}}
\label{tracker-specs}
\end{center}
\end{table}
%%%%%%%%%%%%%%%%%%%%%%%%%%%%%%%%%%%%%%%%%%%%%%%%%%%%%%%%%%%%%%%%%%%%%%%%%

We have designed the tracking detectors with these external constraints: 
a central solenoid of 5~T central field value and a radius available for 
tracking detectors of 25~cm, a new torus with a different aspect ratio 
but with the same number of amp-turns as the present {\tt CLAS} torus, 
an expected background rate consistent with a luminosity of \1035, and a 
separation between the ``forward'' and ``central'' regions defined to be 
at about 40$^{\circ}$; specifically the forward tracking chambers are 
designed to cover scattering angles between 5$^{\circ}$ and 40$^{\circ}$ 
and the central tracker will cover 40$^{\circ}$ to 135$^{\circ}$.  Although 
the torus cryostat will limit the azimuthal coverage to about 50\% at 
5$^{\circ}$, our goal is that the inactive portion of the drift chambers 
not further intrude into the active volume, i.e. the dead areas of the drift 
chambers (endplates, electronics, etc.) will be located in the ``shadow'' 
of the coil as viewed from the target. 

The higher beam energies available to {\tt CLAS12} mean that tracks will 
go more forward and have higher momentum than for the present {\tt CLAS} 
experiments.  We thus require better resolution from the forward drift 
chambers.  Table~\ref{dc-specs} summarizes the specifications for the
forward tracking system.

%%%%%%%%%%%%%%%%%%%%%%%%%%%%%%%%%%%%%%%%%%%%%%%%%%%%%%%%%%%%%%%%%%%%%%%%%
\begin{table}[htbp]
\begin{center}
\begin{tabular} {||c|c||} \hline \hline
{\bf Category}      & {\bf Requirement} \\ \hline
Angular coverage    & 5$^{\circ}$ - 40$^{\circ}$ \\ \hline
Azimuthal coverage & 50$^{\circ}$ of 2$\pi$ at 5$^{\circ}$ \\ \hline
Momentum resolution & 1\%         \\ \hline
Angular resolution ($\theta$)  & 1~mrad  \\ \hline
Angular resolution (sin$\theta d\phi$)  & 2~mrad  \\ \hline
Luminosity     &  10$^{35}$~cm$^{-2}$s$^{-1}$ \\ \hline
\end{tabular}
\caption{\small{Specifications for {\tt CLAS12} drift chamber system.}}
\label{dc-specs}
\end{center}
\end{table}
%%%%%%%%%%%%%%%%%%%%%%%%%%%%%%%%%%%%%%%%%%%%%%%%%%%%%%%%%%%%%%%%%%%%%%%%%

%%%%%%%%%%%%%%%%%%%%%%%%%%%%%%%%%%%%%%%%%%%%%%%%%%%%%%%%%%%%%%%%%%%%%%%%%
\begin{figure}[htbp]
\vspace{6.5cm}
\special{psfile=btoro1.ps hscale=50 vscale=40 hoffset=90 voffset=-5}
\caption{\small{The integral of the B field times path length along
rays from the target at various angles.}}
\label{bdl}
\end{figure}
%%%%%%%%%%%%%%%%%%%%%%%%%%%%%%%%%%%%%%%%%%%%%%%%%%%%%%%%%%%%%%%%%%%%%%%%%

Forward tracks (angles between 5$^{\circ}$ and 40$^{\circ}$) will be 
momentum-analyzed by passing through the magnetic field of the torus.
The magnet provides an integral Bdl of almost 3 T-m at 10$^{\circ}$, 
falling to about 1 T-m at 30$^{\circ}$ (see Fig.~\ref{bdl}).  Such forward 
tracks will first pass through six layers of a forward silicon vertex tracker 
(FVT); a silicon strip tracker with a strip pitch of 150~$\mu$m arranged with 
alternating $U$-$V$ stereo layers with a stereo angle of $\pm$9$^{\circ}$ 
located about 20~cm from the target.  These tracks will then traverse the 
high-threshold {\v C}erenkov counter (HTCC) before entering the Region 1 
drift chamber at a distance of 2.1~m from the target.  The track continues 
through the magnetic field region and its trajectory is measured in two 
more drift chambers, denoted Regions 2 and 3, respectively.  The Region 2 
and 3 chambers are located at 3.3 and 4.5~m from the target, respectively.  
The FVT should localize hits with an estimated accuracy of about 40~$\mu$m 
perpendicular to the strip direction, while the three regions of drift 
chambers are expected to have spatial resolutions of about 200~$\mu$m per 
layer.  The expected momentum resolution from such an assembly is a function 
of angle, ranging from about 0.3\% at 5$^{\circ}$ to about 1.0\% at 
30$^{\circ}$ and nearly constant as a function of momentum.  The angular 
resolution falls rapidly with increasing momentum, but should be better than 
2~mrad at a momentum of 1~GeV (see Section~\ref{simulation} on simulation for 
details).

The momentum and angular resolution goals for the central tracker are 
mainly set by the requirement that we be able to positively identify a 
single missing pion.  This translates into a 5\% fractional momentum 
resolution for a central track with 1~GeV momentum.  A side-view of the 
proposed {\tt CLAS12} detector (cut through the beamline) is shown in 
Fig.~\ref{clas12side}.  A solenoidal magnet contains the target, the 
central silicon vertex tracker (SVT), and the central time-of-flight 
system (CTOF), as well as the M{\o}ller absorber.  Charged particles with 
emission angles greater than 40$^{\circ}$ follow helical paths through the 
8 layers of the SVT, which are arranged into four $U$-$V$ modules with 
``$U$'' and ``$V$'' referring to strip orientations of $\pm$1.5$^{\circ}$, 
respectively.  The time resolution of the CTOF ($\sim$200~ps) will enable 
particle identification of the charged tracks, as well as allowing a very 
efficient rejection of out-of-time accidentals.

%%%%%%%%%%%%%%%%%%%%%%%%%%%%%%%%%%%%%%%%%%%%%%%%%%%%%%%%%%%%%%%%%%%%%%%%%%%
\begin{figure}[htbp]
\vspace{11.4cm}
\special{psfile=clas12_sideview.eps hscale=60 vscale=60 hoffset=-15 voffset=350 angle=-90}
\caption{\small{A side-view of the {\tt CLAS12} detector showing the
different detector subsystems, and highlighting the central and forward
vertex trackers and drift chambers.}}
\label{clas12side}
\end{figure}
%%%%%%%%%%%%%%%%%%%%%%%%%%%%%%%%%%%%%%%%%%%%%%%%%%%%%%%%%%%%%%%%%%%%%%%%%%%

Following the torus-drift chamber assembly is the forward detector,
consisting of a low-threshold {\v C}erenkov counter (LTCC) for charged 
hadron identification, the main time-of-flight (TOF) system, and the 
pre-shower calorimeter (PCAL) and main electromagnetic calorimeter (EC).  
The TOF is used to define the main event start time and to enhance charged 
hadron identification, while the PCAL and EC are used for electron and
photon detection.

