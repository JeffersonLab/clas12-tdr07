%%%%%%%%%%%%%%%%%%%%%%%%%%%%%%%%%%%%%%%%%%%%%%%%%%%%%%%%%%%%%%%%%%%%%%%%%%%
\begin{figure}[htpb]
\vspace{6.2cm} 
\special{psfile=cellsize.eps hscale=50 vscale=50 hoffset=60 voffset=-5}
\caption{\small{Geometry of the R1 drift cells. The centers of the hexagons mark 
the positions of the sense wires and the corner of the hexagons mark the 
positions of the field wires.  Guard wires, surrounding one superlayer of six 
sense wire layers, are not shown.}}
\label{proto:cellsize}
\end{figure}
%%%%%%%%%%%%%%%%%%%%%%%%%%%%%%%%%%%%%%%%%%%%%%%%%%%%%%%%%%%%%%%%%%%%%%%%%%%

\subsection{Drift Chamber Prototyping}

A full-size prototype sector of the Region 1 drift chamber (R1A) is being built 
to study all aspects related to design, assembly, installation and survey, and 
operation.  The R1A prototype can also be used to test different electrical and 
electronic components, as well as service components, like low and high voltage 
cables, readout cables, and gas supply.

The dimension of the detector and its drift cell size is determined by its
position along the beam line.  The present design uses a distance of 2100~mm 
from the target center to the first row of drift cell sense wires at the 
mid-plane of the sector.  The resulting cell geometry is shown in 
Fig.~\ref{proto:cellsize}.

The first step for prototyping Region~1 was the design of the wire supporting 
endplates and their assembly in a frame with open entrance and exit windows.  
The windows will be covered with a gas-tight foil, for example polyimide.
The aluminum endplates are being held in a box frame welded out of stainless 
steel I-beams.  The frame has a small outward bow on its side where the 
endplates fit in.  The bow shape and depth are calculated to straighten 
completely after applying the tension of the wires without collapsing the 
chamber. The whole drift chamber assembly is self-supporting.  By bolting the 
endplate into the frame, the endplate will be pre-bowed according to the 
calculated wire tension.  In Fig.~\ref{proto:r1} the box frame with endplates 
is shown from a three-dimensional drawing.

%%%%%%%%%%%%%%%%%%%%%%%%%%%%%%%%%%%%%%%%%%%%%%%%%%%%%%%%%%%%%%%%%%%%%%%%%%%
\begin{figure}[htpb]
\vspace{9.0cm} 
\special{psfile=r1.eps hscale=50 vscale=50 hoffset=110 voffset=0 angle=0}
\caption{\small{Three-dimensional model of the Region 1 chamber including
some representative STB and HVTB circuit boards attached.}}
\label{proto:r1}
\end{figure}
%%%%%%%%%%%%%%%%%%%%%%%%%%%%%%%%%%%%%%%%%%%%%%%%%%%%%%%%%%%%%%%%%%%%%%%%%%%

Several sections with varying wire length of the assembled prototype chamber
will be wired-up with drift cells.  For the sense wires, 30 $\mu$m-diameter 
tungsten wires will be used, a 50\% increase in diameter in comparison to the 
existing {\tt CLAS} drift chambers.  The wire feedthroughs from the {\tt CLAS} 
Region 1 chamber will be used.  To achieve the same gas gain of $5 \times 10^4$ 
as for the {\tt CLAS} chambers, the voltage between the field and sense wires 
needs to be raised by about 150~V.  The electronics readout boards and HV 
supplying boards will be designed using existing components and the same basic 
layout as for the {\tt CLAS} chambers.  The printed circuit board material will 
be FR-4, as in the past.  In addition, a new polyimide-based printed circuit 
board will be built.  Polyimide boards provide reduced water absorption 
capability and improved dielectric properties, advantageous in view of the 
increased sense wire high voltage.

The main focus of the prototype is on the validation of the overall chamber 
design and its operational stability.  One of the key parameters is the gain 
of the drift cells, designed to be about $5 \times 10^4$. This gain will be 
measured as a function of the high voltage between sense and field wires.
The electron drift velocity inside the drift cell is also going to be increased 
by a mixture change of the drift gas.  Previously, a mixture of 90\% argon with 
10\% CO$_2$ was used and will be replaced by a slightly faster 92/8 mixture of 
the same gases.  To study how the drift gas affects the gas gain, the relative 
mixture of the two gases will be varied.  These measurements will be carried 
out with cosmic rays and radioactive sources. 

The increased high voltages will lead to increased electric fields on the circuit
boards and, hence, potentially to increased leakage currents between the sense 
wire high voltage and the ground of the boards. This leakage current will be 
studied.  The polyimide boards are expected to operate with considerably lower 
leakage currents at comparable distances between circuit board traces and layer 
thicknesses.

Besides the above listed items, the infrastructure to string, manipulate, and 
transport the drift chamber will be built and tested.  This includes clean room 
operation, stringing fixtures, test devices to detect possible electrostatic 
oscillations of the wires via excitation by magnets, transport crates, and 
mounting and handling fixtures.  The same basic techniques as for the assembly 
of the present {\tt CLAS} drift chambers will be used.  Some of the existing 
wire stringing equipment will be either used or modified.

A change to the wire feedthrough design is presently being considered. The 
new design would eliminate the trumpet-shaped metal insert. This insert was
previously being used to reduce the electric field between the feedthrough 
and the conductive endplate. It had the side effect to reduce the wire 
efficiency considerably within a distance of about 1~cm from the feedthrough. 
The all-plastic feedthrough is expected to reduce the inefficient region.  A 
design drawing of the new all-plastic feedthrough is shown in 
Fig.~\ref{proto:newfeed}.  The new all-plastic feedthrough will be tested 
separately.  For this test a small drift chamber with seven drift cells will 
be built.  A three-dimensional drawing of the test chamber is shown in 
Fig.~\ref{proto:babydc}.  The test chamber will be using the new all-plastic 
feedthrough and tested with high intensity electron beams at Idaho State 
University.

%%%%%%%%%%%%%%%%%%%%%%%%%%%%%%%%%%%%%%%%%%%%%%%%%%%%%%%%%%%%%%%%%%%%%%%%%%%
\begin{figure}[htpb]
\vspace{4.5cm} 
\special{psfile=newfeed.eps hscale=35 vscale=35 hoffset=60 voffset=0 angle=0}
\caption{\small{Design drawing of the new all-plastic wire feedthrough.}}
\label{proto:newfeed}
\end{figure}
%%%%%%%%%%%%%%%%%%%%%%%%%%%%%%%%%%%%%%%%%%%%%%%%%%%%%%%%%%%%%%%%%%%%%%%%%%%

%%%%%%%%%%%%%%%%%%%%%%%%%%%%%%%%%%%%%%%%%%%%%%%%%%%%%%%%%%%%%%%%%%%%%%%%%%%
\begin{figure}[htpb]
\vspace{6.5cm} 
\special{psfile=babydc.eps hscale=50 vscale=50 hoffset=60 voffset=-10 angle=0}
\caption{\small{Layout of the small test chamber for testing of the 
drift chamber feedthroughs.
\label{proto:babydc}}}
\end{figure}
%%%%%%%%%%%%%%%%%%%%%%%%%%%%%%%%%%%%%%%%%%%%%%%%%%%%%%%%%%%%%%%%%%%%%%%%%%%
