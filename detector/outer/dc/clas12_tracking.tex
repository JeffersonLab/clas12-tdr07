
\section{Drift Chambers}

The overall tracking requirements (0.5 -  1\% fractional momentum resolution 
at 5~GeV and efficient tracking at a luminosity of \1035) are the main
drivers for drift chamber design.  Because the {\tt CLAS} drift chamber
system~\cite{dcnim} has operated successfully for 8 years, we plan to 
re-use many of the design concepts and most of the utility infrastructure.  
In particular, we plan to re-use the present gas mixing and handling system, 
the high-voltage and low-voltage systems, the FASTBUS TDC system and the 
post-amplifier/multiplexer systems.  The construction project thus consists 
of new chambers, on-board electronics, and on-board jumper cables.

The required better resolution of the tracking system compared to the 
present {\tt CLAS} detector will be achieved by chambers with a smaller 
cell size and thus inherently better spatial resolution, and the smaller 
physical size and thus more accurate placement of the chambers with respect 
to the magnet coils.  The design of the forward chambers is very similar to 
the present {\tt CLAS} chambers.  The cell design is hexagonal and the sense 
wire layers are arranged in 6-layer superlayers as in the present chambers. 
The major difference is that the cells are approximately two thirds as big 
as the present chambers allowing efficient tracking at higher luminosities 
because the accidental occupancy from particles not associated with the 
event is smaller. Table~\ref{fwd-dc-design-parms} lists the design 
parameters and Fig.~\ref{fwddc} shows a single sector of the new Region 1
chamber.  For the purposes of simulating track resolutions, we assumed that 
the position resolution of the individual drift cells would be 250~$\mu$m.  
For reference, the present {\tt CLAS} chambers have resolutions of 310, 315,
and 380 $\mu m$ for R1, R2, and R3, respectively.

%%%%%%%%%%%%%%%%%%%%%%%%%%%%%%%%%%%%%%%%%%%%%%%%%%%%%%%%%%%%%%%%%%%%%%%%%%%
\begin{figure}[htbp]
\vspace{9.0cm}
\special{psfile=r1.eps hscale=47 vscale=47 hoffset=120 voffset=0}
\caption{\small{A view of one sector of the Region 1 drift chamber 
showing the projection of the wires onto the chamber endplate.}}
\label{fwddc}
\end{figure}
%%%%%%%%%%%%%%%%%%%%%%%%%%%%%%%%%%%%%%%%%%%%%%%%%%%%%%%%%%%%%%%%%%%%%%%%%%%

%%%%%%%%%%%%%%%%%%%%%%%%%%%%%%%%%%%%%%%%%%%%%%%%%%%%%%%%%%%%%%%%%%%%%%%%%
\begin{table}[htbp]
\begin{center}
\begin{tabular} {||c|c|c|c||} \hline \hline
&{\bf Region 1}      &  {\bf Region 2} & {\bf Region 3}\\ \hline
dist. from target    & 2.1 m & 3.3 m   & 4.5 m \\ \hline
num. of superlayers  & 2 & 2   & 2 \\ \hline
layers/superlayer    & 6 & 6   & 6 \\ \hline
wires/layer          & 112 & 112   & 112 \\ \hline
cell size            & 0.86 cm & 1.36 cm   & 1.88 cm \\ \hline
assumed resolution per wire  & 0.025 cm & 0.025 cm   & 0.025 cm \\ \hline
\end{tabular}
\caption{\small{Design parameters for the {\tt CLAS12} drift chambers.}}
\label{fwd-dc-design-parms}
\end{center}
\end{table}
%%%%%%%%%%%%%%%%%%%%%%%%%%%%%%%%%%%%%%%%%%%%%%%%%%%%%%%%%%%%%%%%%%%%%%%%%

The chambers differ from the present {\tt CLAS} chambers in a number of 
ways.  Successive superlayers have their wires arranged with a plus or 
minus 6$^{\circ}$ stereo angle; the present arrangement has an axial layer and 
a 6$^{\circ}$ stereo layer.  For the present {\tt CLAS} detector, the $\phi$ 
resolution is about four times larger than the $\theta$ resolution.  To 
have more equal resolution in the two angles, we decided that we needed to 
double our effective stereo angle in order to improve the  $\phi$ resolution.
Unlike the present chambers, all of the wires in one of the superlayers are
strictly parallel, and in a plane perpendicular to the wire direction form
perfect hexagons.  This should allow a more accurate drift velocity 
calibration than the current design with its layer-to-layer increase in 
cell size.  The choice of gas; a 92:08 Argon:CO$_2$ mixture is a small 
departure from our present 90:10 mixture and should result in a higher and 
more constant drift velocity.  We plan to run with a gas gain of $5\cdot10^4$.

Another departure from the present design is to design every chamber (in 
all three regions) to be self-supporting in order to ensure that they are 
easy to install and remove for maintenance.  In the present {\tt CLAS}, the 
Region 1 chambers are all bound together into a single unit in order to 
maintain the wire tension without excessively thick endplates, and the 
Region 2 chambers are actually mounted onto the magnet cryostat with the 
cryostat itself maintaining the internal wire tension.  None of the present 
Region 1 or 2 chambers can be accessed individually without a lengthy 
``tension-transfer'' process.  To avoid this, we are designing all chambers 
to be self-supporting like our present Region 3 chambers.  The key will be 
ultra-stiff endplates which obtain their stiffness by a flanged design.  

A third design change is to use 30~$\mu$m diameter sense wire rather than 
the more common 20~$\mu$m wire. Our choice of wire is 30~$\mu$m diameter, 
gold-plated tungsten for the sense wires, 140~$\mu$m diameter, gold-plated 
aluminum for the field wires and 140~$\mu$m diameter, stainless steel for
the guard wires.   This should make the chamber more robust to wire 
breakages.  Higher voltages will be required to achieve the same gas gain, 
and the resulting higher electric field in the drift cells will result in 
a more nearly constant drift velocity which should be easier to calibrate.
Prototypes are being built to study possible negative side-effects of the 
higher voltage operation such as leakage currents on the circuit boards 
and/or higher rates of cathode emission from the field wire surfaces.
Fig.~\ref{garfield} shows GARFIELD calculations for a Region 3 drift cell
with both a 20~$\mu$m and a 30~$\mu$m diameter sense wire.  Here the
cells with the thicker sense wire will have a significantly higher drift 
velocity which is desirable to reduce the time window, and hence the chamber 
occupancy.

%%%%%%%%%%%%%%%%%%%%%%%%%%%%%%%%%%%%%%%%%%%%%%%%%%%%%%%%%%%%%%%%%%%%%%%%%%%
\begin{figure}[htbp]
\vspace{12.0cm}
\special{psfile=garfield1.eps hscale=30 vscale=27 hoffset=50 voffset=165}
\special{psfile=garfield2.eps hscale=30 vscale=27 hoffset=50 voffset=-5}
\special{psfile=garfield3.eps hscale=30 vscale=27 hoffset=240 voffset=165}
\special{psfile=garfield4.eps hscale=30 vscale=27 hoffset=240 voffset=-5}
\caption{\small{GARFIELD calculations of the electric field lines (top)
and drift time vs. drift distance (bottom) for a Region 3 drift cell.  The 
left plots show the configuration with a 20~$\mu$m diameter sense wire and 
the right plots show the configuration with a 30~$\mu$m diameter sense wire.
The high voltages were set to provide the same gas gain for each
configuration.}}
\label{garfield}
\end{figure}
%%%%%%%%%%%%%%%%%%%%%%%%%%%%%%%%%%%%%%%%%%%%%%%%%%%%%%%%%%%%%%%%%%%%%%%%%%%

\section{Expected Detector Performance}

We studied the position, angle, and momentum resolution for a number of 
possible detector options using MOMRES~\cite{momres}.  Our procedure was 
to produce a MOMRES input file that characterized the detector position, 
material thickness, and estimated hit resolution for a particular track 
angle.  We also produced a B-field file which was a tabulation of the 
B-field strength vs. path length for a particular track angle.  These, 
and the desired range of momenta, were the inputs to MOMRES. MOMRES 
calculated the expected components of the resolutions due to multiple 
scattering and measurement resolution, respectively.  We fit these outputs 
to the expected kinematic form. From these fits we extracted two parameters 
(sig1 and sig2) for each of the three terms ($dp/p$, $d\theta$, $dx$). 
These six parameters summarize the output of MOMRES. In addition, we 
calculated the angle resolution in the non-bend plane in a manner 
analogous to that of MOMRES, using estimates for the effects of multiple 
scattering and measurement error, fitting the resulting smeared trajectory 
by a straight line, and extracting the $\sigma$1 and $\sigma$2 parameters 
that characterize the resolution in this out-of-bend-plane angle.  Thus, 
eight parameters for each value of track angle fully characterize the 
tracking resolution for any one detector option.  Fig.~\ref{30degres} 
shows the momentum, $\theta$, $\phi$, and vertex resolutions for three 
detector options for tracks emitted at 30$^{\circ}$.  Fig.~\ref{90degres} 
shows the momentum, $\theta$, $\phi$, and vertex resolutions for three 
detector options for tracks emitted at 90$^{\circ}$, i.e. into the central 
detector. 

%%%%%%%%%%%%%%%%%%%%%% Figure : 30deg resolution %%%%%%%%%%%%%%%%%%%%
\begin{figure}[htpb]
\vspace{9.0cm}
\special{psfile=30deg_resln.eps hscale=50 vscale=50 hoffset=90 voffset=0}
\caption{\small{Resolution simulated using MOMRES plotted vs. particle 
momentum for 30$^{\circ}$ tracks.  Sub-figures a, b, c, and d show the 
momentum, $\theta$, $\phi$, and vertex resolutions, respectively.  Three 
options are shown: solid- 3 SVT planes + CC + 3 DC planes; dashed- CC + 
3 DC planes; dotted - 3 DC planes.}}
\label{30degres}
\end{figure}
%%%%%%%%%%%%%%%%%%%%%%%%%%%%%%%%%%%%%%%%%%%%%%%%%%%%%%%%%%%%%%%%%%%%%%%%%

Fig.~\ref{resol} provides a few more results of these studies to
parameterize and understand the resolution expected for the {\tt CLAS12}
tracking system.  The plots show the expected angular resolution, the 
expected position resolution, and the expected momentum resolution
for different angle bins, each as a function of momentum.  These plots
were made using the current design of the tracking system.

Fig.~\ref{resplot} shows the reconstructed momentum for 1000 events
each containing a 1~GeV track emitted at a 45$^{\circ}$ polar angle
but at random values of azimuthal angle and with random multiple
scattering.  The dark histogram shows the expected reconstructed
spectrum at a luminosity of 10$^{35}$~cm$^{-2}$s$^{-1}$.  The gray
histogram and the solid line show the resolution with 40 times the
nominal background (consisting of random SVT hits).  The gray histogram
was constructed using the nominal $\pm$1.5$^{\circ}$ $U$-$V$ strip
pitch, and the solid line was made increasing the strip pitch by a
factor of two above the nominal.  The studies serve to validate the
basic SVT design parameters.

%%%%%%%%%%%%%%%%%%%%%% Figure : 90deg resolution %%%%%%%%%%%%%%%%%%%%
\begin{figure}[htbp]
\vspace{8.0cm}
\special{psfile=90deg_resln.eps hscale=50 vscale=50 hoffset=90 voffset=-5}
\caption{\small{Resolution simulated using MOMRES plotted vs. particle 
momentum for 90$^{\circ}$ tracks.  Sub-figures a, b, c, and d show the 
momentum, $\theta$, $\phi$, and vertex resolutions, respectively.  Three 
options are shown: solid- 3 SVT planes; dashed- 16 DC-stereo planes; 
dotted - 8 DC-cathode pad planes.}}
\label{90degres}
\end{figure}
%%%%%%%%%%%%%%%%%%%%%%%%%%%%%%%%%%%%%%%%%%%%%%%%%%%%%%%%%%%%%%%%%%%%%%%%%

%%%%%%%%%%%%%%%%%%%%%%%%%%%%%%%%%%%%%%%%%%%%%%%%%%%%%%%%%%%%%%%%%%%%%%%%%
\begin{figure}[htbp]
\vspace{12.0cm}
\special{psfile=resol_1.eps hscale=35 vscale=30 hoffset=15 voffset=160}
\special{psfile=resol_2.eps hscale=35 vscale=30 hoffset=225 voffset=160}
\special{psfile=resol_3.eps hscale=35 vscale=30 hoffset=115 voffset=-10}
\caption{\small{Simulation results for the {\tt CLAS12} tracking system
showing the expected angular resolution, position resolution, and
momentum resolution for different angle bins as a function of momentum.}
\label{resol}}
\end{figure}
%%%%%%%%%%%%%%%%%%%%%%%%%%%%%%%%%%%%%%%%%%%%%%%%%%%%%%%%%%%%%%%%%%%%%%%%%

%%%%%%%%%%%%%%%%%%%%%%%%%%%%%%%%%%%%%%%%%%%%%%%%%%%%%%%%%%%%%%%%%%%%%%%%%
\begin{figure}[htpb]
\vspace{6.5cm}
\special{psfile=centralrates1.eps hscale=120 vscale=85 hoffset=75 voffset=0}
\caption{\small{A plot of reconstructed momentum (GeV) for 1000 events,
each containing a 1~GeV track emitted at a 45$^{\circ}$ polar angle
but at random values of azimuthal angle and with random multiple
scattering.  The solid histogram is for events with the background 
expected at a luminosity of \1035.  The gray histogram ($\pm$1.5$^{\circ}$
strip pitch) and solid line ($\pm$3$^{\circ}$ strip pitch) were
simulations run increasing the background by a factor of 40.}}
\label{resplot}
\end{figure}
%%%%%%%%%%%%%%%%%%%%%%%%%%%%%%%%%%%%%%%%%%%%%%%%%%%%%%%%%%%%%%%%%%%%%%%%%

\section{Expected Physics Performance}

We used a series of programs to calculate the acceptance and reconstructed 
physics parameters for event types of interest.  The program {\it clasev}
\cite{clasev} served as an event generator and analysis program. 
Depending on the value of input flags, it generates certain types of 
events; that is, it produces a set of 4-momenta for the primary hadrons 
in the hadronic center-of-mass and allows some of them to decay into the 
final-state hadrons and transforms their momenta to the lab system.  For 
each final-state track, it calls {\tt FASTMC} to determine if the track 
falls within a fiducial acceptance window and to determine its final, 
smeared lab momentum.  It then produces selected physics analysis variables 
such as missing mass from calculations involving the smeared momenta of 
those tracks that were accepted. 

Fig.~\ref{massplot} shows the expected missing mass resolution 
expected for {\tt CLAS12} from simulation studies based on the current
design specifications for a number of different reactions.  For all
cases studied the results are quite encouraging in terms of identifying
the missing particle cleanly for each reaction.

%%%%%%%%%%%%%%%%%%%%%%%%%%%%%%%%%%%%%%%%%%%%%%%%%%%%%%%%%%%%%%%%%%%%%%%%%%%
\begin{figure}[htbp]
\vspace{14.0cm}
\special{psfile=mm1.ps hscale=39 vscale=36 hoffset=-10 voffset=160}
\special{psfile=mm2.ps hscale=39 vscale=36 hoffset=225 voffset=160}
\special{psfile=mm3.ps hscale=40 vscale=37 hoffset=-10 voffset=-55}
\special{psfile=mm4.ps hscale=40 vscale=37 hoffset=230 voffset=-55}
\caption{\small{Simulation results highlighting the expected missing
mass resolution of {\tt CLAS12} with the nominal design specifications for
the drift chambers.  Shown are the spectra for the reactions
$ep \to e'\pi+X$ (UL), $ep \to e'K^+X$ (UR), $ep \to e'p\pi^+X$ (LL),
and $ep \to e\rho^+X$ (LR).}}
\label{massplot}
\end{figure}
%%%%%%%%%%%%%%%%%%%%%%%%%%%%%%%%%%%%%%%%%%%%%%%%%%%%%%%%%%%%%%%%%%%%%%%%%%%

\vfil
\eject

\begin{thebibliography}{99}

\bibitem{pcdr} 
Pre-Conceptual Design Report the Science and Experimental Equipment
for the 12~GeV Upgrade of CEBAF, see www.jlab.org/12GeV/collaboration.html.

\bibitem{fastmc}
FASTMC is a fast parametric Monte Carlo of {\tt CLAS12}; the code and 
documentation are contained in CVS under 12GeV/fastmc.

\bibitem{pac} 
JLab PAC30 meeting, Aug. 2006; see www.jlab.org/exp\_prog/PACpage/pac.html.

\bibitem{egs}
R.L. Ford and W.R. Nelson, The EGS code system: Computer program for the 
Monte Carlo simulation of electromagnetic cascade showers, SLAC-210 (1978).
 
\bibitem{dcnim} 
M.D. Mestayer {\it et al.}, Nucl. Inst. and Meth. A {\bf 449}, 81 (2000).

\bibitem{momres} 
B.A. Mecking, The MOMRES package is used as an input to FASTMC and is
a program to estimate the momentum and angle resolution given a table of 
material thicknesses, position resolution, and integral magnetic field.
The progam is in CVS under 12GeV/momres.

\bibitem{clasev} 
H. Avakian, General event generator.  The program is in CVS under
12GeV/fastmc/clasev.

\end{thebibliography}

\end{document}

