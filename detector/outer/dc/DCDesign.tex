\section{Drift Chamber Design}

\subsection{Overview}

The forward tracking system consists of three regions as shown in 
Fig.~\ref{fig:CLAS12 FDC} in each sector; located just before, inside, and 
just outside the torus field volume, and referred to as Regions 1, 2, and 3.  
Each chamber will have its wires arranged in two superlayers of six layers 
each, with the wires in the two superlayers strung with $\pm$6$^\circ$
stereo angles, respectively.  The cell structure will be hexagonal, that is, 
each sense wire is surrounded by six field wires.  This arrangement is 
similar to the present {\tt CLAS} design and offers good resolution with 
very good pattern recognition properties.  Refer to our article on the 
overall {\tt CLAS} detector~\cite{clasnim} and our article on the drift 
chambers themselves~\cite{dcnim} for details of the present detector and 
chambers.

%%%%%%%%%%%%%%%%%%%%%%%%%%%%%%%%%%%%%%%%%%%%%%%%%%%%%%%%%%%%%%%%%%%%%%%%
\begin{figure}[hb]
\vspace{10.6cm}
\special{psfile=umbrella.ps hscale=60 vscale=60 hoffset=60 voffset=0}
\caption{\small{Layout of the {\tt CLAS12} Forward Drift Chambers: Region~1, 
2, and 3.}}
\label{fig:CLAS12 FDC}
\end{figure}
%%%%%%%%%%%%%%%%%%%%%%%%%%%%%%%%%%%%%%%%%%%%%%%%%%%%%%%%%%%%%%%%%%%%%%%%

The major difference is that the cells cover a much smaller solid angle
than those in the present chambers, allowing efficient tracking at higher 
luminosities because the accidental occupancy from particles not associated 
with the event is smaller. Table~\ref{fwd-dc-design-parms} lists the main
design parameters for each region of the {\tt CLAS12} drift chambers.  For 
the purposes of simulating track resolutions, we assumed that the position 
resolution of the individual drift cells would be 250~$\mu$m.  For reference, 
the present {\tt CLAS} chambers have resolutions of 310, 315, and 380~$\mu$m 
for R1, R2, and R3, respectively.

%%%%%%%%%%%%%%%%%%%%%%%%%%%%%%%%%%%%%%%%%%%%%%%%%%%%%%%%%%%%%%%%%%%%%%%%%
\begin{table}[htbp]
\begin{center}
\begin{tabular} {||c|c|c|c||} \hline \hline
&{\bf Region 1}      &  {\bf Region 2} & {\bf Region 3}\\ \hline
dist. from target    & 2.1 m & 3.3 m   & 4.5 m \\ \hline
num. of superlayers  & 2 & 2   & 2 \\ \hline
layers/superlayer    & 6 & 6   & 6 \\ \hline
wires/layer          & 112 & 112   & 112 \\ \hline
cell size            & 0.75 cm & 1.18 cm   & 2.07 cm \\ \hline
active time window   & 150 ns & 250 - 500 ns & 500 ns \\ \hline
assumed resolution per wire  & 0.025 cm & 0.025 cm   & 0.025 cm \\ \hline
\end{tabular}
\caption{\small{Design parameters for the {\tt CLAS12} drift chambers.}}
\label{fwd-dc-design-parms}
\end{center}
\end{table}
%%%%%%%%%%%%%%%%%%%%%%%%%%%%%%%%%%%%%%%%%%%%%%%%%%%%%%%%%%%%%%%%%%%%%%%%%

The chambers differ from the present {\tt CLAS} chambers in a number of 
ways.  Successive superlayers have their wires arranged with a plus or 
minus 6$^{\circ}$ stereo angle; the present arrangement has an axial layer 
and a 6$^{\circ}$ stereo layer.  For the present {\tt CLAS} detector, the 
$\phi$ resolution is about four times larger than the $\theta$ resolution.  
To have more equal resolution in the two angles, we decided that we needed 
to double our effective stereo angle in order to improve the  $\phi$ 
resolution.  Unlike the present chambers, all of the wires in one of the 
superlayers are strictly parallel, and in a plane perpendicular to the wire 
direction form perfect hexagons.  This should allow a more accurate drift 
velocity calibration than the current design with its layer-to-layer 
increase in cell size.  The choice of gas; a 92:08 Argon:CO$_2$ mixture is 
a small departure from our present 90:10 mixture and should result in a 
higher and more constant drift velocity.  We plan to run with a gas gain 
of $5\cdot10^4$.

Another departure from the present design is to design every chamber (in 
all three regions) to be self-supporting in order to ensure that they are 
easy to install and remove for maintenance.  In the present {\tt CLAS}, the 
Region~1 chambers are all bound together into a single unit in order to 
maintain the wire tension without excessively thick endplates, and the 
Region~2 chambers are actually mounted onto the magnet cryostat with the 
cryostat itself maintaining the internal wire tension.  None of the present 
Region~1 or 2 chambers can be accessed individually without a lengthy 
``tension-transfer'' process.  To avoid this, we are designing the
Region~1 and Region~2 chambers to be self-supporting like our present 
Region~3 chambers.  To keep a very thin endplate (to minimize dead area), 
some of the wire tension in the Region~2 chambers will be borne by springs 
mounted to the torus cryostat; but many fewer springs than in the present 
detector.  The key to these improvements will be ultra-stiff endplate 
assemblies which obtain their stiffness by a flanged design.  

A third design change is to use 30~$\mu$m diameter sense wire rather than 
the more common 20~$\mu$m wire. Our choice of wire is 30~$\mu$m diameter, 
gold-plated tungsten for the sense wires, 140~$\mu$m diameter, gold-plated 
aluminum for the field wires and 140~$\mu$m diameter, stainless steel for
the guard wires.   This should make the chamber more robust to wire 
breakages.  Higher voltages will be required to achieve the same gas gain, 
and the resulting higher electric field in the drift cells will result in 
a more nearly constant drift velocity which should be easier to calibrate.
Prototypes are being built to study possible negative side-effects of the 
higher voltage operation such as leakage currents on the circuit boards 
and/or higher rates of cathode emission from the field wire surfaces.

Fig.~\ref{garfield} shows GARFIELD calculations for a Region~3 drift cell
with both a 20~$\mu$m and a 30~$\mu$m diameter sense wire.  Here the
cells with the thicker sense wire will have a significantly higher drift 
velocity which is desirable to reduce the time window, and hence the chamber 
occupancy.

%%%%%%%%%%%%%%%%%%%%%%%%%%%%%%%%%%%%%%%%%%%%%%%%%%%%%%%%%%%%%%%%%%%%%%%%%%%
\begin{figure}[htbp]
\vspace{12.0cm}
\special{psfile=garfield1.eps hscale=30 vscale=27 hoffset=40 voffset=165}
\special{psfile=garfield2.eps hscale=30 vscale=27 hoffset=40 voffset=-5}
\special{psfile=garfield3.eps hscale=30 vscale=27 hoffset=230 voffset=165}
\special{psfile=garfield4.eps hscale=30 vscale=27 hoffset=230 voffset=-5}
\caption{\small{GARFIELD calculations of the electric field lines (top)
and drift time vs. drift distance (bottom) for a Region~3 drift cell.  The 
left plots show the configuration with a 20~$\mu$m diameter sense wire and 
the right plots show the configuration with a 30~$\mu$m diameter sense wire.
The high voltages were set to provide the same gas gain for each
configuration.}}
\label{garfield}
\end{figure}
%%%%%%%%%%%%%%%%%%%%%%%%%%%%%%%%%%%%%%%%%%%%%%%%%%%%%%%%%%%%%%%%%%%%%%%%%%%




