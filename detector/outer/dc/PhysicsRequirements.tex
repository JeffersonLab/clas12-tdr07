
\section{CLAS12 Physics Requirements}

The {\tt CLAS} detector in Hall B is being upgraded to take advantage 
of the increase of the CEBAF beam energy from 6 to 12~GeV, thus the 
new name, {\tt CLAS12}.  There are several broad areas of physics 
enquiry that drive these changes: spectroscopic studies of excited 
baryons, investigations of the influence of nuclear matter on propagating 
quarks, studies of polarized and unpolarized quark distributions, and a 
comprehensive measurement of generalized parton distributions (GPDs).  
Many of the reactions of interest are electroproduction of exclusive and 
semi-inclusive final states.  The cross sections for these processes are 
small, necessitating high-luminosity experiments.  A variety of proposed 
experiments rely on luminosities of \1035 to achieve the desired statistical 
accuracy in runs of a few months duration.  The deep exclusive reactions in 
which an electron scattering at moderate to high values of $Q^2$ results in 
a meson-baryon final state, provide the most stringent requirements for 
the {\tt CLAS12} tracking system.  A final state of a few high-momentum, 
forward-going particles (the electron as well as one or more mesons), 
combined with a moderate-momentum baryon emitted at large angles, is the 
typical event type on which the specifications of the tracking system are 
based.  

In broad strokes, the tracking system must measure forward-going particles 
down to lab angles as small as 5$^{\circ}$ and as large as 
40$^{\circ}$ in order to cover as much of the hadronic center-of-mass 
region as possible.  A silicon strip tracking system will cover the 
40$^\circ$ - 135$^\circ$ angular range.  We require very good momentum and 
angular resolution for the scattered electron (on the order of 
$\Delta p/p = 1\%$ and $\Delta \theta$ = 1~mrad) in order to determine the 
virtual photon flux factor $\Gamma_v$, and hence the cross sections, to an 
accuracy of a few percent.  In addition, a momentum resolution of about 
20 to 50~MeV is necessary in order to positively identify a missing hadron 
in these exclusive reactions.  Finally, good vertex resolution will allow 
detection of secondary decay vertices and serve as a good marker for 
strangeness production.

A tracking system capable of achieving these standards was described
in the PCDR~\cite{pcdr} and quantitatively parameterized in a ``fast'' 
Monte Carlo (FASTMC) program~\cite{fastmc}.  A number of {\tt CLAS} 
collaborators used the existing model of the detector as described in 
{\tt FASTMC} in proposals presented to JLab PAC-30~\cite{pac30} in 
August 2006, the first PAC to consider 12-GeV proposals.

