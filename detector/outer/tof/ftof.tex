\section{FTOF System Overview}

The Forward Time-of-Flight System (FTOF) will be a major component of the 
{\tt CLAS12} forward detector used to measure the time-of-flight of charged 
particles emerging from the target.  The average path length from the target 
to the FTOF counters will be roughly 650~cm.  The requirements for the FTOF 
system include excellent timing resolution for particle identification and 
good segmentation for flexible triggering and prescaling options.  The design
parameters were chosen to allow for separation of pions and kaons up to 
roughly 3~GeV.  The most energetic particles are produced at small angles.  
The system specifications call for a time resolution of $\sigma_{TOF}$=80~ps 
at the more forward angles of {\tt CLAS12} and 150~ps at angles larger than
36$^\circ$.  The system must also be capable of operating in a high-rate 
environment.  The maximum counting rate occurs in the forward direction 
where, at an operating luminosity of $1\times10^{35}$~cm$^{-2}$s$^{-1}$, the 
average rate per scintillator is approximately 250~kHz.

The discriminated scintillator signals will be used in certain situations 
in the {\tt CLAS12} Level-1 trigger.  Therefore, the system must provide 
signals representing a uniform response with adequate granularity to select 
particles reaching the detectors.  The timing in the trigger hardware will 
be limited by the flight time variations between fast particles (e.g. 
electrons) and slow particles (e.g. protons), which can be as large as 50~ns.
Therefore, precise timing information will only be achieved in off-line 
software analysis using the momentum and position measured with the drift 
chamber system.

The FTOF system will also be used for energy-loss measurements in specific 
instances.  Pulse height information, being directly proportional to energy 
deposited, provides an independent means for the identification of slow 
particles.  In this regard, the flight time can provide for a more accurate 
measurement of particle energy than magnetic analysis for slow particles, 
where the effects of multiple scattering are the largest.

In order to meet the requirements for the tight timing resolution, the major 
considerations in the design of the FTOF system are:

\begin{itemize}

\item {\it Scintillator Size:} The overall size of the system will demand 
careful consideration of light collection in order to optimize the time 
resolution of the system.  Also, the width of each scintillator determines
the granularity of the scattering angle definition in the trigger.

\item {\it Geometry:} The projected space behind the coils of the main
{\tt CLAS12} torus is inactive and therefore useful for locating the
light guides, photomultiplier tubes (PMTs), voltage dividers, and cables.  
The remaining area in the forward direction is the fiducial region of the 
detector and must be covered with scintillator counters.

\item {\it Magnetic Field:} The PMTs will have to be properly shielded
from the stray magnetic fields of the {\tt CLAS12} torus.

\item {\it Crossing Tracks:} Particle trajectories from the target can
intersect adjacent TOF counters.  Therefore light from both counters
will have to be summed to optimize particle identification in the data
analysis.

\end{itemize}

Each component of the design must satisfy these design constraints and be
optimized against cost considerations.

In each sector of {\tt CLAS12}, the FTOF system will be comprised of three 
sets of TOF counters, referred to as panels (called panel-1a, 1b, and 2),
in each sector.  Each panel consists of an array of rectangular scintillators
with a PMT on each end.  Panel-1 refers to the sets of counters located at 
forward angles (roughly 5$^\circ$ to 36$^\circ$) (where the two panels are
necessary to meet the 80~ps resolution requirements) and panel-2 refers to 
the sets of counters located at larger angles (roughly 36$^\circ$ to 
45$^\circ$).  The positioning and attachment of the FTOF system panels to 
the forward carriage of {\tt CLAS12} are shown in Fig.~\ref{fwd_car}.

%%%%%%%%%%%%%%%%%%%%%%%%%%%%%%%%%%%%%%%%%%%%%%%%%%%%%%%%%%%%%%%%%%%%%%%%%%%
\begin{figure}[htbp]
\vspace{9.0cm}
\special{psfile=fwd_car.eps hscale=32 vscale=32 hoffset=110 voffset=0}
\caption{\small{View of the FTOF counters for {\tt CLAS12} highlighting
the location of the panel-1 and panel-2 counters.  This figure was generated
from our 3--D CAD model.}}
\label{fwd_car}
\end{figure}
%%%%%%%%%%%%%%%%%%%%%%%%%%%%%%%%%%%%%%%%%%%%%%%%%%%%%%%%%%%%%%%%%%%%%%%%%%%

The panel-1 counters will consist of the current {\tt CLAS} panel-1 TOF 
counters (called panel-1a) and a new set of panel-1 counters (called 
panel-1b).  The existing panel-1a counters consist of 23 scintillators, each 
measuring 5.08-cm thick and 15-cm wide.  The lengths of these counters 
range from roughly 32~cm at the smallest scattering angles to roughly 
375~cm at the largest scattering angles.  The scintillators are constructed 
from Bicron BC-408 and are read out through short acrylic light guides 
to 2-in Thorn EMI-9954A PMTs.  The new panel-1b counters will consist of an 
array of 58 scintillators constructed from Bicron BC-404 scintillator for
the shorter counters and BC-408 for the longer counters, each 6-cm wide and 
6-cm thick with a range of lengths to match the panel-1a counters.  This new 
panel will be mounted off of the forward carriage in front of the existing 
panel-1a counters.  One sector of the FTOF system highlighting the panel-1 
counters is shown in Fig.~\ref{ftof_pic}.  Note that the existing panel-1a
arrays extend down to only 8$^\circ$ in the {\tt CLAS} geometry.  As the
positioning of the forward carriage in {\tt CLAS12} becomes finalized, we
will then decide on the number, lengths, and widths of additional scintillator
bars to add at the small-angle apex of the panel-1a arrays to ensure
acceptance down to 5$^\circ$.

%%%%%%%%%%%%%%%%%%%%%%%%%%%%%%%%%%%%%%%%%%%%%%%%%%%%%%%%%%%%%%%%%%%%%%%%%%%%%
\begin{figure}[htbp]
\vspace{8.8cm}
\special{psfile=panel1_sector.eps hscale=47 vscale=47 hoffset=135 voffset=0}
\caption{\small{View of the FTOF panel-1 counters for {\tt CLAS12} for one
sector from the 3-D CAD model of the detector.}}
\label{ftof_pic}
\end{figure}
%%%%%%%%%%%%%%%%%%%%%%%%%%%%%%%%%%%%%%%%%%%%%%%%%%%%%%%%%%%%%%%%%%%%%%%%%%%%%

The panel-2 counters will consist of the current {\tt CLAS} panel-2 TOF 
counters, which include 11 22-cm wide, 5.08-cm thick scintillators in each 
sector.  The length of these counters ranges from roughly 370~cm to 
430~cm.  The scintillators are constructed from Bicron BC-408 and are 
read out through curved acrylic light guides to 3-in Philips XP4312B PMTs. 
These scintillators are included to give complete acceptance for 
outbending charged particles incident upon the {\tt CLAS12} drift chambers.  
Monte Carlo simulations are currently underway to determine how many of the
panel-2 counters will need to be included in {\tt CLAS12} to match the
acceptance of the drift chamber system.  Note that the exact number of
the panel-2 counters that is required depends directly on the final size 
of the Region~3 chambers (which is still being optimized).  Given the
present design of the outermost {\tt CLAS12} drift chambers, it appears that
no more than 5 to 6 of the existing panel-2 counters will be required in each
sector.

%%%%%%%%%%%%%%%%%%%%%%%%%%%%%%%%%%%%%%%%%%%%%%%%%%%%%%%%%%%%%%%%%%%%%%%%%%%%%
\begin{figure}[htbp]
\vspace{7.5cm}
\special{psfile=panel_layout.eps hscale=70 vscale=70 hoffset=150 voffset=-5}
\caption{\small{View of the forward TOF panel-1 and panel-2 scintillators
for a single sector for the existing {\tt CLAS} TOF system.}}
\label{panel_layout}
\end{figure}
%%%%%%%%%%%%%%%%%%%%%%%%%%%%%%%%%%%%%%%%%%%%%%%%%%%%%%%%%%%%%%%%%%%%%%%%%%%%%

In the current {\tt CLAS} detector, the panel-2 counters are mounted to the
side carriages (called the north and south clam shells).  However, in
the {\tt CLAS12} design these panels will be mounted on the existing
forward carriage.  Work is presently underway to engineer and design the
supports and attachment scheme.  A schematic representation of the layout 
of panel-1 and panel-2 for a single sector of the existing {\tt CLAS} TOF 
system is shown in Fig.~\ref{panel_layout}.

\section{Design Requirements}
\label{design_req}

In {\tt CLAS12}, operated with a maximum beam energy of roughly 11~GeV,
forward-going charged hadrons will have momenta up to roughly 5 - 6~GeV.  
Mounted in front of the FTOF system will be the existing {\tt CLAS} low 
threshold {\v C}erenkov system.  The planned radiator gas for this system will
be C$_4$F$_{10}$ (perfluorobutane), as in the current detector system.  This 
gas has an index of refraction of $n$=1.00153, and thus a threshold for pion
detection of roughly 2.6~GeV.  This threshold sets the momentum range where
the FTOF system must be able to separate kaons from pions.  To separate pions 
from kaons up to 2.6~GeV and pions and kaons from protons up to 5.6~GeV, a 
timing resolution of $\sigma_{TOF} \approx$80~ps has to be achieved for the 
panel-1 counters (see Fig.~\ref{delta_tof}).  This assumes a 4$\sigma$ 
difference in flight-time between the two particles over a 650-cm path 
length and allows identification of a hadron species in the presence of 
other hadrons with up to ten times higher rates.

%%%%%%%%%%%%%%%%%%%%%%%%%%%%%%%%%%%%%%%%%%%%%%%%%%%%%%%%%%%%%%%%%%%%%%%%%%%%%
\begin{figure}[htbp]
\vspace{7.0cm}
\special{psfile=tdiff.ps hscale=70 vscale=70 hoffset=30 voffset=-125}
\caption{\small{Time differences, $\Delta t$, between protons and pions,
between protons and kaons, and between kaons and pions (as indicated) over
the 650-cm path length from the target to the FTOF system.  The 
horizontal lines indicate the $4\sigma$ separation time differences for
the panel-1b and panel-1a counters.}}
\label{delta_tof}
\end{figure}
%%%%%%%%%%%%%%%%%%%%%%%%%%%%%%%%%%%%%%%%%%%%%%%%%%%%%%%%%%%%%%%%%%%%%%%%%%%%%

The requirement of 80~ps for the FTOF timing resolution is expected to be
achievable for the panel-1 system with the planned design.  The TOF counters 
from the existing {\tt CLAS} detector have a timing resolution of about 
150~ps~\cite{smith1}.  The new FTOF system is expected to have an improved 
timing resolution relative to the existing TOF panel (panel-1a) by a factor 
of $\sqrt{15/6}$, or the square root of the ratio of the scintillator widths.  
Thus, due to the improved light collection efficiency, the new TOF panel-1b
should provide a timing resolution of about 95~ps or less.  The resolution 
expected from the two timing measurements in panel-1 is then expected to be 
$\sigma_{TOF} = ((1/150~{\rm ps})^2 + (1/95~{\rm ps})^2)^{-1/2} = 80$~ps.  
Timing resolutions of 60~ps have been achieved with small scintillator 
counters~\cite{kim}.  For counters 200~cm in length, prototypes have achieved 
a resolution of about 70~ps~\cite{chen}.

%%%%%%%%%%%%%%%%%%%%%%%%%%%%%%%%%%%%%%%%%%%%%%%%%%%%%%%%%%%%%%%%%%%%%%%%%%%%%
\begin{figure}[htbp]
\vspace{12.0cm}
\special{psfile=r2_shadow.eps hscale=75 vscale=75 hoffset=60 voffset=5 angle=-2.5}
\caption{\small{View of the projected shadow (in light yellow) created by the 
main torus cryostats and drift chambers as projected on the face of the FTOF 
system (shown in orange and dark red).}}
\label{shadow1}
\end{figure}
%%%%%%%%%%%%%%%%%%%%%%%%%%%%%%%%%%%%%%%%%%%%%%%%%%%%%%%%%%%%%%%%%%%%%%%%%%%%%

As with the design of the current {\tt CLAS} TOF counters, the inactive
components of the detector for the FTOF system must be designed to fit
fully within the projected space (the shadow) behind the superconducting 
coils of the main torus.  Therefore all light guides, PMTs, voltage 
dividers, and cables have tight constraints on how much space they can 
take up.  Fig.~\ref{shadow1} shows the shadow created by the main torus 
cryostats and drift chambers as projected on the face of the FTOF system.  
This picture comes from our CAD model of the {\tt CLAS12} system for one 
particular design assumption for the drift chamber endplate design.  The 
limits of the shadow region are actually defined at the current time by 
the Region~2 drift chamber system.  While the Region~1 and Region~3 
chambers have their endplates, on-board electronics, and readout cables 
located in the shadow of the torus cryostats, the Region~2 chambers are 
located fully between the cryostats.  Thus the Region~2 endplates define
the ultimate shadow for the definition of the FTOF panel-1 inactive
region.  Fig.~\ref{phi_cov} shows the $\phi$ angle coverage for each
counter of the FTOF system.

%%%%%%%%%%%%%%%%%%%%%%%%%%%%%%%%%%%%%%%%%%%%%%%%%%%%%%%%%%%%%%%%%%%%%%%%%%%%%
\begin{figure}[htbp]
\vspace{9.0cm}
\special{psfile=phi_cov.ps hscale=85 vscale=85 hoffset=-35 voffset=-140}
\caption{\small{Calculation of the $\phi$ angle coverage of each counter
in the FTOF system using the nominal design for the FTOF system.  The
discontinuity in the $\phi$ coverage for panel-1b is due to the use of
shorter PMTs for the first six counters (97-mm long Hamamatsu R9779 PMTs), 
whereas the larger angle counters employ slightly longer PMTs (131-mm long 
Photonis XP20D0B PMTs).}}
\label{phi_cov}
\end{figure}
%%%%%%%%%%%%%%%%%%%%%%%%%%%%%%%%%%%%%%%%%%%%%%%%%%%%%%%%%%%%%%%%%%%%%%%%%%%%%

\section{FTOF System Description}

The time-of-flight detectors provide an electronic signal for the data 
acquisition system, which reflects the time a particle passes through the 
scintillator.  Passing through the scintillator, the particle ionizes the 
material and subsequently generates scintillation light; these photons travel 
on various paths inside of the scintillator and the light guide, may 
get absorbed, reflected (internally or on outer coatings), and ultimately 
impinge on the photocathode of the photomultiplier tube (PMT), producing a 
current of photoelectrons.  In several stages this current gets amplified 
within the PMT and is then available as an electronic output pulse.  The 
pulse finally passes through various components of the electronics system, 
including a discriminator and a time-to-digital converter (TDC) for computer 
readout.  These various processes influence and determine the total time 
resolution $\sigma_{TOF}$.  It is convenient to parameterize $\sigma_{TOF}$ 
with the following formula:

\begin{equation}
\sigma_{TOF} = \sqrt{\sigma_0^2 + \frac{\sigma_1^2 + (\sigma_P L/2)^2}
{N_{pe}\exp(-L/2 \lambda)}}.
\end{equation}

The parameters in this formula quantify the characteristics of the detector 
geometry and components (i.e. scintillator, PMTs, and electronics).  In 
particular, $\lambda$ is the attenuation length of the scintillator and $L$ 
its length; $\sigma_0$ represents the intrinsic resolution of the electronics 
and other processes that are independent of the light intensity, $\sigma_1$ 
models the jitter in the combined single-photoelectron response of the 
scintillator and PMT, and $\sigma_P$ accounts for path length variations 
in the light collection.  Path length variations in the scintillator scale 
with the distance from the source to the PMT, which we take to be half the 
length of the counter ($L/2$), since the scintillators are read out at either 
side.  The statistical behavior of the last two terms is indicated by scaling 
the single-photoelectron responses by $\sqrt{N_{pe}}$, where $N_{pe}$ is the 
average number of photoelectrons seen by the PMT of a counter with an 
infinitely long attenuation length.  For scintillators that are several 
meters long, the dominant contribution comes from transit time variations of 
photon paths in the scintillator.  Parameters for the present {\tt CLAS} TOF 
system are given in Ref.~\cite{tof_note}.

\subsection{Geometry}

Fig.~\ref{sigma_tof} illustrates how the detector size (width and length) 
affects the timing resolution.  As a guide to necessary improvements in the 
time resolution of the system, we scale the parameterization of the 
{\tt CLAS} system from the present 15-cm wide counters down to 6~cm in width.
An important consideration is a good match of the detector cross section to 
the size of the PMT entrance window.  How exactly the connection is optimized
(e.g. with or without a light guide), needs to be studied with Monte Carlo 
simulations and detector prototypes, which are being built at the University 
of South Carolina.  Preliminary results indicate that the best time
resolutions are obtained without light guides.  Fig.~\ref{sigma_tof} also 
shows how, by means of a combined measurement from the two detector planes, 
one can improve on the time resolution.

%%%%%%%%%%%%%%%%%%%%%%%%%%%%%%%%%%%%%%%%%%%%%%%%%%%%%%%%%%%%%%%%%%%%%%%%%%%%%
\begin{figure}[t]
\vspace{8.0cm}
\special{psfile=ftof_width_6cm.eps hscale=50 vscale=50 hoffset=420 
voffset=-35 angle=90}
\caption{\small{Expected time resolution for the existing panel-1a counters
(15-cm wide) with an assumed intrinsic electronic resolution of 40~ps.  Also 
shown are predictions for the new panel-1b counters (6-cm wide) and for the
combined panel-1a and panel-1b counters.  The arrow along the bottom of the
plot shows the range of scintillator lengths in the FTOF panel-1 counters.}}
\label{sigma_tof}
\end{figure}
%%%%%%%%%%%%%%%%%%%%%%%%%%%%%%%%%%%%%%%%%%%%%%%%%%%%%%%%%%%%%%%%%%%%%%%%%%%%%

\subsection{Scintillation Material}

The parameterization of $\sigma_{TOF}$ is used to study the possible 
improvements in resolution based on a trade-off between the decay time of 
the scintillator ($\sigma_1$) and the number of photoelectrons ($N_{pe}$) 
arriving at the PMT, which depends on the attenuation length $\lambda$.  The 
bulk attenuation length and the scintillator decay times for three typical 
scintillators are listed in Table~\ref{parms}.  The actual value for the 
attenuation length in a given setup may differ and will be measured on 
prototype detectors. In Fig.~\ref{sc_tof} the expected resolution is plotted 
as a function of counter length for the three scintillators listed in 
Table~\ref{parms}. For the figure we have used bulk attenuation lengths for 
BC-404 and BC-418, while we have used the measured values for BC-408, which 
is used in the current {\tt CLAS} TOF system.  The plot illustrates that 
the overall performance of short counters, less than 200~cm in length, is 
improved by the use of fast scintillators with small decay times $\tau$, 
whereas for long counters, the existing material BC-408 with its larger 
attenuation length is the better choice.  The final choice for the short 
FTOF counters is an open question and will be explored experimentally.  

The new scintillation counters, which are formed in a casting process
against glass surfaces, will be milled using a diamond-edge fly cutting 
technique on two sides of the counters and the ends.  This process has been 
shown to leave the surface typically flatter and smoother than hand-polished 
surfaces.  Costs associated with diamond-tooled finishes on all four
sides of the scintillator bar are significantly more than on two sides only.
The technical representatives at Saint-Gobain Crystals have indicated that
scintillation bars with only two sides diamond-tool finished and two casted
against glass, perform better than bars with all four sides machined.  We
will perform bench tests to verify these claims before making a final decision.

The geometry and materials for the interface between the scintillator and PMT, 
and the wrapping material for the scintillator to provide for light-tightness 
and improved light collection are important design parameters and will be 
studied on prototypes to provide for a maximum of photoelectrons.

%%%%%%%%%%%%%%%%%%%%%%%%%%%%%%%%%%%%%%%%%%%%%%%%%%%%%%%%%%%%%%%%%%%%%%%%%%%%%
\begin{table}[htbp]
 \begin{minipage}{1.5cm}
~~
 \end{minipage}
 \begin{minipage}{8.0cm}
   \begin{tabular} {|c|c|c|} \hline
     Scintillator & Bulk ($\lambda$) (cm) & $\tau$ (ns) \\ \hline
     BC-408 & 380 & 2.1 \\
     BC-404 & 160 & 1.8 \\
     BC-418 & 100 & 1.4 \\ \hline
   \end{tabular}
 \end{minipage}
 \begin{minipage}{8.0cm}
   \begin{tabular} {|c|c|} \hline
     PMT ($2''$) & Rise Time (ns) \\ \hline
     XP20D0B   & 2.5 \\
     XP2020    & 1.5 \\
     XP2020/UR & 1.4 \\
     R9779     & 1.8 \\ \hline 
   \end{tabular}
 \end{minipage}
\caption{\small{Attenuation lengths $\lambda$ and decay times $\tau$ for 
various scintillators (left) and rise times (right) for various photomultiplier
tubes.  The present {\tt CLAS} TOF system utilizes BC-408 scintillator coupled 
to 2-in Thorn EMI9954A PMTs (panel-1a) and 3-in Philips XP4312B PMTs
(panel-2).}}
\label{parms}
\end{table}
%%%%%%%%%%%%%%%%%%%%%%%%%%%%%%%%%%%%%%%%%%%%%%%%%%%%%%%%%%%%%%%%%%%%%%%%%%%%%

%%%%%%%%%%%%%%%%%%%%%%%%%%%%%%%%%%%%%%%%%%%%%%%%%%%%%%%%%%%%%%%%%%%%%%%%%%%%%
\begin{figure}[htbp]
\vspace{8.0cm}
\special{psfile=ftof_att_6cm.eps hscale=50 vscale=50 hoffset=430 voffset=-35 
angle=90}
\caption{\small{Resolution for various scintillation materials showing the 
trade-off between attenuation length and decay time.  Estimates assume 6-cm 
wide scintillators and an intrinsic resolution of the electronics of 40~ps.
The arrow along the bottom of the plot shows the range of scintillator lengths 
in the FTOF panel-1 counters.}}
\label{sc_tof}
\end{figure}
%%%%%%%%%%%%%%%%%%%%%%%%%%%%%%%%%%%%%%%%%%%%%%%%%%%%%%%%%%%%%%%%%%%%%%%%%%%%%

\subsection{Light Guides}
\label{lgs}

Preliminary simulation studies indicate that the best time resolutions 
for the panel-1b counters will be obtained without light guides since 
the number of photons entering the PMT window is higher by a factor of 
two than with the best optimized light guide of non-zero length.  
Eliminating the light guides altogether allows for long enough 
scintillators to cover the full fiducial region given the constraints of 
the PMT, voltage divider, and cable sizes.

\subsection{Photomultiplier Tubes}
\label{sec:pmt}

Prototypes have realized a timing resolution of 70~ps for 200-cm long 
detectors using Philips XP2020/UR PMTs, which have 25\% faster rise times 
than the more standard Philips XP2262 PMTs.  This is achieved with an 
improved transit-time spread across the photocathode.  PMTs contribute to 
a large fraction of the total hardware costs for the FTOF system.  Careful 
prototyping of the detectors is therefore especially important for the 
selection of a specific tube and voltage divider network, as, e.g., the 
XP2020/UR is about twice as expensive as the XP2262.  While we plan to 
perform detailed prototyping efforts to optimize the choice of components, 
we expect that a faster PMT will be required.  In Table~\ref{parms} we 
give the rise times for various common tubes.  Faster PMTs are available, 
but in practice should be matched to the rise times of the combined 
scintillator-light guide system.  

PMT performance must be carefully weighed against the tube geometry.
An important criterion for the new panel-1b counters is that they
cover the same $\phi$ range as the drift chambers.  In other words,
all of the inactive system elements must reside within the projected
shadow of the torus coils/drift chambers.  Thus it is imperative to
select a PMT model that meets this requirement.  The choice of the PMTs 
for the panel-1b counter readout will employ two different PMTs.  For 
the first six counters, we will employ Hamamatsu R9779 PMTs (97-mm long).  
For the remaining counters we will employ Photonis XP20D0B PMTs (131-mm 
long).  The design requirement to match the acceptances of the FTOF system 
to the drift chambers is an important consideration.  The coverage 
specification for the drift chambers is to provide 50\% $\phi$ coverage 
at $\theta$=5$^\circ$.  This specification is only met for the FTOF 
system when using the shorter PMTs.  This $\phi$ coverage of the panel-1b
counters with this design choice is highlighted in Fig.~\ref{phi_cov}.  The 
final number of counters readout with the Hamamatsu PMTs still needs to be 
finalized.  The issue will be a compromise between the physics requirements 
for large $\phi$ coverage at small polar angles and the increased costs
of the Hamamatsu PMTs compared to the Photonis PMTs.  Also the timing
specifications for the Hamamatsu PMTs are slightly worse than for the
Photonis PMTs.  However, this is offset by the improved light collection
for the shorter counters.

\subsection{Voltage Dividers}

A schematic diagram of the high-voltage divider currently used for the
{\tt CLAS} TOF panel-1a and panel-2 readout is shown in Fig.~\ref{divider}
\cite{smith1}.  For the panel-1b counters we are presently planning on
using the Photonis Imaging Systems VD127K/T transistorized hybrid voltage 
divider as shown in Fig.~\ref{divider}.  Each of these active dividers use 
high-voltage transistors to fix the PMT gain by stabilizing the voltage and
to protect the PMT against high light levels by shutting down the circuit 
in an over-current situation.  

The tube-base assembly for the current {\tt CLAS} panel-1a counters is shown 
in Fig.~\ref{tube_base}.  In order to allow the scintillator to span the
maximum area of the detector, the PMT and voltage divider are required to
fit in the shadow of the main torus magnet.  The same constraint will be
imposed on the tube-base assemblies for {\tt CLAS12}.  A design of this
sort meets our requirements of an extremely compact design.

%%%%%%%%%%%%%%%%%%%%%%%%%%%%%%%%%%%%%%%%%%%%%%%%%%%%%%%%%%%%%%%%%%%%%%%%%%%%%
\begin{figure}[htbp]
\vspace{9.5cm}
\special{psfile=divider.eps hscale=43 vscale=43 hoffset=-15 voffset=-40}
\special{psfile=photonis_base.ps hscale=35 vscale=45 hoffset=215 voffset=270 angle=-90}
\caption{\small{(Left) Schematic diagram of the high-voltage divider for the 
EMI-9954A PMT used for the current {\tt CLAS} panel-1 readout.  The last four 
stages of the divider are stabilized by high-voltage FETs.  (Right) Schematic
diagram of the active Photonis VD127K/T dividers planned for the panel-1b 
counters.}}
\label{divider}
\end{figure}
%%%%%%%%%%%%%%%%%%%%%%%%%%%%%%%%%%%%%%%%%%%%%%%%%%%%%%%%%%%%%%%%%%%%%%%%%%%%%

%%%%%%%%%%%%%%%%%%%%%%%%%%%%%%%%%%%%%%%%%%%%%%%%%%%%%%%%%%%%%%%%%%%%%%%%%%%%%
\begin{figure}[htbp]
\vspace{5.2cm}
\special{psfile=tube_base.eps hscale=65 vscale=65 hoffset=60 voffset=-10}
\caption{\small{Side view of the PMT-base assembly mounted on the {\tt CLAS}
panel-1a TOF scintillators.  The compact design is required by space 
limitations in the detector for the 2-in PMT.}}
\label{tube_base}
\end{figure}
%%%%%%%%%%%%%%%%%%%%%%%%%%%%%%%%%%%%%%%%%%%%%%%%%%%%%%%%%%%%%%%%%%%%%%%%%%%%%

\subsection{Assembly}

Each scintillation counter will be individually wrapped, and assembled with
two photomultiplier tubes, one at each end.  The layers are shown in 
Fig.~\ref{wrap} for the existing panel-1a and panel-2 counters.  The outer 
coverings of the scintillator consist of two layers of aluminum foil, one 
strip of 5-mil lead foil on the side facing the target, and finally, one 
layer of black Kapton.  The lead foil was included to shield the scintillators 
from background x-rays produced in the target.  This is the only extra 
material facing the target.

For the new panel-1b counters, studies are underway to decide on the final
wrapping materials, especially with regard to adding a thin lead sheet for
x-ray absorption on the side facing the target.  Currently we plan on
wrapping the counters in a single layer of metalized mylar film and then
a double layer of Tedlar PVF film for light tightness.  Presently we are not 
planning to change the wrapping of the existing panel-1a and panel-2 counters 
as a cost reduction measure.

%%%%%%%%%%%%%%%%%%%%%%%%%%%%%%%%%%%%%%%%%%%%%%%%%%%%%%%%%%%%%%%%%%%%%%%%%%%%%
\begin{figure}[ht]
\vspace{6.5cm}
\special{psfile=panel1a_wrap.eps hscale=70 vscale=70 hoffset=40 voffset=-205}
\caption{\small{Cross section of counter wrapping for the existing {\tt CLAS} 
panel-1a and panel-2 counters. The UV-transmitting fiber is used for the
laser calibration system.  The lead foil is to absorb low-energy background 
x-rays from the target.}}
\label{wrap}
\end{figure}
%%%%%%%%%%%%%%%%%%%%%%%%%%%%%%%%%%%%%%%%%%%%%%%%%%%%%%%%%%%%%%%%%%%%%%%%%%%%%

After wrapping, each panel-1a and panel-2 scintillator was attached to a 
support structure, light guides and photomultiplier tubes were glued in 
place, and an optical fiber installed for the laser calibration system. 
Dymax 3-20262 UV-curing optical cement was used as a bond for the light
guide/scintillator interface, as well as the PMT/light guide interface.
This approach is also planned for the panel-1b counters.

The panel-1a and panel-2 counters are each supported individually by a
composite sandwich structure of stainless steel skins on structural foam 
that is attached to the detector frame only at the two ends. The composite 
structure, which mounts on the scintillator side facing away from the target,
provides uniform material thickness to the scattered particles.  The support 
was undersized so the counters could be placed as close together as allowed 
by the wrapping material.  The assembly with support is shown in 
Fig.~\ref{support}.

%%%%%%%%%%%%%%%%%%%%%%%%%%%%%%%%%%%%%%%%%%%%%%%%%%%%%%%%%%%%%%%%%%%%%%%%%%%%%
\begin{figure}[ht]
\vspace{10.5cm}
\special{psfile=panel1a_mount.eps hscale=70 vscale=70 hoffset=10 voffset=-145}
\caption{\small{Counter and support assembly. a) Top view and enlarged end 
view of the panel-1a and panel-2 assemblies.  b) Side view of the panel-2
mounting assembly.}}
\label{support}
\end{figure}
%%%%%%%%%%%%%%%%%%%%%%%%%%%%%%%%%%%%%%%%%%%%%%%%%%%%%%%%%%%%%%%%%%%%%%%%%%%%%

The panel-1a counters were mounted on 1-in-thick supports to minimize the 
thickness of the package from the standpoint of Coulomb multiple scattering
and energy loss considerations.  The maximum deflection for the installed 
scintillators is 4.4~mm, as estimated from deflection tests and the compound 
angle of each detector,  which relieves the overall support requirements.
The space for the panel-2 counters allowed for 3-in-thick sandwich supports,
which were mechanically much stiffer and resulted in no appreciable 
deflection. 

The mounting of the panel-1b counters in front of the existing panel-1a
counters is being designed to allow for minimal additional material in the
active area of the detectors.  Fig.~\ref{tofsupport1} shows the current 
mounting plan for the panel-1 counters.  Here the panel-1b counters will 
be mounted on a triangular support frame very similar in design to that 
employed for the existing panel-1a counters.  Side views of the panel-1 
counters showing the individual support frames is shown in 
Fig.~\ref{tofsupport2}.  The support frames for the two panels in each 
sector will be mounted to the forward carriage using legs that project 
through the gaps between the PCAL and EC detectors.

%%%%%%%%%%%%%%%%%%%%%%%%%%%%%%%%%%%%%%%%%%%%%%%%%%%%%%%%%%%%%%%%%%%%%%%%%%%%%
\begin{figure}[ht]
\vspace{7.0cm}
\special{psfile=tof_panel1_support.eps hscale=50 vscale=50 hoffset=85 voffset=0}
\caption{\small{Panel-1 counter support structure diagram showing the
angle brackets used to attach the counters in the different panels to their
associated support frames.}}
\label{tofsupport1}
\end{figure}
%%%%%%%%%%%%%%%%%%%%%%%%%%%%%%%%%%%%%%%%%%%%%%%%%%%%%%%%%%%%%%%%%%%%%%%%%%%%%

%%%%%%%%%%%%%%%%%%%%%%%%%%%%%%%%%%%%%%%%%%%%%%%%%%%%%%%%%%%%%%%%%%%%%%%%%%%%%
\begin{figure}[ht]
\vspace{9.5cm}
\special{psfile=support1.eps hscale=50 vscale=50 hoffset=5 voffset=10}
\special{psfile=support2.eps hscale=27 vscale=27 hoffset=180 voffset=30}
\caption{\small{Side views of the panel-1 counters in a single sector 
showing the panel-1a and panel-1b counters mounted on their associated
support frames.  These figures were generated from our 3--D CAD model.}}
\label{tofsupport2}
\end{figure}
%%%%%%%%%%%%%%%%%%%%%%%%%%%%%%%%%%%%%%%%%%%%%%%%%%%%%%%%%%%%%%%%%%%%%%%%%%%%%

\subsection{Magnetic Shielding}

The FTOF PMTs will be located roughly 650~cm from the target in a local
magnetic field that should be less than 24~G (maximum that occurs in
the axial direction with respect to the PMT), given the field map for the 
current design of the {\tt CLAS12} torus.  A magnetic shield for the PMTs 
must be included to reduce both the axial and transverse components of the 
field, especially near the photocathode.  These conditions are similar to 
those of the PMTs in the current {\tt CLAS} TOF system.  In the existing 
panel-1a and panel-2 counters, the magnetic shield consists of a cylinder 
made from 0.020-in thick $\mu$-metal, with the shield extending 2-in beyond 
the front face of the PMT.  For the panel-1b counters, we are planning on a 
design that consists of a cylinder made from 1 to 2-mm thick $\mu$-metal, 
with the shield ending at the front face of the PMT.  The outside face of 
the cylinder will have a $\mu$-metal endcap to improve the shielding
against the axial field and the cable connections to the voltage divider
(anode, dynode, and high voltage) will be made through the shield.  Detailed 
studies of these PMTs will be carried out in various magnetic fields to 
optimize the final design of the magnetic shielding.

\subsection{Laser Calibration System} 
\label{sec:laser}

A system of ultraviolet (UV) lasers is used to test and calibrate the 
existing {\tt CLAS} TOF counters.  This system will continue to be employed
and will be updated to include connections to the new panel-1b counters. 
The UV light is delivered to the center of each scintillator via a silica 
optical fiber.  The fiber core diameter is 200~$\mu$m with a 240-$\mu$m 
cladding.  The TDC and ADC information from the laser pulses is then used
to calibrate the overall timing and pulse-height time-walk. 

The calibration system consists of optical tables located near the 
counters.  Each optical table contains a Laser Photonics LN203C nitrogen 
laser operating at 337~nm enclosed in an aluminum box for RF shielding 
and personnel safety.  The laser beam is directed through an opening in 
the aluminum enclosure to a series of optical and mechanical elements.  
The laser beam first encounters a flat quartz plate that reflects a small 
fraction ($\approx$ 4\%) of the light back to a fast photodiode circuit
that is used as a reference to time the laser to the TOF scintillators.
Most of the laser light passes through a variable neutral-density filter 
with a dynamic range of 1:40.  This filter can be used to attenuate the 
light over a range of values suitable for measuring the time-walk correction 
of the scintillators.  The filter is adjusted by a remotely controlled 
stepping motor.  Downstream of the filter, the beam is expanded by a 
CTR 5$\times$3 diffuser.  The diffused beam incident on the fibers is 
uniform to within 30\%.  The beam can then be partially intercepted by a 
``mask'' controlled by a stepping motor.  Several different hole patterns 
along the ``mask'' can be positioned to illuminate various combinations of 
fiber bundles.  Each bundle consists of seven all-silica 100-$\mu$m-diameter 
fibers (numerical aperture is 0.22) that are 13-m long and distributed to 
the various scintillators.  For the panel-1a counters, there are 24 bundle 
ends that are arranged in a four-by-six rectangular array behind the ``mask'' 
within an area of 3.0~cm$^2$.

%%%%%%%%%%%%%%%%%%%%%%%%%%%%%%%%%%%%%%%%%%%%%%%%%%%%%%%%%%%%%%%%%%%%%%%%%%%%%
\begin{figure}[htbp]
\vspace{9.5cm}
\special{psfile=tof_elec1.eps hscale=60 vscale=58 hoffset=60 voffset=-5}
\caption{\small{Overall schematic of the electronics used in the current
{\tt CLAS} TOF system.}}
\label{electronics_old}
\end{figure}
%%%%%%%%%%%%%%%%%%%%%%%%%%%%%%%%%%%%%%%%%%%%%%%%%%%%%%%%%%%%%%%%%%%%%%%%%%%%%

\subsection{Electronics}

The FTOF counters will generate prompt signals for the {\tt CLAS12} 
Level-1 trigger electronics, as well as signals for pulse-height and 
timing analysis.  The overall layout of the TOF electronics currently used 
in {\tt CLAS} is shown in Fig.~\ref{electronics_old}.  In this design the 
PMT dynode pulses go to a pretrigger circuit to generate the Level-1 trigger.
Each PMT anode pulses go to a LeCroy 1881M FASTBUS ADC and a CAEN VME 1190 
pipeline TDC (100~ps LSB) for readout.  A simplified design using pipeline 
TDCs, flash ADCs (or pipeline ADCs), and improved electronics for forming 
the trigger signals is planned for {\tt CLAS12} (see 
Fig.~\ref{electronics_new}).  With the incorporation of the new ADCs, the 
miles of delay cables employed in the current TOF system can be eliminated.

The intrinsic resolution of the electronics system ($\sigma_0$) must be 
reduced and it has been measured to be as small as 14~ps in various setups
\cite{smith2}.  There are many contributions to this term, and each electronic 
component will be chosen to insure that it meets our specifications.  In 
order to achieve the rate capability at a luminosity of 
$1 \times 10^{35}$~cm$^{-2}$s$^{-1}$, a high-resolution pipeline TDC will be 
used for the new panel-1b counters.  These modules may be the commercially 
available modules used for the readout of the existing {\tt CLAS} TOF 
counters, or may be electronics developed at JLab.  For instance, the JLab 
Fast Electronics group is developing such a TDC based on the COMPASS F1 chip. 
Presently, we assume $\sigma_0$ = 40~ps in our estimates in 
Figs.~\ref{sigma_tof} and \ref{sc_tof}, limited by the resolution of the 
COMPASS F1 TDC chip. 

%%%%%%%%%%%%%%%%%%%%%%%%%%%%%%%%%%%%%%%%%%%%%%%%%%%%%%%%%%%%%%%%%%%%%%%%%%%%%
\begin{figure}[htbp]
\vspace{6.7cm}
\special{psfile=tof_elec2.eps hscale=60 vscale=60 hoffset=70 voffset=0}
\caption{\small{Overall schematic of the electronics planned for the new
{\tt CLAS12} FTOF system.}}
\label{electronics_new}
\end{figure}
%%%%%%%%%%%%%%%%%%%%%%%%%%%%%%%%%%%%%%%%%%%%%%%%%%%%%%%%%%%%%%%%%%%%%%%%%%%%%

\subsubsection{Cables}

Fast timing of signals from the FTOF system requires cables with low 
signal distortion.  Measurements were made of the response of several 
types of coaxial cable for the existing {\tt CLAS} TOF system, including 
RG-58, the usual cable for fast NIM electronics, RG-213, and Belden 9913, 
a low-loss semi-solid polyethylene cable.  The lengths of the cables 
($\sim$400~ns) were chosen to match the time required to form a first-level 
trigger for {\tt CLAS}.  The response of the cables to a NIM logic signal, 
-0.75~V and 10~ns duration, was measured, and the transmission of pulses 
through the cable was simulated.  Note that the cable dispersion in high 
quality Belden 9913 is approximately equal to the dispersion in standard 
RG-213 cable for the same delay because the slower velocity in RG-213 
($\beta = 0.66$) compared to Belden 9913 ($\beta= 0.84$) compensates for 
the larger attenuation.  We plan to keep the Belden 9913 cable from the PMT 
dynode to the trigger electronics and RG-213 from the anode to the ADC and 
discriminator for panels-1a and 2.  The faster trigger cables reduce the 
need for additional signal delay.  Since the panel-1b counters will
typically not be in the trigger for {\tt CLAS12} electron runs, the
preferred choice of readout cable for the new panel-1b counters is to
use short standard RG-58 cables to feed into a nearby pipeline TDC and
ADC readout system.

\subsubsection{High-Voltage Supplies}

The photomultipliers for the FTOF counters will operate at about 2000~V with
negative polarity.  The maximum dark current drawn by the PMTs for the
existing panel-1a counters is about 20~nA and is about 30~nA for the
panel-2 PMTs.  The current {\tt CLAS} TOF system is powered by five 
LeCroy 1458 mainframes that can contain up to sixteen cards, each supplying 
12 independent channels for a maximum of 192 channels per mainframe.  This
system will be reused for the FTOF system, however due to the number of HV
channels in the panel-1b system, additional HV channels will have to
be added to the system.

\section{Simulations and Reconstruction}

The FTOF system is being modeled within our GEANT4 Monte Carlo framework.  
Fig.~\ref{gsim_ftof} shows a representation of the counters in this model,
including panel-1a, panel-1b, and panel-2 in each sector of {\tt CLAS12}.
The output of the Monte Carlo are the ADC and TDC values associated with
the two ends of each counter where there is a hit, modeled with the
appropriate resolution expected for these detectors.  At the current time, 
the coding for the full FTOF reconstruction is being implemented in the 
Monte Carlo.  Studies of the angular coverage of the counters, particle 
identification vs. momentum, reconstruction for tracks that cross multiple 
counters, and the impacts of the panel support frames will be carried out
in the near future.  However, even with the low-level ADC and TDC information
currently available, we can already gain some insight into the expected 
coverage and response of this system.

%%%%%%%%%%%%%%%%%%%%%%%%%%%%%%%%%%%%%%%%%%%%%%%%%%%%%%%%%%%%%%%%%%%%%%%%%%%%%
\begin{figure}[htbp]
\vspace{15.0cm}
\special{psfile=ftof_gemc.eps hscale=80 vscale=80 hoffset=20 voffset=10}
\caption{\small{Model of the FTOF system included in our GEANT3 Monte 
Carlo. (a). Perspective view looking downstream of the six FTOF sectors 
divided into panel-1a, 1b, and 2. (b) Close-up view of (a) focussing in 
on a single sector. (c). View of the back of panel-1a (looking upstream) 
showing details of the definitions of panel-1a (pink) and panel-1b (green).}}
\label{gsim_ftof}
\end{figure}
%%%%%%%%%%%%%%%%%%%%%%%%%%%%%%%%%%%%%%%%%%%%%%%%%%%%%%%%%%%%%%%%%%%%%%%%%%%%%

Fig.~\ref{ftof_acc} shows the acceptance of the {\tt CLAS12} FTOF system
in terms of $\phi$ and $\theta$.  Plots of this sort serve to define the 
shadow region in terms of $\phi$ vs. $\theta$ defined by the main torus 
cryostat and the drift chamber system.  In this shadow, all of the inactive 
components of the detectors must reside, including light guides, PMTs, 
voltage dividers, magnetic shielding, support frames and structures, and 
cables.  Fig.~\ref{ftof_acc} shows the thrown particles in the simulation
and the particles reconstructed by the FTOF system.  The final definition 
of the shadow region serves to define the lengths of the new panel-1b 
counters as a function of polar angle, as well as the number of required
panel-2 counters.

%%%%%%%%%%%%%%%%%%%%%%%%%%%%%%%%%%%%%%%%%%%%%%%%%%%%%%%%%%%%%%%%%%%%%%%%%%%%%
\begin{figure}[htbp]
\vspace{14.0cm}
\special{psfile=theta.eps hscale=49 vscale=47 hoffset=-20 voffset=200}
\special{psfile=phi.eps hscale=44 vscale=48 hoffset=225 voffset=200}
\special{psfile=y_vs_x.eps hscale=50 vscale=55 hoffset=90 voffset=-10}
\caption{\small{Preliminary GEANT4 Monte Carlo output showing the 
geometric coverage of the panel-1b FTOF system in terms of $\theta$ (deg) 
(UL) and $\phi$ (UR) (deg).  Here the black curves are the thrown events 
and the red curves are the reconstructed events.  Also shown for panel-1b
is the reconstructed $y$-coordinate vs. $x$ coordinate (m) (bottom) for the 
full FTOF system.}}
\label{ftof_acc}
\end{figure}
%%%%%%%%%%%%%%%%%%%%%%%%%%%%%%%%%%%%%%%%%%%%%%%%%%%%%%%%%%%%%%%%%%%%%%%%%%%%%

%\section{Engineering and Design: Task List and Time Line}
%
%The engineering and design tasks and time line of the {\tt CLAS12} FTOF 
%system are based in large part on our experiences with the design and
%construction of the existing {\tt CLAS} TOF system.  In this section
%we present an overview of the specific engineering and design tasks and
%time line that are associated with the {\tt CLAS12} FTOF system.  Shown
%below is an overview of the engineering and design time line to complete
%the various tasks by the beginning of FY09.
%
%\begin{tabbing}
%~~~~~$\bullet$ Student Training \hskip 7.0cm                 \= 8/2005 \\
%~~~~~$\bullet$ PMT and Scintillator Acquisition              \> 8/2005 \\
%~~~~~$\bullet$ Assembly Hall Acquisition 		     \> 9/2005 \\
%~~~~~$\bullet$ Data Acquisition System Implementation        \> 6/2006 \\
%~~~~~$\bullet$ Detector Lab Infrastructure	 	     \> 7/2006 \\
%~~~~~$\bullet$ Verification of Electronics Specifications    \> 8/2006 \\
%~~~~~$\bullet$ Calibration of Different TDCs		     \> 7/2006 \\
%~~~~~$\bullet$ Verification of Current TOF Specifications    \> 8/2006 \\
%~~~~~$\bullet$ First Prototype Tests for FTOF	             \> 9/2006 \\
%~~~~~$\bullet$ Prototype Tests for Different Geometries      \> 8/2007 \\
%~~~~~$\bullet$ Prototype Tests with Different PMTs	     \> 8/2007 \\
%~~~~~$\bullet$ Overall Optimized Time Resolution	     \> 8/2007 \\
%~~~~~$\bullet$ Prototype Detector Beam Tests at JLab	     \> 9/2007 \\
%~~~~~$\bullet$ Planning of Final Detector Design	     \> 9/2007 \\
%~~~~~$\bullet$ Design of Detector Assembly Tools	     \> 9/2007 \\
%~~~~~$\bullet$ Detector Calibration System Design	     \> 9/2008 \\
%~~~~~$\bullet$ Complete Detector Design		             \> 9/2008 \\
%~~~~~$\bullet$ Detector Component Acquisition	             \> 9/2008 \\
%~~~~~$\bullet$ Complete Infrastructure for Assembly	     \> 9/2008 \\
%\end{tabbing}
%
%\section{Construction Cost}
%
%The construction costs associated with the FTOF system are provided in
%Table~\ref{costs}.  Here the specific costs associated with the PMTs,
%scintillator material, light guides, voltage dividers, mechanical support,
%signal and high voltage cables, and the laser calibration system are 
%provided.  This table also provides a proposed breakdown of the construction 
%costs on a year-by-year basis over the four-year construction period for
%this system.  Note that the costs (associated with the new panel-1b
%counters) are based on 58 counters/sector $\times$ 6~sectors $\times$ 
%2~ends/counter or 700~units.  All costs are scaled up by 10\% to account for 
%taxes, shipping, and handling.
%
%%%%%%%%%%%%%%%%%%%%%%%%%%%%%%%%%%%%%%%%%%%%%%%%%%%%%%%%%%%%%%%%%%%%%%%%%%%%%
%\begin{table}[htbp]
%\begin{center}
%\begin{tabular}{|l|l|l|l|l|l|} \hline
%                       &           &  \multicolumn{4} {|c|} {Year} \\ \cline{3-6}
%~~~~~~~~~~{\bf Item}   & {\bf Cost}& ~~~1     & ~~~2     & ~~~3     & ~~~4    \\ \hline  
%1. PMTs                & \$630.7k  & \$210.2k & \$210.2k & \$210.2k & \$0     \\ \hline 
%2. Scintillator        & \$345.4k  & \$115.1k & \$115.1k & \$115.1k & \$0     \\ \hline
%3. Light Guides        & \$215.6k  & \$71.9k  & \$71.9k  & \$71.9k  & \$0     \\ \hline
%4. Voltage Dividers    & \$105.5k  & \$35.2k  & \$35.2k  & \$35.2k  & \$0     \\ \hline
%5. Magnetic Shielding  & \$65.5k   & \$21.8k  & \$21.8k  & \$21.8k  & \$0     \\ \hline
%6. Mechanical Support  & \$70.2k   & \$0      & \$0      & \$35.1k  & \$35.1k \\ \hline
%7. Signal Cables       & \$28.7k   & \$9.6k   & \$9.6k   & \$9.6k   & \$0     \\ \hline
%8. High Voltage Cables & \$53.5k   & \$17.8k  & \$17.8k  & \$17.8k  & \$0     \\ \hline
%9. Laser System        & \$41.8k   & \$13.9k  & \$13.9k  & \$13.9k  & \$0     \\ \hline
%~~~~~~~~~~{\bf Total}  & \$1556.9k & \$495.5k & \$495.5k & \$530.6k & \$35.1k  \\ \hline
%\end{tabular}
%\end{center}
%\caption{\small{Breakdown of the construction costs for the {\tt CLAS12} FTOF 
%system, including a proposed cost breakdown of the over the four-year duration 
%of the construction project.}}
%\label{costs}
%\end{table}
%%%%%%%%%%%%%%%%%%%%%%%%%%%%%%%%%%%%%%%%%%%%%%%%%%%%%%%%%%%%%%%%%%%%%%%%%%%%%
%
%Each of the budget items included in Table~\ref{costs} is expanded in more
%detail below to discuss specifically what is included in each budget category,
%what manufacturers and models are being considered, and the source of the
%quote.
%
%\vskip 0.5cm
%
%\noindent
%\underline{PMTs:}
%
%\vskip 0.3cm
%
%The cost estimate for the panel-1b PMTs assumes that we will employ Philips
%XP2020/UR devices.  Our pricing information comes from Photonis Imaging Sensors.
%The have quoted us a price of \$819/PMT.  The budgeted amount for the PMTs is
%given by:
%
%\begin{center}
%\$819/PMT $\times$ 700 PMT $\times$ 1.1 (taxes/S\&H) = \$630.7k
%\end{center}
%
%\noindent
%\underline{Scintillator:}
%
%\vskip 0.3cm
%
%The scintillator cost is based on using BC-404 scintillator for the new panel-1b
%counters.  We have obtained a quote from Saint-Gobain Crystals to manufacture 
%six pieces each of our 58 different length scintillators.  The cost assumes that 
%two surfaces of the rectangular counter are cast against a glass surface and the 
%other two surfaces and ends are diamond-tool finished.  The total number of 
%pieces of 6-cm wide by 6-cm high scintillator ranging from 32~cm to 376~cm in 
%length is 384.  The budgeted amount for the scintillator is given by:
%
%\begin{center}
%\$314k $\times$ 1.1 (taxes/S\&H) = \$345.4k
%\end{center}
%
%\noindent
%\underline{Light Guides:}
%
%\vskip 0.3cm
%
%The light guides that will be constructed for the panel-1b counters are detailed
%in Fig.~\ref{lightg}, and will be constructed from acrylic.  We have obtained
%a quote from Saint-Gobain Crystals to manufacture the 700 light guides for the 
%system.  The budgeted amount for the light guides is given by:
%
%\begin{center}
%\$280/guide $\times$ 700 guides $\times$ 1.1 (taxes/S\&H) = \$215.6k
%\end{center}
%
%\noindent
%\underline{Voltage Dividers:}
%
%\vskip 0.3cm
%
%The voltage dividers that will be employed for the panel-1b counters are detailed
%in Fig.~\ref{divider}.  These units are Photonis Imaging Sensor VD127K/T dividers,
%and the company provided the quote for the 700 dividers.  The budgeted amount for 
%the voltage dividers is given by:
%
%\begin{center}
%\$137/divider $\times$ 700 dividers $\times$ $\times$ 1.1 (taxes/S\&H) = \$105.5k
%\end{center}
%
%\noindent
%\underline{Magnetic Shielding:}
%
%\vskip 0.3cm
%
%The magnetic shielding includes 700 cylinders of 1-mm thick $\mu$-metal.  We
%have obtained a quote from The MuShield Company, Inc. for \$85/unit.  The 
%budgeted amount for the magnetic shielding is given by:
%
%\begin{center}
%\$85/unit $\times$ 700 units $\times$ 1.1 (taxes/S\&H) = \$65.5k
%\end{center}
%
%\noindent
%\underline{Mechanical Support:}
%
%\vskip 0.3cm
%
%The costs for the mechanical supports for the panel-1b scintillators are taken
%directly from the costs associated with the mechanical supports for the panel-1a
%scintillators (from 1994) scaled by an inflation factor of 1.3.  The source of 
%the estimate is Elton Smith of Jefferson Lab who was responsible for the system.  
%The budgeted amount for the mechanical supports is given by:
%
%\begin{center}
%\$54k $\times$ 1.3 = \$70.2k
%\end{center}
%
%\noindent
%\underline{Signal Cables:}
%
%\vskip 0.3cm
%
%We are planning to purchase pre-made signal cables for the panel-1b counters.  
%We have received a quote from Pasternack Enterprises for 72-ft long RG-58 
%cables.  Note that for each PMT there are two signal cables needed, one for 
%the anode and one for the dynode.  The budgeted amount for the signal cables 
%is given by:
%
%\begin{center}
%\$18.65/cable $\times$ 1400 cables $\times$ 1.1 (taxes/S\&H) = \$28.7k
%\end{center}
%
%\noindent
%\underline{High Voltage Cables:}
%
%\vskip 0.3cm
%
%We are planning to purchase pre-made high voltage cables for the panel-1b 
%counters.  We have received a quote from Pasternack Enterprises for 72-ft
%long RG-59B/U cables.  The budgeted amount for the high voltage cables is 
%given by:
%
%\begin{center}
%\$69.42/cable $\times$ 700 cables $\times$ 1.1 (taxes/S\&H) = \$53.5k
%\end{center}
%
%\noindent
%\underline{Laser System:}
%
%\vskip 0.3cm
%
%The costs associated with the laser system include the optical fibers that run 
%from the scintillators to the laser and connector pieces to attach the optical 
%fibers to the fibers attached directly to the counters.  The costs are based on 
%the 1996 purchase order for the {\tt CLAS} panel-1a fibers and connectors scaled 
%by a factor of 1.3 for inflation.  The budgeted amount for the fibers assumes 
%that we will purchase the same fiber bundle arrays used for {\tt CLAS}.  Each 
%fiber bundle includes 7 optical fibers, one of which in each bundle is employed 
%as a spare.  The vendor for the original {\tt CLAS} fibers was RoMack Inc.  In 
%the budget estimates we scale the costs by a factor of 2.52, which represents 
%the ratio of the number of panel-1b scintillators to the number of panel-1a 
%scintillators (i.e. 58/23).  The budgeted amount for the optical fibers is 
%given by:
%
%\begin{center}
%\$11k $\times$ 2.52 $\times$ 1.3 $\times$ 1.1 (taxes/S\&H) = \$39.6k
%\end{center}
%
%For the fiber connectors, the original purchase order for the panel-1a system 
%was with Cameo Electronics.  The cost was \$600 for 100 connectors.  The
%budgeted amount for the fiber connectors is given by:
%
%\begin{center}
%\$600 $\times$ 2.52 $\times$ 1.3 $\times$ 1.1 (taxes/S\&H) = \$2.2k
%\end{center}
%
%\section{Design and Construction Responsibilities} 
%
%This section provides a breakdown of the design and construction manpower
%responsibilities.  The contributing institutions are the University of
%South Carolina (USC) and JLab.  Table~\ref{manpower} provides a breakdown
%of the manpower contributions as a function of year, with years 1 through
%4 representing the FTOF construction period.  Table~\ref{tasks} provides
%further information about which group and which position will be responsible
%for the various design and construction tasks.   Fig.~\ref{usc_tests} 
%shows the test setup currently in place at USC for the testing of the
%{\tt CLAS12} scintillators for the FTOF system.
%
%%%%%%%%%%%%%%%%%%%%%%%%%%%%%%%%%%%%%%%%%%%%%%%%%%%%%%%%%%%%%%%%%%%%%%%%%%%%%
%\begin{table}[htbp]
%\begin{center}
%\begin{small}
%\begin{tabular}{|c|c|c|c|c|c|c|c|c|c|c|} \hline
%                       & \multicolumn{8} {|c|} {Year} & PED & Construction \\ \cline {2-9}
%                       & -2 & -1 & 0 & 1 & 2 & 3 & 4 & Total & Subtotal & Subtotal \\ \hline
%{\bf USC Labor}        & & & & & & & & & &  \\ \cline{1-1}
%Professor              & 20 &  30 &  30 &  30 &  30 &  30 &  30 & 200 &  80 & 120 \\
%Graduate Student       & 15 &  20 &  20 &  20 &  20 &  20 &  20 & 135 &  55 &  80 \\
%Undergraduate Student  & 40 &  50 &  50 &  50 &  50 &  50 &  50 & 340 & 140 & 200 \\
%Staff (Grant)          &  5 &  15 &  15 &  20 &  20 &  20 &  20 & 115 &  35 &  80 \\
%Staff (Other)          &  2 &  15 &   2 &   2 &   1 &   1 &   1 &  24 &  19 &   5 \\
%Contributed University &  0 &   0 &   0 &   0 &   0 &   0 &   0 &   0 &   0 &   0 \\ \hline
%Total                  & 82 & 130 & 117 & 122 & 121 & 121 & 121 & 814 & 329 & 485 \\ \hline \hline
%{\bf JLab Labor}       & & & & & & & & & & \\ \cline{1-1}
%Scientist              & 0 & 2 &  2 &  2 &  2 &  6 & 10 &  24 &  4 & 20 \\
%Mechanical Technician  & 0 & 0 &  0 &  0 & 10 & 10 & 10 &  30 &  0 & 30 \\
%Mechanical Engineer    & 0 & 0 &  4 &  4 &  1 &  1 &  1 &  12 &  8 &  4 \\
%Mechanical Designer    & 0 & 0 & 10 & 10 &  2 &  0 &  0 &  28 & 18 & 10 \\
%Electrical Engineer    & 0 & 0 & 23 & 23 &  0 &  0 &  0 &  45 & 45 &  0 \\
%Electrical Designer    & 0 & 0 & 10 & 10 &  0 &  0 &  0 &  20 & 20 &  0 \\ \hline
%JLab Total             & 0 & 2 & 49 & 49 & 15 & 17 & 21 & 159 & 95 & 64 \\ \hline
%\end{tabular}
%\end{small}
%\end{center}
%\caption{\small{Manpower contributions to the {\tt CLAS12} FTOF design and
%construction tasks as a function of time. All manpower estimates in man-weeks.
%Year 1 marks the start of the construction phase of the project.}}
%\label{manpower}
%\end{table}
%%%%%%%%%%%%%%%%%%%%%%%%%%%%%%%%%%%%%%%%%%%%%%%%%%%%%%%%%%%%%%%%%%%%%%%%%%%%%
%
%
%%%%%%%%%%%%%%%%%%%%%%%%%%%%%%%%%%%%%%%%%%%%%%%%%%%%%%%%%%%%%%%%%%%%%%%%%%%%%
%\begin{table}[htbp]
%\begin{center}
%\begin{tabular}{|l|c|c|c|c|} \hline
%                & \multicolumn{3} {|c|} {JLab} & USC \\ \hline
%{\bf PED Tasks} & Electrical & Scientist & Mechanical & University \\ \hline
%Attach to Carriage                &   & x & x & x \\
%Backing Structure for Scintillator&   & x & x & x \\
%Cable Routing                     &   & x &   &   \\
%Pretrigger Design                 & x & x &   & x \\
%Cross Talk Tests                  &   &   &   & x \\
%Voltage Divider Optimization      & x & x &   & x \\
%Light Guide Optimization          &   &   &   & x \\
%Prototype Detector Optimization   &   &   &   & x \\
%Magnetic Shielding Design         &   & x & x & x \\
%PMT Housing and Grounding         &   & x & x & x \\ \hline
%\end{tabular}
%\end{center}
%\caption{\small{Responsibilities for design and construction tasks at JLab and USC.}}
%\label{tasks}
%\end{table}
%%%%%%%%%%%%%%%%%%%%%%%%%%%%%%%%%%%%%%%%%%%%%%%%%%%%%%%%%%%%%%%%%%%%%%%%%%%%%

%%%%%%%%%%%%%%%%%%%%%%%%%%%%%%%%%%%%%%%%%%%%%%%%%%%%%%%%%%%%%%%%%%%%%%%%%%%%%
%\begin{figure}[htbp]
%\vspace{7.0cm}
%\special{psfile=usc_test.ps hscale=45 vscale=50 hoffset=5 voffset=10}
%\caption{\small{Photograph of the test setup for the {\tt CLAS12} FTOF
%system at the University of South Carolina in March 2007.}}
%\label{usc_tests}
%\end{figure}
%%%%%%%%%%%%%%%%%%%%%%%%%%%%%%%%%%%%%%%%%%%%%%%%%%%%%%%%%%%%%%%%%%%%%%%%%%%%%

%\begin{tabbing}
%~~~~~~~~~~{\bf Key Milestones for Construction} ~~~~~\= Start   ~~~~~~~  \= Finish\\
%~~~~~~~~~~Ordering all detector components      \> 9/2008	\> 1/2009\\
%~~~~~~~~~~Testing all detector components       \> 9/2008	\> 9/2010\\
%~~~~~~~~~~Staged assembly of TOF counters       \> 9/2008	\> 1/2011\\
%~~~~~~~~~~Testing Electronics                   \> 5/2009	\> 9/2010\\
%~~~~~~~~~~TOF counter frame                     \> 1/2010	\> 1/2011\\
%~~~~~~~~~~Mounting and cabling		        \> 1/2011	\> 7/2011\\
%~~~~~~~~~~Readout and final testing	        \> 2/2011	\> 9/2011
%\end{tabbing}

\section{FTOF System Quality Assurance Procedures}

This section provides a list of quality assurance (QA) steps for
the construction of the FTOF system for {\tt CLAS12}.  The information
given here is based on experience from construction of the TOF
system for the current {\tt CLAS} detector system.

\vskip 0.5cm

{\it Scintillator Material:}  Upon arrival of the scintillator material
from the manufacturer, all scintillator should be unwrapped and
inspected with UV light for scratches, marks, and defects.  All pieces
should be measured for tolerances.  Any plastic outside of acceptable
size, clarity, or condition should be shipped back to the manufacturer.

\vskip 0.5cm

{\it Scintillator Handling:} In all steps of handling the scintillator
material, white gloves and lab coats should be worn.  Whenever possible,
observe clean room-like procedures.  Naked plastic should be covered if 
it needs to be left exposed for any period of time.  The scintillator 
material is very susceptible to damage from scratches, heat, cold, and 
shock.  Care must be taken at each step of the construction process to 
monitor the environmental conditions and to be sure that the detectors 
and materials are handled with care.

\vskip 0.5cm

{\it Workmanship:} Accuracy in workmanship is very important as
mishandling and poor construction procedures can reduce the detector
resolution noticeably.  All measurements have to be precise and all
construction procedures and tolerances have to be rigidly followed.

\vskip 0.5cm

{\it Construction Steps:}  The steps involved in the construction of
the FTOF scintillators are listed below along with general QA
guidelines.

\begin{itemize}

\item Preparation of wrapping layers: The dimensions of the wrapping
materials must be precise to avoid any gaps that could cause local
efficiency problems or introduce light leaks.  Proper cutting implements
must be employed with sharp blades always in use.  All wrapping materials
must be handled using gloves to avoid surface deposits.  A clean work
area is required to avoid any debris adhering to the surface.  Finally
all wrapping steps must ensure smooth material finishes to ensure optimal 
resolution.  

\item Fiber preparation and gluing: Fibers must be cut so that the ends
are perfectly flat.  Any ridges in these fibers will give rise to a
surface for reflections that could impact the calibrations.  Also the
cut ends must be properly polished and cleaned.  Care must be taken 
when gluing the fibers into position to ensure that no glue gets on the
ends of the fibers.  Care in handling and routing of the fibers is
required to ensure that they are not bent in such a manner that they
could crack or break.

\item Wrapping: Strict procedures must be followed when wrapping the
aluminum and Tedlar about the scintillator material.  The wrapping
materials and the tape employed tend to curl or wrinkle when under
stress.  When smoothing the materials during the wrapping process care
must be taken not to rip the materials or introduce scratches in the
material layers or the scintillator itself.

\item Support structure preparation and mounting: Careful visual inspection 
must be employed of any tape that is being employed to attach the counters
to the backing structure to be sure it is free from defects.  Also
procedures must be followed carefully to be sure that any tape employed
has good adhesion to all contact surfaces.  Sufficient measurements
must be made when positioning the scintillators on the support structure
to ensure that all alignment tolerances are met.

\item Mounting the PMTs: Care must be taken to ensure that instructions 
for mixing epoxy used to attach the PMTs are precisely followed.  Procedures 
to eliminate or reduce air bubbles in the epoxy must also be followed.

\end{itemize}

\vskip 0.5cm

{\it PMT Handling:} The PMTs are susceptible to damage from shock and
care must be taken to handle them with care both before and after they
are mounted.  In addition, care must be taken to avoid any scratches on
the PMT face during handling and installation.

\vskip 0.5cm

{\it Detector Testing:} HV checks of the detectors should be performed
and the dark current should be measured for each PMT.  Any measured
dark current more than 100~nA is an indication of a possible light
leak or bad PMT.  Attempts to find light leaks should continue to ensure 
that the measured currents are the same in the testing area with lights on 
and lights off.  Attenuation length measurements should be performed on
selected detectors to ensure that they are within the manufacturers
stated specifications.

\vskip 0.5cm

{\it Installation:} Quality assurance during installation must involve
procedures for proper and safe handling of the detector structures to
be sure that they are not subject to any shocks or stresses, as well as
to ensure that support frames and lifting structures do not put any
stresses on the detector, the light guides, or the PMTs.  Care must be
taken that the installation procedure does not affect the integrity of
the surface wrapping.  Finally, procedures must be followed regarding
voltage and signal connections to the detectors to be sure that all
cables are labeled and sufficient strain-relief or slack is allowed for
so no stresses are put on the detectors or readout electronics. 

\section{FTOF System Safety Issues}

During construction there are three important areas of safety concern.  
These include:

\begin{itemize}

\item The construction procedure involves the use of glue, epoxy, and
paints that can cause irritation if they come in contact with skin
or eyes.  Safety precautions dictate the requirements to use gloves when
mixing or handling glues, epoxies, or paints, as well as appropriate eye 
protection.

\item A sizeable number of construction tasks involve use of scissors
and Exacto blades.  Extreme care must be taken when undertaking activities
using these instruments to avoid personal injury and cuts.

\item During construction the scintillator bars and scintillator
assemblies will have to be handled and manipulated.  Care must be taken
when lifting or supporting loads to avoid personal injury.

\end{itemize}

There are four main areas of safety concern for installation and operation
of the FTOF system.  These include:

\begin{itemize}

\item Grounding scheme - necessary to prevent electrical shock.

\item Elevated work areas - access to the FTOF system for installation
and repairs will be via man-lifts.  Operators will employ appropriate
harnessing and fall protection.

\item Staging and installation - special procedures will have to be 
detailed for the installation of these large and heavy detector panels.

\item Cable installation - floor grating will be removed and signage
and personnel barriers will be installed to prevent trip and fall
hazards.

\end{itemize}

It is expected that the safety issues involved with this work involve
low risk for personnel injury or equipment damage, especially with the
use of appropriately planned and supervised work activities.

\section{Collaboration}

The design, prototyping, and construction of the FTOF system is the
primary responsibility of the University of South Carolina and
Jefferson Laboratory.  USC is currently in the stage of R\&D to optimize 
the design of the panel-1b counters and to demonstrate that the combined 
panel-1a/panel-1b system will meet the design specification of 80~ps 
timing resolution. 
