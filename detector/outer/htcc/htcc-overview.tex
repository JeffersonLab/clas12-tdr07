




The main function of HTCC is to aid in the identification of electrons and 
pions. A light gas such as $CO_2$ will be used as radiator. At atmospheric 
pressure this will provide a threshold for the detection of charged pions of 
4.9 GeV/c. In combination with the CLAS FEC, this will allow highly efficient 
electron identification. A pion rejection factor of more tha 2000 can be 
achieved 
for the momentum range up to 4.9 GeV, and of  more than 100  for momenta 
greater than
5 GeV/c. The HTCC is located in front of the Torus magnet and the first forward
tracking chamber. 
Figure ~\ref{fig:htcc_optics}
illustrates the optics of the HTCC. The detector is arranged in six sectors 
to match the geometry of the other components of the forward detector. 
In each sector the optics is symmetric about the mid-plane. The polar angle 
coverage from 5 degrees to 40 degrees is achieved by segmented mirrors that 
focus the Cerenkov light towards two sets of eight 5-inch photomultiplier tubes. 
Most Cerenkov photons will directly hit the photocathode area in the 5" 
photomultiplier tubes, those outside are collected in Winston cone mirrors 
around the PMTs. The PMTs are located in the fringe field of the Solenoid 
magnet and will be magnetically isolated with a multi-layer magnetic shield. 
Such magnetic shields have been used successfully in the CLAS Cerenkov detector
. The expected response in terms of the number of collected photoelectrons
 has been simulated using the measured properties of the mirror system in the 
CLAS Cerenkov counter, and photomultipliers with known photocathode 
sensitivities and quartz windows. In the polar angle range from 5 to 35 degrees the 
$N_{pe}$  for $CO_2$  as radiator gas is between 8 and 12, slightly dependent 
on the polar angle. $N_{pe}$ is independent of the azimuthal angle $\phi$ . 


\begin{figure}[tb]
\begin{center}
\epsfig{file=HTCC.epsi,angle=270, width=\linewidth}
%\vspace{12cm}
%\special{psfile= HTCC.ps hscale=160 vscale=160 
%hoffset=70 voffset= 20}
\caption{\it{\small Optics of the High Threshold Cerenkov Detector.
Cerenkov photons are generated in the gas volume beginning after 
the microstrip detector and ending at the mirror system. 
The mirror system consists of
two sets of 8 mirror segments in each sector arranged symmetrically 
about the mid plane which reflect 
the light toward the outside 
region.}}
\label{fig:htcc_optics}
\end{center}
\end{figure}



















