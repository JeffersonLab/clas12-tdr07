\section{HTCC System Quality Assurance Procedures}

This section provides a list of quality assurance (QA) steps for the 
construction of the HTCC counter for {\tt CLAS12}.  The information
given here is based on experience from construction of the {\v C}erenkov 
counter for the existing {\tt CLAS} detector and on the HTCC R\&D results.

\subsection{QA Procedures for the Combined Mirror Construction}

Quality assurance procedures for the HTCC combined mirror construction
include several items:

\begin{itemize}

\item {\it Mirror Substrates}: All constructed mirror substrates (96 
segments) will be checked to ensure all dimensions and tolerances meet
the design requirements.  The substrates will then be sent to the vendor
for the reflector and protection coating deposition;

\item {\it Reflectivity}: All coated mirror segments will be tested in 
several randomly chosen points to provide input control to ensure the
reflectivity is as specified in the design requirements;

\item {\it Assembly}: The assembly of the sections of the mirror sector 
will be done separately using positioning tables and a class-2 laser for 
orientation checks;

\item {\it Preparation for Assembly}: The overall dimensions of the portions 
will be checked one more time before final assembly;

\item {\it Final Assembly}: Assembly will be performed using a foam
assembly table that will consist of 12 separate identical portions of 
smaller size, aligned and attached to the flat metal base plate.  A metal 
ring with inner diameter equal to the outer diameter of the combined mirror 
will be used as a part of the final assembled product.  The alignment 
(position and orientation) for each of the left and right mirror portions 
will be controlled with a laser.

\end{itemize}

\subsection{PMT Quality Control}

\begin{itemize}

\item {\it PMT Handling}: The PMTs are susceptible to damage from shock,
and care must be taken when handling them both before and after they
are mounted.  In addition, care must be taken to avoid any scratches on
the PMT face during handling and installation;

\item {\it PMT HV Testing}: HV checks of the detectors will be performed
and the dark current will be measured for each PMT.  Any measured
dark current more than 100~nA is an indication of a bad PMT;

\item {\it PMT Testing}:  The PMT output signals will be checked using 
light emitting diodes (LEDs).  During these measurements, the HV will 
be adjusted to make sure that the one-photoelectron signals have the same 
pulse height.  The output signal will be checked by studying the ADC 
spectrum.  All ADC spectra will be written to a database for future 
reference. 

\end{itemize}

\subsection{QA for Magnetic Shielding}

The PMT magnetic shielding consists of several layers of magnetic material.
It is very important to check that the magnetic properties of the PMT 
shielding correspond to specifications.  The assembled magnetic shielding 
should be tested at the magnetic test stand.  The PMT signal from an LED 
should be within specifications at a magnetic field of 50~G.

\section{HTCC System Safety Issues}

There is potential shock hazard if the high voltage is contacted or if 
grounding fails.  The safety concerns include:

\begin{itemize}

\item Grounding scheme - necessary to prevent electrical shock;

\item Every PMT should have a cover that electrically protects the PMT  
from the support structure;

\item The HV to the phototubes should be turned off if maintenance of 
the tubes or based required;

\item An interlock on the HV should be implemented to protect personnel and 
equipment;

\item Access to the test area and dark box should be key protected;

\item Eyes should be protected when handling PMTs;

\item Under normal operation the HV should be controlled by a slow-controls 
system.  The control should indicate an {\it off for maintenance} option;

\item The HV power supply should have low current trip limits on each PMT 
line, thus sustained currents are not possible;

\item Only HV-rated connectors and coaxial cables should be used;
 
\item Elevated work areas - access to the HTCC system for installation
and repairs will be via man-lifts.  Operators will employ appropriate
harnessing and fall protection;

\item Staging and installation - special procedures will have to be 
detailed for the installation of the mirrors and PMTs.

\end{itemize}

It is expected that the safety issues involved with this work involve
low risk for personnel injury or equipment damage, especially with the
use of appropriately planned and supervised work activities.
