\section{Collaboration}

Several institutions are involved in the design, prototyping, and
construction of the PCAL.  The main contributions by these institutions
will be in the manpower needed for the pre-design simulations, as
well as for the design, testing, and prototyping.  In the construction 
stage, individual institutions can take over separate tasks or participate 
in different activities.  The main groups are: 

\begin{itemize}

\item Yerevan Physics Institute (YerPhI) -- will provide scientists
  experienced in the detector design, construction, and simulation, as 
  well as students to help during the assembly and testing of the PCAL
  modules.  It is expected also that some parts needed for the
  prototyping and construction can be made at YerPhI facilities
  (e.g. adapters for the fiber-PMT connection). The YerPhI collaboration
  already has made significant contributions in the simulations, design,
  and prototyping of the PCAL.

\item James Madison University (JMU) -- will provide scientists and
  students to participate in the design, prototyping, and construction
  of the PCAL. The JMU group already has made significant contributions 
  to the testing of the PCAL components.

\item Institut de Physique Nucleaire d'Orsay (INP) -- will provide
  engineers to participate in the design of the PCAL.  The INP group 
  has already made significant contributions to the design of the PCAL.

\item Ohio University (OU) -- will provide scientists and students to 
  participate in the design, prototyping, and construction of the PCAL. 
  The OU group has already made contributions to the design of the PCAL 
  components.

\item Norfolk State University (NSU) -- will provide scientists and
  students to participate in the prototyping and construction
  of the PCAL. 

\item Jefferson Lab (JLAB) -- will manage the project, and will lead the
  design, prototyping, and construction of the PCAL.  JLab will provide
  infrastructure for the project, work space for different activities,
  and engineering and technical help.

\end{itemize}