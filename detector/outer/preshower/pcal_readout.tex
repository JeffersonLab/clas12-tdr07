\section{Signal Readout and Triggering}

The electrical connections of the PCAL PMTs will be as follows: the
voltage dividers for the PCAL PMTs will have high voltage (HV) as input and 
the anode signal as an output.  Each PMT will be furnished with a separately 
regulated HV power source.  The charge and the time of the anode signals for 
each PMT will be measured using a flash ADC (FADC) and a multi-hit TDC (MTDC),
respectively.  The PMT anode signal will be sent to a splitter through about 
15~m of RG-58 cable, with approximately a $1:3$ split.  The larger portion 
of the split, $\sim$75\%, will be sent to a discriminator input.  Outputs from 
the discriminator will feed the MTDC and a scaler. The second output of the 
splitter, $\sim$25\% of the signal, will go to the FADC.  The PCAL will be 
included in the trigger system of {\tt CLAS12}.  The trigger signals from 
the PCAL modules will be formed using the fast readout of the FADCs and the 
FPGA programming of the mainframe controller.  This type of trigger 
organization will allow for more sophisticated and robust trigger 
configurations compared to the total energy sum trigger configuration that
is available in the current {\tt CLAS} detector.  

The energy uniformity of the calorimeter response is one of the important 
aspects in the configuration of the trigger system.  The response of the 
PCAL to electromagnetic energy deposited in the scintillator material across 
the calorimeter front face depends upon several factors.  Assuming that the 
same amount of energy deposited in the scintillator generates the same amount 
of light in the scintillating fibers independent of the position across the
calorimeter, the light attenuation along in the fiber remains the major
factor that determines the PMT response for a given scintillator stack
in the $U$, $V$, and $W$ views.  A typical light attenuation length for 
green fibers is $L_0$ = 300 - 400~cm.  Depending upon the $X-Y$ position at 
the calorimeter face, the light may travel up to nearly 500~cm before it
reaches the photomultiplier.  Therefore, the signal may be attenuated by a 
factor of $\sim$4 when reaching the PMT photo-cathode.  Therefore the
response of the individual $U$, $V$, or $W$ stacks can vary by large factors,
making the response highly non-uniform.  However, the triangular structure of 
the PCAL with nearly equal side lengths, and with the scintillators oriented 
approximately 120$^\circ$ relative to each other, reduces this non-uniformity 
drastically when the signals from all strips are added together.  In fact, in 
the case of linear attenuation, the sum of the signals from the $U$, $V$, and 
$W$ strips is independent of the $X-Y$ position at the calorimeter face. 
However, in a more realistic case, the signal drops like an exponential 
function of the distance, i.e. $I(L) = I_0 \times \exp(-L/L_0)$, where $L$ is 
the distance along the fiber from where the signal was generated.  Adding all 
signals up will generate significantly higher sums ($\sim$40\%) at the
edges and corners of the calorimeter than in the center.  While this effect 
can be corrected in the offline analysis, it could, however, significantly 
bias the calorimeter response towards lower energy deposition near the edges 
when used in a total energy trigger.  To avoid this non-uniformity, the PCAL 
response can be made more uniform by giving lower weight factors to the 
shorter strips than to the longer strips in the trigger logic using the FADCs 
and the FPGA programming.  The non-uniformity can be reduced to a few percent 
across the entire PCAL face.  
