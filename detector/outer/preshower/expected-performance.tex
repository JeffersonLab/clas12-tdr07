\section{ Expected Performance }

Many aspects of the performance of the PCAL system has already 
been described in the previous sections.  In this section, we 
simply summarize the information already given above. 

From Fig. \ref{fig:Eres}, the energy resolutions for the 
combined EC plus PCAL systems shows the expected linear 
form for the energy resolution, $\sigma$, as:
\begin{equation}
  \frac{\sigma}{E} \simeq \frac{1}{\sqrt{E}}
\end{equation}
where the constant of proportionality is determined by the 
geometry and light collection efficiency of the calorimeter. 
This resolution was obtained by fitting a gaussian to the 
sampling fraction distribution over many events, for each 
energy bin (i.e., point on the graph) shown, from simulations 
using GEANT.  Based on past performance of the EC, the 
performance of the PCAL+EC system contains the shower, 
giving satisfactory resolution over the momentum (energy) 
range expected in CLAS12 experiments.

The efficiency for resolving two-photon clusters for the 
PCAL+EC system is found to be greater than about 85\% over 
the full range of momenta for $\pi^0$ decay photons 
expected at CLAS12 as shown in Fig. \ref{fig:???}.  
This performance could be improved 
slightly if more readout channels were available, but 
this would increase the cost.  The current design, as 
described above, is for 15 layers in the PCAL with 
45 mm wide strips covering half of linear height of each 
of the three stereo views, and double-wide strips over the 
remaining area, as shown in Fig. \ref{fig:finalgeom}. 

A prototype of the PCAL is being built.  Prelimimnary 
studies suggest that 11 photoelectrons per MeV can 
be obtained using 45 mm strips with 3 WS fibers in 
the FNAL extruded scintillator grooves, going into 
a Hamamatsu 6096 PMT.  Further studies of the prototype 
will refine this estimate of photoelectrons.

