\section{Construction}

Construction of the PCAL consists of several quasi-independent
processes.  Each process requires independent manpower and work
space.  Most of these processes require a clean environment.  The
handling of many items, such as scintillator strips, fibers, 
$\mu$-metal shields, lead, PMTs, etc., requires the use of gloves. 
Below are the steps/processes involved and the resources needed in 
the construction of the PCAL.

\begin{enumerate}

\item Preparation of scintillator strips with WLS fibers requires a
semi-clean area with a 4.5-m long table and open shelves for about 
3000~m of scintillator strips.  A table should be instrumented with 
fixtures for gluing.  This process requires tools for measuring 
dimensions and tools for cutting and gluing the optical fibers. 
Preparation of the scintillator-fiber assembly will be done for one 
sector at a time.  The work will include:  

  \begin{itemize}
    \item visual inspection of the scintillator strips;
    \item cut the strips to the correct length and measure dimensions;
    \item cut and inspect the fibers;
    \item glue the fibers into the grooves;
    \item inspect the gluing quality.
  \end{itemize}

\item Assembly and tests of the PMTs and dividers requires a room with 
a dark box and storage shelves. The process requires tools for mechanical 
assembly, an oscilloscope, a HV system for the PMTs, cables, electronics,
and a DAQ system.  Work will include:  

  \begin{itemize}
    \item assembly of dividers to the end-caps of the PMT housing;   
    \item assembly of the PMT housing, cleaning the plastic tubes
          and end-caps, and installation of the $\mu$-metal shield;   
    \item full assembly of the PMT, divider, and housing system;
    \item measurement of the relative sensitivity to green light (500~nm) 
          and the gain of each PMT using a photo-diode, and compare with 
          specifications.
  \end{itemize}

\item Stacking of the scintillator and lead layers requires a semi-clean
room $8\times 8$~m$^2$ in area.  The room should have access to an overhead 
crane or other lifting device to move the lead sheets.  The process will 
require wide paper or Teflon sheets to place between the lead and scintillator
layers.  The stacking will be done one sector at a time.  The work will 
include:

  \begin{itemize}
    \item cleaning and assembly of the backplate and the side walls of 
          the box;
    \item stacking scintillator and lead layers;
    \item after scintillator layer is in place, add fixtures and shims to
          secure the position of the scintillator layer from all sides; 
          cover the layer with paper or Teflon sheets and add the lead sheet;
    \item after all layers are stacked, put enough paper or Teflon sheets
          on top of the last layer to have a tight, uniform compression of 
          the scintillator-lead layers between the two endplates;
    \item put on the front-plate and bolt it to the side walls;
    \item inspect that all fibers are in place and that there are no broken 
          fibers;
    \item move the PCAL module where a high-load crane is available and
          rotate it to have the front-plate down;
    \item move the module to the area where the PMTs will be assembled.
  \end{itemize}

\noindent
The plate that holds the weight of the PCAL should have a strong-back
support to prevent bowing.  There should be at least two strong-backs,
one for stacking and a second for the PMT assembly.

\item Mounting of fiber-adapters and PMTs requires a semi-clean room of
$6 \times 6$~m$^2$ area, with gluing fixtures, polishing instruments, 
black masking tape, tools for mechanical assembly, an oscilloscope, a HV 
power supply, and HV and signal cables.  The work will include:

  \begin{itemize}
    \item mount shelves and fiber-PMT adapters;
    \item position fibers inside the adapters according to the readout 
          channel distribution and glue them into place;   
    \item cut and polish the ends of the fibers;
    \item clean shelves and adapters;
    \item mount aluminum sheet covers;
    \item mount PMTs;
    \item check for light leaks; if there are light leaks use black masking 
          tape to cover;
    \item move the PCAL module to the test area.
  \end{itemize}

\item Test of the PCAL module with cosmic particles requires a room of about 
$6 \times 6$~m$^2$ area, that is air conditioned, and includes cables and
electronics for one module and a working DAQ system.  The work will include 
cabling, checking each channel, data taking, and analysis.

\end{enumerate}
