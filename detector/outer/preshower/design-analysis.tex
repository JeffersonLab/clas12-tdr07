\section{Conceptual Design}

The design parameters of the PCAL were established using the full GEANT 
simulations of the PCAL-EC system.  As a tool, a modified GEANT simulation 
computer program for the {\tt CLAS} detector was used.  The PCAL was 
positioned in front of the current EC as shown in Fig.~\ref{fig:clas12fwd}.  
These studies are described in detail in Refs.~\cite{natalia,whitlow} and are 
summarized below.  The mechanical design depends on the number of 
scintillator-lead layers, on the angular coverage of the PCAL, and on the 
size of the readout segmentation.  These parameters were determined by the 
physics requirements for the detection and identification of high-energy 
electrons, photons, and $\pi^0$s via $2\gamma$ decay.  

%%%%%%%%%%%%%%%%%%%%%%%%%%%%%%%%%%%%%%%%%%%%%%%%%%%%%%%%%%%%%%%%%%%%%%%%%
\begin{figure}[ht]
\vspace{100mm}
\special{psfile=../preshower/clas12fwd.epsi hscale=55 vscale=50 hoffset=70 
voffset=-60}
\caption{\small{Particle ID detector package of the {\tt CLAS12} forward
region.  The low threshold {\v C}erenkov counter is shown in magenta, two
layers of FTOF counters are shown in green and yellow, the PCAL is shown 
in red, and the EC in blue.}}  
\label{fig:clas12fwd} 
\end{figure}
%%%%%%%%%%%%%%%%%%%%%%%%%%%%%%%%%%%%%%%%%%%%%%%%%%%%%%%%%%%%%%%%%%%%%%%%%

Initial simulations were carried out with 15 layers of lead and 
scintillator (similar to the inner part of the EC), using 35-mm wide 
segmentation for the scintillator layers, corresponding to about 108 
readout channels in each stereo view.  Events were generated in a uniform 
distribution of $\pi^0$ and photon events at the target with momenta up 
to 12~GeV.  Reconstruction of clusters was done using the standard cluster 
reconstruction algorithm of the EC, but applied to both the PCAL and EC.  
As shown in Fig.~\ref{fig:ecpcal}, the combined PCAL and EC system retains 
good energy resolution, $\sigma_E \sim 0.1\sqrt E$, and a constant 
efficiency for two-cluster reconstruction up to the highest momenta. 

%%%%%%%%%%%%%%%%%%%%%%%%%%%%%%%%%%%%%%%%%%%%%%%%%%%%%%%%%%%%%%%%%%%%%%%%%
\begin{figure}[ht]
\vspace{70mm}
\special{psfile=../preshower/ecpcal_res.eps hscale=45 vscale=45 hoffset=20 
voffset=-15}
\special{psfile=../preshower/eff_12gev.epsi hscale=45 vscale=45 hoffset=220 
voffset=-90}
\mbox{
\begin{picture}(-10,-10)(350,0)
\put(520,170){\large {\bf (a)}}
\put(765,170){\large {\bf (b)}}
\end{picture}}
\caption{\small{Performance of the combined PCAL-EC detector at high 
energies, shown with blue points. a) The energy resolution of the 
{\tt CLAS12} electromagnetic calorimetry system as a function of the 
inverse square root of energy. b) The efficiency of reconstruction of two 
clusters from $\pi^0$ decay photons as a function of pion momentum. Red 
symbols correspond to the EC performance presented in 
Fig.~\ref{fig:ecsim}.}}  
\label{fig:ecpcal} 
\end{figure}
%%%%%%%%%%%%%%%%%%%%%%%%%%%%%%%%%%%%%%%%%%%%%%%%%%%%%%%%%%%%%%%%%%%%%%%%%

Keeping the number of layers at 15, the width of the readout segments 
(strips) was varied from 35~mm (108 channels) up to 60~mm (65 channels). In 
Fig.~\ref{fig:lw}a, the two-cluster reconstruction efficiency is presented 
for different strip widths.  The magenta points correspond to a 60-mm strip 
width, the blue points are for 50-mm segmentation, the green points for 43-mm
strip width, and the red points are for 35-mm width.  The efficiency 
decreases with energy for wider strips, but remains reasonably high if the 
strip size is kept at less than 50~mm. 

Other simulations were performed using 9 or 12 layers in the PCAL design, 
clearly showing a loss of two-cluster reconstruction efficiency.  In
Fig.~\ref{fig:lw}b, the two-cluster reconstruction efficiency is shown as a 
function of $\pi^0$ momentum for 9 (red), 12 (green), and 15 (blue) layers. 
Clearly, the 15 layer configuration has the highest efficiency.  The reduced 
efficiency in the other cases is mainly due to insufficient radiation 
thickness for high-energy photons to convert and deposit sufficient energy 
for the shower to be detected.  Simulations were repeated for the 12-layer 
case, by doubling the thickness of the first 3 layers of lead.  This 
configuration had a comparable two-cluster reconstruction efficiency as the 
original 15-layer design, but poorer energy resolution. 

%%%%%%%%%%%%%%%%%%%%%%%%%%%%%%%%%%%%%%%%%%%%%%%%%%%%%%%%%%%%%%%%%%%%%%%%%
\begin{figure}[ht]
\vspace{70mm}
\special{psfile=../preshower/str_width-1.epsi hscale=45 vscale=45 hoffset=30 
voffset=-15}
\special{psfile=../preshower/mod-3-4-5.epsi hscale=45 vscale=45 hoffset=230 
voffset=-88}
\mbox{
\begin{picture}(-10,-10)(350,0)
\put(520,30){\large {\bf (a)}}
\put(745,30){\large {\bf (b)}}
\end{picture}}
\caption{\small{Performance of different configurations of the PCAL.  a) 
The reconstruction efficiency for two clusters from $\pi^0$ decay photons 
as a function of pion momentum for different readout segmentation.  The 
magenta points correspond to 60-mm strip width, the blue points are for 
50-mm segmentation, the green points for 43-mm strip width, and the red 
points are for 35-mm width.  b) The reconstruction efficiency for two 
clusters for different numbers of scintillator-lead layers. The red points 
correspond to a 9-layer configuration, the green points are for 12 layers, 
and the blue points are for 15 layers.}}  
\label{fig:lw} 
\end{figure}
%%%%%%%%%%%%%%%%%%%%%%%%%%%%%%%%%%%%%%%%%%%%%%%%%%%%%%%%%%%%%%%%%%%%%%%%%

Additional simulations were performed using variable segmentation of the 
scintillator layers.  Keeping constant the total number of readout channels 
per sector, it was found that the maximum efficiency can be obtained if 
half of each stereo layer is equipped with 45-mm wide strips and half with 
90-mm wide strips (double-strip readout). The triangular stereo layers
overlap such that there is always a region with 45-mm wide strips in one of 
the stereo layers, as shown in Fig.~\ref{fig:vwidt}.  There is only a small 
loss of two-cluster efficiency at the highest momenta for this geometry 
compared with 45-mm wide strips in all stereo layers, see 
Fig.~\ref{fig:vwidteff}.  It should be noted that at forward angles (short 
$U$-strips), where most of the high-energy $\pi^0$s are produced, all three 
stereo readout views have small readout segmentation. 

%%%%%%%%%%%%%%%%%%%%%%%%%%%%%%%%%%%%%%%%%%%%%%%%%%%%%%%%%%%%%%%%%%%%%%%%%
\begin{figure}[ht]
\vspace{90mm}
\special{psfile=../preshower/vwidth.eps hscale=60 vscale=60 hoffset=40 
voffset=10}
\caption{\small{Variable segmentation for different stereo readout planes 
($U$, $V$, and $W$).  There is always a region with single strip (45-mm 
segmentation) readout in one of the stereo layers.}}  
\label{fig:vwidt}
\end{figure}
%%%%%%%%%%%%%%%%%%%%%%%%%%%%%%%%%%%%%%%%%%%%%%%%%%%%%%%%%%%%%%%%%%%%%%%%%

%%%%%%%%%%%%%%%%%%%%%%%%%%%%%%%%%%%%%%%%%%%%%%%%%%%%%%%%%%%%%%%%%%%%%%%%%
\begin{figure}[ht]
\vspace{9.3cm}
\special{psfile=../preshower/eff_64pmt.eps hscale=60 vscale=60 hoffset=80 
voffset=-10}
\caption{\small{Various combinations of 45-mm and 90-mm readout 
segmentation.  For the black histogram only 45-mm readout is used, amounting
to a total of 258 channels per sector.  The other histograms correspond to 
cases when part of the layer was read out in 90-mm segments, keeping the 
total number of channels to 192 per sector.}}  
\label{fig:vwidteff} 
\end{figure}
%%%%%%%%%%%%%%%%%%%%%%%%%%%%%%%%%%%%%%%%%%%%%%%%%%%%%%%%%%%%%%%%%%%%%%%%%

The proposed design of the PCAL covers the full angular range of the EC. 
The PCAL has 15 scintillator and 14 lead layers, confined between two 
endplates.  A variable width of the readout segmentation will be used to
minimize the number of readout channels, while retaining sufficient
resolution for the separation of close clusters. It is proposed to
use 45-mm segmentation for the short strips in $U$, and for the long
strips in the $V$ and $W$ stereo readout planes, at least half of the 
height of the layer.  For the remaining part, 90-mm wide segmentation can 
be used (two strips).  The total number of readout channels will be limited 
to 192 per sector. 
