\section{PCAL Quality Assurance}

In each process of the PCAL construction, appropriate steps will be taken
to ensure the quality of the final product.  Quality controls will
include -- accuracy of mechanical assembly of the box, quality of
scintillator strips, quality of fiber gluing, and checks of the PMTs and 
dividers.  Quality assurance of the mechanical assembly will include 
verification of the dimensions of the parts, and the assembly will be
checked to be within the specified tolerances.  During the assembly, 
after installation of each layer, mechanical dimensions will be measured 
to ensure correctness of the assembly procedure.  As for quality checks 
for the gluing process and for the scintillators and fibers, we will 
follow practices used during the assembly of the existing {\tt CLAS}
calorimeters.  The scintillator strips will undergo visual and dimensional 
checks.  The glued fibers will be checked using a UV light to identify 
any cracks on the cladding or bad gluing. The frequency of these checks 
will be defined after we have more experience with the scintillators and
fibers, and the gluing process is determined.  Each PMT will be tested to 
verify the required QE for green light and the required gain.  After a 
module is assembled, the light yield and time response of each channel will 
be tested using cosmic particles.

\section{PCAL Safety Issues}

There are no major hazards in use of materials, high-power electrical 
supplies, mechanical tools, or radioactive materials.  Construction of 
the PCAL includes work with lead sheets, adhesives, HV power supplies 
for the PMTs, and some heavy lifting. In current operation of the 
{\tt CLAS} detector, work with lead, heavy lifting, and work with adhesives 
are done routinely. These types of work practices are covered by the
JLab EH\&S manual and routine monitoring practices.

Lead work will be handled by personnel who have "Lead Worker Training"
(SAF 136). Lead work practice includes use of appropriate PPEs, wet washing 
the lead or using HEPA vacuums to clean oxides from the surfaces, 
appropriate cleansing of personnel in contact with the lead, appropriate
TOSPs or OSPs and task hazard analysis, and using HEPA vacuums to clean 
the work areas after lead handling.

Adhesives will be used to glue the fibers.  Gluing will be done in a clean,
well-ventilated room.  All personnel working on the gluing process will be 
trained in the use of adhesives.  Different types of adhesives are routinely 
used in Hall B operation, and all technical staff and users are familiar
with the associated hazards and follow the JLab ES\&H manual with regards
to chemical hygiene and respiratory protection.  Material Safety Data Sheets 
(MSDS) for all chemicals in use will be available, and every worker will be 
familiarized with the hazards in the MSDS.

Moving the PCAL modules will be done by qualified personnel using
appropriate lifting and moving tools and procedures.  One of the
requirements for the work area is access to an overhead crane for this
purpose. 

