\section{Prototyping and Component Testing}

The design goals for the PCAL detailed in the previous sections are based 
on the proposed geometry for {\tt CLAS12}, the combined performance of the 
PCAL, the existing EC for {\tt CLAS12}, and experience gained from the 
construction followed by 10 years of successful operation of the EC~\cite{ec}. 
In addition, several factors have established the preliminary design.
These principle considerations include: 

\begin{itemize}
\item Comparable geometric coverage for the PCAL with respect to the EC;
\item Good resolution/calorimetry coverage for up to 10~GeV photons and 
electrons;
\item Improved particle identification ability to enhance final state
reconstruction at higher energies; 
\item Information on the longitudinal shower development for $e/\pi$
discrimination;
\item Fast calorimeter response for use in the Level-1 trigger;
\item Sufficient position information to resolve $\pi^0 \to \gamma \gamma$; 
\item Compatibility with the present EC and other {\tt CLAS} components;
\item Mechanical stability;
\item Mechanical support viability;
\item Constructibility (reasonable facilities, manpower and resources
to assemble the detectors); 
\item Reasonable options for testing components in order to establish
PCAL operational parameters. 
\end{itemize}

In the construction of the existing EC system, the components were 
carefully characterized (e.g. scintillator attenuation, light transmission, 
photo-electron efficiencies). Calibration procedures were developed and 
continue to be improved to update the relevant properties (e.g. PMT gains 
and pedestal).  This information has been used to construct algorithms for 
extracting particle identification, particle position, and particle energy 
\cite{ecrec}.  An EC electronic sum is available as part of the current 
trigger options for {\tt CLAS} and has proven to be essential for limiting 
the amount of recorded data, while identifying useful final states.  The 
calorimeter has operated reliably, providing critical information on reaction 
final states, with sound performance in terms of maintenance and stability. 
The mechanical structures for the calorimeter movement and support work well. 
Thus part of the PCAL design has been based on the existing EC.

To match the above goals, a lead-scintillator sampling calorimeter with a
wavelength shifting fiber readout was chosen.  Scintillator light readout 
systems with embedded fibers are well known.  This technique was used, for
example, in the MINOS FAR detector~\cite{MINOS}, and many other ``tile''
calorimeters. 

The main design features of the PCAL are:

\begin{itemize}
\item Triangular shape with sides of order 4.0~m in length;
\item Sampling structure with Pb as the radiator and scintillator as the 
      active medium;
  \begin{itemize}
  \item Lead 2.2~mm (available sheets) [simulation verifies this choice of 
        sheet depth];
  \item Scintillator strips:
      \begin{itemize}
      \item Scintillator 10-mm longitudinal depth;
      \item Strip width will be important in determining position resolution;
      \item Simulation suggests that 4.5-cm width is adequate to reach 
            required position resolution;
      \item $U$, $V$, and $W$ readout;
      \end{itemize}
   \end{itemize}
\item Light readout needs to reach several photo-electrons/MeV of deposited 
      energy;
\item Components need to be combined to reduce the required number of PMTs;
\item Wavelength-shifting fibers are the chosen method for light collection.
\end{itemize}

Component testing was primarily focused on finding the components that did 
not compromise the design goals, while minimizing costs.  Tests were 
conducted to find the optimal:
 
\begin{itemize}
\item Wavelength-shifting fibers;
\item Photomultiplier tubes;
\item Number of fibers per strip;
\item Grooved scintillators;
\item Glue that binds fibers and scintillators and improves optical properties.
\end{itemize}

Tests were also designed to characterize combinations of components in
terms of the number photo-electrons/MeV and to verify the expected 
attenuation lengths for the scintillator and the fiber.

For these test measurements, several different types of scintillator,
wavelength-shifting fibers (WLS - single and multi-clad), and PMTs were 
studied, see Table~\ref{tab:req}.  Extruded scintillators from Fermilab (FNAL) 
and Kharkov, as well as commercial scintillators from Eljen were compared on 
the basis of light yield and cost.  The FNAL and Kharkov strips are grooved 
during the extrusion by the die~\cite{fnalex,yerex}, whereas the grooves for 
the Eljen scintillators were machined by the manufacturer.  The grooves 
provide insets for the WLS fibers.  The fibers are held in place by either a 
UV-cured glue or an epoxy.  Light absorption by the fiber is influenced by 
the optical properties of the glue.  Preliminary results indicate that an 
optical epoxy is adequate, both in terms of bonding and light transmission. 
The more expensive UV-cured optical glues were used during these tests.  The 
FNAL and Kharkov scintillator strips were coated with a reflective material 
(1-mm titanium dioxide, excluding the inside surface of grooves) to improve 
light collection.  Tests on the Eljen scintillator were done with and without 
an additional aluminized-mylar surface. The Hamamatsu R1450-13 and R6095 PMTs 
were tested because of their high quantum efficiency at 500~nm.  R1450-13
and R6095 PMTs were chosen with QE $>$ 18\ and $>$16\% at 500~nm and studied 
as possible options.  Light yield measurements were done with Kuraray 1.5-mm 
and 2-mm diameter Y11 fibers to determine the dependence of the light yield 
on the fiber diameter.  Finally, a sufficient sample of scintillator-WLSF-PMT 
combinations was studied to determine the optimal combination.

%%%%%%%%%%%%%%%%%%%%%%%%%%%%%%%%%%%%%%%%%%%%%%%%%%%%%%%%%%%%%%%%%%%%%%%%%
\begin{table}[hbt!]
\begin{center}
\begin{tabular}{|c||c|c|c|} \hline 
PMT           & Type    & Photo-cathode & \# of stages \\
              &         &               &              \\ \hline
Hamamatsu     & R7899EG & 25 mm         & 10 \\
              & R1450-13& 19 mm         & 10 \\
              & R6095   & 28 mm         & 11 \\ \hline 
ElectronTubes & 9124B   & 30 mm         & 11 \\ \hline 
Photonis      & XP1912  & 19 mm         & 10 \\
              & XP2802  & 19 mm         & 10 \\ \hline \hline 
WLS fibers    & Type    & Diameter      & Cladding \\ \hline 
Kuraray       & Y11     & 2 mm          & Single   \\
              & Y11     & 1.5 mm        & Single   \\
              & Y11     & 1 mm          & Single   \\
              & Y11     & 1 mm          & Multi    \\ \hline 
Bicron        & BC-91A  & 1 mm          & Single   \\ 
              & BC-92   & 1 mm          & Single   \\ \hline \hline 
Scintillators & Type    & Cross section &\# of grooves \\ \hline 
Eljen Tech.   & EJ-204  & $3\times 1$ cm$^2$ & 4  \\
              & EJ-204  & $3\times 1$ cm$^2$ & No grooves \\ \hline 
FNAL          & MINOS   & $4\times 1$ cm$^2$ & 1  \\ \hline 
Amcrys-Plast, Kharkov   &               & $2.6\times 1$ cm$^2$ & 2 \\
                        &               & $2.6\times 1$ cm$^2$ & 3 \\ \hline  
\end{tabular}
\end{center} 
\caption{\small{Pre-shower calorimeter readout components used in the test.}} 
\label{tab:req} 
\end{table} 
%%%%%%%%%%%%%%%%%%%%%%%%%%%%%%%%%%%%%%%%%%%%%%%%%%%%%%%%%%%%%%%%%%%%%%%%%

\subsection{Setup}

Measurements were performed in the semi-clean room in the EEL building at 
JLab using a 4-m long dark box (see Fig.~\ref{fig:dbox}).  The box was 
instrumented with a moving cart.  In Fig.~\ref{fig:setup}, a schematic view 
of the setup is shown.  Scintillator strips with fibers were secured inside 
the box.  The trigger PMT was attached directly to the end of the 
scintillator strip through an acrylic light guide.  For the 
scintillator-light guide and the light guide-PMT connections, Bicron BC-630 
optical grease was used. 

%%%%%%%%%%%%%%%%%%%%%%%%%%%%%%%%%%%%%%%%%%%%%%%%%%%%%%%%%%%%%%%%%%%%%%%%%
\begin{figure}[!tbh]
\vspace{110mm} 
\special{psfile=../preshower/dbox.eps hscale=50 vscale=50 hoffset=35 voffset=0}
\caption{\small{Picture of the dark box. The moveable cart is mounted on 
rails.}}
\label{fig:dbox}
\end{figure}
%%%%%%%%%%%%%%%%%%%%%%%%%%%%%%%%%%%%%%%%%%%%%%%%%%%%%%%%%%%%%%%%%%%%%%%%%

WLS fibers were glued inside the scintillator grooves using Dymax UV-curable 
optical glue OP-52.  The fibers were extended about 40~cm from the end of the 
scintillator in order to connect to the photo-cathode of a test PMT.  The 
test PMT was installed inside a plastic housing, a tube with $\mu$-metal 
shielding inside.  The housing tube had two end-caps, one with connectors for 
the HV and signal cables, and the another with an adapter for the fiber 
connection~\cite{adapter}.  To secure the fiber on the photo-cathode of the 
test PMT, several plastic adapters were built with thin metallic tubes as 
inserts.  The inner diameter of the tubes was chosen to have a tight fit for 
the fibers.  The tubes were aligned along the axis of the PMT and guided the 
fibers in the direction perpendicular to the photo-cathode.  Bicron BC-630 
optical grease was used between the fiber and the photo-cathode for good 
optical contact.     

%%%%%%%%%%%%%%%%%%%%%%%%%%%%%%%%%%%%%%%%%%%%%%%%%%%%%%%%%%%%%%%%%%%%%%%%%
\begin{figure}[!t]
\vspace{50mm} 
\special{psfile=../preshower/setup.eps hscale=70 vscale=60 hoffset=70 voffset=0}
\caption{\small{Schematic view of the setup for the light yield measurements.}}
\label{fig:setup}
\end{figure}
%%%%%%%%%%%%%%%%%%%%%%%%%%%%%%%%%%%%%%%%%%%%%%%%%%%%%%%%%%%%%%%%%%%%%%%%%

The readout electronics of the system consisted of a LeCroy 1881M ADC and a 
Philips 704 discriminator.  As a gate for the ADC, the discriminated pulse 
from the trigger PMT was used.  Signals of the test and the trigger PMTs were 
delayed and connected to the ADC inputs, see Fig.~\ref{fig:setup}.  The
ADC information was read out using the standard {\tt CLAS} data
acquisition (DAQ) software.

\subsection{Determination of the Number of Photo-electrons} 

For the analysis of the photo-electron statistics, the method developed in
Ref.~\cite{yield} was used.  Each ADC spectrum was fit with a sum of a 
Poisson distribution, $P_i(N_{pe})$, convoluted with a Gaussian function,
$C_i(n_{ch})$.  Poisson distributions describe the photo-electron 
distribution, while the Gaussians are used to describe the PMT response. 
The predicted ADC spectrum for a given average number of photo-electrons 
will be:

\begin{eqnarray}
\label{eq:fit}
A&=&c\cdot\sum_i P_i(N_{pe})\times C_i(n_{ch}) \\
\label{eq:pi}
P_i(N_{pe})&=&\frac {N^i_{pe}\cdot e^{-N_{pe}} } {i!} \\
\label{eq:ci}
C_i(n_{ch})&=&\frac {1} {\sigma_1\cdot\sqrt{i}} \cdot
\exp \left( -(\frac {n_{ch}-(a_1+(i-1)\cdot a_0)} 
{\sigma_1\cdot\sqrt{2i}})^2 \right). 
\end{eqnarray}
 
In eq.(\ref{eq:fit}), the summation goes over the possible number of
photo-electrons in the spectrum, $i$.  The coefficient $c$ is for the 
overall normalization and $N_{pe}$ is the average number of photo-electrons. 
In eq.(\ref{eq:ci}), $n_{ch}$ is the pedestal-subtracted ADC channel number. 
The parameters $a_1$ and $\sigma_1$ are the position and the standard 
deviation of the single photo-electron peak in units of ADC channel. 
$a_0$ is the distance between two adjacent photo-electron peaks in units 
of ADC channel (not necessarily equal to $a_1$). The fit parameters were 
$c$, $N_{pe}$, and $a_0$. 

The parameters $a_1$ and $\sigma_1$ were defined from fits to a single
photo-electron distribution for each of the test PMTs.  Two examples of 
single photo-electron peaks are shown in Fig.~\ref{fig:sing}.  The left 
plot shows the response of the Photonis XP2802 PMT as the light intensity 
is lowered.  The right graph shows a single photo-electron peak for the 
Hamamatsu R6095 PMT.

%%%%%%%%%%%%%%%%%%%%%%%%%%%%%%%%%%%%%%%%%%%%%%%%%%%%%%%%%%%%%%%%%%%%%%%%%
\begin{figure}[tb]
\vspace{90mm} 
\special{psfile=../preshower/rel_spec_xp2802.eps hscale=60 vscale=60 hoffset=35
voffset=0}
\special{psfile=../preshower/r6095_spe_fit.eps hscale=42 vscale=50 hoffset=235
 voffset=0}
\caption{\small{(Left) ADC distribution of the XP2802 PMT corresponding to a 
few and single photo-electrons.  (Right) Fit to the single photo-electron 
distribution of the Hamamatsu R6095 PMT.  The fit was done using the sum of 
two Gaussians, one for the ADC pedestal and a second one for the single 
photo-electron distribution.}}
\label{fig:sing}
\end{figure}
%%%%%%%%%%%%%%%%%%%%%%%%%%%%%%%%%%%%%%%%%%%%%%%%%%%%%%%%%%%%%%%%%%%%%%%%%

\subsection{Determination of the Absolute Light Yield}

Since the measurements were carried out with a $\beta$-source, the amount 
of energy deposited for a given event will vary significantly.  Most of 
the measurements were done with a $^{90}$Sr source that provided a 2.3-MeV 
$\beta$ based on the decay $^{90}{\rm Sr} \to 0.546~{\rm MeV} \beta^- +
^{90}{\rm Y}  \to 2.28~{\rm MeV} \beta^-$.  The measured responses will,
however, depend on the part of the $\beta$ spectrum that was selected by the 
trigger PMT.  A typical spectrum is shown in Fig.~\ref{fig:trigpm}.  To reduce 
the systematics due to variations in the trigger conditions and to estimate 
the deposited energy, events from the endpoint of the $\beta$ spectrum 
($\sim$2~MeV) were used in the comparisons.  The endpoint was identified by 
increasing the trigger threshold until the average $N_{pe}$ remained constant. 
In Fig.~\ref{fig:b14}, fits to the ADC distributions of the Hamamatsu R7899EG 
and R6095 PMTs for different trigger settings (ADC channel 550 to 600) are 
presented.  The dependence of $N_{pe}$ on the trigger PMT ADC channel is 
shown in Fig.~\ref{fig:npe}.  As one would expect, the number of 
photo-electrons increases with increasing trigger ADC channel and flattens 
out at the end.  The number of photo-electrons at the endpoint was taken as 
the light yield corresponding to a $\sim$2~MeV energy deposition.

%%%%%%%%%%%%%%%%%%%%%%%%%%%%%%%%%%%%%%%%%%%%%%%%%%%%%%%%%%%%%%%%%%%%%%%%%
\begin{figure}[h!t]
\vspace{130mm} 
\special{psfile=../preshower/trig_test_adc.eps hscale=70 vscale=80
hoffset=55 voffset=-5} 
\caption{\small{ADC spectra of the trigger and test PMTs. The test PMT ADC
distributions for different slices of the trigger PMT ADC were fitted to get 
the photo-electron statistics.}} 
\label{fig:trigpm}
\end{figure}
%%%%%%%%%%%%%%%%%%%%%%%%%%%%%%%%%%%%%%%%%%%%%%%%%%%%%%%%%%%%%%%%%%%%%%%%%

%%%%%%%%%%%%%%%%%%%%%%%%%%%%%%%%%%%%%%%%%%%%%%%%%%%%%%%%%%%%%%%%%%%%%%%%%
\begin{figure}[h!t]
\vspace{130mm} 
\special{psfile=../preshower/bin14_7899_6095.eps hscale=80 vscale=80 
hoffset=35 voffset=-5} 
\caption{\small{Fit to the ADC spectra of the Hamamatsu R7899EG and R6095
PMTs with eq.(\ref{eq:pi}). The spectra correspond to the trigger PMT ADC 
channels 550 to 600.}} 
\label{fig:b14}
\end{figure}
%%%%%%%%%%%%%%%%%%%%%%%%%%%%%%%%%%%%%%%%%%%%%%%%%%%%%%%%%%%%%%%%%%%%%%%%%

%%%%%%%%%%%%%%%%%%%%%%%%%%%%%%%%%%%%%%%%%%%%%%%%%%%%%%%%%%%%%%%%%%%%%%%%%
\begin{figure}[ht!]
\vspace{11.5cm} 
\special{psfile=../preshower/npe_7899_6095.eps hscale=70 vscale=70 hoffset=75 
voffset=0} 
\caption{\small{Dependence of $N_{pe}$ on the trigger PMT ADC channel.  The 
closed squares are $N_{pe}$ for the R7899EG PMT and the open squares are for 
the R6095 PMT.  The solid line curve is the $\chi^2$ distribution for the 
fit to the spectra of the R7899EG PMT.}} 
\label{fig:npe}
\end{figure}
%%%%%%%%%%%%%%%%%%%%%%%%%%%%%%%%%%%%%%%%%%%%%%%%%%%%%%%%%%%%%%%%%%%%%%%%%

\subsection{Light Yield with Single-Fiber Readout}

Studies of the PMT response were performed with the same FNAL scintillator 
and the single-clad 1-mm diameter Y11 WLS fiber (WLSF).  Sample results are 
shown in Fig.~\ref{fig:diffpm}.  The response was extrapolated to the endpoint 
energy by increasing the trigger thresholds.  All of the fits show the 
expected flattening at the end of the spectrum.  The tabulated yields 
(see Table \ref{tab:diffpm}) are based on the endpoint extrapolation to 
$\sim$2~MeV energy deposition. 

%%%%%%%%%%%%%%%%%%%%%%%%%%%%%%%%%%%%%%%%%%%%%%%%%%%%%%%%%%%%%%%%%%%%%%%%%
\begin{figure}[ht!]
\vspace{130mm} 
\special{psfile=../preshower/diff_pmt_npe.eps hscale=60 vscale=60 hoffset=75 
voffset=0}
\mbox{
\begin{picture}(-120,-120)(350,0)
\put(450,7){\large {\bf Trigger PMT ADC bin center}}
\put(380,210){\large {\bf $N_{pe}$}}
\end{picture}}
\caption{\small{Dependence of $N_{pe}$ on the trigger PMT ADC channel.  The
closed squares are $N_{pe}$ for R7899EG (red), R6095 (black), XP2802 (blue), 
and R1450 (green) PMTs.  The open squares are the $\chi^2$ distributions for 
the fits (in the same color coding).}} 
\label{fig:diffpm}
\end{figure}
%%%%%%%%%%%%%%%%%%%%%%%%%%%%%%%%%%%%%%%%%%%%%%%%%%%%%%%%%%%%%%%%%%%%%%%%%

%%%%%%%%%%%%%%%%%%%%%%%%%%%%%%%%%%%%%%%%%%%%%%%%%%%%%%%%%%%%%%%%%%%%%%%%%
\begin{table}[ht!]
\begin{center}
\begin{tabular}{|c||c|c|c|c|c|c|} \hline 
PMT & R7899EG & R6095 & XP2802 & R1450 & 9124B \\ \hline
$N_{pe}$ & 10.3 & 7.6 & 7.3 & 6.5 & 4.7   \\ \hline 
$\sigma_{N_{pe}}$ & 0.53 & 0.57 & 0.39 & 0.17 & 0.14 \\ \hline 
$\chi^2$ & 1.7 & 0.92 & 1.16 & 1.23 & 0.7 \\ \hline  
\end{tabular} 
\end{center} 
\caption{\small{The number of photo-electrons for different PMTs
corresponding to a 2~MeV energy deposition in the FNAL extruded scintillator 
with one groove.  Readout employed a 1-mm diameter, single-clad Y11 WLS 
fiber.}}
\label{tab:diffpm} 
\end{table} 
%%%%%%%%%%%%%%%%%%%%%%%%%%%%%%%%%%%%%%%%%%%%%%%%%%%%%%%%%%%%%%%%%%%%%%%%%

The Hamamatsu R7899EG (green extended photo-cathode) PMT has the highest 
light yield, followed by the Hamamatsu R6095 PMT with $\sim$26\% fewer 
photo-electrons.  The other PMTs, the PHOTONIS XP2902, the ElectronTubes 
9124B, and the Hamamatsu R1450, did not perform as well. 

%%%%%%%%%%%%%%%%%%%%%%%%%%%%%%%%%%%%%%%%%%%%%%%%%%%%%%%%%%%%%%%%%%%%%%%%%
\begin{figure}[htb!]
\vspace{120mm} 
\special{psfile=../preshower/diff_scint.eps hscale=60 vscale=60 hoffset=75 
voffset=0} 
\mbox{
\begin{picture}(-120,-120)(350,0)
\put(450,7){\large {\bf Trigger PMT ADC bin center}}
\put(380,210){\large {\bf $N_{pe}$}}
\end{picture}}
\caption{\small{Dependence of $N_{pe}$ on the trigger PMT ADC channel.  The
closed squares are $N_{pe}$ for the FNAL extruded scintillator strip (red), 
Kharkov extruded scintillator strip (green), and the Eljen diamond-cut strip 
(blue).  Note that the FNAL and Kharkov scintillator strips have a reflective
coating, while the Eljen scintillator was wrapped only in aluminized mylar.}} 
\label{fig:diffsc}
\end{figure}
%%%%%%%%%%%%%%%%%%%%%%%%%%%%%%%%%%%%%%%%%%%%%%%%%%%%%%%%%%%%%%%%%%%%%%%%%

Comparison of the scintillators was done using an R6095 PMT.  All of 
the strips had a 1-mm, single-clad Y11 fiber glued in the groove on the 
surface of the strip.  The EJ204 scintillator did not have a reflective 
coating, therefore two measurements with and without wrapping were done. 
The results are tabulated in Table~\ref{tab:scint} and are shown in 
Fig.~\ref{fig:diffsc}.  The FNAL extruded scintillator showed the highest 
light yield.

%%%%%%%%%%%%%%%%%%%%%%%%%%%%%%%%%%%%%%%%%%%%%%%%%%%%%%%%%%%%%%%%%%%%%%%%%
\begin{table}[ht!]
\begin{center}
\begin{tabular}{|c||c|c|c|c|} \hline 
         & FNAL & Eljen & Eljen   & Kharkov \\
         &      &       & wrapped &         \\ \hline 
$N_{pe}$ & 7.6  & 2.5   & 5.7     & 6.8     \\ \hline 
$\sigma_{N_{pe}}$ & 0.57 & 0.55 & 0.33 & 0.23 \\ \hline 
$\chi^2$ & 0.92 & 1.08 & 1.1 & 1.02 \\ \hline  
\end{tabular} 
\end{center} 
\caption{\small{The number of photo-electrons for 2~MeV of energy deposition 
in different scintillator strips. Readout was with 1-mm, single-clad Y11 WLS
fibers and an R6095 PMT.}} 
\label{tab:scint} 
\end{table} 
%%%%%%%%%%%%%%%%%%%%%%%%%%%%%%%%%%%%%%%%%%%%%%%%%%%%%%%%%%%%%%%%%%%%%%%%%

The candidate WLS fibers were studied. Different fiber types and fiber
diameters were glued into the FNAL scintillator strip and the photo-electron 
statistics were measured with the Hamamatsu R7899EG PMT.  The relative light 
yield for different fibers is presented in Fig.~\ref{fig:wfs}.  The light 
yield for the Bicron G92 fiber is not presented because it was too low (not a 
surprise since G92 is designed to have a fast response time and has a small 
attenuation length).

The number of photo-electrons was observed to increase with the diameter of 
the fiber.  A 1.5-mm fiber has $\sim$25\% more light than a 1-mm diameter 
fiber, and a 2-mm fiber has $\sim$40\% more light.  The light yield is also 
higher for multi-clad fibers compared to single-clad by about 20\%.  Bicron 
G91A 1-mm diameter, single-clad fiber has about 10\% less light yield than 
Kuraray 1.0-mm, single-clad Y11 fiber.

%%%%%%%%%%%%%%%%%%%%%%%%%%%%%%%%%%%%%%%%%%%%%%%%%%%%%%%%%%%%%%%%%%%%%%%%%
\begin{figure}[ht!]
\vspace{110mm} 
\special{psfile=../preshower/diff_fibers.eps hscale=70 vscale=70 hoffset=70 
voffset=-20} 
\caption{\small{Light yield for different WLS fiber readout relative to 1-mm
diameter, single-clad Y11 fiber. In all cases, the fibers were glued to the
FNAL extruded fiber and the Hamamatsu R7899 PMT was used for the readout.}} 
\label{fig:wfs}
\end{figure}
%%%%%%%%%%%%%%%%%%%%%%%%%%%%%%%%%%%%%%%%%%%%%%%%%%%%%%%%%%%%%%%%%%%%%%%%%

\subsection{More Measurements}

Additional measurements have been performed to check the systematics of the 
results.  The same scintillator-WLSF-PMT combinations were measured at several 
settings.  Results from different measurements were consistent to within a 
few percent.  One of the major sources of uncertainty is the determination
of the single photo-electron peak position and the Gaussian width for a given
PMT.  Therefore, measurements for one PMT but several different voltages were
performed.  In Fig.~\ref{fig:diffhv}, the dependence in the measured number of 
photo-electrons vs. the trigger PMT response for different HV settings is
shown.  The maximum difference in the estimated number of photo-electrons for 
the extrapolated end point is $<$10\%.

%%%%%%%%%%%%%%%%%%%%%%%%%%%%%%%%%%%%%%%%%%%%%%%%%%%%%%%%%%%%%%%%%%%%%%%%%
\begin{figure}[ht!]
\vspace{120mm} 
\special{psfile=../preshower/636_637_638.eps hscale=60 vscale=60 hoffset=70 
voffset=0} 
\mbox{
\begin{picture}(-120,-120)(350,0)
\put(450,7){\large {\bf Trigger PMT ADC bin center}}
\put(380,220){\large {\bf $N_{pe}$}}
\end{picture}}
\caption{\small{Light yield for the Hamamatsu R6095 PMT operated at different
HV settings.  The blue squares correspond to 800~V, the red squares are at 
850~V, and the green squares are at 900~V.  The open squares correspond to 
the $\chi^2$ distributions of the fits.}}
\label{fig:diffhv}
\end{figure}
%%%%%%%%%%%%%%%%%%%%%%%%%%%%%%%%%%%%%%%%%%%%%%%%%%%%%%%%%%%%%%%%%%%%%%%%%

Measurements were made with cosmic ray muons in order to verify the above 
results.  The light yield for the R6095 PMT with FNAL scintillator and 
1-mm diameter single-clad Y11 fibers was determined.  As a gate for the ADC, 
the coincidence of the trigger PMT and the scintillator counter positioned
above the scintillator strip was used.  The counter had 1-cm thick, 
$1.5 \times 4$~cm$^2$ scintillator.   A fit to the test PMT spectrum, 
selected with a cut on the trigger PMT ADC in the range of MIP energy 
deposition, yielded $\sim$9 photo-electrons.  There is a difference of 
$\sim$20\% compared to the number of photo-electrons obtained with the 
$^{90}$Sr source ($\sim$2~MeV energy deposition). This difference can be 
explained by the fact that in the cosmic setup, the average path of particles 
is $>$1~cm.  Also the energy loss in the 1-mm titanium dioxide coating 
would lower the $\beta$ energy deposited, but it would not influence the 
cosmic ray muon energy deposition.  More precise measurements of the absolute 
photo-electron yield will be done with a box prototype (see below).

\subsection{Multiple Fiber Readout}

In the final design of the PCAL, each scintillator strip will be read out 
with three WLS green fibers embedded in the grooves on the surface of the
scintillator strip.  To estimate the expected light yield for the three 
fiber readout, the Amcrys-Plast, Kharkov extruded scintillator strips with
three grooves were used with Kuraray 1-mm diameter, single-clad Y11 fibers. 
In Fig.~\ref{fig:13f}, the light yield dependences on the trigger PMT ADC 
bin center are shown for one, two, and three fiber readout. The source 
($^{90}$Sr) position was unchanged during these measurements.  As one can 
see, the number of photo-electrons is proportional to the number of fibers. 
In the figure the open squares represent the $\chi^2$ distribution for the 
fit to the two fiber readout case.

%%%%%%%%%%%%%%%%%%%%%%%%%%%%%%%%%%%%%%%%%%%%%%%%%%%%%%%%%%%%%%%%%%%%%%%%%
\begin{figure}[ht!]
\vspace{12.0cm} 
\special{psfile=../preshower/1_to_3f.eps hscale=70 vscale=70 hoffset=60 
voffset=0}
\caption{\small{Dependence of $N_{pe}$ on the trigger PMT ADC channel for one 
(closed squares), two (closed upward triangles), and three (closed downward 
triangles) WLS fiber readout from the Kharkov scintillator strip with three 
grooves on the surface. The source was $^{90}$Sr and the readout PMT was the 
R6095.  The open squares represent the $\chi^2$ distributions for the fit to 
the two fiber readout case.}} 
\label{fig:13f}
\end{figure}
%%%%%%%%%%%%%%%%%%%%%%%%%%%%%%%%%%%%%%%%%%%%%%%%%%%%%%%%%%%%%%%%%%%%%%%%%

The source position dependence was studied using three fiber readout.  The
source was moved on the surface of the strip from one side to the other, and 
the measurements of the light yield were performed at four points.  The 
results are shown in Fig.~\ref{fig:coord}.  No dependence on the position 
of the source is observed.  However, it should be noted that the multi-fiber 
readout and the position dependence should be checked with the final strip 
geometry since the Kharkov scintillators are only 2.63-cm wide, while for 
the pre-shower, 4.5-cm wide scintillators will be used.

%%%%%%%%%%%%%%%%%%%%%%%%%%%%%%%%%%%%%%%%%%%%%%%%%%%%%%%%%%%%%%%%%%%%%%%%%
\begin{figure}[ht!]
\vspace{110mm} 
\special{psfile=../preshower/coord_dep.eps hscale=70 vscale=70 hoffset=60 
voffset=-5}
\caption{\small{Dependence of $N_{pe}$ on the trigger PMT ADC channel for 
the three fiber readout of the Kharkov three-groove scintillator strip. 
Different color symbols correspond to different position of the $^{90}$Sr 
source. For light readout, the R6095 PMT was used.}} 
\label{fig:coord}
\end{figure}
%%%%%%%%%%%%%%%%%%%%%%%%%%%%%%%%%%%%%%%%%%%%%%%%%%%%%%%%%%%%%%%%%%%%%%%%%

\subsection{Summary of the Test Measurements}

The light yield for several different types of scintillator strips, WLS
green fibers, and PMTs was measured.  The purpose of these measurements
was to select the best combination of the scintillator-WLSF-PMT based on
the performance and price.  Systematic uncertainties of the relative
light yield measurements of different combinations of
scintillator-WLSF-PMT were $<$10\%. For the absolute light yield, the 
estimated systematic uncertainty was $\sim$20\%.

The best results were obtained with the FNAL extruded scintillator, Kuraray 
Y11 fibers, and the Hamamatsu R7899EG PMT. The Hamamatsu R6095 PMT, selected 
with QE $>$16\% at 500~nm, showed only $\sim$25\% less light yield, while in 
price it is about 33\% times less expensive.  Multi-clad fiber readout showed 
$\sim$20\% more light than single-clad fiber readout, but it is about 35\% 
more expensive. 

Based on the measurement results and the available price estimates, the 
choice for the pre-shower will be: the FNAL extruded scintillator, Kuraray
1-mm diameter, single-clad Y11 fiber, and the Hamamatsu R6095 PMT,
selected with QE $>$16\% at 500~nm.  It should be noted that by the
performance and price, extruded scintillators from Amcrys-Plast, Kharkov 
(Ukraine), Bicron G91A wavelength-shifting fibers, and the Hamamatsu 
R1450 PMT, selected to have $>$18\% quantum efficiency at 500~nm, were not
too far from the best choice set and generally meet the requirements
for the pre-shower.     

%%%%%%%%%%%%%%%%%%%%%%%%%%%%%%%%%%%%%%%%%%%%%%%%%%%%%%%%%%%%%%%%%%%%%%%%%
\begin{table}[htbp]
\begin{center}
\begin{tabular}{||c||c||} \hline 
 Component    & Optimal Choice \\ \hline 
 WLS Fiber    & Kuraray 1-mm diameter, single clad WLS fiber (Y11) \\ \hline 
 PMT          & Hamamatsu R6095 PMT, selected with QE $>$16\% at 500~nm \\ \hline  
 Scintillator & FNAL extruded scintillator \\ \hline  
\end{tabular} 
\end{center} 
\caption{\small{Final choice for the pre-shower calorimeter components.}}
\label{tab:choice} 
\end{table} 
%%%%%%%%%%%%%%%%%%%%%%%%%%%%%%%%%%%%%%%%%%%%%%%%%%%%%%%%%%%%%%%%%%%%%%%%%

\clearpage

\subsection{Prototype of the PCAL Module}

A small-size prototype has been designed and constructed from existing
materials.  The cross section of the prototype has a rectangular shape and 
is on the order of 30~cm.  The prototype contains 15 layers of scintillator 
and 14 layers of lead.  Top and side views of this prototype are shown in 
Fig.~\ref{fig:boxpro1}.  The scintillator layers were read out on three 
sides ($X1$, $Y$, $X2$), so as to mimic the final ($UVW$) readout planned 
for the PCAL.  In each view there were five readout strips channels. 
The scintillator and lead layers were stacked using horizontal supports to 
secure them inside the box.  There were openings made in these supports and 
on the side walls to allow the WLS fibers to come out of the box.  To connect 
to the PMTs, the fibers were glued inside holders located on shelfs that were 
attached along the side plates, outside of the box.  The ends of the fibers 
were polished and the PMTs were attached to the fibers using an optical gel. 
Only $X1$ and $Y$ views were furnished with Hamamatsu R6095 PMTs. On $X2$, 
5 different PMTs were used.  The full five module longitudinal depth in each 
view of the prototype will allow accurate characterization of the shower and 
a fairly complete determination of the combined response of the components in 
terms of energy and timing.  The prototype will be placed in the Hall B beam 
for direct studies with electrons.

%%%%%%%%%%%%%%%%%%%%%%%%%%%%%%%%%%%%%%%%%%%%%%%%%%%%%%%%%%%%%%%%%%%%%%%%%
\begin{figure}[htb]
\vspace{8.5cm} 
\special{psfile=../preshower/presh_box_1.eps hscale=40 vscale=40 hoffset=40 
voffset=0}
\special{psfile=../preshower/presh_box_3.eps hscale=60 vscale=60 hoffset=245 
voffset=10}
\caption{\small{Two views of the box prototype currently under construction. 
The full 5 module longitudinal depth will allow study of the behavior of 
shower development and actual sampling features using cosmic rays or
possibly from in-beam tests.}}
\label{fig:boxpro1}
\end{figure}
%%%%%%%%%%%%%%%%%%%%%%%%%%%%%%%%%%%%%%%%%%%%%%%%%%%%%%%%%%%%%%%%%%%%%%%%%

The first measurements with the prototype were performed using cosmic ray
muons.  In Fig.~\ref{fig:cosmset}, a schematic of the test setup is shown. 
As a trigger, a coincidence of a scintillator counter, located on top of
the prototype, with an OR of one of the prototype layers was used.  The
signals from each PMT were split into two parts with a ratio of $2:1$.  The 
larger part was sent to the discriminator and then to the trigger logic 
and TDCs.  The smaller part was delayed and sent to the ADC inputs. 

%%%%%%%%%%%%%%%%%%%%%%%%%%%%%%%%%%%%%%%%%%%%%%%%%%%%%%%%%%%%%%%%%%%%%%%%%
\begin{figure}[htb]
\vspace{13.cm} 
\special{psfile=../preshower/pcalDAQ.epsi hscale=57 vscale=57 hoffset=-10 
voffset=0}
\caption{\small{Schematic layout of the cosmic test setup of the PCAL 
prototype.}}
\label{fig:cosmset}
\end{figure}
%%%%%%%%%%%%%%%%%%%%%%%%%%%%%%%%%%%%%%%%%%%%%%%%%%%%%%%%%%%%%%%%%%%%%%%%%

The main purpose of the cosmic test was to determine the average number of
photo-electrons per MeV of energy deposition, and to compare this with the
single strip measurements presented above. In Fig.~\ref{fig:XADC}, ADC 
spectra of all five PMTs on the $X1$ view are shown.  In these
distributions, events when only a single PMT is fired in all three views 
are selected.  This selection ensures an MIP crossing through the system. 
The ADC distributions in the figure correspond to the MIP energy
distribution and have a characteristic Landau shape.  Fits were performed
using two Gaussian functions and the peak position of the Gaussian with
the narrower width was taken as the point of 2~MeV energy deposition in a
single scintillator strip.  The corresponding distribution of the $Y$ view 
is shown in Fig.~\ref{fig:YADC}.  

%%%%%%%%%%%%%%%%%%%%%%%%%%%%%%%%%%%%%%%%%%%%%%%%%%%%%%%%%%%%%%%%%%%%%%%%%
\begin{figure}[htb]
\vspace{14.cm} 
\special{psfile=../preshower/adc_x1.epsi hscale=60 vscale=60 hoffset=40 
voffset=0}
\caption{\small{ADC spectra of the $X1$ view PMTs corresponding to the MIP 
energy deposition.}}
\label{fig:XADC}
\end{figure}
%%%%%%%%%%%%%%%%%%%%%%%%%%%%%%%%%%%%%%%%%%%%%%%%%%%%%%%%%%%%%%%%%%%%%%%%%


%%%%%%%%%%%%%%%%%%%%%%%%%%%%%%%%%%%%%%%%%%%%%%%%%%%%%%%%%%%%%%%%%%%%%%%%%
\begin{figure}[htb]
\vspace{14.cm} 
\special{psfile=../preshower/adc_y.epsi hscale=60 vscale=60 hoffset=40 
voffset=0}
\caption{\small{ADC spectra of the $Y$ view PMTs corresponding to the MIP 
energy deposition.}}
\label{fig:YADC}
\end{figure}
%%%%%%%%%%%%%%%%%%%%%%%%%%%%%%%%%%%%%%%%%%%%%%%%%%%%%%%%%%%%%%%%%%%%%%%%%

To determine the photo-electron statistics, for each PMT position, the
single photo-electron peak at a given voltage was measured using a
setup with a trigger PMT and a LED.  For accurate determination of the
single photo-electron peak position, the gain vs. voltage for each PMT was
measured in the single photo-electron regime and in the regime when a large
amount of light was directed to the photo-cathode.  Corrections to the
light emission stability of the LED were applied using the trigger PMT
pulse.  In Table~\ref{tab:pestat} the obtained photo-electron statistics for
all 10 R6095 PMTs are presented.  The average number of photo-electrons 
obtained is in good agreement with the expected number of photo-electrons 
from previous test measurements with single strips.

%%%%%%%%%%%%%%%%%%%%%%%%%%%%%%%%%%%%%%%%%%%%%%%%%%%%%%%%%%%%%%%%%%%%%%%%%
\begin{table}[hbt!]
\begin{center}
\begin{tabular}{|c|c|c|c|c|c|} \hline 
    & &  & ADC peak & ADC
peak & \# of p.e. \\ 
Channel & \# of fibers & PMT HV (V)  & for MIP (1/3) & for s.p.e.
& for 1 MeV \\ 
\hline
X1-1 & 15 & 786 & $82.9\pm0.5$& 3.3& 7.5 \\
X1-2 & 15 & 791 & $95.4\pm0.2$& 2.9& 9.7 \\
X1-3 & 15 & 841 & $105.5\pm0.3$& 4.3& 7.2 \\
X1-4 & 15 & 825 & $104.\pm0.3$& 3.8& 8.1 \\
X1-5 & 14 & 855 & $116.3\pm0.6$& 5.1& 6.7 \\
Y-1 & 15 & 865 & $128.3\pm0.7$& 3.4& 11.3 \\
Y-2 & 14 & 822 & $102.9\pm0.5$& 3.7& 8.1 \\
Y-3 & 15 & 818 & $96.4\pm0.3$& 2.2& 12.8 \\
Y-4 & 15 & 762 & $87.4\pm0.2$& 2.9& 8.8 \\
Y-5 & 15 & 733 & $78\pm0.6$&1.9& 11.9 \\
\hline
\end{tabular}
\end{center} 
\caption{\small{Results of cosmic test measurements for the $X1$ and $Y$
PMT views.  The ADC value for the MIP energy deposition corresponds to 
1/3 of the PMT signal after the splitter.}} 
\label{tab:pestat} 
\end{table} 
%%%%%%%%%%%%%%%%%%%%%%%%%%%%%%%%%%%%%%%%%%%%%%%%%%%%%%%%%%%%%%%%%%%%%%%%%

\clearpage