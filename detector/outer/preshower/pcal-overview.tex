\section{Overview and Physics Requirements}

The primary goal of experiments using the {\tt CLAS12} detector at energies
up to 11~GeV is the study of internal nucleon dynamics by accessing the 
nucleon's generalized parton distributions (GPDs).  This is accomplished 
through the measurement of deeply virtual Compton scattering (DVCS), deeply 
virtual meson production (DVMP), and single spin asymmetries (SSA). Towards 
this end, the detector has been tuned for studies of exclusive and 
semi-inclusive reactions in a wide kinematic range. 

The {\tt CLAS12} program of experiments goes further than just GPDs and 
includes experiments such as the space-time characteristics of 
hadronization. Detection of $\pi^0$s and $K^0$s is important to 
complement the measurements of nuclear attenuation seen for charged pions
and kaons.  These experiments depend on the ability to detect neutral and
charged pions at high momentum.

Copious amounts of high-energy particles, both charged and neutral, will be 
produced at experiments to be done at {\tt CLAS12}.  Electromagnetic
calorimeters for the {\tt CLAS12} detector should have sufficient radiation
length to absorb the full energy of the electromagnetic showers produced by 
high-energy electrons and photons.  High-energy neutral pions present a 
challenge as well, as they decay immediately into two photons with an 
opening angle that decreases as the $\pi^0$ momentum increases.  Unless 
there is sufficient position resolution in the electromagnetic calorimeter 
of the {\tt CLAS12} detector, the two photons from $\pi^0$ decay could be 
seen as a single high-energy photon.

The separation of single high-energy photons from the photons of $\pi^0$ 
decay is very important to the deeply virtual Compton scattering (DVCS) 
experiments that represents a major physics program planned for {\tt CLAS12}.
A single high-energy photon is produced in the reaction $ep \to ep\gamma$
and the largest background to this process is from single $\pi^0$ 
production, $ep \to ep\pi^0$.  Clearly, good $\pi^0$ detection is crucial 
to separate these two processes.  In addition, direct $\pi^0$ production 
complements the DVCS measurements by accessing GPDs at low and high 
momentum transfer $|t|$.

Simulations have shown that the existing electromagnetic calorimeter (EC)
of {\tt CLAS}~\cite{ec} will not be able to absorb the full energy of the
electromagnetic showers produced by electrons and photons with momenta 
above 5~GeV.  The leakage from the back of the calorimeter will diminish 
the energy resolution, see Fig.~\ref{fig:ecsim}a.  For the simple 
kinematics of $\pi^0$ decay, above a momentum of 5.5~GeV the opening angle 
of the decay photons becomes too small to be resolved with the existing 
EC system, which is at a distance of about 6~m from the target.  The readout 
segmentation of the EC is only $\sim$10~cm.  Simulations with the GEANT 
software for the {\tt CLAS12} geometry show that these pions would be seen 
as a single cluster (i.e., one photon) by the event reconstruction 
software, see Fig.~\ref{fig:ecsim}b.

%%%%%%%%%%%%%%%%%%%%%%%%%%%%%%%%%%%%%%%%%%%%%%%%%%%%%%%%%%%%%%%%%%%%%%%%%
\begin{figure}[ht]
\vspace{70mm}
\special{psfile=../preshower/ecres.eps hscale=45 vscale=45 hoffset=30 
voffset=-15}
\special{psfile=../preshower/2clusteff.eps hscale=45 vscale=45 hoffset=245 
voffset=-15}
\mbox{
\begin{picture}(-10,-10)(350,0)
\put(520,170){\large {\bf (a)}}
\put(740,170){\large {\bf (b)}}
\end{picture}}
\caption{\small{Performance of the EC at high energies. a) The energy 
resolution of the EC as a function of the inverse square root of energy. 
b) The efficiency of reconstruction of two clusters from $\pi^0$ decay
photons as a function of pion momentum, filled red squares.  The open
symbols correspond to the probability that a single cluster is 
reconstructed with the same energy as the simulated pion.}} 
\label{fig:ecsim} 
\end{figure}
%%%%%%%%%%%%%%%%%%%%%%%%%%%%%%%%%%%%%%%%%%%%%%%%%%%%%%%%%%%%%%%%%%%%%%%%%

To reconstruct the energy of high-energy showering particles and to
separate high-energy $\pi^0$s and photons, a pre-shower detector (PCAL), 
with finer granularity, will be built and installed in front of the 
current EC.  The PCAL will have a similar geometry as the current EC.  It 
will consist of a lead-scintillator sandwich with three stereo readout planes. 
Initial simulations have shown~\cite{natalia,whitlow} that 15 layers of 
1-cm thick scintillator layers, segmented into 4.5-cm wide strips, 
sandwiched between lead sheets of 2.2-mm thickness, corresponds to about 
5.5 radiation lengths, and will be sufficient to address issues arising at 
high energies.

In summary, accurate reconstruction of high-energy electromagnetic showers 
and detection of $\pi^0$s, and in particular the ability to distinguish 
single high-energy photons from the two-photon clusters from $\pi^0$ decay, 
is essential to the experimental program using the {\tt CLAS12} detector at 
high energies that will be part of the GPD program.  Efficient detection of 
high-momentum $\pi^0$s is also needed for a variety of other {\tt CLAS12} 
proposals.  The physics dictates the need for a pre-shower calorimeter to 
be placed in front of the existing EC.  Simulations show that good $\pi^0$ 
identification can be obtained with full coverage of the EC front surface. 
In addition to the improved performance in the reconstruction of 
electromagnetic showers, the increase of the overall scintillator thickness 
in the {\tt CLAS12} forward electromagnetic calorimeters will increase the 
detection efficiency for neutrons. 


