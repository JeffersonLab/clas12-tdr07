\section{Mechanical Design}

Experience gained in the design and construction of the {\tt CLAS}
electromagnetic calorimeters, followed by 10 years of successful operation 
of these detectors, forms the basis of the PCAL design.  A calorimeter of 
sampling structure with lead as the radiator and scintillator as the active 
medium is an economical way of covering a large detection area.  The triangular
shape of the EC with sides of order of 4.5~m in length, matches well the 
forward-angle acceptance of {\tt CLAS12}.  In the EC, three stereo readout 
planes were used, segmented into 36 transverse stacks and into two 
longitudinal parts.  The calorimeter showed good performance in terms of 
energy and position reconstruction of showering particles.  The proposed 
geometry of the PCAL is similar to the geometry of the existing EC, however 
the mechanical design will have a few essential differences. 

The 15 layers of the PCAL will have all the same size, i.e. no pointing 
geometry as in the EC. This will simplify the design of the container, 
including the individual elements inside, and the assembly process.  The 
triangular boxes will contain scintillators, lead, and support elements, and
will have 1.5-in thick aluminum side walls and two composite endplates.  The 
endplates will be constructed from 2-in thick composite foam, sandwiched 
between 2-mm thick stainless steel sheets connected with aluminum bars. 
Initial FEA calculations showed that the deflection of the plates is less 
than 1.5~mm in the center in the installed position~\cite{philippe}, see 
Fig.~\ref{fig:plate}.  

%%%%%%%%%%%%%%%%%%%%%%%%%%%%%%%%%%%%%%%%%%%%%%%%%%%%%%%%%%%%%%%%%%%%%%%%%%%
\begin{figure}[ht]
\vspace{8.5cm}
\special{psfile=../preshower/endplate.eps hscale=90 vscale=90 hoffset=80 
voffset=-5}
\caption{\small{FEA simulation of the endplate deflection in the position of 
the PCAL in {\tt CLAS12} Sector 2.  The maximum deflection with 2-in thick
Rohacell foam sandwiched between 2-mm thick stainless steel sheets is
less than 1.5~mm.}}
\label{fig:plate} 
\end{figure}
%%%%%%%%%%%%%%%%%%%%%%%%%%%%%%%%%%%%%%%%%%%%%%%%%%%%%%%%%%%%%%%%%%%%%%%%%%%

%%%%%%%%%%%%%%%%%%%%%%%%%%%%%%%%%%%%%%%%%%%%%%%%%%%%%%%%%%%%%%%%%%%%%%%%%%%
\begin{figure}[ht]
\vspace{110mm}
\special{psfile=../preshower/pcal_scint.eps hscale=50 vscale=50 hoffset=30 
voffset=10}
\caption{\small{View of the opening of one PCAL sector. Three subsequent
scintillator layers have strips oriented parallel to one of the sides of
the triangle.}}
\label{fig:pcals} 
\end{figure}
%%%%%%%%%%%%%%%%%%%%%%%%%%%%%%%%%%%%%%%%%%%%%%%%%%%%%%%%%%%%%%%%%%%%%%%%%%%

Light from the scintillator strips will be transported to a photo-detector 
via wavelength-shifting fibers embedded in grooves on the surface of 
the scintillator strips.  There will be no need for optical connections 
inside the box.  This will allow for use of simple bars to secure the 
scintillator and lead layers in place inside the box, see 
Fig.~\ref{fig:pcals}.  Spacers will be used to position the scintillators 
and lead sheets inside the box within 1-2~mm tolerances. 

%%%%%%%%%%%%%%%%%%%%%%%%%%%%%%%%%%%%%%%%%%%%%%%%%%%%%%%%%%%%%%%%%%%%%%%%%%%
\begin{figure}[ht]
\vspace{110mm}
\special{psfile=../preshower/pcal_all_6.eps hscale=50 vscale=50 hoffset=30 
voffset=10}
\caption{\small{Six sector view of the PCAL.  The PMTs are shown in red and
magenta.  The PMTs for two sides are aligned along the openings between 
the sectors.  On the third side, the side perpendicular to the beam direction, 
the PMTs will occupy the space along the side where no fixtures for mounting 
other detectors exist.}}  
\label{fig:pcal6} 
\end{figure}
%%%%%%%%%%%%%%%%%%%%%%%%%%%%%%%%%%%%%%%%%%%%%%%%%%%%%%%%%%%%%%%%%%%%%%%%%%%

The fibers will be brought to the outside via feedthroughs in the side walls. 
There will be only a few feedthroughs, spaced along each side of the 
triangle.  Fibers from several scintillator strips will be routed to a single
feedthrough inside the box and will be spread out to the PMT adapters on the 
outside.  There will be shelves mounted along the side walls at the level of 
the backplate that will hold adapters for the fiber-PMT connections.  The 
shelves for the two sides will be located in the space between neighboring 
sectors to allow access to PMTs from the forward carriage (from behind the 
EC), see Fig.~\ref{fig:pcal6}.  For the side perpendicular to the beam 
direction, the location and the length of the shelves will be defined by the 
space free from the mounting arms of the low threshold {\v C}erenkov counter.  
There will be thin aluminum covers along the shelves to make the PCAL 
container light tight.



