\section{Event Reconstruction}


The main goal of the event reconstruction program is to provide track parameters and particle identification on an event basis, to any physics analysis. These events will in practice either come from real data, or from the Monte Carlo simulation. The reconstruction program should also be able to perform specific tasks, like sending hits and corresponding fit results to the event display service.
The program is currently developed in C, but will eventually be coded in C++, with an object oriented structure. Even if the tracking techniques (track finding and track fitting) can be used in a detector-independent form, their implementation will be adapted to the geometry of the CLAS12 spectrometer, and therefore split in 2 parts:

-	The central tracking: this part provides the reconstruction of particles detected in the Silicon Vertex Tracker, located inside the high solenoid magnetic field. Due to the � approximate - phi symmetry of this region, we implemented a Kalman Filter algorithm using cylindrical coordinates. A track finding procedure has also been developed to separate real tracks from the background, and some results of these techniques are presented in Section 2.8.

-	The forward tracking: particles produced at small angles are detected in 2 different subsystems, the Forward Vertex Tracker and the Drift Chambers. A key issue of the tracking program is to provide an algorithm that will efficiently match track segments found in these detectors, in the presence of background. As for the central part, we also implemented a Kalman Filter algorithm whose preliminary results can also be found in Section 2.8.


The last task of the tracking program will be to link tracks found in these 2 regions, that will include a vertex fitting algorithm, thus providing a full event reconstruction. Different outputs (not only the structure, but also the contents) will be produced, depending on the incoming request.
