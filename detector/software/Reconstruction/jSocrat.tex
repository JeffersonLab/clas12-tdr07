\subsection{Tracking in Java (jSocrat)}

As described above, tracking in CLAS12 is done in three detectors, the Drift Chambers (DV), and the Forward and Barrel regions of the Silicon Vertex Tracker.  Event detector information is passed to the tracking program, {\it jSocrat}, via the standard EVIO file format. {\it jSocrat}  contains two separate components:  forward and central tracking.  Central tracking deals with the barrel region of the SVT, while forward tracking connects the forward region of the SVT with the drift chambers.  The central and forward tracking can be run independently from one another, and therefore designed to be run on separate threads.
{\it jSocrat} is largely based on Sebastien Procureur�s program, Socrat, developed within the CERN Root C-Interpreter, and borrows most of the algorithms from Socrat. 
Hit-linking in the forward tracking is accomplished separately for the drift chamber and the forward SVT.  In the drift chamber, the task is to find tracks among the hits:  first, grouping hits within each superlayer to form �clusters�, then linking pairs of clusters to produce �segments�, and then linking the �segments� together to produce tracks.  Using similar criteria, track segments are also constructed by hits in the forward SVT.  (For a more detailed description of the algorithms, see the javadoc of jSocrat in the CLAS12 source code repository). 
For each of the tracks found in the DC, a Kalman filter is run to calculate the track .  During the Kalman filter, the path is tracked through the magnetic field backwards from the DC, until the track reaches the plane of intersection with the closest plane of the FSVT.  The nearest FSVT track segment to the extrapolated track is used for continuing the tracking as far upstream as possible.
The process is similar for the central tracking.  The barrel SVT hits are linked together to form strip intersections.  Then these intersections are linked together to form tracks.  For each track, intersection positions are recalculated using a helical approximation that utilizes the other three (or two) intersections in the track.  (For a more detailed description of the algorithms for the BSVT, see the javadoc of jSocrat in the CLAS12 source code repository).   This is done to get a better estimate of where the particle hit the strips in the BSVT. Then, a Kalman filter is used for estimating the tracking parameters.  Because the track has to be swum backwards a shorter distance in the central tracking than in the forward tracking, the Kalman filter algorithm for the central tracking takes less time.  Both Kalman filters use the same time-step. 

The most time-consuming task in jSocrat is running the Kalman Filter on the forward tracks.  The Kalman filter can thus be multi-threaded.  In the Kalman filter, jSocrat utilizes the Clas12 detector�s field map files, which it loads in at start time from a query to the Magnetic Field Service.  

	The output in jSocrat is an EVIO file, the banks of which are described in the javadocs of jSocrat.

Currently, vertex finding is being developed.
