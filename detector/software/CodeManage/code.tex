\section{Code Development and Distribution}

\subsection{Code Management}

For CLAS 12 GeV, we have elected to use the widely adopted and free (open source) �subversion� revision control system. Subversion is the open source software community�s replacement for cvs. It has many of the same features and employs the same no-lockout paradigm. (That is, conflicts are resolved through merging rather than avoided through code locks�the latter generally found to be too draconian and a hindrance to productivity.) In addition, subversion plug-ins are available for the popular integrated development environments, such as the widely used eclipse. This allows one to check in, check out, track changes, and merge differences with mouse-clicks in a development environment rather than through a command line.

In CLAS 6 GeV, all users, whether or not they were developers, accessed the repository, downloaded code and build scripts, and tried, with mixed results, to build the complex packages. We have abandoned that approach. In CLAS 12 GeV, we have decided to implement a three tiered code distribution system. The first level will be access to the subversion repository. Only developers will access code in this manner. The second level will be code releases, in the form of archives, and intelligent build scripts that do not rely on environment variables. In CLAS 6 GeV the user wanting to use an application accessed the repository and downloaded the latest code, which may include bugs checked in the night before. In CLAS 12 GeV the user will go to a web page and download a specific, tested released. A third tier of release, for limited systems (probably only for whatever linux system JLab is supporting) is to distribute binaries.

This in an area in which we expect and have already obtained substantive student involvement.

\subsection{Code Release} 

The consensus in the software group is to base our software process on what is know in the software community as �agile programming.� Part of agile programming is a rapid release schedule based on development cycles called �sprints.� The exact frequency has not yet been determined, but the canonical sprint duration is one month: two weeks of development and two weeks of testing and bug fixing. So every month a new version of all software is released, typically with modest changes from the previous release. Functionality is advance incrementally as opposed to infrequent but massively different updates.

\subsection{Software Tracking}

The software group recognized that complicated software development is aided by requirements, task, and bug tracking. To this end, Christopher Newport University has deployed a web-based commercial package called Gemini for the use of the CLAS 12 GeV effort. Gemini will allow us to enter projects corresponding to the main development efforts, such as gemc for the Geant-4 simulation. Then resources (developers) are assigned to the project. The time development will match the code release sprint cycles. For the next cycle, the new tasks will be entered as well the bugs that have to be fixed. Developers will enter estimates regarding the time it will take to complete the tasks and fix the bugs. Project managers can see if the estimated time fits with the cycle duration�if not he or she can adjust according by postponing some tasks or adding new tasks. Developers can log their time to a task or bug so as to develop better intuition for estimating. The subversion revision control system can be set up so that any code checked in must have a comment that ties it, by ID to a task or bug in Gemini.
