\subsection{\label{sec:software.services.magneticfield}Magnetic Field}
The magnetic field service is implemented in \texttt{C++} using extensive use of the standard library's \texttt{map} construct. The entire field map is divided into individual maps corresponding to various magnets such as the solenoid and main torus. Each of these maps inherit all capabilities of \texttt{std::map}. In addition, it has methods to get the magnetic field at a certain point, check for consistency within the map, and interpolate values inside the defined grid spacing.

Each map can be tailored to the field it holds. For instance, the main torus is defined in cylindrical coordinates, with $\phi\in[0,\frac{\pi}{6}]$. The class holding the main torus map then has an algorithm to calculate the field at any point $\phi\in[0,2\pi]$. However, this is only good for ideal fields. For measured fields, the class can be easily extended to handle a case where the field is known precisely in the entire region. Furthermore, the dimensions and coordinates of the map's position and field need not be same. Inside the solenoid field for instance, the position is stored in ($r$, $\phi$) while the magnetic field is stored in ($x$, $y$, $z$).

A separate \emph{mother} class is responsible for loading in and storing the maps from a database or file. This class is aware of the volumes of the individual maps and sums the fields where appropriate.

The final layer on this system is the magnetic field \texttt{ClaRA} service. This is where the mother class is initialized and held in memory. Several mother classes can be held; i.e. one for the ideal fields, one for the measured fields, and one mix of these two. The service registers itself with the \texttt{ClaRA} system and can provide the various field maps in several formats depending on the consumer's preference. As an example, the service can be polled for an entire map or for an individual position.
