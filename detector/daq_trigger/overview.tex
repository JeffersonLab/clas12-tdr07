\section{Overview and Requirements}

\subsection{Requirements}

The following parameters are required for the {\tt CLAS12} Data 
Acquisition (DAQ) System: at least 10~kHz event rate, at least 100~MB/s 
data rate at the Event Recorder level, and not more than a 15\% dead 
time under the above listed conditions.

\subsection{Current DAQ System Analysis}

Currently the {\tt CLAS} DAQ system runs at an 8-10~kHz 
event rate and a 25-35~MB/s data rate at a dead time of not more than 
15\%.  Since it will be partially re-used in {\tt CLAS12}, we will 
discuss its main features and limitations, along with improvements 
necessary to achieve the {\tt CLAS12} requirements.

\subsubsection{Front-End Electronics: ADCs}

The currently used FASTBUS 1881M ADCs set a limit for the entire DAQ 
system: 12~$\mu$s conversion time projected into 12\% dead time at 10~kHz 
event rate.  It can be improved by switching to the low-resolution mode of
these units, which sets a 9~$\mu$s conversion time.  In this case the 
event rate limit can be pushed up to about 15~kHz.  This means that from 
a performance point of view, the 1881M ADCs can be re-used, however, the 
number of FASTBUS ADCs must be doubled to equip all new detectors, and 
delay cables and pretrigger discriminators must be purchased as well.

We have decided to replace all existing ADCs and equip all new detectors with
Flash ADC boards. That decision provides higher performance for the traditional
DAQ system and free-running DAQ compatibility for future upgrades.  Flash ADC
boards with on-board processing units will be used as the first stage of the
Level-1 trigger, so no pretrigger discriminators are required (as in the
current {\tt CLAS} DAQ system.  The delay cables will be eliminated as well.

It should be mentioned that removing the 1881Ms will increase the Level-1 
trigger decision time from hundreds of nanoseconds to at least 3~$\mu$s, 
making it more powerful.  In addition, Flash ADCs can deliver more 
detailed information to the trigger logic than the pretrigger discriminators.

\subsubsection{Front-End Electronics: TDCs and Discriminators}

All fast detectors in {\tt CLAS} are already equipped with new pipeline TDC
boards, both low (85~ps) and high (35~ps) resolution, and they will be
re-used in {\tt CLAS12}.  All new detectors will be equipped with the same
types of boards.  We are planning to purchase new VME-based JLAB-made
discriminators to equip all channels, eliminating the existing CAMAC-based 
LeCroy discriminators.

\subsubsection{Front-End Electronics: TDCs and ADBs in Drift Chambers}

The drift chamber readout is equipped with FASTBUS 1877 TDCs, and according 
to our plan, they will be re-used in {\tt CLAS12}.  All FASTBUS crates
needed to accommodate those TDCs were recently equipped with new remotely 
controllable power supplies and fan units, which will extend their lifetime. 
The LeCroy 1877 TDC board has the following basic parameters: 500~ns
resolution, 20~ns double-pulse resolution, up to 32~$\mu$s full scale, 
1 to 512~$\mu$s fast clear window with 250~ns fast clear settling time, 
conversion time 750~ns + 50~ns/hit (1.6~$\mu$s minimum). The conversion time
and readout speed of the 1877s will not limit the DAQ performance until about 
15~kHz, however, we may see aging effects and the necessity to increase the 
event rate beyond 15~kHz in the future.  In that case, all FASTBUS 1877
TDC boards can be replaced with pipeline TDCs. Replacement will not require 
any additional changes in other subsystems such as the trigger electronics.

The Amplifier-Discriminator board (ADB) electronics will be partially re-used,
but new ADB crate backplanes, multiplexer (MUX) boards, and trigger boards 
(segment finders) will be designed.  The number of ADB crates will be decreased 
from 30 to 18, leaving a significant number of spare power supplies and fan units.

\subsubsection{Trigger System}

The current {\tt CLAS} Level-1/Level-2 trigger system will be completely 
replaced with a new Level-1/Level-2 system capable of a few microseconds 
decision time.  The new trigger system will include segment and track 
finding capabilities in the drift chambers, cluster finding in the 
calorimeters, and matching capabilities between different detectors.  In
addition, a Level-3 trigger will be running between the Event Builder and 
the Event Recorder for additional event rate reduction and/or data
filtering.  Level-3 will perform full event reconstruction, including
hit-based tracking and time-based tracking, if required.  The trigger system, 
in general, will be flexible to accommodate different running conditions, 
with the main goal to search for events with electrons.

\subsubsection{Online Monitoring}

Four practically independent monitoring systems are currently used in 
{\tt CLAS}: Nagios-based computer and network monitoring, SmartSocket-based 
DAQ and Online monitoring, EPICS-based slow controls monitoring, and data 
monitoring.  In addition, a few smaller hardware-specific systems are used. 
The data monitoring system includes visualization only, with almost no alarm 
capabilities.

{\tt CLAS12} will have one integrated monitoring system, with a standard 
low-level interface compatible with the {\tt CLAS12} equipment.  It will 
also be compatible with the accelerator control and monitoring system on 
some level.  It will have not only monitoring, but also control capabilities. 
The implementation of such a system is under discussion and concept 
development will take at least one more year.  However, with our experience in 
developing and maintaining such systems, we are not expecting any problems in 
developing the new integrated system.

\subsubsection{Online Calibration}

The online calibration system must provide prompt information for the
trigger system and online monitoring, and will be used for preliminary
offline data processing.  Such a system does not exist in {\tt CLAS} and 
will be implemented in {\tt CLAS12}.

\subsection{`Free-Running' vs. Traditional DAQ Systems}

Since we will employ some old FASTBUS boards in the readout, the new
DAQ system cannot be run in `free-running' mode, however, all new components 
will be compatible with the `free-running' concept.  This means that the
{\tt CLAS12} DAQ system will become `free-running' after all FASTBUS boards 
are replaced in the future.

\subsection{SVT Readout}

The Silicon Vertex Tracker (SVT) is not a part of the {\tt CLAS12} trigger 
system.  The SVT readout is not included in the {\tt CLAS12} DAQ section, 
and is described in detail in the SVT section of this document (see
Section~\ref{svt:daq}.  However, it should be mentioned that the current SVT 
readout design sets a limit for the Level-1 trigger decision time at 3~$\mu$s, 
which may effect its ability to make the desired decisions.  Two solutions in 
the SVT readout are considered to avoid that problem: new readout chips and
a 16~$\mu$s Level-2 FIFO buffer.  More details can be found in the SVT section.









