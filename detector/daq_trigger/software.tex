\section{Software}
\label{sec:software}

\subsection{CODA Data Acquisition System}

{\tt CLAS12} will use the CODA data acquisition software~\cite{coda}, similar 
to what we are using in {\tt CLAS}.  The {\tt CLAS} version of CODA was 
modified in recent years to improve performance and reliability, and those 
changes will be incorporated into the future version of CODA, along with 
appropriate redesign and cleanup.  The most important changes were `parallel' 
front-end readout, multi-processor ROC support, and a multi-threaded Event 
Builder.  As a result, {\tt CLAS} is running at an 8-10~kHz event rate, 
which is close to the {\tt CLAS12} requirements.  We are expecting to get 
new software from the CODA group by the time of {\tt CLAS12} startup.  Until
then, we will use the current {\tt CLAS} version of CODA for detector 
commissioning and test runs.

Recent upgrades have eliminated most of bottlenecks from the DAQ software, but 
there are still places where improvements can be made.  One of them is 
multi-event readout and corresponding data coping inside the ROC memory. 
That modification will improve DAQ performance significantly for the 
free-running DAQ mode, but it will not have a big impact while {\tt CLAS12}
is using FASTBUS 1877 TDCs. Those TDCs have 6 event buffers and need to be
readout on an event-by-event basis to avoid back-logs.  When the FASTBUS 
electronics are replaced, we plan on utilizing multi-event readout.

\subsection{{\tt CLAS12}-Specific Components Running as CODA Extensions}

Several CODA extensions were developed for the {\tt CLAS} detector, and some 
of them will be re-used in {\tt CLAS12}.  The CODA fragment format was 
expanded by adding an additional data bank header; this way several data 
banks with different formats could be created and processed by the same ROC. 
This feature may be implemented into the new CODA version, or we can continue
to use it as it is implemented now.

The currently used BOS data format will be obsolete, and the CODA data format 
will be used instead for all output data files, but data will be organized 
based on the detectors, not on the readout hardware layout.  Again, it can 
be done by expanding the CODA format or by {\tt CLAS12}-specific extensions.

The data translation procedure running in the ROCs will be re-used, with 
possible data format modifications.  Two advantages of this method should 
be mentioned: the data volume reduction (especially in the drift chambers) 
and the ability to access data using the `offline' format on the very first 
stage of the data acquisition process.  Also the ROC-level histogram package 
played a critical role in the DAQ performance analysis, and it will be reused 
after adjustment to the new data format.  The DAQ performance profiler will 
be reused, although it may be part of the new CODA version as well.

The Event Transfer (ET) system in {\tt CLAS} was extended by adding a new 
package to handle multi-threaded attachments to a single ET station, with 
special care taken for preserving the event order.  This method was 
successfully used by the Event Builder, Level-3, and the fast online 
reconstruction software running on an 8-CPU Sparc Server, and will be 
reused in some form in {\tt CLAS12}.

\subsection{Level-3 Trigger}
\label{daq_l3t}

A multi-threaded Level-3 (software) trigger was implemented in {\tt CLAS}, 
but was used in tagging-mode only. We will redesign that component for 
{\tt CLAS12}, implementing full reconstruction up to at least the hit-based 
tracking level.  A single multi-core machine is used right now to run the
Level-3 trigger, and a computer farm solution can be used for {\tt CLAS12} 
if needed.

\subsection{Online Monitoring}

Online Monitoring will be addressed in {\tt CLAS12} as a part of the
Experiment Control System (ECS) concept.  This approach was used in 
previous years by other collaborations.  We will use their experience to 
design the {\tt CLAS12} ECS.

ECS will collect all information coming from various {\tt CLAS12} components
into one processing center, which will perform analysis and report problems 
with maximum possible precision.  To make the whole system work successfully,
all information from the low-level components will be reported in standard 
form, EPICS-channel access format is currently being considered.  Special 
care will be taken to avoid false alarms.

The generic scheme for the {\tt CLAS12} Experiment Control System is shown 
in Fig.~\ref{fig:MON1}.  Low-level hardware and data quality control 
components will communicate with top-level state machines using EPICS channel 
access and/or similar protocol.  The data acquisition components will have 
a DAQ-specific communication system.  Most state machines will be 
detector-specific, with one central component supervising the entire 
{\tt CLAS12} setup. A set of logic diagrams will be developed to handle 
different situations during {\tt CLAS12} experiments, such as production 
data taking, calibration runs, test runs, troubleshooting, recovery, etc. 
These diagrams will be processed by the state machines.

%%%%%%%%%%%%%%%%%%%%%%%%%%%%%%%%%%%%%%%%%%%%%%%%%%%%%%%%%%%%%%%%%%%%%%
\begin{figure}[ht]
\vspace{100mm}
\special{psfile=CLAS12_MON_1.eps hscale=64 vscale=64 hoffset=0 voffset=0}
%%\mbox{
%%\begin{picture}(-10,-10)(350,0)
%%\end{picture}}
\caption{\small{Generic scheme of the {\tt CLAS12} Experiment Control System.}}
\label{fig:MON1} 
\end{figure}
%%%%%%%%%%%%%%%%%%%%%%%%%%%%%%%%%%%%%%%%%%%%%%%%%%%%%%%%%%%%%%%%%%%%%%

The Experiment Control System for {\tt CLAS12} is currently under development.
We are looking into several frameworks that can be used to build the
{\tt CLAS12} ECS, such as CERN PVSS (used by all major CERN experiments), 
AFECS (under development at JLab), and others.

It may take a while to complete the design because of the dependency on 
the accelerator's control system, which we have to be compatible with on 
some level.  Several approaches are being considering by other JLab groups, 
and we are planning to participate in joint efforts to obtain the system
most suitable for {\tt CLAS12}. 

\subsection{Online Calibration}

The main goal of the online calibration system is to deliver the calibration 
parameters for the trigger components.  Normally, special runs will be taken 
and analyzed online before production data taking starts.  Those runs will be 
processed and analyzed by software included into the Level-3 trigger 
component running in tagging mode.
