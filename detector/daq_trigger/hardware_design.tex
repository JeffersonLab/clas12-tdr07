\section{Hardware Design}

\subsection{Overview}

The generic scheme for the {\tt CLAS12} DAQ system hardware is shown in 
Fig.~\ref{fig:HARDWARE1}.  This system will be described in detail in 
the following sections.

%%%%%%%%%%%%%%%%%%%%%%%%%%%%%%%%%%%%%%%%%%%%%%%%%%%%%%%%%%%%%%%%%%%%%%
\begin{figure}[ht]
\vspace{130mm}
\special{psfile=CLAS12_HARDWARE_1.eps hscale=60 vscale=60 hoffset=0 
voffset=0}
\caption{\small{Generic scheme for the {\tt CLAS12} DAQ system hardware.}}
\label{fig:HARDWARE1} 
\end{figure}
%%%%%%%%%%%%%%%%%%%%%%%%%%%%%%%%%%%%%%%%%%%%%%%%%%%%%%%%%%%%%%%%%%%%%%

\subsection{Front-End for Fast Detectors}

{\tt CLAS12} will have two calorimeters, two time-of-flight systems, and 
two {\v C}erenkov counters with more than 3700 total PMT channels. Every 
PMT will be connected to a Flash ADC and to a TDC through a discriminator 
(see Figs.~\ref{fig:TDC1} and \ref{fig:ADC1}).  Splitter panels will be used 
to deliver signals to both the ADC and TDC subsystems.  Four VME64X and 
nineteen VXS crates will be used to accommodate the 56 TDC and 244 ADC boards. 
Sixteen VME crates will contain the 262 discriminator boards.

The 16-channel JLab-made discriminators have a controllable threshold and 
output signal width.  They have built-in scalers, and every VME crate with
discriminators will have a readout controller (ROC), so those scalers can be 
read out.  The discriminators were successfully used in the {\tt CLAS} 
tagger system.  A new revision of the discriminators for {\tt CLAS12} will 
have a test input for calibration purposes and a channel mask.  This is
currently under development.

In {\tt CLAS12}, 128-channel 85~ps resolution and 32-channel 35~ps resolution 
pipeline TDCs will be used.  The first channel of every board will be 
connected to a reference signal through a discriminator to provide a timing 
reference.  The {\tt CLAS} system has been using these TDC boards for several 
years.  A recent firmware version for those TDCs has reached 160~MB/s
readout speed due to advanced 2eSST data transfer protocols.

%%%%%%%%%%%%%%%%%%%%%%%%%%%%%%%%%%%%%%%%%%%%%%%%%%%%%%%%%%%%%%%%%%%%%%
\begin{figure}[ht]
\vspace{130mm}
\special{psfile=CLAS12_TDC_1.eps hscale=62 vscale=62 hoffset=10 voffset=10}
\caption{\small{Planned scheme for the TDC readout in {\tt CLAS12}.}}
\label{fig:TDC1} 
\end{figure}
%%%%%%%%%%%%%%%%%%%%%%%%%%%%%%%%%%%%%%%%%%%%%%%%%%%%%%%%%%%%%%%%%%%%%%

The 16-channel JLab-made Flash ADCs are still under development and were 
not available for testing in {\tt CLAS} at the time of writing of this
document.  However, based on our tests with commercially available Flash 
ADC boards (Struck SIS3320, see Refs.~\cite{pozd,smith}), {\tt CLAS12} can 
be equipped with 12-bit 200~MHz boards. We are expecting that the JLab 
FADCs will satisfy those requirements.

The FADCs are a significant part of the trigger system and will send
prompt information to the Cluster Finder board installed in the same
crate. The cluster information will include energy for every channel above 
threshold.  A threshold will be applied to the energy sum of several 
consecutive hits.  Information will be sent to the Cluster Finder over the 
backplane serial bus every 16~ns.  The Cluster Finder board is described 
in Section~\ref{l1trig}, along with the energy summing algorithm.

%%%%%%%%%%%%%%%%%%%%%%%%%%%%%%%%%%%%%%%%%%%%%%%%%%%%%%%%%%%%%%%%%%%%%%
\begin{figure}[ht]
\vspace{130mm}
\special{psfile=CLAS12_ADC_1.eps hscale=60 vscale=60 hoffset=0 voffset=0}
\caption{\small{Planned scheme for the ADC readout in {\tt CLAS12}.}}
\label{fig:ADC1} 
\end{figure}
%%%%%%%%%%%%%%%%%%%%%%%%%%%%%%%%%%%%%%%%%%%%%%%%%%%%%%%%%%%%%%%%%%%%%%

\subsection{Front-End for the Drift Chambers}

The existing {\tt CLAS} preamplifier hybrid boards will be re-used for
sense wire signal amplification.  The existing MUX boards will be redesigned 
and the ADB crates will be equipped with new backplanes.  The ADB crates will 
be reused, probably with new power supplies installed.  Fig.~\ref{fig:ADB1} 
shows the new ADB electronics layout that is planned.  One ADB crate will have 
14 96-input preamplifier boards serving a complete drift chamber region, with 7 
boards per superlayer (112$\times$6 wires = 96$\times$7 channels).  The new MUX 
boards will send data to the new Segment Finder using a 96-wire bus and ECL
outputs to the TDCs after a 2-to-1 multiplexing.  The ADB crate backplane has 
two 96-line buses, one per superlayer, ending in the middle of crate, where the 
two-unit-wide Segment Finder is installed.  Communication between the various 
components is described in Section~\ref{l1trig}.

%%%%%%%%%%%%%%%%%%%%%%%%%%%%%%%%%%%%%%%%%%%%%%%%%%%%%%%%%%%%%%%%%%%%%%
\begin{figure}[ht]
\vspace{12.0cm}
\special{psfile=CLAS12_ADB_1.eps hscale=62 vscale=64 hoffset=15 voffset=0}
\caption{\small{Planned layout of an ADB crate for {\tt CLAS12}.}}
\label{fig:ADB1} 
\end{figure}
%%%%%%%%%%%%%%%%%%%%%%%%%%%%%%%%%%%%%%%%%%%%%%%%%%%%%%%%%%%%%%%%%%%%%%

18 ADB crates will be connected to 9 FASTBUS crates containing 14 LeCroy 
1877 TDC boards each.  The FASTBUS crates will be upgraded with new Wiener 
power supplies that include remote control.  In the future, the FASTBUS
electronics can be replaced by 6 VME64X crates containing 16 v1190
CAEN TDC boards each (see Fig.~\ref{fig:DC1}).

%%%%%%%%%%%%%%%%%%%%%%%%%%%%%%%%%%%%%%%%%%%%%%%%%%%%%%%%%%%%%%%%%%%%%%
\begin{figure}[ht]
\vspace{13.0cm}
\special{psfile=CLAS12_DC_1.eps hscale=64 vscale=64 hoffset=15 voffset=5}
\caption{\small{Schematic layout of the planned readout configuration
for the {\tt CLAS12} drift chambers.}}
\label{fig:DC1} 
\end{figure}
%%%%%%%%%%%%%%%%%%%%%%%%%%%%%%%%%%%%%%%%%%%%%%%%%%%%%%%%%%%%%%%%%%%%%%

In addition to aging problems, the FASTBUS TDCs may set another limitation 
for the {\tt CLAS12} DAQ system.  Currently they are programmed to accept 
up to 4 hits during a 2~$\mu$s interval, which corresponds to the maximum 
drift time in the Region~3 {\tt CLAS} drift chamber. The {\tt CLAS12} drift 
chambers will have similar drift times, so the TDC window will be set to similar 
values as with the current {\tt CLAS} configuration.  Unfortunately, with the 
FASTBUS 1877 TDCs, we can only program the window size against a common stop 
signal.  This signal is derived from the Level-1 trigger signal, and can be 
delayed a few microseconds to the TDC inputs.  In that situation, we have to open 
the TDC window by a few more microseconds, which may lead to more background 
hits.  In that case, we may need to allow 8 or 16 hits per channel to make 
sure that we are not losing useful hits.  That may increase the TDC conversion 
time (which is hit-dependent) and the data rate from the FASTBUS crates, which
would lead to a decrease in the overall DAQ performance.  Our backup VME-based 
solution will be activated if the FASTBUS electronics will not be usable for 
any reason.

\subsection{{\tt CLAS12} Trigger}

{\tt CLAS12} will have a three-level trigger system.  The Level-3 trigger 
is a software-based component and will be discussed in Section~\ref{sec:software}.  
The following sections contain a detailed description of the hardware-based 
Level-1 and Level-2 trigger components.

\subsubsection{Trigger Efficiency Study}

As a first step in the {\tt CLAS12} trigger system design, we studied the
efficiency of the current {\tt CLAS} trigger system.  In particular, we
studied its efficiency for electrons, which is important for most of the
physics program.  Detailed results can be found in Ref.~\cite{mikh}.
We found that by increasing the {\v C}erenkov counter pretrigger 
discriminator threshold up to 2 photoelectrons, we can make the electron 
trigger much more selective.  However, {\tt CLAS} has never run with that 
threshold because of the possibility of losing electrons with low energy 
deposition in the {\v C}erenkov counter.  {\tt CLAS12} will have two 
{\v C}erenkov counters, so even with a lower threshold, it is expected that
electron selection will be improved.  We also found that many non-electron 
events can be rejected by doing pattern recognition in the drift chambers 
and matching tracks with the fast detectors.  The new {\tt CLAS12} trigger 
system was designed to be flexible enough to exercise different approaches, 
depending on the particular run requirements.

\subsubsection{Trigger System Timing and Layout}

The maximum trigger decision time is currently set to 3~$\mu$s for Level-1
and 4~$\mu$s for Level-2.  The first number is defined by the SVT readout
chip, and the sum of both times must be less then 8~$\mu$s, which is defined 
by the FADC units.  Our preliminary FPGA algorithm development shows that 
3~$\mu$s for Level-1 may not be enough, for example, to complete the cluster 
finding in the calorimeters using energy corrections.  Those time constraints 
can be mitigated by using newer SVT readout chips and bigger FPGAs in the FADC
boards.

It is important to make sure that logically completed parts of the detector 
will not be wired to the several FPGAs inside the Cluster Finder boards, which 
will make cluster reconstruction much more difficult.  The current Cluster
Finder board design contains two FPGAs, which collect information from 8 FADC 
modules each.  That layout will satisfy all possible Level-1 algorithms.

\subsubsection{Level-1 Trigger}
\label{l1trig}

The Level-1 trigger system is based on fast detectors equipped with Flash
ADCs.  The first stage components of the trigger hardware are incorporated 
into the Data Acquisition elements (VXS crates), while the final decision is 
made in a specialized trigger crate containing Level-1 Subsystem
Processors. The same crate also contains Level-2 Road Finders and
Level-1/Level-2 matching logic incorporated into the Global Trigger
Processor, as well as Trigger Supervisor boards as 
shown in Fig.~\ref{fig:TRIGGER1}.  The system is driven by a 16~ns global
clock with an internal FPGA 4~ns clock.

%%%%%%%%%%%%%%%%%%%%%%%%%%%%%%%%%%%%%%%%%%%%%%%%%%%%%%%%%%%%%%%%%%%%%%
\begin{figure}[ht]
\vspace{145mm}
\special{psfile=CLAS12_TRIGGER.eps hscale=55 vscale=55 hoffset=0 voffset=0}
\caption{\small{Trigger system logic diagram for {\tt CLAS12}.}}
\label{fig:TRIGGER1} 
\end{figure}
%%%%%%%%%%%%%%%%%%%%%%%%%%%%%%%%%%%%%%%%%%%%%%%%%%%%%%%%%%%%%%%%%%%%%%

The VXS crates shown in Fig.~\ref{fig:ADC1} contain two trigger system
components: processing units on the FADC boards (Fig.~\ref{fig:TRIGGER_FADC})
and Cluster Finder (CF) boards that collect data from the Flash ADCs over fast 
serial lines.  The CF boards search for clusters in the electromagnetic 
calorimeters and other detectors, and send the results to the following 
stage of the trigger system (see Fig.~\ref{fig:TRIGGER_CF}).  A signal 
processing algorithm running in the FADC board processing unit is shown in 
Fig.~\ref{fig:TRIGGER_FADC_1}.

%%%%%%%%%%%%%%%%%%%%%%%%%%%%%%%%%%%%%%%%%%%%%%%%%%%%%%%%%%%%%%%%%%%%%%
\begin{figure}[ht]
\vspace{12.0cm}
\special{psfile=CLAS12_TRIGGER_FADC.eps hscale=62 vscale=64 hoffset=0 voffset=5}
\caption{\small{Processing unit on the FADC board.}}
\label{fig:TRIGGER_FADC} 
\end{figure}
%%%%%%%%%%%%%%%%%%%%%%%%%%%%%%%%%%%%%%%%%%%%%%%%%%%%%%%%%%%%%%%%%%%%%%

%%%%%%%%%%%%%%%%%%%%%%%%%%%%%%%%%%%%%%%%%%%%%%%%%%%%%%%%%%%%%%%%%%%%%%
\begin{figure}[ht]
\vspace{10.0cm}
\special{psfile=CLAS12_TRIGGER_CF1.eps hscale=20 vscale=25 hoffset=15 voffset=0}
\caption{\small{Cluster Finding diagram on the Cluster Finder board.}}
\label{fig:TRIGGER_CF} 
\end{figure}
%%%%%%%%%%%%%%%%%%%%%%%%%%%%%%%%%%%%%%%%%%%%%%%%%%%%%%%%%%%%%%%%%%%%%%

%%%%%%%%%%%%%%%%%%%%%%%%%%%%%%%%%%%%%%%%%%%%%%%%%%%%%%%%%%%%%%%%%%%%%%
\begin{figure}[ht]
\vspace{10.0cm}
\special{psfile=CLAS12_TRIGGER_FADC_1.eps hscale=60 vscale=60 hoffset=15 voffset=0}
\caption{\small{Signal processing algorithm on the FADC board.}}
\label{fig:TRIGGER_FADC_1} 
\end{figure}
%%%%%%%%%%%%%%%%%%%%%%%%%%%%%%%%%%%%%%%%%%%%%%%%%%%%%%%%%%%%%%%%%%%%%%

\subsubsection{Level-2 Trigger}

The ADB crates shown in Fig.~\ref{fig:ADB1} contain the Segment Finder 
(SF) boards that collect data from the set of drift chamber MUX boards
corresponding to an entire drift chamber region. The SF boards search for
track segments and send them onto the following stage.  The MUXs send data 
with a 15~ns strobe, so in a 105~ns interval, all data from both superlayers
is injected into a programmable delay (see Fig.~\ref{fig:TRIGGER2}).  The 
Level-1-driven switch and latch units accumulates the drift chamber hits 
during a 2~$\mu$s interval and passes the result to the Segment Finder board 
for processing.

The Segment Finder searches for segments in both superlayers simultaneously, 
then performs a region-based segment search, and finally builds a list of 
region-based segments.  The system works as a pipeline and sends sets of 
segments to the Level-2 Road Finders. Every segment occupies 24 bits: 7 bits for 
wire numbers in $U$ and 5 bits for the $U$-$V$ difference for the first and 
last wires in the region-based segment.

Six Level-2 Road Finder boards are installed in the trigger crate along with 
six Level-1 Subsystem Processor boards, a Global Trigger Processor (GTP) board, 
and a Trigger Supervisor (TS) board. The system collects all information from 
the fast detectors and drift chambers, along with the information from the 
central detectors.  It performs drift chamber-based track finding and matches 
the tracks with clusters from the fast detectors.  The Global Trigger Processor 
collects the data from the Level-1 Subsystem Processors and Level-2 Road Finders 
and produces the Level-2 trigger signals (see Fig.~\ref{fig:TRIGGER1}).

The Trigger Supervisor generates the Level-1/Level-2 pass/fail and other 
signals, and controls the entire DAQ system readout through the Trigger 
Interface units.  The Trigger Interface (TI) units installed in every 
crate participate in the readout process.

\subsubsection{Level-1/Level-2 Timing Diagram}

The generic scheme for the Level-1/Level-2 trigger timing is shown in 
Fig.~\ref{fig:TRIGGER2}.  The entire trigger system can be considered as a
three-stage process.

The first stage is the Level-1 sector-based trigger, which includes the FADC 
on-board processing (T1), data transfer over the VXS backplane serial bus (T2), 
cluster finding (T3), data transfer to the trigger crate (T4), sector-based 
Level-1 decision (T5), and data transfer to the following trigger stages.  
During that period of time (T1+T2+T3+T4+T5+T6+T7), the drift chamber signals are 
delayed by a programmable delay on the MUX boards in the ADB crates.  The T3 and 
T5 time intervals can be different for different run periods, so the programmable 
delay must be reprogrammed accordingly.

The second stage starts when the Level-1 signal arrives at the ADB crate, 
initiating the drift chamber hit-latching process (T8).  The typical T8 
timing is about 2~$\mu$s and is defined by the maximum drift time.  Three 
latch units will be used for every channel, so up to three Level-1 triggers 
can be processed in parallel.  If a fourth Level-1 signal arrives while 
the first one is still latching, the whole trigger system will be stopped, 
however, such events will be extremely rare based on the expected trigger
rate.  After the latching process is finished, the latched hits will be 
transferred to the Segment Finder (T9), processed there (T10), and the list
of segments will be transferred to the Level-2 Road Finders (T11) installed in 
the trigger crate, and processed there (T12).  During that period of
time (T8+T9+T10+T11+T12), the Level-1 information (clusters from the
Level-1 Subsystem Processors) will be delayed by another programmable delay.

Stage three is the final Level1/Level2 trigger processing stage in the
Global Trigger Processor (T13+T14) and Trigger Supervisor (T15). The
total trigger decision time must be less than 7~$\mu$s to satisfy
the Flash ADC range.

%%%%%%%%%%%%%%%%%%%%%%%%%%%%%%%%%%%%%%%%%%%%%%%%%%%%%%%%%%%%%%%%%%%%%%
\begin{figure}[ht]
\vspace{12.0cm}
\special{psfile=CLAS12_TRIGGER_2.eps hscale=58 vscale=58 hoffset=10 voffset=0}
\caption{\small{Level-1 and Level-2 trigger timing diagram.}}
\label{fig:TRIGGER2} 
\end{figure}
%%%%%%%%%%%%%%%%%%%%%%%%%%%%%%%%%%%%%%%%%%%%%%%%%%%%%%%%%%%%%%%%%%%%%%

\subsection{Readout Controllers}

Readout Controllers (ROCs) are installed in every FASTBUS, VME, VME64X, 
and VXS crate.  The ROCs collect data from the front-end boards, process it, 
and send it to the Event Builder over the network.  Currently we employ 
mvme6100 controllers with a prpmc880 or pmc280 co-processor module.  That 
configuration is fast enough to meet the {\tt CLAS12} requirements and more 
co-processor modules can be installed if necessary.  By the time of 
{\tt CLAS12} startup, a new generation of ROCs (including multi-processor and 
multi-core units) will be available.

\subsection{Computing and Network}

A Foundry 1500 switch is currently used as the backbone of the DAQ system, 
and a similar switch will be used for {\tt CLAS12}.  The Event Builder, 
Event Recorder, and other critical DAQ components for {\tt CLAS} are 
running on 4-CPU Opteron-based servers, and that configuration is sufficient 
for {\tt CLAS12} as well.  ROCs are linked to the 1500 Foundry switch
through smaller switches with up to four GBit uplinks (see Fig.~\ref{fig:NET1}).  
The {\tt CLAS} data storage system (RAID5) is sufficient for up to a 150~MB/sec 
data rate.

All available equipment is good enough to satisfy the {\tt CLAS12} 
requirements, but most likely we will upgrade most of the components 
because of aging and maintenance considerations.  The performance of the 
link to the JLab Computer Center tape storage must be increased
up to at least 200~MB/s.

%%%%%%%%%%%%%%%%%%%%%%%%%%%%%%%%%%%%%%%%%%%%%%%%%%%%%%%%%%%%%%%%%%%%%%
\begin{figure}[ht]
\vspace{12.0cm}
\special{psfile=CLAS12_NET_1.eps hscale=64 vscale=64 hoffset=10 voffset=5}
\caption{\small{Planned scheme for {\tt CLAS12} networking.}}
\label{fig:NET1} 
\end{figure}
%%%%%%%%%%%%%%%%%%%%%%%%%%%%%%%%%%%%%%%%%%%%%%%%%%%%%%%%%%%%%%%%%%%%%%
