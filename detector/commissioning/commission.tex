\section{Checkout \& Quality Assurance Before Detector Installation} 

\subsection{Objectives}

In general, the objectives of the {\tt CLAS12} detector checkout before beam
is delivered to Hall~B are to eliminate dysfunctional or weak components
in the detector system that are instrumental for the overall operation of
the individual detector systems and can be spotted without using the
electron beam.  This will include the following aspects of the detector
systems.

After completion of the systems checkout without beam, the overall
functioning of the {\tt CLAS12} components after installation will have been
established.

\begin{itemize}

\item Checkout of the cable routing schemes of the signal readout without
high voltages or low voltages applied to the detectors.  For the
drift chamber system, this will be carried out with digital signals
injected channel-by-channel into the front-end electronics.  Similarly,
cable routing of the electromagnetic calorimeters (PCAL, EC), the forward
and central time-of-flight (FTOF, CTOF) arrays, the {\v C}erenkov
counters (LTCC, HTCC), and the silicon vertex tracker (SVT), will be
checked.  This procedure will weed out false signal cablings and
faulty signal connections.

\item The high voltage (HV) system of the drift chambers will be tested
by reading out electronic noise in the drift chambers when the high
voltage has been applied.  This will be done in a ``self-triggering'' mode,
where the discriminator thresholds of the front-end electronics are set
to sufficiently low values to allow electronics noise to pass the
threshold.  This procedure allows checking that all channels are properly
supplied with the voltage required to operate the preamplifier and
amplifier-discriminator boards (ADBs).  Similarly, the HV distribution
to the photomultiplier tube (PMT) based detectors (PCAL, EC, HTCC, LTCC,
FTOF, CTOF) will be checked by turning the HV supplied to individual
channels ON and OFF and observing the appearance and disappearance of
noise and cosmic ray signals induced in the scintillators coupled to
the PMTs.

\item Low Voltage: The functionality of the low voltage (LV) distribution
to the front-end electronics will be tested by turning the supplies to
clusters of channels ON and OFF and observing the appearance and
disappearance of related channels in the occupancy plots.  This will be
accomplished by using the {\tt CLAS12} data acquisition system run in a 
free-running mode.

\item Slow Controls: The slow controls procedures to adjust the critical
parameters of the equipment will be checked out once the full system has
been installed in the experimental hall.  It will include checkout of
all systems controlled by EPICS.

\item Trigger and Data Acquisition: Part of the {\tt CLAS12} online system
(Trigger and DAQ), as well as the response of the new FTOF and PCAL
detectors will be checked without using beam.  Cosmic rays will be used
to form a trigger from the existing EC and TOF scintillator arrays to
determine the response of the new FTOF and PCAL detectors.  This will
allow a complete checkout and calibration of the two new detector
systems that are part of the {\tt CLAS12} forward detector system.

\item Laser Calibration TOF: The forward TOF system will be instrumented
with a laser-based calibration system.  A fast laser pulse will be
injected into the center of each TOF paddle, and the response of the
scintillator measured at both ends.  This allows rapid checkout of the
functionality of all FTOF paddles, and will provide accurate timing
calibration and time-walk corrections for the PMT-discriminator-TDC
chain.

\item PCAL, EC, and FTOF systems: Checkout and calibration will proceed
with cosmic rays.  The three detector systems are supported together
on the forward carriage.  After mounting the three detector systems in
one of the six sector, the detector packages can be conveniently tested
together using cosmic rays.  The procedure will be similar to what is
currently using in CLAS for calibrating the EC and TOF systems.  A
trigger will be setup that uses the stereo readout of the inner and outer
parts of the EC to form a trigger that selects cosmic rays pointing
toward specific regions of the PCAL and FTOF.  The analysis of such
events allows a detailed mapping of the responses of these detectors
across their entire active areas.

\item Cosmic ray calibration of the drift chamber systems: The
reconstruction efficiency of the DC system can be mapped out using
a cosmic ray trigger made by a coincidence of two test scintillation
counters oriented such as to sample tracks passing through the DC
system and are oriented towards the target region.  Efficiency of
single DC layers (1 out of 36) or of a single superlayer (1 out of 6) can
be mapped by excluding the layer (or superlayer) to be tested from the
fit, and measuring the hit probability in that same region by
projecting the reconstructed track into that particular region.

\item Single photon calibration for HTCC/LTCC: The standard procedure for 
calibrating the low threshold Gas {\v C}erenkov counter (LTCC) response has 
been to use the single photo-electron ``noise'' signal and adjust the PMT 
HV to equalize the output signal.  This technique uses a self-triggering 
procedure that allows calibrating all 216 channels simultaneously.  For 
the HTCC, a similar procedure will be used.  To check the sensitivity of 
the HTCC PMTs to the fringe field of the solenoid magnet, the calibration 
will be repeated with the magnetic field ramped up in steps of 1/10 of the 
maximum field.  The (possibly shifted) single photo-electron peak will be 
recorded, and then the HV will be adjusted to restore the no-field peak 
position.   

\item CTOF cosmic ray calibration: After complete assembly and integration 
into the Solenoid magnet, a single small scintillation start counter will 
be placed in the center of the CTOF barrel for triggering. A signal in the 
start counter and a hit in one of the CTOF scintillators, which indicates 
passage of a cosmic ray muon, will be used to calibrate the CTOF 
scintillators on the opposite side of the CTOF barrel.  The Solenoid will 
then be powered up in steps of 1/10 of maximum current, and the response of 
the CTOF PMTs will be measured.   

\item SVT (barrel part BST) cosmic ray calibration: After complete assembly 
and installation of the central detector system, cosmic ray triggers by the 
additional start counter in the CTOF barrel will be used to measure the 
efficiency of the SVT response and the track reconstruction, first with 
straight tracks without magnetic field, and then with the Solenoid magnet 
ramped up to 1/2 and full fields to test the accuracy of track 
reconstruction in the barrel part.  For this the track will be reconstructed 
in the 8 SVT layers in the upper hemisphere, and checked against the track 
parameters reconstructed in the 8 SVT layers of the lower hemisphere. 
Agreement within multiple scattering effects is expected.      

\item Solenoid and Torus magnet checkout: The Solenoid magnet checkout in 
Hall~B will occur after the initial quality assurance test and field 
mapping outside of the Hall.  It will consist of ramping the field to a 
moderate level to ensure integrity of the magnet after transport to the 
Hall, and the functionality of the cryogenics interfaces and controls and 
of the power supply.  The Torus magnet will be assembled in Hall~B by the 
vendor from pre-fabricated components.  After assembly, the magnet will be 
interfaced with the cryogenic system in the hall, cooled down, and ramped 
up.  A magnet quench will be initiated to verify the quench detection and 
protection system.  This will be part of the vendor's responsibility.  Once 
the magnet is accepted and in its final location, magnetic field tests will
be carried out to determine deviations from the ideal symmetry.

\end{itemize}

\section{Checkout with Beam}

\subsection{Objectives}

In general, the objectives of the Hall~B and {\tt CLAS12} commissioning 
procedure are to achieve reliable beam transport through thin and extended 
targets to the beam dump, to verify the optics design of the {\tt CLAS12} 
Torus and Solenoid magnets, and to determine the alignment and operational 
performance of the {\tt CLAS12} detector systems using beam interactions.  
Operation of the {\tt CLAS12} detector system in the presence of magnetic 
fields, and in conjunction with the trigger and data acquisition system and 
other ancillary systems will be studied as well. 

After completion of the commissioning, the functionality of the Hall~B 
instrumentation, including the beam line, the {\tt CLAS12} superconducting 
Torus and Solenoid magnetic field distribution, all {\tt CLAS12} detectors, 
the trigger and data acquisition systems, as well as the {\tt CLAS12} online 
software will have been verified.  

\subsection{Beam Line Operation}

\begin{itemize}

\item Optimize the beam transport through the hall and {\tt CLAS12} to the 
Faraday cup (FC) and/or the beam dump at various beam currents.  Determine 
the temperature rise of the FC for beam currents of 5, 10, 15, 20, 30, and
50~nA, and 6.6, 8.8, and 11~GeV beam energies.

\item Verify the accuracy, stability, and reproducibility of the beam monitors.

\item Verify beam parameters such as spot size and angle dispersion using 
the beam position monitors.

\item Establish background levels from the beam dump and/or Faraday cup for 
optimized beam parameters.  Develop optimal shielding conditions of the 
tunnel at various running conditions.

\item Calibrate the current integrator (Faraday cup) at various beam energies 
and beam currents. 

\end{itemize}

\subsection{{\tt CLAS12} Operation}

\begin{itemize}

\item Study the performance of the {\tt CLAS12} detector systems, as well 
as the trigger and data acquisition system at different beam/target 
conditions.  This includes the measurement of rates in the various detector 
systems, the FTOF, the HTCC, the PCAL, and the R1, R2, and R3 drift chamber 
systems in all six {\tt CLAS12} sectors. This will be done at different 
levels of the solenoid field (50\%, 75\%, 100\%), and at different levels 
of the Torus current (25\%, 50\%, 75\%, 100\%).   

\item Verify the detector alignments, especially the 3 regions of drift 
chambers, by operating {\tt CLAS12} without magnetic fields (both Solenoid and 
Torus magnets will be OFF) and at very low beam currents ($<$0.1~nA) to 
avoid the intense M{\o}ller background.  Using thin targets will provide a 
point-like source of charged tracks emerging from the nominal {\tt CLAS12} 
center. 

\item Calibrate the various CLAS detector systems using beam interactions in 
a thin target and with various field settings in the {\tt CLAS12} Torus and 
Solenoid magnets.  This includes establishment of trigger and detection 
efficiencies of the various detection systems.

\item Measure the timing resolution of the FTOF and CTOF systems using the 
RF beam structure for accurate information on the start time of charged 
particles. 

\item Checkout the {\tt CLAS12} calibration and online reconstruction 
software. 

\item Study drift chambers occupancies as a function of beam current and 
test the online reconstruction efficiencies. 

\item Determine the LTCC and HTCC efficiencies with high-energy electrons 
identified by large energy deposition in the EC and PCAL.  

\item Measure the rates of the FTOF and CTOF detectors for different beam 
currents and different target materials, and determine possible saturation 
effects.  

\end{itemize}


