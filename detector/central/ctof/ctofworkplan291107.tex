\documentclass[12pt]{article}
\input epsf
\usepackage{graphicx}
%
\def\beqn{\begin{eqnarray}}
\def\eeqn{\end{eqnarray}}
%
\textwidth  6.5in
\textheight 9.0in
\topmargin -0.5 in
\headheight 0.45in
\headsep 0.25in
\oddsidemargin 0in
\evensidemargin 0in
\parsep 0 in
\pagenumbering{arabic}
%\pagestyle{plain}
%
\begin{document}
%
\title{Tentative 2008 Plan for the Cooperative R\&D on the Central TOF/CAL 
Detector}

\author{F. Barbosa, V. Baturin, S. Boiarinov, V. Burkert, R. DeVita, \\
L. Elouadrhiri, C. Cuevas, M. Guidal, W. Kim, V. Kuznetsov, \\
M. MacCormick, S. Majewski, M. Mirazita, G.S. Mutchler, A. Ni, \\
S. Niccolai, S. Pereira, V. Popov, M. Ripani, J. Riso, P. Rossi, \\
A. Starostin, E. Voutier, M. Ungaro, C. Zorn}

\maketitle  

\noindent
Source file: {\small {\it /12gev/TDR/TDR2007/main/detector/central/ctof/ctofworkplan291107.tex}}

\nopagebreak

\tableofcontents
\listoftables

\newpage

\section{R\&D Overview} 

The {\tt CTOF} counter, which is being designed at JLab, will be capable
of accommodating several kinds of PMTs at the final stage of design.  Due 
to their different sensitive areas ($S_a$) and tolerated magnetic fields 
$(B_m)$, the corresponding light guides have various transmission lengths
($L_t$) and diameters ($d$).  For the time being, we consider the set of
PMTs given in Table~\ref{table1}\footnote{Unfortunately, Hamamatsu has 
recently discontinued the most appropriate fine-mesh PMT (R6504) with a 
photocathode diameter of 51~mm.}.

%%%%%%%%%%%%%%%%%%%%%%%%%%%%%%%%%%%%%%%%%%%%%%%%%%%%%%%%%%%%%%%%%%%%%%%%%
\begin{table}[htpb]
\begin{center}
\begin{tabular}{|c|c|c|c|c|c|c|}  \hline 
Hamamatsu & Type       & $d$   & $S_a$    & $L_t$  &  $B$   & Shield \\
model     &            & (mm)  & (mm$^2$) & (cm)   & (mT)   & (kg)   \\ \hline
R7761     & fine mesh  & 27    &  573     & 75     & 700    &  0  \\
R5924     & fine mesh  & 39    & 1194     & 75     & 700    &  0 \\
R6504     & fine mesh  & 51    & 2042     & 75     & 700    &  0 \\ 
R2083     & ordinary   & 46    & 1662     & 160    &  50    & $\approx 5-10$ \\
H8500     & metal channel& 49$\times$49 & 2400   & 90     & 300    & $\approx 10$ \\  \hline
\end{tabular}
\caption{List of the various PMTs considered for the central detector.}
\label{table1}
\end{center}
\end{table}
%%%%%%%%%%%%%%%%%%%%%%%%%%%%%%%%%%%%%%%%%%%%%%%%%%%%%%%%%%%%%%%%%%%%%%%%%

In order to provide sufficient information for the right choice of PMT,  
we plan to build three prototypes.  Each will consist of three counters,
six long light guides of realistic design, and six PMTs.  These projects 
are supposed to be performed in parallel.  Thus in a shorter research time  
we will have extended studies of the {\tt CTOF} time resolution in realistic 
environments.  The prototyping is also important for the JLab designers 
working on the {\tt CTOF} mechanical design. 

The R\&D projects are described briefly in the remainder of this document,
along with the measurement plans and the cooperative research plans.  

\subsection{Prototyping with 27-mm Fine-Mesh PMTs}

The prototype with the fine-mesh R7761-70 PMT is being developed at 
Kyungpook National University (KNU).  The advantage of the R7761 PMT is 
that, due to its smaller diameter, a better acceptance for {\tt CLAS12} 
may be achieved.

This prototype will be composed of three counters, trapezoidal in cross 
section, with a double-sided readout via straight 0.7 to 0.9~m light guides
of a corresponding design.  Six light guides may be produced with the 
assistance of JLab.  The optical axis of the light guides will have a 
corresponding pitch to the axis of the scintillator.

\paragraph{Current Status:}  Significant progress has been achieved in 
2006-2007.  The triplet without light guides is already in operation, and
an encouraging time resolution was obtained recently with this prototype.  
However, a systematic study of coordinate, angular, and gain dependence 
will require 10 to 20 times more statistics than have been collected to
date.

\subsection{Prototyping with Regular and Metal-Channel PMTs}

This prototype is being developed at JLab.  The main advantage of the R2083 
and H8500 PMTs is their rapid rise time of $\le$1~ns.  This system will
contain a triplet of adjacent counters, trapezoidal in cross section, with a 
double-sided readout via $\approx$1.6~m (for the R2083 PMT) and $\approx$1~m
(for the H8500 PMT) long bent light guides.  $U$-shaped light guides of a 
complex design will be coupled to the PMTs on one side and to the adjacent 
scintillators on the other.  Magnetic shielding for the R2083 PMT against
$\approx$30~mT and for the H8500 PMT against $\approx 300$~mT is required.

\paragraph{Current Status:}  Conventional R2083 PMTs were systematically 
studied at KNU in 2004-2006 with a triplet of counters instrumented with 
1~m long light guides. Encouraging time resolution results were obtained 
with high statistic under different conditions.  Another triplet without 
light guides, but with six soft-steel magnetic shields, was assembled at 
KNU in 2005.  This prototype is ready for tests in a magnetic field.  The 
initial R\&D plan\footnote{The preliminary R\&D program was published as 
CLAS-Note 2006-011.} scheduled such tests to be performed at JLab in 2007. 
Additional advantages of the H8500 PMT is that its sensitive area is wider 
and it can operate at $\approx$20--50~mT without shielding.  However, we plan 
to design and fabricate the shield against 200--300~mT in order to use such 
PMTs with shorter, $\approx$1~m long light guides.
 
\subsection{Prototyping with 39-mm Fine-Mesh PMTs} 

The triplet of counters with R5924-70 PMTs will be similar to the design 
with the R2083/H8500 PMTs, but the light guides will be straight and 
0.7 to 1~m long.  Although the diameter of the acrylic light guides will be  
39~mm, they may be easily adopted by the R2083/H8500 prototype, without 
changing its mechanical support, and tested for time resolution.  This work  
is in the tentative plan of Hall B, however, in order to save research time,  
this prototype could be sent to an outside group for assembly and testing 
as a kit of scintillators, light guides, PMTs, and supporting drawings.

\paragraph{Current Status:} The gain and resolution of different fine-mesh 
PMTs from Hamamatsu in a magnetic field was thoroughly studied by Bonesini 
{\it et~al.} in August 2006\footnote{See the {\tt CLAS12} Technical Design 
Report and the paper NIM A {\bf 567}, 200 (2006).}.  In these studies, a time
resolution of 60~ps was measured with laser flashes equivalent to 300 
photoelectrons.  Thus the measured resolution remains unchanged up to 0.6~T.
However, these results have to be verified with realistic scintillators 
using the aforementioned prototype. 

\subsection{Detector Simulation}
  
The main goal of the Monte Carlo simulations for the {\tt CTOF} is to study 
the transmittance of the light guides and scintillators as a function of 
their relevant parameters, such as length, pitch, diameter, etc.  These data 
will be used for extrapolating the prototype resolutions to the final light 
guide geometry.  Other parameters, such as machining angles, cross sections, 
refractive indexes, surface quality, and attenuation length, may be optimized 
via simulation.  Note that the effects of {\v C}erenkov light in the light 
guides and its effect on the counting rates has yet to be addressed.

\subsection{Optical Properties of Scintillators and PMTs}
 
Another important activity is to optimize the scintillators.  The attenuation 
length of acrylic seems to be significantly higher in blue wavelengths than
in green.  The idea is to find a scintillator similar to BC408 in timing,
but with greener emission spectrum, such as RP-200, BC-412, or EJ-260.
Photomultipliers with improved spectral sensitivity for green wavelengths
are available on the market.  

\subsection{Bent Scintillators}

To achieve better {\tt CTOF} acceptance, we plan to use scintillators 
that curve upwards slightly at the downstream end of the detector.  
Fabrication of such scintillators with traditional methods of machining 
and/or bending may be challenging.  We will therefore study the possibility 
for casting such scintillators against Teflon and/or glass from four sides.
For this purpose, a Teflon-coated mold with partitions should be fabricated  
in close cooperation with the relevant companies.  This may result in a 
significantly higher light transmittance of the scintillators.

\subsection{Optical Properties of Light Guides}
 
Simple methods have been developed at JLab for optical studies
\footnote{Results to be published soon as a CLAS-note.}. The attenuation 
length of different acrylics has been measured at JLab.  Different reflective 
materials will be tested in the near future, and polishing technologies 
will also be studied.  The transmittance of the final {\tt CTOF} light 
guides will be measured with these methods.

\subsection{{\tt CTOF} Test Setup at JLab}
 
During the period preceding the installation of the {\tt CTOF} detector in
the solenoid, the fully assembled and instrumented detector will be tested 
for its performance in the EEL building.  The detector will be tested
extensively using cosmic ray muons.  The time resolution of each {\em logical}
triplet will be compared with the measured reference values and optimized.  
With this purpose in mind, we plan to develop the {\tt CTOF} test setup at 
JLab. 

As a first step, we plan to assemble a data acquisition (DAQ) setup for the
prototypes described above.  We plan to optimize their time resolutions with 
the new readout electronics (TDCs, ADCs) and with various discriminators
(LEDs,CFDs) in the JLab environment.  This setup will be gradually expanded
from testing a triplet of counters to the full-scale {\tt CTOF} detector. 
After completion, this readout system will be incorporated into the global 
readout system of {\tt CLAS12}.

\section{Measurements and Criteria}
\label{mescre}

The primary goal of each project is to determine the counter time 
resolutions with realistically long light guides.  The dependencies on the 
track coordinates, angles, counting rate, and magnetic field have to be 
addressed as well.  The time resolution, averaged over the angular and 
coordinate span of the prototype, may be used for reference.

Below we describe the methods that may be implemented for the reference 
measurements of the various prototype characteristics. 

\subsection{Time Resolution vs. Track Coordinates and Angles} 

The time resolution as a function of the track coordinates and angles may 
be studied with cosmic rays.  The experience of from {\tt CLAS} confirms 
that cosmic rays provide adequate data for this purpose. 
 
Although the measurement methods with the singlet configuration (one 
double-sided readout counter) and radioactive source was developed at KNU 
in 2003, the method of tracking with triplets was later selected (in 2004)
as the main reference method\footnote{Information on the methods and the 
drawings for the singlet and triplet configurations may be found in the 
{\tt CLAS12} Technical Design Report.}.  Taking data with cosmic rays allows
for a data rate of about 0.5~tracks/s and about $10^5$ events in a few days.
 
Note that the technique of using the triplet configurations will be 
extensively used for the final {\tt CTOF} detector testing and calibration
in the assembly area.

\subsection{Time Resolution vs. Counting Rate \& Gain} 

Studies of scintillator timing resolution as a function of counting rate
and gain may be addressed with an LED pulser and a singlet configuration 
without light guides.  Such measurement methods were used at KNU in 
2003-2004.  The gain dependence has to be addressed, as well, especially for 
fine-mesh PMTs.

\subsection{Time Resolution vs. Magnetic Field}  

Studies of the timing resolution as a function of magnetic field may also
be done with an LED pulser and a singlet configuration without light guides. 
Enclosed into a pipe, such a setup may be placed into a solenoid or other 
suitable magnet available at JLab.  Drawings for such setup are available.
The real setup with six R2083 PMTs is available at KNU.

\section{{\tt CTOF} Projects in Detailed Cooperative Presentation}
\label{coopro}

In order to organize the effective cooperation below, we describe the 
projects in more detail.  A brief motivation for each project is followed 
by a table, in which the project is divided into smaller tasks.  For each 
task, the goal, the status, and the executive group are listed.  We suggest 
that small research tasks may be solved by an outside group in cooperation 
with the ``host'' group.  Some tasks may be solved with the support of JLab 
expertise and equipment.  Any group may join any task, depending on their  
research interests, manpower, equipment, funds, etc. 

\subsection{Prototype with 27-mm Fine-Mesh PMTs at KNU} 

\paragraph{Motivation:}
The cooperative plan for this project is given in Table~\ref{table2}.  The 
main advantages of the 27-mm fine-mesh PMTs are they can operate in magnetic 
fields as high as 0.6~T and their small size allows for improved acceptance
in the forward direction for {\tt CLAS12}. 

\paragraph{Current Status and Future Plans:}
The triplet with six R7761-70 PMTs directly attached to the scintillators 
is available at KNU.  An encouraging time resolution was obtained with this 
setup.  Magnetic field tests and counting rate tests are planned for 2008.
In addition to studies performed without light guides, KNU is planning the 
following:

\begin{enumerate}
 \item purchase 6 straight light guides 0.7--0.8~m long, 27~mm in diameter, 
tapered at the scintillator side;
 \item purchase from EJ 3 scintillators trapezoidal in cross section, 0.66~m 
long;
 \item assemble the triplet of counters instrumented with straight 0.7~m 
long light guides with a corresponding pitch to the scintillators;
 \item perform time resolution measurements with cosmic rays and other 
studies of counting rate capability, etc.
\end{enumerate}

\paragraph{Expected Results:} 
With this prototype we will estimate the time resolution of the {\tt CTOF} 
detector with 27-mm fine-mesh PMTs.

%%%%%%%%%%%%%%%%%%%%%%%%%%%%%%%%%%%%%%%%%%%%%%%%%%%%%%%%%%%%%%%%%%%%%%%%%
\begin{table}[htbp]
\begin{center}
\begin{tabular}{|l|l|c|l|l|r|} \hline
Component                   & Quantity  & Expenses    & Status & Exec. Groups & Comments \\ 
or Activity                 & or Goal   & (k\$)       &        &              &          \\  \hline
\multicolumn{6} {|l|} {Logistics} \\ \hline
Pyramid LGs           & 6~pc  & 2.6 & planned     & JLab/KNU          & straight \\ 
                      &       &         &             & Plastic Craft     & 0.8 mm        \\ \hline
Scint. LJ200          & 3~pc  & 0.7   & done        & KNU               & box                \\
                      &       &         &             &                   &                    \\ \hline
VM2000 film           & 1m$^2$& 0.4   & done        & KNU               & wrapping           \\ 
                      &       &         &             &                   &                    \\ \hline
Setup-3               & 1~pc  & 7.5   & in prog. & KNU               & triplet for        \\
                      &       &         &             &                   & $\sigma(z,\theta)$ \\ \hline
Setup-1               & 1~pc  & 1.0   & planned     & KNU               & singlet for        \\ 
                      &       &         &             &                   & $\sigma_t(B)$ test \\ \hline
PMT R7761-70          & 2+4~pc& 15.0  & done        & KNU               &                    \\ 
                      &       &         &             &                   &                    \\ \hline
Magnet 1 T            &       &         & in prog. & KNU               &                    \\ \hline 
DAQ  setup            &       &         & done        & KNU               &                    \\ \hline
\multicolumn{6} {|l|} {Projects} \\ \hline
Setup-1 in B-Field    &$\sigma_{R7761}(B)$ &   & planned & KNU    & $B_{max}$=600 mT           \\ \hline
Setup-1 Count rate &$\sigma_{R7761}(f)$ &   & planned & KNU    & $f_{max}$=1 MHz            \\ \hline
Setup-3 Resol.        &$\sigma_{R7761}(z,\theta)$&     & planned  & KNU    & $\theta\pm40^\circ$ \\
                      &                          &     &          &        & $z\pm25$ cm         \\ \hline
Detector Sim.         &LG trans &       & planned  & KNU      &           \\ \hline
Optical tests         &LG trans &       & planned  & JLab     &           \\ \hline
\end{tabular}
\end{center}
\caption{Research plans for the R7761-70 PMT.  Setup-1: single counter with 
two unshielded PMTs attached directly to the scintillator. Setup-3: triplet 
of counters with two PMTs coupled to the scintillator via straight 0.8~m 
long light guides without pitch.  Approximate costs are listed for guidance.}
\label{table2}
\end{table}
%%%%%%%%%%%%%%%%%%%%%%%%%%%%%%%%%%%%%%%%%%%%%%%%%%%%%%%%%%%%%%%%%%%%%%%%%

\subsection{Prototype with R2083/H8500 PMTs at JLab}

\paragraph{Motivation:}
The tentative plan for the cooperative efforts for this project is given 
in Table~\ref{table3}.  Long light guides are unavoidable elements of the 
{\tt CTOF}.  In order to achieve the desired time resolution of 50~ps, the  
transmittance of the light guides has to be improved by a factor of 
$\approx$1.5.  The improvement has to be even higher for the case of  
fine-mesh PMTs because of their smaller sensitive area.

Cast acrylic may have an almost ideal surface for a liquid.  Therefore, we 
plan to investigate this technology and study both optical and mechanical  
characteristics of the manufactured light guides, regardless of what kind of 
PMTs will be used in the final design.  The manufacturing technologies for 
more complex light guides and their technical solutions for implementation in 
the {\tt CTOF} have to be studied, as well. 

\paragraph{Goals:} Fabricate the triplet of adjacent barrel counters,  
trapezoidal in cross section.  Six light guides are to be constructed as
bent cylinders with the scintillator's ends machined as pyramids.  Mounted 
on the appropriate mechanical support, this triplet emulates a 3/50
fraction of the {\tt CTOF} barrel.  Two PMTs in this triplet will be 
shielded against external magnetic fields of $\approx$50~mT.

With this triplet we plan to perform the following studies: 

\begin{enumerate}
  \item light guide transmittance measurements;
  \item magnetic shield tests;
  \item counting rate tests;
  \item triplet resolution tests with cosmic rays and different 
        discriminators (CFD,LED);
  \item mechanical design tests.
\end{enumerate}

\paragraph{Current Status and Future Plans:} 
We have selected the manufacturer of the most transparent acrylic rods in 
the U.S., Professional Plastic Inc.  The Plastic Craft company has started  
fabrication of the light guides.  We have also determined the company that 
will fabricate the pilot magnetic shields.

\begin{enumerate}
  \item (in progress) Purchase 2 magnetic shields for the H8500/R2083 PMTs;
  \item (in progress) Purchase 2+4 H8500 PMTs from Hamamatsu;
  \item (in progress) Purchase 2+4 H2431-50 PMTs from Hamamatsu;
  \item (in progress) Purchase 2+4 H5924-70 PMTs from Hamamatsu;
  \item (20\% progress) Purchase six bent light guides of high optical 
        quality from the Plastic Craft;
  \item (20\% progress) Purchase additional cast acrylic rods from several 
        manufacturers for attenuation length measurements. Select the best 
        supplier for the Plastic Craft; 
  \item (80\% progress) Design the {\tt CTOF} test bench;
  \item (Done) Purchase 3 EJ200 scintillators shaped as trapezoids in cross 
        section;
  \item (90\% progress) Develop methods and tools for measuring the  
        transmittance and optical quality of acrylic components with 1\% 
        accuracy.
  \item (90\% progress) Measure the bulk attenuation length of acrylic rods 
        from various manufacturers;
  \item (30\% progress) Measure/improve the acrylic surface quality for 
        different polishing technologies.
\end{enumerate}

\paragraph{Expected Results:} 
We will have three pilot samples for the {\tt CTOF} counters with cast acrylic 
light guide(s) of complex shape with improved surface and transparency. 
Such light guides may be used for further resolution tests with any type of 
PMTs, including fine-mesh and metal-channel PMTs.  Resolution tests with 
conventional H8500/R2083 PMTs will provide a final estimate of the 
{\tt CTOF} timing resolution.

%%%%%%%%%%%%%%%%%%%%%%%%%%%%%%%%%%%%%%%%%%%%%%%%%%%%%%%%%%%%%%%%%%%%%%%%%
\begin{table}[htbp]
\begin{center}
\begin{tabular}{|l|l|c|l|l|r|} \hline
Component              & Quantity       & Expenses  & Status   & Groups & Comments         \\ 
or Activity            & or Goal        & (k\$)     &          &        &                  \\  \hline
\multicolumn{6} {|l|} {Logistics} \\ \hline
Pyramid LGs            & 6~pc           & 1.6       & done     & JLab,         & bent, 1.5 m long \\ 
                       &                &           &          & Plastic Craft & for R2083        \\ \hline
Pyramid LGs            & 6~pc           & 1.6       & planned  & JLab,         & 1.0m long        \\ 
                       &                &           &          & Plastic Craft & for H8500        \\ \hline
Scint. LJ200           & 3~pc           & 0.7       & done     & JLab          & trapezoid        \\ \hline
Scint.                 & 3~pc           & 0.7       & planned  & JLab          & trapezoid        \\
                       &                &           &          &               & bent             \\ \hline
VM2000                 & 1 m$^2$        & 0.38      & done     & JLab          & wrapping         \\ \hline
Magnetic shield        & 2+4~pc         & 6.0       & in prog. & JLab          & +funds           \\ \hline
PMT H2431-50           & 2+4~pc         & 14.4      & in prog. & JLab          & +funds/equipment  \\ \hline
PMT H8500              & 2+4~pc         & 14.4      & in prog. & JLab          &                  \\ \hline 
PMT R5924              & 2+4~pc         & 16.4      & in prog. & JLab          &                  \\ \hline 
Lab space              & 36 m$^2$       &           & in prog. & JLab          &                  \\ \hline
DAQ  setup             & Table~\ref{table6}&        & in prog. & JLab          & +equip.          \\ \hline
Setup-1                & 1~pc           & 1.0       & planned  & JLab          & singlet for      \\ 
                       &                &           &          &               & $\sigma_t(B)$ test\\ \hline
Setup-3                & 1~pc           & 7.5       & in prog. & JLab          & triplet for       \\
                       &                &           &          &               & $\sigma(z,\theta)$ \\ \hline
\multicolumn{6} {|l|} {Projects} \\ \hline
Mag. Shield test       & $\sigma_{R2083}(B)$ &      & planned  & JLab,         & $B_{max}$=50 mT    \\
with Setup-1           &                &           &          & KNU           &                    \\ \hline
Mag.Shield test        & $\sigma_{H8500}(B)$ &      & planned  & JLab          & $B_{max}$=400 mT   \\
 with Setup-1          &                &           &          &               &                    \\ \hline
Resolution test        & $\sigma_{R2083}(z,\theta)$ & & planned & JLab,        & $\theta\pm40^\circ$ \\
 with Setup-3          &                &           &          & KNU           & $z\pm25$ cm         \\ \hline
Resolution test        & $\sigma_{H8500}(z,\theta)$ & & planned & JLab         & $\theta\pm40^\circ$ \\
 with Setup-3          &                &           &          &               & $z\pm25$ cm        \\ \hline
Resolution test        & $\sigma_{R5924}(z,\theta)$ & & planned & JLab, KNU    & $\theta\pm40^\circ$ \\
 with Setup-3          &                &           &          &               & $z\pm25$ cm         \\ \hline
Resolution test        & $\sigma_{R5924}(f)$        & & planned & JLab, KNU    & $f\approx1$MHz      \\
 with Setup-3          &                &           &          &               & $z\pm25$ cm         \\ \hline
Det. Sim.              & LG trans.      &           & planned  &               &           \\ \hline  
Optical tests          & LG trans.      &           & in prog. & JLab          &           \\ \hline
\end{tabular}
\end{center}
\caption{Research plans for the R2083/H8500 PMTs.  Setup-1: single counter 
with two shielded PMTs attached directly to the scintillator.  Setup-3: 
triplet of counters with two PMTs coupled to the scintillator via 1.5~m long 
bent light guides. Approximate expenses are given for guidance.} 
\label{table3}
\end{table}
%%%%%%%%%%%%%%%%%%%%%%%%%%%%%%%%%%%%%%%%%%%%%%%%%%%%%%%%%%%%%%%%%%%%%%%%%

\subsection{Prototype with 39 mm Fine-Mesh PMTs} 

\paragraph{Motivation:}  The contents of this project are listed in 
Table~\ref{Table4}.  At a given distance from the light source, the 51-mm  
PMTs, due to a higher solid angle, collect 3.6 times more light than the
27-mm PMTs.  By the same reasoning, the 39-mm PMT collects twice as much light 
as the 27-mm PMT.  Therefore, their time resolutions may be 2.0 and 1.44 
times better, respectively.  

\paragraph{Current Status and Future Plans:} 
Unfortunately, Hamamatsu has discontinued its fine-mesh R6504 PMT with a 
photocathode diameter of 51~mm, and they have no plans to make any more.
However, the 39-mm PMTs are still available.  A quote on these devices
was recently received from Hamamatsu.

Although the R5924 PMT is an adequate replacement for the R6504 PMT, we 
plan the following:

\begin{enumerate}
  \item purchase 6 39-mm fine-mesh photomultipliers; 
  \item purchase from Plastic Craft 6 straight light guides, trapezoidal 
        in cross section from the scintillator side, 0.7 to 0.8~m long, and
        39~mm diameter.
  \item purchase from EJ 3 scintillators, trapezoidal in cross section, 
        0.66~m long;
  \item assemble the triplet of counters instrumented with straight, 0.7-m  
        long light guides;
  \item perform time resolution measurements with cosmic rays and/or 
        radioactive source at KNU.
\end{enumerate}

\paragraph{Expected Results:} 
With such prototypes we will estimate the best possible time resolution of 
the {\tt CTOF} detector with fine-mesh PMTs.  The total estimated time is
2 to 3 months starting from the PMT delivery date.

%%%%%%%%%%%%%%%%%%%%%%%%%%%%%%%%%%%%%%%%%%%%%%%%%%%%%%%%%%%%%%%%%%%%%%%%%
\begin{table}[htbp]
\begin{center}
\begin{tabular}{|l|l|c|l|l|r|} \hline
Component                   & Quantity       & Expenses  & Status   & Exec. Groups  & Comments                \\ 
or Activity                 & or Goal        & (k\$)     &          &               &                 \\  \hline
\multicolumn{6} {|l|} {Logistics} \\ \hline
Pyramid LGs                 & 6~pc           & 2.6       & in prog. & JLab          & straight, 0.8 m long \\ 
                            &                &           &          & Plastic Craft & no pitch             \\ \hline
Scint. EJ200                & 3~pc           & 0.7       & done     & JLab          & trapezoid            \\ \hline
VM2000 film                 & 1 m$^2$        & 0.38      & in prog. & JLab          & wrapping             \\ \hline
PMT R5924-70                & 2+4~pc         & 15.0      & in prog. & JLab          &  + funds             \\ \hline
Lab space                   & 36 $m^2$       &           & in prog. & JLab          &                      \\ \hline
Magnet 1T                   &                &           & done     & JLab          & Solen (5T), PS (1T)  \\ 
                            &                &           &          &               & Mini Torus(?)        \\ \hline
Setup-1                     & 1~pc           & 1.0       & planned  & JLab          & singlet for          \\ 
                            &                &           &          &               & $\sigma_t(B)$ test   \\ \hline
Setup-3                     & 1~pc           & 7.5       & in prog. & JLab          & triplet for          \\
                            &                &           &          &               & $\sigma(z,\theta)$   \\ \hline
\multicolumn{6} {|l|} {Projects} \\ \hline
Mag. Shield                 & $\sigma_{R5924}(B)$ &      & planned  & JLab          & $B_{max}$=600 mT     \\
with Setup-1                &                &           &          & KNU           &                      \\ \hline
Counting rate               & $\sigma_{R5924}(f)$ &      & planned  & JLab          & $f_{max}$=1 MHz      \\
with Setup-1                &                &           &          & KNU           &                      \\ \hline
Resolution                  & $\sigma_{R5924}(z,\theta)$ & & planned & JLab,        & $\theta\pm40^\circ$  \\
with Setup-3                &                &           &          & KNU           & $z\pm25$ cm          \\ \hline
\end{tabular}
\end{center}
\caption{Research plans for the R5924-70 PMTs.  Setup-1: single counter with 
two unshielded PMTs attached directly to the scintillator. Setup-3: triplet 
of counters with two PMTs coupled to the scintillator via straight 0.8-m long 
light guides without pitch.}
\label{Table4}
\end{table}
%%%%%%%%%%%%%%%%%%%%%%%%%%%%%%%%%%%%%%%%%%%%%%%%%%%%%%%%%%%%%%%%%%%%%%%%%

\subsection{Prototype with Silicon PMTs for Timing and Calorimetry}

\paragraph{Motivation:} 
Hamamatsu Corporation is making rapid progress in the fabrication of silicon 
photomultipliers (SiPMs), which are insensitive to magnetic fields.  The 
S10362-11 series has an active area of 1~mm$^2$ and is available in three 
pixel counts: 100, 400, and 1600.  The typical gain is $\approx$10$^6$.  The 
peak photon detection efficiency at 400~nm ranges between 25\% and 70\%, 
depending on the pixel pitch\footnote{These values were obtained via 
measuring the output current normalized to the known light power. Thus, it 
includes after pulses, cross talk, etc.}. 
  
At the current time, Hamamatsu is developing a SiPM with a 3$\times$3~mm$^2$ 
sense area, and pilot samples are available for tests.  It may be available 
in larger lots in 2008.  Such PMTs may be used for both timing and 
calorimetry.  For this purpose, a space of 120 to 130~mm in radius is 
reserved inside the {\tt CLAS12} solenoid for a future calorimeter/neutron
detector.  Also the space for the readout communication lines is reserved.  
  
Due to the very high intrinsic noise of SiPMs\footnote{A common output 
readout from the 3$\times$3~cm$^2$ scintillator requires 100 SiPMs with 100 
times higher noise rates.} the most realistic way to implement them would
be to attach a pair of such PMTs to a scintillating fiber of appropriate 
cross section (3$\times$3~mm$^2$).  Thus, the amount of light may not be 
enough for an accurate time measurement via double-sided readout.  Below we  
estimate the number of primary avalanches, $n_{pa}$, and then $\sigma_{tof}$, 
assuming that:

\begin{equation}
\sigma_{tof} \propto \frac{1}{\sqrt{n_{pa}}}.
\end{equation}

From our reference measurements we know that, in the R2083 PMT prototype with 
Bicron-408 scintillators 50$\times$3$\times$2~$m^3$ in size, the number of
 primary photoelectrons in the two PMTs is $\approx$700 for the minimum
ionizing particles (MIPs) with a 3~cm range\footnote{As follows from our 
CLAS notes, this number corresponds to $\sigma_{tof}\approx$40~ps.}.  This 
value has to be scaled down by 10 times due to a correspondingly smaller 
sensitive area, then doubled, due to the higher quantum efficiency of SiPMs
(up to 50\% according to Hamamatsu specifications).  Thus, we obtain 
$n_{pa}\approx 140$, which is an overestimated number of primary avalanches
in two SiPMs.  The corresponding TOF resolution for a double-sided readout 
yields $\sigma_{tof}\approx$90~ps.  Using 4 to 5 layers of Bicron-408 
3$\times$3mm$^2$ fibers, we restore the TOF resolution to the desired value 
of 50~ps.   

Note that these results are in agreement with a direct estimation of $n_{pa}$
via:

\begin{equation}
n_{pa}=n_{ph} \times t \times g \times a_{fib} \times g_{SiPM}=180,
\end{equation}

\noindent
where $n_{ph}$=16000 photons per cm produced in BC-408, $t$=0.3~cm is the
thickness of the fiber, $g$=0.5 is the experimental estimate for the 
light attenuation in our scintillators,
$a_{fib}=\frac{1}{2}\times(1-\cos\theta_{max})\approx0.15$ is the acceptance
of the BC-408 fiber, and $g_{SiPM}$ is the effective quantum efficiency of 
the SiPM $(\leq 0.5)$ at 420~nm. 

Thus, a time resolution of $\approx$50~ns may be achieved with 10 layers 
with a double-sided readout, provided that only the photon statistics limit 
the time resolution.

The central calorimeter/neutron detector may be a hollow, multi-layer 
cylinder with a stereo angle between neighboring layers.  One layer will 
have more than 500 3$\times$3mm$^2$ fibers.  Provided that 30 to 40 layers 
form such a detector, a time resolution below 50~ps may be achieved.  
However, up to 40000 readout channels will be required to instrument such a
detector.   
  
The alternative solution of collecting the light from the 
$\approx$30$\times$30~mm$^2$ surface of the scintillator will require a 
complex assembly of 100 SiPMs with common readout.  For such a case, the 
noise rate may be as high as $10^7$s$^{-1}$ per scintillating channel. 

Our preliminary R\&D plan consists of the following steps:

\begin{enumerate}
  \item Develop a setup of 9 fibers with 18 SiPMs;    
  \item Perform resolution tests with cosmic rays;
  \item Perform SiPM life time tests.    
\end{enumerate}

\paragraph{Current Status:} Hamamatsu has been developing the 3mm SiPMs.

\paragraph{Funding:} A listing of the costs associated with this R\&D
plan are provided in Table~\ref{table5}.

%%%%%%%%%%%%%%%%%%%%%%%%%%%%%%%%%%%%%%%%%%%%%%%%%%%%%%%%%%%%%%%%%%%%%%%%%
\begin{table}[htbp]
\begin{center}
\begin{tabular}{|l|c|r|} \hline
Part & Amount& Price (k\$)                               \\ \hline 
SiPM  3mm                   & 20 pc             &   2.0  \\ \hline
Amplifier                   & 20 ch             &   6.0  \\ \hline
Scintillating fibers        & 9  pc             &   0.2  \\ \hline
Mechanical support          & 6  pc             &   1.0  \\ \hline
\end{tabular}
\end{center}
\caption{{\tt CTOF} + calorimeter/neutron detector prototyping with SiPMs.}
\label{table5}
\end{table}
%%%%%%%%%%%%%%%%%%%%%%%%%%%%%%%%%%%%%%%%%%%%%%%%%%%%%%%%%%%%%%%%%%%%%%%%%

\subsection{Test Setup at JLab}

\paragraph{Motivation:}
The list of required components is given in Table~\ref{table6}.  This setup 
is initially intended for the tests with the R2083/H8500/R5924 PMTs.  This
setup will then be developed further from 3 counters to 50 counters in order 
to create a test station for the final {\tt CTOF} detector.  The 
assembling/testing period of about 1.5~years is required prior to the final 
installation.  We emphasize that achieving $\sigma_{TOF}$=50~ps in the 
environment of JLab is not a trivial task, even with successful tests at 
outside labs.  As a third step, this setup will be included into the 
{\tt CLAS12} DAQ system.

Therefore, it would be reasonable to invest a part of the DAQ funds 
into the test setup, keeping in mind that the most expensive equipment,
such as VME crates and modules, high voltage power supplies, and readout
boards will be utilized in {\tt CLAS12}.  In addition, we will gain 
experience with the new readout electronics.

%%%%%%%%%%%%%%%%%%%%%%%%%%%%%%%%%%%%%%%%%%%%%%%%%%%%%%%%%%%%%%%%%%%%%%%%%
\begin{table}[htbp]
\begin{center}
\begin{tabular}{|c|c|c|r|l|r|r|r|} \hline
 Type  & Part                & Qnt.& Unit  & ($\$$)  & Cmnt. & Group\\ \hline
NIM \\ \hline
1      &  crate              &  2  & pc    &         &       &\\ \hline
2      &  logic Unit         &  3  & slot  &         &       &\\ \hline
3      &  Gate Generator     &  1  & slot  &         &       &\\ \hline
4      &  scalar             &  1  & slot  &         &       &\\ \hline
5      &  ORTEC  CF8000      &  1  & slot  & 8ch CFD &       &\\ \hline 
6      &  ORTEC  935         &  2  & slot  & 4ch CFD &fast   &\\ \hline
VME \\  \hline
7      &  Crate              &  1  & pc    &         &       &\\ \hline
8      &  host computer      &  1  & pc    &         &       &\\ \hline
9      &  TDC  25ps          &  1  & slot  & avail.  &       &JLAB\\ \hline
10     &  ADC  Gated         &  1  & slot  & avail.  &       &JLAB\\ \hline
11     &  CAEN V812          &  1  & slot  & 16ch CFD&       &\\ \hline
Conn.Cables\\  \hline
12     &  LEMO  1m           & 60  & pc    &         &       &\\ \hline
13     &  RG58  BNC          &  8  & pc    &         &       &\\ \hline
14     &  SHV HV             &  8  & pc    &         &       &\\ \hline
15     &  BNC-Lemo           & 10  & pc    &         &       &\\ \hline
16     &  Lemo-Lemo, T-pc    & 10  & pc    &         &       &\\ \hline
17     &  Lemo-Lemo, I-pc    & 10  & pc    &         &       &\\ \hline 
Misc.Equip.\\ \hline
18     & CAEN PS HV 3000V    &  8  & ch    & common MF &       &JLAB\\ \hline
18     &Electr. Rack         &  3  & pc    &         & ok    &JLAB\\ \hline 
20     &DAQ computer         &  1  & pc    &         &       &JLAB\\ \hline
21     &Digital scope $\ge$250 MHz&  1  & pc    &         &       &JLAB \\ \hline
\end{tabular}
\end{center}
\caption{{\tt CTOF} minimal test setup at JLab.}
\label{table6}
\end{table}
%%%%%%%%%%%%%%%%%%%%%%%%%%%%%%%%%%%%%%%%%%%%%%%%%%%%%%%%%%%%%%%%%%%%%%%%%

\end{document}

\newpage
\section{Further Possible Studies with Micro-Channel PMTs}

\paragraph{Motivation:} Manufacturers (Burle) of the micro-channel plate 
(MCP) PMTs are making progress in the design of these devices. This may 
result in higher counting rates and higher magnetic field immunity.  Burle 
is developing 5~$\mu$m and 10~$\mu$m MCP PMTs such as the 18-mm 85104
with an increased quantum efficiency of 20-30\%.

\begin{enumerate}
  \item Develop a setup of 2 10 (5) micron MCP PMTs with on-board 
        preamplifiers.    
  \item Perform resolution/counting rate tests with 2 MCP PMTs.
\end{enumerate}

\paragraph{Current Status:} Preliminary discussions with the JLab Detector
Group have taken place and a series of initial measurements have been
carried out.  We plan to publish in NIM our recent results obtained with 
Burle 85011 PMTs.  This paper will include a description of the on-board 
preamplifier, resolution measurements, and the magnetic field tests performed 
by the Detector Group.

\paragraph{Labor:} 40 man-days for preamplifier/HV dividers by the  
Detector Group.
\paragraph{Equipment:} Electronic lab of JLab for preamplifier/HV dividers.
\paragraph{Materials:} 2 MCP PMTs and HV components for the voltage dividers.
\paragraph{Funding:} \$8000 (from Detector Group?) for 2 MCP PMTs and 
electronic components.
