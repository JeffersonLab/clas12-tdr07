\documentclass[12pt]{article}
\input epsf
\usepackage{graphicx}
%
\def\beqn{\begin{eqnarray}}
\def\eeqn{\end{eqnarray}}



\textwidth  6.in
\textheight 8.5in
\topmargin 0. in
\oddsidemargin 0in
\evensidemargin 0in
\pagenumbering{arabic}
%
\newcommand{\Thgg}{$\theta_{\gamma^*\gamma}~$}
\newcommand{\Phgg}{$\phi_{\gamma^*\gamma}~$}
\newcommand{\Epg}{$ep~\rightarrow~ep\gamma~$}
\newcommand{\Eppiz}{$ep~\rightarrow~ep\pi^o~$}
\newcommand{\Enpip}{$ep~\rightarrow~en\pi^+~$}
\newcommand{\EppiD}{$ep~\rightarrow~e\pi \Delta~$}
\newcommand{\Epeta}{$ep~\rightarrow~ep\eta~$}
\newcommand{\Epr}{$ep~\rightarrow~ep\rho~$}
\newcommand{\EpX}{$ep~\rightarrow~epX~$}
\newcommand{\EpKY}{$ep~\rightarrow~eKY~$}
\newcommand{\vEpg}{$\vec ep~\rightarrow~ep\gamma~$}
\def\gevc2{(GeV/c)$^2$}
\begin{document}
\pagestyle{plain}
%\indent
%\input{central_introduction.tex}
%\input{central_solenoid.tex}
%\input{central_tracker.tex}
%\input{central_tof.tex}
%\input{central_ec.tex}%
%\begin{thebibliography}{99}
%% \bibitem{CLAS12G}

\bibitem{BONUS} JLab E03-12, The Structure of the Free Neutron Via Spectator 
Tagging. H. Fenker, C. Keppe, S. Kuhn and W. Melnitchouk spokespersons.  

\bibitem{ZHENG}
X. Zheng {\it et al.},
Phys. Rev. Lett. {\bf 92}, 012004 (2004);
Phys. Rev.  {\bf C70} 065207 (2004).

\bibitem{VIPULI} K.V. Dharmawardane {\it et al.},
Phys.Lett. {\bf B641} 11 (2006).

\bibitem{BONUS12} JLab  E12-06-113, The Structure of the Free Neutron at 
Large x-Bjorken. S. Bueltmann, M. Christy, H. Fenker, K. Griffioen, 
C. Keppel, S. Kuhn, W. Melnitchouk, V. Tvaskis spokespersons.  

\bibitem{EG12} E12-06-109 The Longitudinal Spin Structure of the Nucleon. 
D. Crabb, A. Deur, K. Dharmawardane, T. Forest, K. Griffioen, M. Holtrop, 
S. Kuhn, Y. Prok

\bibitem{KUHL}
S.~Kuhlmann {\em et al.},
Phys. Lett. B {\bf 476}, 291 (2000).

\bibitem{MT}
W.~Melnitchouk and A.~W.~Thomas,
Phys. Lett. B {\bf 377}, 11 (1996).

\bibitem{Leader:2005ci}
  E.~Leader, A.~V.~Sidorov and D.~B.~Stamenov,
  %``Longitudinal polarized parton densities updated,''
  Phys.\ Rev.\ D {\bf 73}, 034023 (2006)
  [arXiv:hep-ph/0512114].
  %%CITATION = HEP-PH 0512114;%%

\bibitem{Dharmawardane:2006zd}
  K.~V.~Dharmawardane, S.~E.~Kuhn, P.~Bosted and Y.~Prok  [the CLAS
                  Collaboration],
  %``Measurement of the $x$- and $Q^2$-Dependence of the Asymmetry $A_1$ on the
  %Nucleon,''
  to be published in Phys. Lett., arXiv:nucl-ex/0605028.
  %%CITATION = NUCL-EX 0605028;%%


 \bibitem{EG1a}
R. Fatemi {\it et~al.} [CLAS Collaboration],
Phys. Rev. Lett. {\bf 91}, 222002 (2003),
J. Yun  {\it et~al.} [CLAS Collaboration],
Phys. Rev. C {\bf 67}, 055204 (2003).

\bibitem{rev_mom}
J.-P. Chen, A. Deur, Z.-E. Meziani, Mod. Phys. Lett. \textbf{A20}, 2745 (2005);
M. Osipenko {\it et al.}, Phys. Rev. D {\bf 71}, 054007 (2005).


\bibitem{Bj}
J. D. Bjorken, Phys. Rev. \textbf{148}, 1467 (1966).

\bibitem{GDH}
S. D. Drell and
A. C. Hearn, Phys. Rev. Lett. \textbf{16}, 908 (1966);
S. Gerasimov, Sov. J. Nucl. Phys. \textbf{2}, 430 (1966).

\bibitem{JiSR}
X. Ji and J. Osborne, J. Phys. {\bf G27} 127 (2001).

\bibitem{HERMES}
HERMES collaboration: A. Airapetian \emph{et al.}, 
Eur. Phys. J. {\bf C26}, 527 (2003).

\bibitem{E143}
E143 collaboration: K. Abe \emph{et al.},
Phys. Rev. Lett. \textbf{78}, 815 (1997); 
K. Abe \emph{et al.}, 
Phys. Rev. D {\bf58}, 112003 (1998).

\bibitem{E155}
P.L. Anthony {\it et al.,} Phys. Lett. {\bf B493}, 19 (2000).

\bibitem{AO}
V. D. Burkert and B. L. Ioffe,
Phys. Lett. \textbf{B296}, 223 (1992);
J. Exp. Theor. Phys. {\bf 78}, 619 (1994).

\bibitem{BT} N. Bianchi and E. Thomas,
Nucl. Phys. Proc. Suppl. \textbf{82}, 256 (2000).

\bibitem{BjHT} 
A. Deur {\em et al.}, Phys. Rev. Lett. {\bf 93} 212001 (2004)

\bibitem{E97110} J.P. Chen, A. Deur and F. Garibaldi, JLab experiment
E97-110

\bibitem{E03006} M. Battaglieri, A. Deur, R. De Vita and M. Ripani,
JLab experiment E03-006

\bibitem{BG}
E.~D.~Bloom and F.~J.~Gilman,
Phys. Rev. Lett. {\bf 16}, 1140 (1970);
Phys. Rev. D {\bf 4}, 2901 (1971).

\bibitem{NIC}
I.~Niculescu {\em et al.},
Phys. Rev. Lett. {\bf 85}, 1182, 1186 (2000).

\bibitem{d2n} M. Amarian {\em et al.}, 
Phys. Rev. Lett. {\bf 92} 022301 (2004)


\bibitem{BOSTED EG1b}P.E. Bosted {\it et al.},
hep-ph/0607283


\bibitem{E01-012} P. Solvigon, Contribution to the proceedings of the First 
Workshop on Quark-Hadron Duality and the Transition to pQCD. A. Fantoni, 
S. Luiti and O. Rondon ed. World Scientific, 2006.


\bibitem{ERIC}
M.~E.~Christy {\em et al.},
E94-110 Collaboration, in preparation.

\bibitem{osipenko} 
M. Osipenko {\em et al.}, Phys. Rev. {\bf D67} 09201 (2003)

\bibitem{noi} 
S. Simula and M. Osipenko, Nucl. Phys. {\bf B675} 289 (2003)

\bibitem{alpha} S. Bethke, $\alpha_S$ 2002 High-Energy Physics Int'l 
Conference in Quantum Chromodynamics, Montpellier (France) (2002)

\bibitem{GDH04} K. Helbing. Talk given at the GDH04 symposium. 
www.physics.odu.edu/GDH2004/Proceedings/Helbing.pdf

\bibitem{SLACGDH} SLAC experiment E159. P. Bosted and D. Crabb spokespersons.
www.slac.stanford.edu/exp/e159/prop.pdf

 
\bibitem{DGP}
A.~de~R\'ujula, H.~Georgi and H.~D.~Politzer,
Ann. Phys. {\bf 103}, 315 (1975).

\bibitem{DUALMODEL}
N.~Isgur, S.~Jeschonnek, W.~Melnitchouk and J.~W.~Van Orden,
Phys. Rev. D {\bf 65}, 054005 (2001).


%\end{thebibliography}
%\end{document}
\title
{
\vspace{-1.4cm}
\begin{flushright}
\normalsize{Draft; source file /u/home/baturin/DocCVS/12gev/TDR/TDR2007/main/detector/central/ctof/ctofmagshield.tex}
%\bigskip
%\bigskip
%\bigskip
%\bigskip
%\bigskip
%\bigskip
%\bigskip
%\bigskip
\nopagebreak
\end{flushright}Magnetic shielding for the CLAS12 Central TOF detector.}
\author
{
{}\\
%{}\\
%{}\\ 
%{}\\
%{}\\
%{}\\
\mbox
{J.Ball, V.Baturin, V.Burkert, D.Carman} \\
{ L.Elouadrhiri, P.Fazilleau ,D Grilli, M.Guidal,} \\
{G.S.Mutchler,} \\
{J.Riso,  A.Starostin. }
%{}   \\
%{V.Popov, C.Zorn, V.Kuznetsov(KNU), A.Nee(KNU) }   \\
%{}\\
%{}\\
\mbox 
%{}\\
%{}\\FAZILLEAU Philippe
%{}\\
%{}\\
{}\\
}
\maketitle  
\tableofcontents
\listoftables
\newpage
\section{Introduction}
 
The Central Time Of Flight detector ({\tt CTOF}) 
 is being designed at JLab with the  ordinary PMTs, coupled 
to the scintillators via $\approx1.6m$-long light guides of complex design
\footnote{We  anticipate that  at
final stage of designing   the regular PMTs may be replaced
with fine mesh PMTs with twice shorter light guides. These PMTs
 are insensitive  up to 5000Gs.}.
 At the nominal length of light guides the
 ordinary  PMTs  requires a robust magnetic shield against  of $\leq500Gs$.
The   optional metal channel PMTs H8500  may operate at 200-500Gs  without magnetic shielding.
However, with a powerful enough  shielding both kinds of PMTs could operate at significantly 
higher field. In such a case a  shorter light guides may  be used, 
 that improves  significantly the  time resolution.

%The  magnetic shield for  R2083/H8500  against
%  $\approx 30/300~mT$ respectively is  required.
%However, we anticipate that  at
% the final stage of designing  the regular PMTs may be replaced
%with fine mesh PMTs, which are insensitive to magnetic fields up to 700mT.
%  Due to their  different sense areas($S_a$) and   
%tolerated magnetic fields $(B_m)$ the corresponding
%light guides has various  transmission lengths($L_t$) and diameters.  
%The time being we consider %, as the most realistic candidates,  
%the following PMTs\footnote{Unfortunately, recently 
% Hamamatsu has discontinued   the most appropriate
%fine mesh PMT R6504 with the photo cathode diameter $51mm$}:
%However, the more efficient is the shield  
%the shorter may be light guides, thus the better may be  the time resolution.

%The main  goal of this note is to 
% estimate  a  possible shield's   dimensions and  tolerated fields
%using  simple methods suggested by different manufacturers of magnetic shields.
%  Then we  
The main goal of this note is to formulate a research plan  
for  the further   FEA  simulation  in the  inhomogeneous fringe  magnetic fields.
Using  the  results of FEA  calculations we'll  design a shield for R2083 and H8500
and 
optimize  the length of light guides for  for a  better time resolution.

The relevant characteristics of PMTs are listed below:
\begin{center}
\begin{tabular}{|c|c|c|c|c|c|c|}  \hline 
Hamamatsu& Type         &$D_{max}$& $S_a$    &  $L_{max}$  &  $B_{max}$& M        \\
model    &              & $(mm)$  & $(mm^2)$ & $(mm)$ &    $(Gs)$ &   $(Kg)$    \\ \hline
R2083    & ordinary     & 54      & 1662     & 120    &    0.1    & $\approx 5$    \\
H8500    & metal channel& 72      & 2400     & 15     &   200-500 & $\approx10$    \\ \hline
\end{tabular}
\end{center}
where $D_{max}$ is the maximum diameter, $S_a$ - sensitive area,
 $L_{max}$- maximum length to protect against the magnetic field,
$B_{max}$ - maximum tolerated field by ``naked'' PMT and $M$ is the expected 
shielding mass.
\bigskip
Most likely  R2083 PMTs will be used in the assembly H2431, enclosed into 
the $\mu$-metal shielding $0.8mm$ thick, $60mm$ in external diameter and $200mm$ long.
However, we plan to ask Hamamatsu to change the  H2431 design  in order to arrange a 
overhung of $30-60mm$ from  the photo cathode side.  It is required for a 
 more efficient shielding\footnote{The internal field drops significantly at the depth of one radius of the shield.}.
%\verb|http://www.mushield.com/design-guide.shtm|


\section{Principles for  magnetic shield design}

Two  kinds of PMT shielding are possible:
 passive and active shielding. 
Active shielding makes use of  magnetic fields 
produced by a  solenoid  around a PMT to cancel the 
fringing magnetic field inside  PMT's.
We do not plan to use active shielding 
 because in our case the  harmful filed
is  not solenoid-al. In addition such shield  will be 
 more complex and more costly.

In a passive shielding  the diamagnetism 
of superconducting cylinder could be   used 
to block the magnetic field  inside the cylinder.
Such  kind of shielding looks very complex and expensive.

In a traditional ferromagnetic PMT  shielding 
field lines are concentrated in the bulk of a 
ferromagnetic cylinder, reducing the fringing fields inside it.  
%
%Unfortunately, the ferromagnetic shield around PMTs
%  need holes in both ends. 
%These holes allow the shield to be slid onto the PMT,
%Active shielding makes use of the magnetic field 
%produced by a  coil to cancel the 
%fringing magnetic field in the PMT's area. 
%Active shielding may be  relatively light weight,
% compared to ferromagnetic shielding. 
%


\subsection{How Magnetic Shields Works}
%% estimate, roughly,  a  possible shield's   dimensions and  tolerated fields
%using  simple methods suggested by different manufacturers of magnetic shields.
%  Then we  
Let's consider a simplified  model for a PMT  magnetic shield, 
i.e. a ferromagnetic($\mu>>1$)
cylinder of outer/inner  diameters $D_+/D_-$ , thickness 
$t=\frac{1}{2}(D_+-D_-)$ and finite length $L\approx 4D_+$, which is 
 placed into a uniform axial  field $B_o$. Assume that the last  is created
 inside an infinitely long  solenoid of inner diameter  $2D_+$.
The conservation of the magnetic flux in such system implicates:
 \begin{equation}
4B_o \pi D_+^2 \approx B_{in} \pi D_-^2 + B_m \pi (D_+^2-D_-^2)
\label{eq004}
\end{equation}
Neglecting $B_{in}$ compared $B_o$ one can  
estimate, roughly,  the field in the media, $B_m$, and inside the shield, $B_{in}$, as follows:
 \begin{equation}
B_m  \approx B_0\frac{D+}{t}=\mu(B_m)B_{in}
\label{eq003}
\end{equation}

The problem of the  infinite  hollow ferromagnetic cylinders in the uniform 
transverse magnetic fields 
may be solved analytically, as well as the problem  for 
 ellipsoids in axial fields.
  Therefore,  
the following formulas\footnote{
Recommended by  the Magnetic Shield 
 Corporation.\\\verb |http://www.magnetic-shield.com/dynamics/works.html} 
 are available for preliminary
 estimations of the magnetic shield efficiency
\begin{equation}
S=\frac{B_o}{B_{in}}\approx \mu({B_m})\frac{t}{D_+}~~~;~~~B_m\approx B_o\frac{D_+}{0.8t}~~~;~~~t\approx B_o\frac{D_+}{0.8B_m}
\label{eq000}
\end{equation}
%
where $S$ stands for the shielding factor of a cylinder,\\$\mu(B_m)$ - the permeability in function 
on the field in the shielding material $B_m$
 $B_o$ - the external  field,\\ $B_{in}$ - the 
 field inside ferromagnetic,\\
$t$ - the thickness of a shielding material and \\
$D_+$ - the external diameter of the cylinder.\\ 
%This  formula is 
% mostly  applicable for a  transverse field. 
%
For a multilayer shielding of $n$ coaxial cylinders
the resulting shielding factor is:
\begin{equation}
S=S_1 \times S_2 \times...\times S_n
\label{eq777}
\end{equation}
%
where $S_i, i=1,...,n$ are shielding factors for $i$'s 
cylinder,  estimated via Eq.\ref{eq000}.

Although the above formulas are recommended  for rough estimations 
of the shield design at perpendicular direction 
they may be used for axial fields, provided the 
cylinder lengths exceeds  four  diameters.  

 Hence,  our shielding  will be designed  as  2-3 coaxial 
cylinders,  fabricated from  ferromagnetic  materials.
However, prior to implementing  this simplified  approach it would be useful to understand,at least qualitatively,
the dynamics of magnetic shielding. 

Ferromagnetic properties of materials may be 
reproduced assuming that  there are randomly oriented internal current loops,
 responsible for a local  magnetization.  
 The external  field
 just correlates   these  current loops along its direction.
Due to such correlation the  fields  inside   
ferromagnetic  amplifies  by  orders of magnitude.
However,  outside the ferromagnetic  it  drops, since outside the  oriented  current loops produce
 a field  of   opposite  direction.  
That is the basic  mechanism of  magnetic  shielding.

   Effectively, local currents cancel each other in the bulk
of the ferromagnetic, thus,  resulting in surface currents, only. 
%In our case of a magnetic field applied along the axis of a  
%hollow cylinder, 
Due to a cylindrical symmetry,
  the surface  currents will run in $\phi$-directions, 
in which connection
the inner surface current 
will be opposite to the outer one.
      Hence, our shield  may be approximated with  
two  thin coaxial solenoids.
The outer solenoid has the diameter $D_+$, while the inner diameter 
$D_-=D_+-t$. At that point one may forget about the ferromagnetic 
media, which is replaced now by currents in vacuum.

Due to a well known formula for a 
field at the axis $z$  of a thin  finite  solenoid,
the magnetic field created by the outer surface, $B_+(z)$,   is:
\begin{equation}
%B_+=+\mu_o j_s \frac{L}{\sqrt{L^2+D_+^2}}
B_{+}(z)=+\mu_o j_s(Cos~\alpha^+(z)-Cos~\beta^-(z)~)
\label{eq31}
\end{equation}
where $j_s$ is the absolute value of surface current density,
 $\alpha_^+(z)$ and $\beta_^+(z)$
are  the two  angles,  created by $z$-axis and two  beams,
 running from   point $z$ to the corresponding ends  of  a  cylinder.
Due to  opposite direction of  inner surface 
currents the corresponding field is opposite to that of external cylinder:
\begin{equation}
%B_-=-\mu_o j_s \frac{L}{\sqrt{L^2+D_-^2}}   
B_-(z)=-\mu_o j_s(Cos~\alpha^-(z) - Cos~\beta^-(z)~)
\label{eq32}
\end{equation}
%At  $t.D$  the angles 
%$\alpha^+\approx \alpha^-$ and  
Thus, the resulting vacuum field inside 
the   shield $B_{in}$ , may  be evaluated as
\begin{equation}
%B_{in}(z)=B_o + B_+ - B_-\approx B_o-\mu_o j_s \frac{t}{L}  
B_{in}(z)=B_o + B_+ - B_-\approx B_o- \mu_o j_s \frac{t}{D_-}g(z),
\label{eq1}
\end{equation}
where  factor 
%$g(z)\approx 2(Sin^2(\alpha_1)Cos(\alpha_1)-Sin^2(\alpha_2)Cos(\alpha_2))$%
$g(z)\approx Sin(\alpha^+ + \alpha^-)-Sin(\beta^+ + \beta^-)$.
%
In  the Maxwell equation
\begin{equation}
rot~\textbf{H} = \textbf{j} ~
\label{eqHJ}
\end{equation}
$\textbf{j}$ %=$\textbf{j_{ext}}$+$\textbf{j_s}$ 
is the summ of external $\textbf{j}_e$  and
 surface  $\textbf{j}_s$ current densities.
Integrating Eq.\ref{eqHJ} 
%along the axial direction $z$ 
over a rectangular loop, 
enclosing the inner surface element, 
%at the $center$ of the solenoid 
we find :  
\begin{equation}
H_m^z =H_{is}^z+j_s   
\label{eqBJ1}
\end{equation}
where $H_m^z$ is the field inside the ``media'', 
$H_{is}^z$ is the field at the 
inner surface of the shield ,$j_s$ is the $\phi$ component of the 
surface current density.
Therefore, neglecting   radial components, one obtains:
\newline
\begin{equation} 
B_m = B_{is}+\mu_o j_s=\mu(B_m) B_{is}\approx \mu(B_m)\kappa B_{in}  
\label{eqBJ}
\end{equation}
where $\mu(B_m)>>1$ is the permeability at $B_m$, 
$\kappa=\frac{B_{is}}{B_{in}}\leq1$ 
accounts for radial non-uniformity
 of the axial field inside  short solenoids.
 
An important rule  may  be concluded  from the  Eq.\ref{eqBJ}. 
This equation relates
the internal ferromagnetic field to the surface currents. Assume that  
a piece of ferromagnetic is placed  inside at   very close contact to the  shield's surface.
In this case  the magnetization of this piece  has to be  equal to that of  
the outer layer. Otherwise surface currents are not identical, that contradicts to charge  conservation.

Therefore, on order to provide independent functioning of inner layer( according to Eq.\ref{777}), the close  
surface contacts has to be excluded.

 
Using  Eq.\ref{eqBJ}  to exclude  the term $\mu_o \j_s$ from
Eq.\ref{eq1} we find:
%
\begin{equation}
B_{in}=B_o-B_{in}(\mu(B_m)-\kappa)\frac{tg}{D_-}
%=B_o+\mu(B_m)\frac{tg}{D_-}B_{in}  
\label{eq11}
\end{equation}
%
and neglecting $\kappa$ compared to $\mu$ we find
%
\begin{equation}
B_{o}
%=(1+\mu(B_m)\frac{tg}{D_-})B_{in}
\approx \mu(B_m)\frac{tg}{D_-} B_{in}
=\kappa^{-1}\frac{tg}{D_-}B_{m}~~~;~~~B_{m}=B_o\frac{\kappa D_-}{tg}
%=\frac{tg}{D_-} B_{m} 
\label{eq12}
\end{equation}
%
The final relation between external and internal fields is:
%
\begin{equation}
B_o \approx \mu(B_o \frac{\kappa D}{gt}) \frac{gt}{D} B_{in}
\label{eq13}
\end{equation}
%
%where $\alpha$ is the angle between the axial line and the beam running from the 
%center of the cylinder to  its  most distant point. 
At the center of a cylinder  $4D_-$ long  
  $g\approx 0.96$,   $\kappa\approx 1$. Therefore
\begin{equation}
B_o \approx \mu(B_o \frac{D}{t}) \frac{t}{D} B_{in}
\label{eqfinal}
\end{equation}
which  is  close  to the recommended  Eq.\ref{eq000}.


%Magnetic shield compny 
%alloys-perm-graph.ps.gz



\begin{figure}[htbp]%#1
\begin{center}
%\includegraphics[width=13.4cm,clip=true,bb=-100 -40 730 850]{./figures_llg/CLASTOF.ps.gz}
\includegraphics[width=16cm,clip=true,bb=0 0 500 500]{./magshieldGrilli-2.ps.gz}
\end{center}
\caption{FEA of the magnetic flux density $B$ inside the ``Russian Doll'' shield,
  composed of two cylinders.  The external field 3000Gs(300mT). 
Vertical scale is for  $B(Gs)$. Horizontal scale - transversal distance $D(in)$;
 $D=5.04$ corresponds to the axial line of  cylinders. 
The outer shield is a Netic cylinder $136mm$  
external diameter, 1'' thick and  $250mm$ long.
The second layer is a Hyperm-49 cylinder $84mm$ in outer  diameter
%1/16''
$1.6mm$ thick.
 Minimum $B$ inside the cylinder is  $\approx0.23Gs$, only.
This figure was 
 kindly presented  by David Grilli from the Mu-Shield Company. 
\label{shieldGrilli}}
\end{figure}
\clearpage




\begin{figure}[htbp]%#1
\begin{center}
%\includegraphics[width=16.4cm,clip=true,bb=-100 -40 730 850]{./figures_llg/CLASTOF.ps.gz}
\includegraphics[width=16cm,clip=true,bb= 0 0 500 700]{./Netic-conetic.ps.gz}
\end{center}
\caption{Permeability of NETIC and CO-NETIC from the Magnetic Shield Corp. 
\label{muneco}}
\end{figure}
\clearpage



\subsection{Ferromagnetic materials}
Available ferromagnetic shielding materials come 
in two types: those with high saturation and 
those with high permeability.
%, in other words, 
%there is always trade-off. 
High permeability materials are useful where the fields are small, 
therefore,  high permeability materials are to ``wrap '' PMTs. 
In high field regions, high saturation materials 
should be used.  
%A low saturation material 
%in a high field region would need to be excessively thick.

As a  high saturation ferromagnetic material
 a  silicon steel (2.25% Si, 0.40% Al, balance Fe) may be implemented.
 It has been  widely used  by  industry for relays and motors. 
%specifically Allegheny Ludlum Relay Steel #5. 
%The composition of silicon steel is 2.25% Si, 0.40% Al, balance Fe 
It  is easy to form and  has moderately high saturation  $1.56/1.96T$ 
for a  non-oriented/oriented grain makes, respectively.
The more advanced and expensive  Netic alloy saturates at $2.14T$.

The high permeability material is Conetic, 
which contains 80.6% Ni and 14% Fe and is in the same 
family metallurgically as Mu-metal.
 Conetic may be  used for the inner  shield layer around PMTs.


% about 10 cm away from the top flange of the magnet 
%in order to catch most of the residual 
%leakage field from silicon\u2013iron with 
%its high permeability.








\verb|http://www.mushield.com/design-guide.shtml|

\section{Estimations}

For our preliminary estimations of the shield parameters
  we will use both  Eq.\ref{eq000}.
% and  
% Eq.\ref{eqfinal}. 
We set the external magnetic field $B_o$ to be tolerated by  the shielding.
First we estimate the filed inside the ferromagnetic,$B_m$.
Then  we read the value of $\mu(B_m)$ from the Fig.\refn{muneco}.
Next we  determine the magnetic field inside the shield, $B_{in}$. 
This procedure will be repeated for each next 
layer using  as input $B_o=B_{in}$ from the previous
step, starting from the outer layer.
The stages  of   such estimations are listed in Table~\ref{ca001}.
In this table  the initial external field is  3000Gs. 
Although the field inside th ferromagnetic is close to saturation, after the first layer
the internal field is only $27.4Gs$. In such low field the permeability of the second 
layer of the same ferromagnetic is 3300 that results in 0.4Gs field inside 
the second layer. This rough  calculations are in good agreement with the FEA 
calculations($0.23Gs$) performed for us by D.Grilli from the Mu-shield 
company (see Fig.\ref{shieldGrilli}).  
%%%%%%%%%%%%%%%%%%%%%%%%%%%%%%
\begin{table}[htbp]
\begin{center}
\begin{tabular}{|c|c|c|c|c|c|c|c|c|c|} \hline
Cyl&$B_{o}$ & $D_+$ & $D_-$ & $t$ & $B_m$& $\approx\mu$&$S$      &$B_{in}$     & Comm.   \\ 
\# &  $Gs$  & $mm$  & $mm$  & $mm$  & $Gs$ &           &         &$Gs$         &         \\ \hline  
1 &3000     &  136  &   86  & 25    & 18462&      600  & 110     & 27.4        & Netic   \\ \hline
2 &27.4     &   84  &   80.8& 1.6   & 1797 &     3300  &  63     & 0.4         & Netic   \\ \hline
3 &0.4      &  61.6 &  60   & 0.8   &  11  &    250000 &0.006$   &$\leq0.001$  & E989-05 \\ \hline
\end{tabular}                                                      
\end{center}
\caption{         \label{ca001}}
\end{table}
\clearpage
%3&0.2       &  61.6 &  60  & 0.8   &  11  &     250000&0.006$   &  $\leq\times10^{-3}$ & E989-05 \\ \hline
%%%%%%%%%%%%%%%%%%%%%%%%%%%%%%%%%%%%%%%%%%%%%%%%%%%%%%%%%%%
\begin{table}[htbp]
\begin{center}
\begin{tabular}{|c|c|c|c|c|c|c|c|c|c|} \hline
Cyl&$B_{o}$& $D_+$ & $D_-$ & $t$  & $B_m$  & $\mu$   & $S$    &$B_{in}$         & Comm. \\ 
\#&$Gs$    & $mm$  & $mm$  & $mm$ & $Gs$   &         &        &$Gs$             &       \\ \hline  
1 &3000    &  128  &  74   & 27   &  17600 &  1000   &  214   &  14             & Netic \\ 
2 &14      &  74   &  70.8 & 1.6  &  810   &  1800   &  39    & 0.4             & Netic  \\ \hline \hline 
1&3000     &   124 &  74   & 25   & 18560  & 600     & 120    & 24.8            & Netic \\ 
2&24.8     & 73.6  &  72   & 0.8  & 2868   & 4500    & 48.6   & 0.5             & Netic \\ \hline
% 2&14     &   -   &   -   &   -  &   -    &  300000 & $10^4$ & 1.5\times10^{-3}& Conetic \\ 
% 3&0.2    &  61.6 &  60   & 0.8  &   329  &  150000 &$1935$  & 0.1\times10^{-3}& E989-05 \\ \hline
\end{tabular}                                                      
\end{center}
\caption{The H8500 shielding   against vs  $3000Gs$  at maximum   
filed inside  ferromagnetic  set $B_m=17600, 18500$. \\
$\#$ - layer number starting from the external layer, \\
$B_o$ - external field, \\
$B_m$ - fields in the ferromagnetic determined via   Eq.\ref{eq000}, \\
$t$ - ferromagnetic thickness  determined via   Eq.\ref{eq000}, \\
$\mu$ - permeability, determined from Fig.\ref{muneco},\\
$B_{in}$ -fields inside the shielding.\\
External diameter $128mm$    \label{H8500at17600} }
\end{table}
\clearpage

%%%%%%%%%%%%%%%%%%%%%%%%%%%%%%
\begin{table}[htbp]
\begin{center}
\begin{tabular}{|c|c|c|c|c|c|c|c|c|c|} \hline
Cyl&$B_{o}$&$D_+$&$D_-$&$t$   &$B_m$  &$\approx\mu$&$S$   &$B_{in}$          & Comm.    \\ 
\#&  $Gs$  &$mm$ &$mm$ &$mm$  &$Gs$   &            &      &$Gs$              &          \\ \hline  
1    & 3000& 128 & 74  & 27   & 17600 & 1000       & 214  & 14               & Netic    \\  \hline 
2    & 14  & 74  & 71  & 1.6  & 810   & 1800       & 39   & .35             & Netic    \\ \hline \hline 
1    & 3000& 115 & 66  & 24.5 & 17600 & 1000       & 213  & 14               & Netic    \\  \hline 
2    & 14  & 60  & 58.4& 0.8  & 1312  & 300000     & 4000 & .0035           & E989-05 \\ \hline \hline 
1    & 2000& 92  & 66  & 13.1 & 17600 & 1000       & 142  & 14               & Netic    \\ \hline
2    & 14  & 64  & 63.4& 0.8  & 1400  & 350000     & 4375 & .004             & Conetic  \\ \hline
2    & 14  & 64  & 63.4& 0.8  & 1400  & 3500       & 44   & .4               & Netic    \\ \hline \hline 
1    &3000 & 128 & 72  & 28   & 10900 & 4500       & 1200 & 2.5             & Netic    \\ \hline
2    &2.5  &   - &   - &   -  & 140   & 120000     & 1920 &$.8\times10^{-3}$ & Conetic \\ \hline
3    &0.2  &61.6 &  60 & 0.8  & 11    & 250000     & 6000 &$\leq\times10^{-3}& Conetic \\ \hline\hline 
1    & 1000& 86  & 66  & 9.94 & 10800 & 5000       & 872  & 1.2    & Netic    \\ \hline
2    & 1.2 & 64  & 63.4& 0.8  &  120  & 100000     & 1250 & 0.001  & E989-05  \\ \hline \hline
1    & 500 & 74.6& 66  & 4.3  & 10800 & 5000       & 250  & 2.0    & Netic    \\ \hline
2    & 2.0 & 60  & 58.4& 0.8  & 187   & 150000     & 2000 & 0.001  & E989-05  \\ \hline
\end{tabular}                                                      
\end{center}
\caption{R2083 multilayer  shielding at external fields  
$B_{o}=3000,2000,1000,500Gs$. \\
$\#$ - layer number starting from the external layer, \\
$B_m$ - field  in the ferromagnetic determined via  Eq.\ref{eq000},  \\
$\mu$ - permeability, determined from Fig.\ref{muneco},\\
$B_{in}$ -fields inside the shielding.\\
%External diameter $92mm/86$ at $2000Gs/1000Gs$\\
\label{cal2}}
\end{table}
\clearpage


%%%%%%%%%%%%%%%%%%%%%%%%%%%%%%
\begin{table}[htbp]
\begin{center}
\begin{tabular}{|c|c|c|c|c|c|c|c|c|c|} \hline
Cyl&$B_{o}       $ & $D_+$ & $D_-$ & $t$    &  $B_m$ & $\approx\mu$ &  $S$  & $B_{in}$          & Comm. \\ 
\# &      $Gs$     & $mm$  & $mm$  & $mm$   &   $Gs$ &              &       & $Gs$              &       \\ \hline  
1& 3000            & 105   &  66   &  19.8  &  20000 &     250      &  47.1 &  63.7             & Netic \\ \hline
2& 63.7            &  64   &  63.4 &  0.8   &   6370 &   80000      &  1000 &  0.064            & Conetic      \\ 
2& 63.7            &  64   &  60.8 &  1.6   &   3185 &  450000      & 11000 &  0.007            & Conetic      \\ \hline
2& 63.7            &  64   &  63.4 &  1.6   &   3185 &    4500      &  110  &  0.7              & Netic       \\
3& 0.064           &  60   &  58.4 &  0.8   &     6  & 60000        &   800 &      0            & E989-05      \\ \hline
3& 0.7             &  60   &  58.4 &  0.8   &    60  &      80000   &   1000 & 0.7\times10^{-3} & E989-05      \\ \hlin
\end{tabular}                                                      
\end{center}
\caption{R2083 shield   against of  $3000Gs$ at   ferromagnetic  $B_m=20000Gs$ ;
 $\mu(B_m)=250$. \\
$B_{o}$ - external field(s) \\
$t$   - the ferromagnetic thickness  determined via  Eq.\ref{eq000})  \\
$\mu$ - permeability, determined from Fig.\ref{muneco} \\
$B_{in}$ -fields inside the  shielding cylinder.\\
The external shield diameter $105mm$.  
.\label{ca20000}}
\end{tablemu250}
\end{document}


\begin{table}[htbp]
\begin{center}
\begin{tabular}{|c|c|c|c|c|c|c|c|c|c|} \hline
Cyl&$B_{o}       $ & $D_+$ & $D_-$ & $t$    &  $B_m$ & $\approx\mu$ &  $S$  & $B_{in}$          & Comm. \\ 
\# &      $Gs$     & $mm$  & $mm$  & $mm$   &   $Gs$ &              &       & $Gs$              &       \\ \hline  
1& 3000            & 115   &  66   &  24.5  &  17600 &   1000       &  213  &  63.7             & Netic \\ \hline
2& 63.7            &  64   &  63.4 &  0.8   &   6370 &   80000      &  1000 &  0.064            & Conetic      \\ 
2& 63.7            &  64   &  60.8 &  1.6   &   3185 &  450000      & 11000 &  0.007            & Conetic      \\ \hline
3& 0.064           &  60   &  58.4 &  0.8   &     6  & 60000        &   800 &      0            & E989-05      \\ \hline
\end{tabular}                                                      
\end{center}
\caption{R2083  shield  against of  $3000Gs$ at  maximum ferromagnetic   $B_m=17600 G$ and 
 $\mu(B_m)\approx 1000$. \\
$B_{o}$ - external field(s) for the \#'s-layer  \\
$t$   - the ferromagnetic thickness,  determined via  Eq.\ref{eq000})  \\
$\mu$ - permeability, determined from Fig.\ref{muneco} \\
$B_{in}$ -fields inside the  shielding cylinder .\\
The external shield diameter $115mm$
\label{ca17600}}
\end{tablemu250}
\end{document}



\begin{table}[htbp]
\begin{center}
\begin{tabular}{|c|c|c|c|c|c|c|c|c|c|} \hline
Cyl&$B_{o}       $ & $D_+$ & $D_-$ & $t$    &  $B_m$ & $\approx\mu$ &  $S$  & $B_{in}$          & Comm. \\ 
\# &      $Gs$     & $mm$  & $mm$  & $mm$   &   $Gs$ &              &       & $Gs$              &       \\ \hline  
1& 50              & 164.75&156.75 &     4  &  2578  &     4500     & 110   &  0.46             & Netic \\ \hline
2& 0.46            & 151.37&149.77 &  0.4   &  2185  &     400000   & 530   &  0.001            & Conetic      \\ 
%3&                 & 144.0 & 142.4 &  0.4   &   3185 &  450000      & 11000 &  0.007            &       \\ \hline
\end{tabular}                                                      
\end{center}
\caption{HTCC  shield  against of  $50Gs$.
$B_{o}$ - external field(s) for the \#'s-layer  \\
$t$   - the ferromagnetic thickness,  determined via  Eq.\ref{eq000})  \\
$\mu$ - permeability, determined from Fig.\ref{muneco} \\
$B_{in}$ -fields inside the  shielding cylinder .\\
The external shield diameter $115mm$
\label{ca17600}}
\end{tablemu250}
\end{document}

























\newpage
\section{Further  possible studies with micro channel PMTs}


\paragraph{Motivation} manufacturers($Burle$) of the micro channel 
plate PMTs are making progress
          in the design of the MCP PMs. This may result in higher 
          counting rate and higher magnetic field immunity, as well.
          $Burle$ is developing 10- an 5-micron MCP PMs such as 18mm 85104
	  with increased QE of $20-30\%$ .     
\begin{enumerate}
    
     \item
Develop a setup of 2 10(5) micron MCP PMs with on board 
          pre-amplifiers.    

     \item
perform resolution/counting rate  tests with 2 MCP PMTs
\end{enumerate}

\paragraph{Current status} preliminary publications are done  
                and  preliminary discussions  with the detector group members.
We plan  to  publish  in NIM our recent results obtained with Burle 
          85011 PMs. This paper will include the description of the 
          on-board preamplifier, resolution measurements and, perhaps ,
          magnetic field tests performed by the detector group.

\paragraph{Labor}     $40 man\times day$ for  Pre-Amp/HV dividers by the  
Detector Group.
\paragraph{Equipment} Electronic lab of JLAB for  Pre-Amp/HV dividers.
\paragraph{Materials} 2 MCP  PMTs, HV components  for voltage dividers.
\paragraph{Funding}   8000\$(from Detector Group?) for 2 MCP  PMTs and electronics 
components




\newpage
\end{document}
