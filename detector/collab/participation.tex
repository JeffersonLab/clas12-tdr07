\section{Technical Participation of Research Groups to the Hall B 12-GeV 
Upgrade}

This section lists the responsibilities and technical participation of
the outside laboratory and University group to the Hall B 12-GeV Upgrade.
Table~\ref{work_summary} provides a summary of these contributions.

\vskip 0.4cm

\noindent
$\diamondsuit$ {\bf Argonne National Laboratory} \\[0.2ex]

\noindent
The Argonne National Laboratory Medium Energy group is actively involved 
in this proposal, as well as in the color transparency proposal using 
{\tt CLAS12}.  Among {\tt CLAS12} baseline equipment, the group intends to 
take responsibility for the design, prototyping, construction, and testing 
of the high threshold {\v C}erenkov counter.  Three research staff and two 
engineers are likely to work at least part time on this project in the next 
few years.  Funding for the group is from DOE. Additional sources of funding 
will be sought as appropriate.  Beyond the baseline equipment, the group is 
also interested in exploring the possibility of building a RICH detector for 
{\tt CLAS12}.

\vskip 0.4cm

\noindent
$\diamondsuit$ {\bf Arizona State University} \\[0.2ex]

\noindent
The Arizona State University group is actively involved in this proposal.
The ASU group is planning to contribute to the development of the {\tt CLAS12} 
software and hardware with regard to the slow controls system.  One professor,
two research professors, and several graduate and undergraduate students are
likely to work at least part time on this project in the next few years.
ASU is also planning to recruit undergraduate students who can contribute.

\vskip 0.4cm

\noindent
$\diamondsuit$ {\bf University of Connecticut} \\[0.2ex]

\noindent
The University of Connecticut (UConn) group is actively involved in this 
proposal using {\tt CLAS12} baseline equipment.  Among the {\tt CLAS12} 
baseline equipment, the UConn group is sharing the responsibility for the 
design, prototyping, construction, and testing of the high threshold 
{\v C}erenkov counter (HTCC).  One faculty member (Kyungseon Joo), one 
post-doc, and four graduate students are already, or will be, working on 
this project.  The University of Connecticut Research Foundation (UCRF) 
already funded \$32,000 for the equipment purchase for the HTCC prototyping 
project.  UConn is also providing a clean room facility for this project and 
providing funding for half a post-doc and half a graduate student for the 
next two years to work on this effort.  The group is currently funded by the 
DOE. Additional sources of funding will be sought as appropriate.
Beyond the baseline equipment, the group is also deeply involved in software
planning and development for {\tt CLAS12}. The group was recently awarded
a DOE SBIR/STTR Phase I grant with a software company, CyberConnect EZ, to 
develop a software framework to archive a large-scale nuclear physics 
experiment data base.

\vskip 0.4cm

\noindent
$\diamondsuit$ {\bf DAPNIA/SPhN-Saclay, France} \\[0.2ex]

\noindent
The DAPNIA/SPhN-Saclay group is actively involved in this proposal.  Among 
{\tt CLAS12} baseline equipment, the group intends to take responsibility 
for the design, prototyping, construction, and testing of the central tracker 
(both the cylindrical part and the forward part). The group has started 
working on an option based on cylindrical Micromegas detectors. Provided this
is shown to work as designed, the group anticipates that this option will be 
examined in comparison with the Silicon Vertex Tracker toward the end of 2007 
or the beginning of 2008.  Four research staff members and four 
technicians/engineers are likely to work at least part time on this project 
in the next few years.  Funding for the group is from CEA-France.  Additional 
sources of funding (ANR-France, European Union 7th PCRD) will be sought as 
appropriate.  In case the Micromegas option is not suitable, or not selected 
for valid reasons, the group would study other technical participation in the 
{\tt CLAS12} baseline equipment.  Beyond the baseline equipment, the group is 
also interested in exploring neutral particle detection (mostly neutrons) in 
the central detector of {\tt CLAS12}, in the so far empty space between the 
CTOF scintillators and the solenoid cryostat.

\vskip 0.4cm

\noindent
$\diamondsuit$ {\bf Fairfield University} \\[0.2ex]

\noindent
Fairfield University is actively involved in the DVCS proposal, as well as 
other {\tt CLAS12} proposals.  The FU group is planning to contribute to 
the development of the {\tt CLAS12} software, such as the GEANT4 simulation
effort.  FU is also planning to recruit undergraduate students who can work 
on hardware projects over the summer at JLab or in collaboration with other 
groups.

\vskip 0.4cm

\noindent
$\diamondsuit$ {\bf Florida International University} \\[0.2ex]

\noindent
The Florida International University group is actively involved in this 
proposal.  The FIU group will take responsibility of the beamline devices,
detectors, readout electronics, targets, and software associated with the 
Hall B M{\o}ller polarimeter.  FIU is also planning to recruit graduate
and undergraduate students who can work on this project. 

\vskip 0.4cm

\noindent
$\diamondsuit$ {\bf Hampton University} \\[0.2ex]

\noindent
Hampton University (HU) is actively involved in this proposal, as well as
in the original BONUS experiment.  The HU group will continue to support 
development, augmentation, and use of the BONUS target and detector system 
for {\tt CLAS12}.  The HU nuclear experimental suite consists of over 1,400 
square feet of lab space with an electronic lab station, mechanical lab 
station, computer/graphic processing bay, and a dedicated radiation hot lab. 
The physics department, furthermore, has a 1,300 square foot class-10,000
clean room for component preparation and module construction. 
Research support for the Hampton University nuclear experimental group
comes predominantly from the NSF.

\vskip 0.4cm

\noindent
$\diamondsuit$ {\bf University of Glasgow, United Kingdom} \\[0.2ex]

\noindent
The Glasgow group plans to contribute to the design, prototyping,
construction, and testing of the following {\tt CLAS12} baseline equipment:
the silicon vertex tracker electronic readout system, data acquisition,
GRID computing, and the photon beamline. Beyond the baseline equipment, the
group is also interested in building a RICH detector for kaon identification 
in {\tt CLAS12}.  Seven faculty members and research staff of four 
technicians/engineers are likely to work at least part time on this project 
in the next few years. Funding for the group is from the UK Engineering and 
Physical Sciences Research Council, EPSRC.  Additional sources of funding 
will be sought as appropriate.

\vskip 0.4cm

\noindent
$\diamondsuit$ {\bf Idaho State University} \\[0.2ex]

\noindent
The Idaho State University group is actively involved in this proposal.
Among {\tt CLAS12} baseline equipment, the group intends to take 
responsibility for the design, prototyping, construction, and testing of 
the Region~3 Drift Chambers. Three faculty, an engineer, and several students
are likely to work at least part time on this project in the next few years. 
ISU has also provided significant high bay laboratory space with clean room 
capabilities for this project.  Detector prototype testing will utilize the
existing Idaho Accelerator Center and related infrastructure.  The group will 
seek other sources of funding as appropriate.

\vskip 0.4cm

\noindent
$\diamondsuit$ {\bf Institute for Theoretical and Experimental Physics} \\[0.2ex]

\noindent
The ITEP group is actively involved in this proposal and in the {\tt CLAS12} 
upgrade.  The ITEP team is responsible for the design, prototyping,
construction, and testing of the superconducting magnets (torus and solenoid), 
the tracking system, and the high threshold {\v C}erenkov counter (HTCC). 
Oleg Pogorelko is serving as a coordinator for the magnet design.  Alex 
Vlassov is doing the Monte Carlo program for the {\v C}erenkov counter, 
and Sergey Kuleshov will participate in the tracking system design and will 
improve the parameters of the Inner Calorimeter.  Nikolay Pivnyuk and Ivan 
Bedlinsky are committed to significant contributions in the development of 
the {\tt CLAS12} software.

\vskip 0.4cm

\noindent
$\diamondsuit$ {\bf James Madison University} \\[0.2ex]

\noindent
The James Madison University group is actively involved in this proposal.
Among {\tt CLAS12} baseline equipment, the group intends to be involved 
as a major contributor with the design, prototyping, construction, and 
testing of the pre-shower calorimeter.  Laboratory space is currently 
being set up at the university for the testing of photomultiplier tubes 
and fibers.  The group is supported by a grant from the NSF and has a strong 
undergraduate research component.

\vskip 0.4cm

\noindent
$\diamondsuit$ {\bf Kyungpook National University, Republic of Korea} \\[0.2ex]

\noindent
The Kyungpook National University group has been actively participating
in the {\tt CLAS} collaboration. The group has been contributing to the laser
calibration of the TOF system and playing a central role in developing 
the prototype for the 12-GeV central TOF detector.  The group is committed to 
design, construct, test, and maintain the 12-GeV  CTOF detector.  The group 
also intends to participate in data analysis.

\vskip 0.4cm

\noindent
$\diamondsuit$ {\bf Moscow State University} \\[0.2ex]

\noindent
The Moscow State University Group (MSU) is actively involved in development 
of {\tt CLAS12} base equipment needed for proposed experiments.  In 
particular, the MSU group will participate in development of the simulation
(GEANT4) and reconstruction software, and the trigger and data acquisition.
The MSU group takes responsibility for the maintenance and development of 
the special database needed for $N^*$ studies in coupled-channel analyses. 
This project will be developed jointly with Hall B and EBAC.  MSU personnel 
will also participate in the development of the pre-shower calorimeter, the 
HTCC, and the drift chambers under supervision of the Hall B staff.  At 
least 4 staff scientists and 5 graduate students will be involved in base 
equipment development.

\vskip 0.4cm

\noindent
$\diamondsuit$ {\bf University of New Hampshire} \\[0.2ex]

\noindent
The University of New Hampshire Nuclear Physics group is committed to 
significant contributions in the development of the {\tt CLAS12} software. 
Maurik Holtrop is currently chair of the {\tt CLAS12} GEANT4 simulation 
group and has a post-doc contributing to the effort as well.  Since currently 
the main software efforts for {\tt CLAS12} are in the area of simulation, the
UNH group is part of and contributing to the general {\tt CLAS12} software 
group.  Current manpower commitments to this effort are 0.15 FTE of a faculty 
and 0.4 FTE of one post-doc.  They expect to increase this effort as their
{\tt CLAS} activities wind down and their {\tt CLAS12} activities pick up.
They plan on attracting some talented undergraduate students to this project.
Among the {\tt CLAS12} baseline equipment, the group intends to take 
responsibility for the design, prototyping, construction, and testing of the 
silicon vertex detector and perhaps the inner detector's silicon tracking 
detectors.  Faculty member Maurik Holtrop is likely to work at least part 
time on this project in the next few years and is likely to be joined by 
Jim Connel, a cosmic ray experimentalist with a background in nuclear 
physics, who is very interested in joining the vertex detector project. He 
has considerable experience with silicon detectors for space observations. 
Funding for the group is from DOE and additional sources of funding will be 
sought for this project to bring aboard Prof. Connel.  If funded they would
like to attract a post-doc, a graduate student, and one or two undergraduate 
students to this project.  Beyond the baseline equipment, the group is also 
interested in exploring an extended inner calorimeter for {\tt CLAS12}.

\vskip 0.4cm

\noindent
$\diamondsuit$ {\bf Norfolk State University} \\[0.2ex]

\noindent
Norfolk State University is actively involved in this proposal using 
{\tt CLAS12}.  Among {\tt CLAS12} baseline equipment, the group intends to 
make important contributions to the prototyping, construction, and testing 
of the pre-shower calorimeter.  Two faculty members and several students are 
likely to work at least part time on this project in the next few years.  
Funding for the group is from NSF.  Additional sources of funding will be 
sought as appropriate. 

\vskip 0.4cm

\noindent
$\diamondsuit$ {\bf Ohio University} \\[0.2ex]

\noindent
Ohio University is actively involved in this proposal using {\tt CLAS12}. 
Among {\tt CLAS12} baseline equipment, the group intends to make important 
contributions to the prototyping, construction, and testing of the pre-shower 
calorimeter.  One faculty member and a graduate student are likely to work 
at least part time on this project in the next few years.  Funding for the
group is from NSF.  Additional sources of funding will be sought as 
appropriate. 

\vskip 0.4cm

\noindent
$\diamondsuit$ {\bf Old Dominion University} \\[0.2ex]

\noindent
The Old Dominion University group is actively involved in this proposal, as 
well as several other proposals using {\tt CLAS12}.  Among {\tt CLAS12} 
baseline equipment, the group intends to take responsibility for the design, 
prototyping, construction, and testing of the Region~1 Drift Chamber. Five 
faculty (including one research faculty) and one technician are likely to 
work at least part time on this project in the next few years. Funding for 
the group is from DOE and from the university (75\% of research faculty 
salary + one regular faculty summer salary + 50\% of the technician).
ODU has also provided 6,000 square feet of high bay laboratory space with
clean room capabilities for this project.  The group will seek other sources 
of funding as appropriate.  Gail Dodge is the chair of the {\tt CLAS12} 
Steering Committee and the user coordinator for the {\tt CLAS12} tracking 
technical working group.  Beyond the baseline equipment, the group is also 
interested in exploring improvements to the BONUS detector and a future RICH 
detector for {\tt CLAS12}.

\vskip 0.4cm

\noindent
$\diamondsuit$ {\bf Rensselaer Polytechnic Institute} \\[0.2ex]

\noindent
The RPI group is actively involved in this proposal using the {\tt CLAS12} 
base equipment.  Paul Stoler is a member of the {\tt CLAS12} Steering 
Committee.  Among the {\tt CLAS12} baseline equipment, the RPI group is 
involved in the design, prototyping, construction, and testing of the high 
threshold {\v C}erenkov counter (HTCC) and the modification of the low 
threshold {\v C}erenkov counter (LTCC).  Currently, Paul Stoler is serving 
as a coordinator of the HTCC and LTCC groups involved in the effort. 
Research scientist Valery Kubarovsky is designing and building the apparatus 
for testing the prototype components.  Two undergraduates are spending the 
summer at JLab working on prototype mirror fabrication and computer-aided 
optics design and simulation.  The group will continue to be fully involved 
as needed. The group is currently funded by NSF and RPI. Additional sources 
of funding will be sought as appropriate.

\vskip 0.4cm

\noindent
$\diamondsuit$ {\bf University of Richmond} \\[0.2ex]

\noindent
The University of Richmond group is actively involved in this proposal using 
{\tt CLAS12}.  Among {\tt CLAS12} baseline equipment, the group intends to 
take responsibility for the design, prototyping, development, and testing of 
software for event simulation and reconstruction.  One faculty member along 
with 2-3 undergraduates each year are likely to work at least part time on 
this project in the next few years.  The group has a 100-CPU computing 
cluster solely for nuclear physics supported by a linux-trained, technical 
staff member.  The cluster was funded by NSF and the University. The 
University also supports routine travel to JLab and undergraduate summer 
stipends.  Funding for the group is from DOE.  Additional sources of funding 
will be sought as appropriate.

\vskip 0.4cm

\noindent
$\diamondsuit$ {\bf University of South Carolina} \\[0.2ex]

\noindent
The University of South Carolina group is actively involved in this proposal 
using {\tt CLAS12} base equipment.  Ralf Gothe is a member of the {\tt CLAS12}
Steering Committee.  Among the {\tt CLAS12} baseline equipment, the USC 
group has taken responsibility for the design, prototyping, construction, 
and testing of the forward time-of-flight detector.  Ralf Gothe is currently 
heading FTOF technical working group.  Three USC faculty members (R. Gothe, 
S. Strauch, and D. Tedeschi), one post-doc, three graduate students, and 
two undergraduate students are already working on this project.  The USC 
nuclear physics group is committed to carry out this project and will 
continue to be fully involved as needed.  The group is currently funded by 
NSF. USC is providing a detector assembly hall for the duration of the 
project and has funded \$60,000 for the initial infrastructure needs. 
Additional sources of funding will be sought as appropriate.  Beyond the 
baseline equipment, the group has also been deeply involved in software 
planning and development for {\tt CLAS12}.

\vskip 0.4cm

\noindent
$\diamondsuit$ {\bf Temple University} \\[0.2ex]

\noindent
The Temple University group has one faculty member (Z.-E. Meziani), a 
post-doc, a research associate, and several graduate students.  The major 
source of funding for the group is DOE, however, the research associate is 
only paid half by DOE the other half of his salary is provided by the 
University.  The intended contribution is 1 FTE-year for the {\v C}erenkov 
counter installation and commissioning, as well as the analysis of the 
experiment.

\vskip 0.4cm

\noindent
$\diamondsuit$ {\bf University of Virginia} \\[0.2ex]

\noindent
The University of Virginia Polarized Target Group is actively involved in 
this proposal, as well as other proposals using {\tt CLAS12}.  The group's 
contribution to the {\tt CLAS12} baseline equipment will be the design, 
construction, and testing of the longitudinally polarized target discussed 
in this proposal.  The target will use a horizontal $^4$He evaporation 
refrigerator with a conventional design and similar to ones built and 
operated at JLab in the past.  The refrigerator will be constructed in the 
Physics Department workshop; the workshop staff have experience with 
building such devices.  Testing will be done in our lab where all the 
necessary infrastructure is on hand.  Two research professors (75\% of 
salary from UVA, 17\% from DOE), two post-docs, and two graduate students, 
all supported by DOE, will spend their time as needed on this project. 
Other funding will be pursued as necessary.  Outside the base equipment 
considerations, one member of the group (DGC) has started working with 
Oxford Instruments on a design for an optimized transverse target magnet to 
be used for transverse polarization measurements.

\vskip 0.4cm

\noindent
$\diamondsuit$ {\bf Virginia Polytechnic Institute and State University} \\[0.2ex]

\noindent
Virginia Tech is actively involved in this proposal.  The VT group is 
planning to contribute to the development of the {\tt CLAS12} software, such 
as the GEANT4 simulation effort.

\vskip 0.4cm

\noindent
$\diamondsuit$ {\bf The College of William and Mary} \\[0.2ex]

\noindent
The College of William and Mary group is actively involved in this proposal, 
as well as several other proposals using {\tt CLAS12}.  Among {\tt CLAS12} 
baseline equipment, the group is committed to building part of the forward 
tracking system, but the exact tasks have not yet been determined.  At least 
one faculty member, two graduate students, half a post-doc and several 
undergraduates are likely to work at least part time on this project in the 
next few years.  Funding for the group is from the DOE and from the NSF.
Additional funding will be sought for building the base equipment.
Facilities at William and Mary include a clean room suitable for drift-chamber
construction, and, on the time scale of a few years, ample space for detector 
construction and testing.

\vskip 0.4cm

\noindent
$\diamondsuit$ {\bf The Yerevan Physics Institute} \\[0.2ex]

\noindent
The Yerevan Physics Institute group is actively involved in this
proposal, as well as several other proposals using {\tt CLAS12}.  Among
{\tt CLAS12} baseline equipment, the group is participating in the upgrade
of the Electromagnetic Calorimeter.  This includes designing, prototyping, 
constructing, calibrating, and commissioning the EC preshower detector (under 
the supervision of S. Stepanyan).  It also includes writing and implementing 
the EC Preshower software package.  Three faculty or staff members and 
several graduate students are likely to work at least part time on this 
project in the next few years.

\newpage

%%%%%%%%%%%%%%%%%%%%%%%%%%%%%%%%%%%%%%%%%%%%%%%%%%%%%%%%%%%%%%%%%%%%%%%%%%%
\begin{table}
\begin{small}
\begin{center}
\begin{tabular}{||l||l||} \hline \hline
Institution                            & Commitment to {\tt CLAS12} \\ \hline 
Argonne National Laboratory            & High Threshold {\v C}erenkov \\ \hline 
Arizona State University               & Slow Controls \\ \hline
University of Connecticut              & High Threshold {\v C}erenkov \\ 
                                       & University Infrastructure \\ \hline
DAPNIA/SPhN-Saclay, France             & Central Tracker \\ \hline
Fairfield University                   & Software Development \\ \hline
Florida International University       & M{\o}ller Polarimeter \\ \hline
University of Glasgow, United Kingdom  & SVT readout, DAQ, GRID Computing \\ \hline
Hampton University                     & BONUS Detector \\ \hline
Idaho State University                 & Region 3 Drift Chambers \\
                                       & University Infrastructure \\ \hline
Institute for Theoretical and Experimental Physics & SC Magnets: Torus and Solenoid  \\
                                       & Software development, Inner Calorimeter \\ \hline
James Madison University               & Preshower Calorimeter \\ \hline
Kyungpook National University, Korea   & Central TOF, Laser Calibration System \\ \hline
Moscow State University                & Software Development, DAQ \\ \hline 
University of New Hampshire            & SVT, Software development \\ \hline
Norfolk State University               & Preshower Calorimeter \\ \hline
Ohio University                        & Preshower Calorimeter \\ \hline 
Old Dominion University                & Region 1 Drift Chambers \\ 
                                       & University Infrastructure \\ \hline
Rensselaer Polytechnic Institute       & High Threshold {\v C}erenkov \\ 
                                       & Modification to the Low Threshold {\v C}erenkov \\ \hline
University of Richmond                 & Software development \\ \hline 
University of South Carolina           & Forward TOF System \\ 
                                       & University Infrastructure \\ \hline 
Temple University                      & High Threshold {\v C}erenkov  \\ \hline
The College of William and Mary        & Drift Chambers \\ \hline
University of Virginia                 & Polarized Target \\ \hline
Virginia Tech                          & Software Development \\ \hline
Yerevan Physics Institute              & Preshower        \\ \hline \hline
\end{tabular}
\end{center}
\end{small}
\caption{\small{Summary table of the contributions of research groups
to the {\tt CLAS12} upgrade project.}}
\label{work_summary}
\end{table}
%%%%%%%%%%%%%%%%%%%%%%%%%%%%%%%%%%%%%%%%%%%%%%%%%%%%%%%%%%%%%%%%%%%%%%%%%%%

