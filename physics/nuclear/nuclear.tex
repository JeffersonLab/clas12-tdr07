\chapter{Properties of QCD from the Nuclear Medium}

\section{Introduction}

While the strong interaction seems to be well described by QCD at high
energies, in the non-perturbative domain it remains largely unsolved
and untested. Further, little is known from experiment about the
space-time characteristics of QCD at any energy scale. Exploration of
the QCD phase diagram for hot, dense matter is a large effort in QCD
physics, and these studies can take advantage of pQCD, however, a
QCD-based description of cold, dense matter still represents a 
formidable challenge. Lattice QCD, in combination with chiral effective 
theory, will partly address the non-perturbative physics and the hot, 
dense matter, but space-time processes and non-zero baryon density are 
still inaccessible via lattice techniques.

New experimental access to a number of properties and consequences of
QCD will become available at 12~GeV. Properties of deconfined quarks,
such as their lifetimes and energy loss, can be extracted. The time
evolution of $q\bar{q}$ pairs can be deduced from studies of color
transparency. The process in which hadrons are formed out of energetic 
quarks can be characterized, both in terms of formation times and 
formation mechanisms. The properties of strongly interacting nucleons 
and their connections to cold, dense matter can be accessed with 
precision over a broad kinematic range, potentially connecting to 
astrophysical systems such as neutron stars. These are exciting 
prospects that will provide profound new insights into the space-time 
characteristics of fundamental QCD processes and the nature of cold 
QCD matter.

\boldmath
\subsection{$p_T$ Broadening and the Lifetime of Deconfined Quarks}
\unboldmath

The confinement of quarks into hadrons is the most important
manifestation of the non-Abelian character of QCD. Achieving a
quantitative understanding of confinement is one of the highest
priority endeavors of nuclear and hadronic physics, and ranks as 
one of the great quests of modern science. The effort to understand
confinement is multi-pronged, typically involving hadron spectroscopy
interpreted through the use of models and lattice calculations, which
aim to characterize the effective potential between quarks. In
addition to the effective potential, another important piece of
confinement is the process of color neutralization, wherein a
deconfined colored quark finds colored partners, such that the
resulting system is a color singlet. Experimental access to the
characteristics of the deconfined quark can be obtained using
semi-inclusive deep inelastic scattering on nuclei in specific
kinematic regions.

In deep inelastic scattering (DIS) kinematics with $x>0.1$, quark-pair 
production by the virtual photon is suppressed~\cite{DDBH} and the target 
quark absorbs all of its energy and momentum. Thus, neglecting the 
intrinsic quark momentum, the initial quark energy is $\nu$ and its 
direction is given by the direction of the virtual photon. The quark 
propagates for some distance until its color is neutralized, at which 
point it is contained in a ``pre-hadron'' that subsequently evolves into 
a fully formed hadron~\cite{KNPH}. In the event that the neutralization 
takes place outside of a nuclear target, the interactions of the struck 
quark with the nuclear medium are limited to partonic-level multiple 
scattering, primarily due to the emission of medium-stimulated gluons
\cite{XGXNW}. This multiple scattering broadens the distribution of 
momentum transverse to the virtual photon direction, which is observable 
by measurement of the final state hadron's $p_T^2$ distribution. 
Subtracting the intrinsic quark $p_T^2$ distribution as measured in 
deuterium yields the basic observable of $p_T$ broadening: 

\begin{equation}
\Delta p_T^2=p_T^2(A)-p_T^2(^2H).
\end{equation}

One property of the deconfined quark that can be accessed is its
lifetime, known as the production time $\tau_p$~\cite{SJBAHM88}. 
This is accessed by studying the dependence of $p_T^2$ on $\nu$ for 
several nuclei. Since $\tau_p$ is time dilated in proportion to $\nu$,
the $\nu$ dependence of $p_T^2$ for a series of nuclei of known
dimensions can be deconvoluted to reliably yield an estimate of
$\tau_p$. While measurements with a 5-GeV electron beam have
demonstrated the feasibility of the method~\cite{EG2,Haf06}, with an 11-GeV 
beam a much wider range in $\nu$ will be accessible. An example of a
measurement of this type is shown in Fig.~\ref{pt2}, where
$\Delta p_T^2$ is shown for three nuclei in a range of $\nu$ from 2 to
9~GeV. 

%%%%%%%%%%%%%%%%%%%%%%%%%%%%%%%%%%%%%%%%%%%%%%%%%%%%%%%%%%%%%%%%%%%%%%%%%%%
\begin{figure}[htbp]
\vspace{7.5cm}
\special{psfile=../nuclear/pt2.ps hscale=60 vscale=60 hoffset=45 voffset=-10} 
\caption{\small{A plot of $\Delta p_T^2$ vs. $\nu$ for three different 
nuclei.  Plateaus are observed in the carbon and iron targets for larger 
values of $\nu$, indicating the production length $\tau_p$ is longer than 
the full thickness of either nucleus, but still less than that of the lead 
nucleus.}}
\label{pt2}
\end{figure}
%%%%%%%%%%%%%%%%%%%%%%%%%%%%%%%%%%%%%%%%%%%%%%%%%%%%%%%%%%%%%%%%%%%%%%%%%%%

A second property of the deconfined quark is the rate of its loss of
energy due to gluon emission. This partonic energy loss has a simple
relationship to $\Delta p_T^2$ under certain simplifying assumptions.
This energy loss can be written as:

\begin{equation}
{dE\over{dx}} \approx \frac{3}{4} \alpha_s \Delta p_T^2.
\end{equation}

\noindent
This energy loss has been estimated previously using the Drell-Yan
reaction~\cite{DY1,DY2}, although theoretical ambiguities have hampered the
extraction.  In addition, energy loss by gluon radiation is the primary 
evidence for the quark-gluon liquid discovered at RHIC~\cite{RHIC} 
through the phenomenon known as jet suppression or mono-jet production
\cite{STAR}. Thus these measurements have important interdisciplinary 
connections. In a simple picture, the energy loss can be estimated
simply from data such as that shown in Fig.~\ref{pt2} from the functional 
form and the known properties of the nucleus.  A picture such as that 
shown in Fig.~\ref{pt2} implies the observation of the novel result that 
the total energy loss is proportional to the square of the distance 
traveled through the medium, a result long predicted~\cite{BDPS}.  This 
implies a coherence behavior in QCD analogous to the QED effect known as 
the Landau-Pomeranchuk-Migdal (LPM) effect, in which electron
bremsstrahlung is suppressed by coherent scattering from multiple
scattering centers~\cite{LPM1,LPM2,Migdal}. 

Models for the energy loss of quarks typically contain a parameter called 
the transport coefficient or jet quenching parameter, $\hat{q}$.  While 
difficult to calculate in lattice QCD, this parameter can be calculated 
for a hot medium~\cite{LRW} using the AdS/CFT correspondence that 
connects nonperturbative phenomena in hot, strongly coupled gauge theories 
onto calculable problems in a dual gravity theory~\cite{ADSCFT1,ADSCFT2,ADSCFT3}.
While this technique cannot be applied to a cold medium, an extrapolation 
procedure such as that employed by Baier {\it et al.}~\cite{BAIER1,BAIER2} could 
potentially extrapolate from hot to cold media. 

\boldmath
\subsection{Color Transparency and the Time Evolution of $q\bar{q}$ Pairs}
\unboldmath

The color transparency (CT) phenomenon illustrates the power of exclusive 
reactions to isolate simple elementary quark configurations.  For a hard 
exclusive reaction, such as vector meson electroproduction on the nucleon, 
the scattering amplitude at large momentum transfer is suppressed by 
powers of $Q^2$ if the hadron contains more than the minimal number of 
constituents.  This is derived from the QCD-based quark counting rules. 
Therefore, the hadron containing valence quarks only, participates in the 
scattering.  Moreover, each quark, connected to another one by hard gluon 
exchange carrying momentum of order $Q$, should be found within a distance 
of order $1/Q$. Therefore, at large $Q^2$ one selects a very special 
configuration of the hadron wave function where all connected quarks are 
close together, forming a small size color neutral configuration called 
a Point Like Configuration (PLC).  Such an object is unable to emit or 
absorb soft gluons.  Therefore, its strong interaction with the other 
nucleons becomes significantly reduced, and then the nuclear medium 
becomes more transparent.  The nucleus offers a unique laboratory to study 
quark dynamics.  Indeed, the nucleus can be used as a revealing medium of 
the evolution in time of elementary configurations in the hadron wave 
function.  The time necessary for a quark to cross distances typical of 
the confined systems is of the order of 1~fm.  By taking into account the 
relativistic time dilation factor, the characteristic time scale 
corresponds to length scales on the order of a few fm.  The only medium 
available at this scale is the nucleus, offering to us a new generation of 
experiments where the nucleus functions as bubble chamber!

Experimentally, we would like to understand this spectacular phenomenon by 
studying the hadron attenuation as it propagates through the nuclear 
medium.  These measurements will allow us to not only access the special 
configuration of the hadron wave function, but also to study how this 
configuration dresses with time to form the fully complex asymptotic wave 
function of the hadron.  This puts us in the heart of the dynamics of 
confinement.  Furthermore, the onset of CT is related to the onset of 
factorization, which is an important requirement for accessing Generalized 
Parton Distributions (GPDs) in deep exclusive meson production. The 
$\rho^0$ meson is our hadron of choice because it offers many advantages. 
It is believed that the onset of CT is expected at lower $Q^2$ in the 
($q\bar{q}$) system than in the ($qqq$) system, as it is much more probable 
to produce a small-size system of two quarks than one of three quarks
\cite{Blat}.  In addition, the $\rho^0$ is a vector meson similar to 
the virtual photon. Therefore its production mechanism is fairly well 
understood because the virtual photon fluctuates into a ($q\bar{q}$) pair 
which then materializes into the $\rho^0$ meson. The size of the 
produced ($q\bar{q}$) can be directly connected to the virtuality of the 
photon. Therefore, smaller sizes can be reached at larger $Q^2$.

More than two decades of experimental investigations have lead to only 
one clear signal of CT at high energies.  It was observed in experiment 
E791~\cite{Aita} at Fermilab.  The $A$-dependence of the diffractive 
dissociation into di-jets of 500~GeV pions scattering coherently from 
carbon and platinum targets was measured.  It was found that the cross 
section can be parameterized as $\sigma = \sigma_{0} A^{\alpha}$, with 
$\alpha$ = 1.6.  This result is quite consistent with theoretical 
calculations~\cite{Bert1,Bert2,Bert3} including CT and obviously 
inconsistent with a cross section proportional to $A^{2/3}$, which is 
typical of inclusive pion-nucleus interactions.  At moderate energies, 
the situation is more complicated.  No evidence of CT was found in 
quasi-free $A(e,e'p)$ reactions~\cite{eep1,eep2,eep3,eep4} even for $Q^2$ 
values as large as 8~GeV$^2$.  While the outcome of quasielastic $(p,2p)$ 
scattering from nuclei~\cite{p2p1,p2p2,p2p3} 
was very controversial due to the fact that the results do not support a 
monotonic increase in transparency with $Q^2$ as predicted by CT, the 
transparency increases for $Q^2$ from 3 to 8~GeV$^2$, but then decreases 
for higher $Q^2$, up to 11~GeV$^2$.  This subsequent decrease was 
explained as a consequence of soft processes that interfere with 
perturbative QCD in free $pp$ scattering but which are suppressed in the 
nuclear medium~\cite{ral}.  Other measurements studied the attenuation 
of the $\rho^0$ vector meson in the nuclear medium via exclusive $\rho^0$ 
lepto-production off nuclei.  The results from the two measurements
\cite{meso1,meso2} are very suggestive of a CT signal, but they are 
statistically limited.

The exclusive, diffractive, incoherent electroproduction of vector mesons 
off nuclei has been suggested~\cite{kop02} as a sensitive way to detect 
CT.  In the laboratory frame, the photon fluctuation can propagate over 
a distance $l_c$ known as the coherence length.  The coherence length can 
be estimated relying on the uncertainty principle and Lorentz time dilation 
as $l_c = 2 \nu/(Q^2 + M^2_{q\bar{q}})$, where $\nu$ is the energy of the 
virtual photon in the laboratory frame, (-$Q^2$) is its squared mass, and 
$M_{q\bar{q}}$ is the mass of the ($q\bar{q}$) pair.  In the case of 
exclusive $\rho^0$ electroproduction, the mass of the ($q\bar{q}$) is 
dominated by the $\rho^0$ mass.  The produced small-size, colorless 
hadronic system will then propagate through the nuclear medium with 
reduced attenuation because its cross section is proportional to its size. 
The effect of the nuclear medium on the particles in the initial and final 
states can be characterized by the nuclear transparency $T_A$.  $T_A$ is 
defined as the ratio of the measured exclusive cross section to the cross 
section in the absence of initial and final state interactions.  It can be 
measured by taking the ratio of the nuclear per-nucleon ($\sigma_A/A$) to 
free nucleon ($\sigma_N$) cross sections:

\begin{equation}
T_A = \frac{\sigma_A}{A\sigma_N}.
\end{equation}

The signal of CT would be an increase of the nuclear transparency as 
$Q^2$ increases.  It was shown by the HERMES collaboration~\cite{acker} 
that the nuclear transparency increases when $l_c$ varies from long to 
short compared to the size of the nucleus.  This is due to the fact that 
the nuclear medium seen by the ($q\bar{q}$) fluctuation becomes shorter. 
Thus the ($q\bar{q}$) pair interacts less.  This situation occurs when 
$Q^2$ increases at fixed $\nu$.  This so-called coherence length effect 
(CL) can mimic the CT signal.  Therefore one should keep $l_c$ fixed while 
measuring the $Q^2$ dependence of the nuclear transparency.

{\tt CLAS12} is the ideal detector for such measurements.  It offers large 
acceptance, good particle identification, and high luminosity. Using 
an 11-GeV electron beam, one can extend the $Q^2$ region up to
5.5~GeV$^2$. An indication of the quality of the data that can be
obtained is shown in Fig.~\ref{res}. 
The nuclear transparency for several targets (C, Fe, and Sn) could be 
measured.  This is important in order to study the formation time of the 
hadron as it propagates through different nuclear sizes.  These 
measurements would also be a natural extension of a previous {\tt CLAS} 
experiment~\cite{koko}, where the preliminary results indicate a clear 
evidence of a CT signal despite the limited $Q^2$ range to 2~GeV$^2$.

%%%%%%%%%%%%%%%%%%%%%%%%%%%%%%%%%%%%%%%%%%%%%%%%%%%%%%%%%%%%%%%%%%%%%%%
\begin{figure}[htbp]
\vspace{9.0cm}
\special{psfile=../nuclear/tr_err_fe.eps hscale=70 vscale=65 hoffset=40 voffset=0}
\caption{\small{A plot indicating the expected error bars for the 11-GeV 
measurements of nuclear transparency for $\rho_0$ production on Fe from 
incoherent $\rho$ electroproduction (12 days of beam time), and the 
predictions of Ref.~\cite{kop02}.}}
\label{res}
\end{figure}
%%%%%%%%%%%%%%%%%%%%%%%%%%%%%%%%%%%%%%%%%%%%%%%%%%%%%%%%%%%%%%%%%%%%%%%

\section{Hadron Attenuation and the Formation Time of Color Fields}

In the preceding discussion, the focus was on the kinematics in which the 
color of the quark is neutralized outside the nucleus.  In those kinematics 
one learns about the properties of the propagating quark through its 
partonic-level multiple scattering in the medium. In this section the focus 
is on the case when the color neutralization happens within the nuclear 
medium, creating a color singlet object referred to as a pre-hadron. The 
pre-hadron evolves to a full hadron over a period of time $\tau_f$, the 
formation time.  The pre-hadron and the final hadron can both interact with 
the nuclear medium and the most probable interaction is a highly inelastic 
reaction. Relative to the same kinematics on deuterium, in a heavy nucleus 
these interactions move hadron flux at high energies to lower energies (or
lower $z$) and higher multiplicities. This is referred to as hadron 
attenuation. It is quantitatively measured by the hadronic multiplicity 
ratio $R_M^h$:
    
\begin{equation}
R_M^h = \frac{(N_h^{DIS,A})/(N_e^{DIS,A})}
{(N_h^{DIS,^2H})/(N_e^{DIS,^2H})},
\end{equation}

\noindent
where $N_h^{DIS,A}$ is the number of hadrons of species $h$ produced
in DIS kinematics on nucleus $A$, $N_e^{DIS,A}$ is the number of DIS
electrons from nucleus $A$, and the denominator refers to the same
quantities for deuterium $^2H$~\cite{HERMES}. $R_M^h$ may in principle
depend on a number of kinematic variables such as $Q^2$, $\nu$, $z$,
$p_T$, and $\phi$. Isolating the multi-variable dependence of $R_M^h$
is a powerful discriminator between models based on different physical
pictures. At the current time, two different physical pictures
dominate model descriptions. The first~\cite{WANG,ARLEO} assumes that 
hadronization takes place outside the nucleus. The second
\cite{KNPH,MOSEL} assumes that hadronization can take place inside the 
nucleus, and cite the interaction of the pre-hadron as the main cause of 
the attenuation.  Both approaches provide adequate descriptions of the 
HERMES data, which is statistics limited to one or at most two dimensional
analyses. A full multi-dimensional analysis spanning a wide kinematic
range is feasible with {\tt CLAS12}, and such an analysis will be certain 
to strongly constrain theoretical descriptions. In addition to spanning a
range of kinematic variables, it is feasible to measure a wide range
of hadron masses with varied flavor content, as may be seen in
Table~\ref{table:hadron_list}, which lists hadrons with $c\tau$ greater
than nuclear dimensions for which measurements with {\tt CLAS12} are
feasible. With a dataset of this quality and breadth, a comprehensive
program to extract hadron formation lengths is practical to carry
out. Such a program will yield insights into the systematic behavior
of how hadrons form, a heretofore unknown sector of nonperturbative
QCD in the spacetime domain.  

%%%%%%%%%%%%%%%%%%%%%%%%%%%%%%%%%%%%%%%%%%%%%%%%%%%%%%%%%%%%%%%%%%%%%%%
\begin{table}[htbp]
\begin{center}
\begin{tabular}{||c|c|c|c|c|c||} \hline \hline
hadron & $c\tau$ & mass & flavor  & detection & Production rate\\
       &         &(GeV) & content &  channel &  per 1k DIS events\\ \hline \hline
$\pi^0$ & 25 nm & 0.13 & $u\bar{u}d\bar{d}$ & $\gamma\gamma$ & 1100 \\ \hline
$\pi^+$ & 7.8 m & 0.14 &   $u\bar{d}$ & direct & 1000 \\ \hline
$\pi^-$ & 7.8 m & 0.14 &   $d\bar{u}$  & direct & 1000 \\ \hline
$\eta$ & 0.17 nm & 0.55 & $u\bar{u}d\bar{d}s\bar{s}$&$\gamma\gamma$ & 120 \\ \hline
$\omega$ & 23 fm & 0.78 &  $u\bar{u}d\bar{d}s\bar{s}$ & $\pi^+\pi^-\pi^0$ & 170 \\ \hline
$\eta'$ & 0.98 pm & 0.96 &  $u\bar{u}d\bar{d}s\bar{s}$ & $\pi^+\pi^-\eta$ & 27 \\ \hline
$\phi$ & 44 fm & 1.0 &  $u\bar{u}d\bar{d}s\bar{s}$ & $K^+K^-$ & 0.8 \\ \hline
$f1$ & 8 fm & 1.3 &  $u\bar{u}d\bar{d}s\bar{s}$ & $\pi\pi\pi\pi$ & - \\ \hline
$K^+$ & 3.7 m & 0.49 &  $u\bar{s}$ & direct & 75 \\ \hline
$K^-$ & 3.7 m & 0.49 &  $\bar{u}s$ & direct & 25 \\ \hline
$K^0$ & 27 mm & 0.50 &  $d\bar{s}$ & $\pi^+\pi^-$ & 42 \\ \hline
$p$ & stable & 0.94 &  $ud$ & direct & 530 \\ \hline
$\bar{p}$ & stable & 0.94 &  $\bar{u}\bar{d}$ & direct & 3 \\ \hline
$\Lambda$ & 79 mm & 1.1 &  $uds$ & $p\pi^-$ & 72 \\ \hline
$\Lambda(1520)$ & 13 fm & 1.5 &  $uds$ & $p\pi^-$ & - \\ \hline
$\Sigma^+$ & 24 mm & 1.2 &  $us$ & $p\pi^0$ & 6 \\ \hline
$\Sigma^0$ & 22 pm & 1.2 &  $uds$ & $\Lambda\gamma$ & 11 \\ \hline
$\Xi^0$ & 87 mm & 1.3 &  $us$ & $\Lambda\pi^0$ & 0.6 \\ \hline
$\Xi^-$ & 49 mm & 1.3 &  $ds$ & $\Lambda\pi^-$ & 0.9 \\ \hline \hline	
\end{tabular}
\end{center}
\caption{\small{Final-state hadrons potentially accessible for formation 
length and transverse momentum broadening studies in {\tt CLAS12}. The 
rate estimates were obtained from the LEPTO event generator for an 11-GeV 
incident electron beam. (The criteria for selection of these particles 
was that $c\tau$ should be larger than the nuclear dimensions, and their 
decay channels should be measurable by {\tt CLAS12}.)}}
\label{table:hadron_list}  
\end{table}
%%%%%%%%%%%%%%%%%%%%%%%%%%%%%%%%%%%%%%%%%%%%%%%%%%%%%%%%%%%%%%%%%%%%%%%

\boldmath
\subsection{$D(e,e'p_s)$ and the Quark Structure of Neutrons in a Cold
Dense Medium}
\unboldmath

For a complete understanding of QCD at hadronic scales, we need to 
learn more about the interplay between the internal (quark) structure 
of nucleons and the interaction between two nucleons. In particular, it 
is of high interest whether nucleons in close proximity to each other
(effectively at non-equilibrium high density) change their internal 
structure or perhaps even lose their separate identity to fuse into a 
``six quark cluster''~\cite{Carlson}.  Some less dramatic modifications 
of the nucleon structure that have been proposed include off-shell 
effects~\cite{MT}, $Q^2$ rescaling effects, and the suppression of
small-size configurations (PLCs) in the nucleon wave function
\cite{FS81,FS85}.  Deuterium is the optimal system to study such 
``tightly bound pairs'', since there are no additional nucleons 
interacting with the pair under study and the pair is at rest in the lab, 
with completely defined kinematics.  While the probability for a small 
inter-nucleon distance configuration in deuterium is rather small 
compared to heavier nuclei, such configurations can be ``tagged'' by the 
emission of a fast proton in the backward hemisphere relative to the 
momentum transfer vector.  We therefore propose to measure the reaction 
$D(e,e'p_b)X$ with coincident detection of the scattered electron in the 
forward part of {\tt CLAS12} and the fast (above 300~MeV) backwards 
proton in the central detector.

In the simple spectator picture, the backwards-moving proton does not 
participate in the scattering process and can serve as a tag of the 
initial state momenta of both nucleons.  By measuring the momentum of
this backward proton, we can correct the observed electron kinematics 
for the initial motion of the unobserved struck neutron and extract the 
modified neutron structure function $F_2^{n(eff)}(x, Q^2, p^2)$.
The emphasis here is not on nearly on--shell neutrons, but rather on
the opposite kinematic extreme of fast--moving neutrons, where off-shell
effects and other internal structure changes are much more pronounced.
We can extract the dependence of the structure function 
$F_2^{n(eff)}(x, Q^2, p^2)$ at fixed $x$ and $Q^2$ on the spectator
momentum $p$ in the range from about 70~MeV to 700~MeV.  We will 
simultaneously cover a large range in $x$ and $Q^2$, allowing us to make 
detailed comparisons with the different models mentioned above, including 
the rather striking change in the shape of the structure function $F_2$ 
predicted for a non-trivial six quark configuration~\cite{Carlson}.

For the proposed experiment, we will use {\tt CLAS12} in the standard
configuration, with a liquid-deuterium target and the fully 
instrumented central detector to tag the backward proton. We estimated 
the expected number of counts for a 20-day run with full luminosity
(10$^{35}$ cm$^{-2}$s$^{-1}$). The results are shown in Fig.~\ref{deeps}
as a function of the ``ordinary'' Bjorken variable $x = Q^2/2m\nu$
in the lab and for several bins in the light cone fraction $\alpha$ of
the backward proton. One can clearly see the kinematic shift due to the 
motion of the struck neutron, which we can fully correct using the proton 
kinematics.  We clearly will have good statistics for a large range in 
$x$ and in $\alpha$ (the highest bin corresponds to more than
600~MeV momentum opposite to the direction of the $q$ vector),
drastically extending the kinematic coverage and statistical precision
of the existing data from the analog experiment at 6~GeV (E94-102)
\cite{e94102}.

%%%%%%%%%%%%%%%%%%%%%%%%%%%%%%%%%%%%%%%%%%%%%%%%%%%%%%%%%%%%%%%%%%%%%%%
\begin{figure}[htbp]
\vspace{8.6cm}
\special{psfile=../nuclear/deeps.eps hscale=65 vscale=55 hoffset=35 voffset=-15}
\caption{\small{Kinematic coverage in Bjorken--$x$ and proton light-cone 
fraction $\alpha_S$ for the proposed experiment. The count rates have
been estimated for a 20-day run with the standard {\tt CLAS12} 
configuration.}}
\label{deeps}
\end{figure}
%%%%%%%%%%%%%%%%%%%%%%%%%%%%%%%%%%%%%%%%%%%%%%%%%%%%%%%%%%%%%%%%%%%%%%%

\section{$x>1$ and the Properties of Cold Dense Matter} 

Measurements with $x>1$ have been demonstrated to provide insight into
the properties of nucleon-nucleon correlations~\cite{EGIYAN,ARRINGTON}. 
With {\tt CLAS12}, the combination of increased luminosity and large 
acceptance promises to permit investigation of the hadronic final states 
associated with $x>>1$, which means that the properties of cold nuclear 
matter fluctuating ephemerally to conditions of high density can be 
accessed experimentally.  Together with theoretical extrapolations to 
equilibrium conditions, one can hope to explore such exotica as stable 
strange matter as an equilibrium component of neutron stars. A
possible experimental avenue to explore these ideas is to measure
strangeness production on a series of nuclei in $x>1$ kinematics. A
significant enhancement in kaon production increasing with $x$ may be
a signature that can be connected to equilibrium conditions at high
nuclear density which have a persistent strangeness component. These
studies complement investigations in heavy-ion reactions~\cite{GSI}
where kaon yields have recently been shown to be consistent with
thermal production in high-density nuclear matter. These exclusive or
semi-inclusive studies with $x>1$ are the natural next step following
the inclusive experiments to date. 

\subsection{The Polarized EMC Effect and the Quark Structure of Nuclei}

The well known ``EMC effect'', which shows a modification of the
$F_2$ structure function in nuclei relative to the proton, has been
a puzzle for over two decades.  Nearly 1000 papers have been written
on the subject to try to explain the EMC effect and its associated
phenomena.  The best models generally use a modification of the nucleon 
structure within the nuclear medium to explain the effect.  It is
known from lattice QCD that the region surrounding a baryon or a meson
has a suppressed chiral condensate, and it is reasonable to infer that
the nuclear medium continues this pattern.  In a quantum field theory
picture of the nucleus, the structure of bound baryons is dominantly
affected by the scalar field of the nucleus, which essentially
polarizes the nucleon and modifies its structure, particularly that of
the lower component of the relativistic wave function.  Calculations of
the polarized EMC effect indicate it is approximately twice as big as
the unpolarized effect~\cite{CLOET}.  A plot indicating the achievable
uncertainties is shown in Fig.~\ref{p_emc}.

%%%%%%%%%%%%%%%%%%%%%%%%%%%%%%%%%%%%%%%%%%%%%%%%%%%%%%%%%%%%%%%%%%%%%%%
\begin{figure}[htbp]
\vspace{6.5cm}
\special{psfile=../nuclear/REMC.eps hscale=60 vscale=60 hoffset=70 voffset=-15}
\caption{\small{A plot of the polarized EMC effect for an 11-GeV beam, 40\% 
target polarization, 80\% beam polarization, and 70 PAC days measured in 
{\tt CLAS12}. The two curves are for the two dominant structure function 
multipoles for this ($J^{\pi}=3/2^-$) nucleus; the solid line is for $K=1$ 
and the dashed line is for $M=J$.}}
\label{p_emc}
\end{figure}
%%%%%%%%%%%%%%%%%%%%%%%%%%%%%%%%%%%%%%%%%%%%%%%%%%%%%%%%%%%%%%%%%%%%%%%
