\chapter{Baryon Form Factors}
\label{sec:title}

\section{Introduction}
\label{sec:intro}

The structure of the nucleon is a defining problem for nuclear physics.  
The most basic observables that reflect the composite nature of the 
nucleon are its EM form factors (FFs).  Indeed, historically the first 
direct indication that the nucleon is not elementary came from 
measurements of these quantities in elastic $ep$ scattering~\cite{HOF}. 
The elastic electric and magnetic FFs characterize the distributions of 
charge and magnetization in the nucleon as a function of spatial resolving 
power. The transition FFs reveal the nature of the excited states of the 
nucleon.  Further, these quantities can be described and related to other 
observables through the use of generalized parton distributions (GPDs).  
Therefore, this topic connects strongly to other thrusts of the 12-GeV 
program and is central to answering one of the questions in the DOE 
Nuclear Physics Long Range Plan, `What is the structure of the nucleon?'
\cite{scirev}.

\section{Context and Motivation}
\label{sec:motivation}

The nucleon elastic FFs are defined via matrix elements of the EM current, 
$J_{\mu} = \overline \psi \gamma_{\mu} \psi$, as:

\begin{equation}
\langle N(P') | J_{\mu}(0) | N(P) \rangle
= \overline u(P')
\left( \gamma_{\mu} F_1(Q^2)
     + {i\sigma_{\mu\nu} q^{\nu} \over 2 M} F_2(Q^2)
\right) u(P),
\end{equation}

\noindent
where $P$ and $P'$ are the initial and final nucleon momenta, and 
$q = P - P'$ is the 4-momentum transferred to the nucleon, with 
$Q^2 = -q^2$.  The Sachs electric and magnetic FFs are defined in terms 
of $F_1$ and $F_2$ as:
\vskip -0.6cm
\begin{eqnarray}
G_E(Q^2) &=& F_1(Q^2) - (Q^2/4M^2)\ F_2(Q^2),	\\
G_M(Q^2) &=& F_1(Q^2) + F_2(Q^2)\ .
\end{eqnarray}
\vskip -0.3cm
\noindent
EM transition FFs may be similarly defined.  In this case the final state 
is no longer a nucleon, but rather may be a resonance state: 
$\langle R(P') | J_{\mu}(0) | N(P) \rangle$.

The elastic FFs at low $Q^2$ approximately follow a dipole form, 
$G_D(Q^2) \propto 1/(1 + Q^2/Q_0^2)^2$, with $Q_0^2 \approx 0.71$~GeV$^2$.  
This behavior can be qualitatively understood in a VMD picture where the 
virtual photon interacts with the nucleon after fluctuation into a virtual 
vector meson.  Deviations from this form have been observed (see 
Fig~\ref{fig:ff_fig1}), and it is important to understand their nature.
At asymptotically large $Q^2$, the elastic FFs can be described 
in terms of perturbative QCD~\cite{LB}.  Here the short wavelength of the 
highly virtual photon enables the quark substructure of the nucleon to be 
cleanly resolved.

Just where the perturbative behavior sets in is still an open question.
Evidence from recent experiments at JLab and elsewhere suggests that 
non-perturbative effects still dominate the FFs for $Q^2 < 10$~GeV$^2$.
The $Q^2$ dependence of $G_E$ and $G_M$, which is expected to be the same 
in perturbative QCD, is observed to be rather strong in the $G_E/G_M$ 
ratio for the proton out to $Q^2 \approx 6$~GeV$^2$~\cite{GEMJL1,GEMJL2}.

Understanding the transition from the low to high $Q^2$ regions is vital
not just for determining the onset of perturbative behavior.  FFs 
in the transition region are very sensitive to mechanisms of spin-flavor 
symmetry breaking, which cannot be described {\em in principle} within 
perturbation theory.  A classic example is the electric FF of the 
neutron, $G_E^n$~\cite{GEN}, which is identically zero in a simple valence 
quark picture, and whose non-zero value can only be understood in terms of
non-perturbative mechanisms, such as the hyperfine interaction between
quarks~\cite{IKS} or a pion cloud~\cite{THOMAS}.

Theoretical guidance on the FFs in the transition region can
be obtained from lattice QCD.  These calculations will have achieved a 
high degree of accuracy by the time {\tt CLAS12} is taking data~\cite{LQCD}
and challenging these fundamental calculations with high-precision data 
for both the proton and the neutron out to high $Q^2$ will be an important 
test.

%%%%%%%%%%%%%%%%%%%%%%%%%%%%%%%%%%%%%%%%%%%%%%%%%%%%%%%%%%%%%%%%%%%%%%%
\begin{figure}[htbp]
\vspace{5.7cm}
\special{psfile=../formfactors/gmn_highlight_plot.eps hscale=55 vscale=50 hoffset=85 voffset=-5}
\caption{\small{The normalized elastic proton and neutron magnetic form
factors (${G_M}_n/G_{D}\mu_n$, ${G_M}_p/G_{D}\mu_p$)~\cite{BOSTED95,chw1}. 
The proton data have been shifted upward by 0.3 for clarity.  Note the lack 
of high-quality data for the neutron at large $Q^2$.  The red circles are 
preliminary results from the {\tt CLAS} E5 data set.  The dark green circles 
are anticipated results for a 45-day {\tt CLAS12} measurement.}}
\label{fig:ff_fig1}
\end{figure}
%%%%%%%%%%%%%%%%%%%%%%%%%%%%%%%%%%%%%%%%%%%%%%%%%%%%%%%%%%%%%%%%%%%%%%%

The $Q^2$ dependence of the nucleon elastic FFs reflects the dynamics of 
the quark constituents in a region where confinement plays an important 
role.  Because the EM current couples to the charged quark constituents, 
one can decompose the FFs into a sum over the quark contributions 
$G_{E,M}(Q^2) = \sum e_q\ G^{(q)}_{E,M}(Q^2)$, where $e_q$ is the quark 
charge and the sum is over the valence quarks.  To determine 
$G^{(q)}_{E,M}$ from each individual quark flavor requires measurement of 
the FFs of both the proton and neutron. However, at present, precision 
data at high $Q^2$ exists only for the proton (see Fig.~\ref{fig:ff_fig1}), 
which extends beyond 30~GeV$^2$.  

Recent work on GPDs has provided a unifying framework within which both 
FFs and structure functions can be simultaneously embedded
\cite{JI,Radyushkin:1997ki}.  
GPDs hold the promise of providing a 3D image of the nucleon (two spatial, 
one momentum), but require a variety of measurements.  The elastic and 
transition FFs can be related to each other in dynamical quark models of 
the nucleon, and more rigorously, in the large $N_C$ limit of QCD. 
Therefore, within this new framework, they measure different combinations 
of the same set of GPDs and provide an essential constraint on theory.  In 
particular, the elastic FFs are related to the $0^{th}$ moments of the 
GPDs, {\it e.g.} $H^q(x,\xi,t)$ is given by~\cite{JI}:

\begin{equation}
\int_{-1}^1 \sum_q H^q(x,\xi;t) dx = F_1(t),
\end{equation}

\noindent
where $t$ is the momentum transfer, $x$ is the longitudinal momentum 
fraction, $\xi$ is the skewness, and the sum is over the valence quarks.
The various FF measurements accessible in {\tt CLAS12} are therefore 
interrelated and can be interpreted within a unified analysis.  They 
complement and support other deeply virtual exclusive reactions that 
are part of the JLab 12-GeV program.

From another perspective, the interplay between FFs and structure 
functions is central to the phenomenon of quark-hadron duality
\cite{BG,Niculescu:2000tk,RUJ,JU,JM}, and the transition from quark to 
hadron degrees
of freedom in QCD.  FFs obtained in exclusive reactions can be related 
through local quark-hadron duality to deep-inelastic structure functions 
measured in inclusive processes.  For elastic scattering, the FFs can be 
used to predict the behavior of structure functions in the limit $x \to 1$
\cite{BG,DY,WEST,MEL}, which is a difficult region to access experimentally.  
For example, the $F_1$ structure function of the nucleon at large $Q^2$ 
\cite{BG,MEL} is proportional to $dG_M^2(Q^2) / dQ^2$.  Conversely, from 
data on structure functions at very large $x$, one can extract the elastic 
FFs as a function of $Q^2$ and compare them directly with data
\cite{Niculescu:2000tk,RUJ}.  One can similarly use quark-hadron duality 
to study not 
just the elastic case, but the entire spectrum of excited final states, 
and more generally, the transition from resonance production to scaling in 
deep-inelastic scattering~\cite{CM}.  

Nucleon ground and excited states represent different eigenstates of the 
Hamiltonian, therefore to understand the interactions behind nucleon 
formation from the fundamental constituents, the structure of both the 
ground and excited states should be studied. Nucleon resonances are clearly 
seen in inelastic inclusive structure functions off nucleons.  Moreover, 
the $Q^2$ evolution of the non-resonant parts of the inclusive structure 
functions may be described reasonably well by QCD-based approaches
\cite{Me05}, while the evolution of the $N^*$ excitation strength with 
$Q^2$ strongly depends on the quantum numbers of the excited state
\cite{Bur1}.  Therefore, $N^*$ states contain important 
information on binding interactions, which is complementary to that 
obtained from the studies of the ground state.  

Obviously the data on the $Q^2$-evolution of $N^*$ electrocouplings offer 
just phenomenological information.  Lattice simulations represent the most 
straightforward way to relate this phenomenology to fundamental QCD. Lattice 
studies of $N^*$ states are making steady progress 
\cite{Ri05,Me03,Burc04,alexandrou03,alexandrou04}, and on the time scale for 
the 12-GeV upgrade, we may expect lattice predictions for the $Q^2$-evolution 
of the electrocouplings of the lightest $N^*$s in each partial wave. This is 
an important goal in the current activity of the Theory Center at JLab.

Recent QCD calculations on the lattice~\cite{ichie02} show evidence 
for a ``Y-shape'' color-flux tube, indicating a genuine 3-body force 
for baryons with stationary quarks as shown in Fig.~\ref{fig:mercedes}.
This 3-body force is a unique feature of a 3-quark baryon system in QCD.
Lattice simulations~\cite{Ta04} also relate the fundamental QCD Lagrangian 
to quark confinement potentials, which are responsible for the formation 
of the ground state and a variety of excited nucleon states, and determine 
the behavior of the $N^*$ EM FFs as a function of $Q^2$.  These FFs may 
be estimated in various models~\cite{Az90,Ca95,aiello98,CRob06,me02}, which 
are potentially capable of relating them to the binding potential and 
quark-quark interactions at various distance scales. 

%%%%%%%%%%%%%%%%%%%%%%%%%%%%%%%%%%%%%%%%%%%%%%%%%%%%%%%%%%%%%%%%%%%%%%%%%%%%%%%
\begin{figure}[ht]
\vspace{5.1cm}
\special{psfile=../formfactors/epsfigs/ichie_top.eps hscale=135 vscale=130 hoffset=40 voffset=-115}
\caption{\small{Lattice QCD calculation of the 3D color-flux distribution 
for a baryon. The calculation was carried out to study the abelian color-flux 
distribution in a static 3-quark system. The ``Y-shape'' configuration is 
evident, indicating the presence of a genuine 3-body force. The graph shows 
high density at the quark locations and in the center. The $\Delta$-shaped 
flux configuration, characteristic of 2-body forces, would have a depletion 
in the center.}}
\label{fig:mercedes}
\end{figure}
%%%%%%%%%%%%%%%%%%%%%%%%%%%%%%%%%%%%%%%%%%%%%%%%%%%%%%%%%%%%%%%%%%%%%%%%%%%%%%%

The {\tt CLAS} $N^*$ program with energies up to 6~GeV has already 
provided valuable data on resonance electrocouplings
\cite{Bur1,Bu05}.  Further extension of this research with 
{\tt CLAS12} in the $Q^2$ region up to 10~GeV$^2$ will open new 
opportunities to study $N^*$ states at still unexplored distance scales, 
where considerable contributions from the bare quark core are expected.

Studies of {\tt CLAS} data on nucleon resonance transitions at increasingly 
short distances have resulted in strong empirical evidence of large 
meson-baryon dressing contributions to the resonance excitations at large 
and medium distances. This is particularly evident in the region of the 
$\gamma^*N\Delta$ transition where constituent quark models using point-like 
$\gamma^*$-$q$ couplings are unable to explain the much larger strength of 
the magnetic dipole transition from what is predicted from quark 
contributions alone~\cite{Bur1}.

The JLab Excited Baryon Analysis Center (EBAC) is developing an approach
capable of evaluating contributions of meson-baryon dressings to $N^*$ 
electrocouplings based on coupled-channels analyses of the world data 
\cite{Lee06}.  Isolating hadronic dressing effects, we eventually get 
information on bare quark core contributions.  Preliminary studies have 
shown that these contributions become increasingly important as $Q^2$ 
increases.
    
%%%%%%%%%%%%%%%%%%%%%%%%%%%%%%%%%%%%%%%%%%%%%%%%%%%%%%%%%%%%%%%%%%%%%%%%%%%%%
\begin{figure}[ht]
\vspace{5.3cm}
\special{psfile=../formfactors/epsfigs/asym_1.ps hscale=42 vscale=30 hoffset=115 voffset=-49}
\caption{\small{Helicity amplitudes $A_{1/2}(Q^2)$ scaled by $Q^3$ for the 
Roper $P_{11}(1440)$ (triangles), $S_{11}(1535)$ (circles), $D_{13}(1520)$ 
(squares), and $F_{15}(1680)$ (stars). At the highest $Q^2$ the dependence is 
consistent with a flat behavior.}}
\label{fig:asym_1}
\end{figure}
%%%%%%%%%%%%%%%%%%%%%%%%%%%%%%%%%%%%%%%%%%%%%%%%%%%%%%%%%%%%%%%%%%%%%%%%%%%%%

In Fig.~\ref{fig:asym_1} we show recent results from {\tt CLAS} data on 
$\vec{e} p \rightarrow en\pi^+$ in terms of the leading, helicity-conserving 
$A_{1/2}$ amplitudes to the excitation of the $P_{11}(1440)$, $D_{13}(1520)$, 
$S_{11}(1535)$, and $F_{15}(1680)$.  The amplitudes are multiplied by $Q^3$, 
the expected dependence for point-like coupling to the quarks.  The quantity 
$Q^3 \cdot A_{1/2}(Q^2)$ is consistent with a constant for all resonances 
above $Q^2$=3~GeV$^2$, an encouraging sign that a considerable contribution 
from the bare quark core may be reached with the energy upgrade.

This regime in $N^*$ excitations is governed by light front wave 
functions (LFWFs)~\cite{Br04}. The data on $N^*$ electrocouplings at high 
$Q^2$ enable us for the first time to access LFWFs for excited nucleon states. 
Substantial progress has been made in calculations of LFWFs from first 
principles in QCD, including lattice gauge theory~\cite{Del00}, transverse 
lattice~\cite{MBu02}, and discretized light front quantization~\cite{Pa85}. 
All of these developments offer promising avenues to relate data on $N^*$ 
electrocouplings at high $Q^2$ to fundamental QCD.   

The studies of $N^*$ electrocouplings at high $Q^2$ are of considerable
interest for various 12-GeV upgrade research programs.  Data on $N^*$
electrocouplings are needed in the studies of inclusive structure functions 
at large $x_b$ for reliable treatment of resonant contributions. Credible 
estimates of the resonant parts in various meson electroproduction amplitudes 
at high $Q^2$ may allow us to extend GPD studies toward the $N^*$-excitation 
region.

\section{Form Factor Measurements}
\label{sec:clas}

The program we describe here is part of a broad assault on the nucleon
elastic and transition FFs at JLab~\cite{chw1,GEMJL1,GEMJL2}.  All four 
elastic FFs are needed to untangle the physics including the different 
quark contributions.  In the upgraded {\tt CLAS12} we have proposed 
measuring $G_M^n$ out to $Q^2$=13~GeV$^2$ using a ratio method described 
below.  A proposal (PR12-07-104) for this experiment was approved by the 
JLab PAC in 2007 \cite{PAC32}.  Another proposal to measure the ratio 
$G_E^p/G_M^p$ in the upgraded Hall A (PR12-07-109) was also approved to 
push our knowledge of $G_E^p$ out to $Q^2$=13~GeV$^2$~\cite{PAC32}.  There 
are plans to measure $G_E^n$ in the upgraded Hall A out to $Q^2$=7.2~GeV$^2$ 
and to explore the baryon transition FFs with {\tt CLAS12} and in Hall C.  
The $N^*$ program in {\tt CLAS12} will measure the electrocouplings of 
almost all the known $N^*$ states up to $Q^2$=10~GeV$^2$.  The {\tt CLAS12} 
detector, coupled with the unprecedented quality of the upgraded CEBAF beam, 
will be the only facility worldwide capable of accessing the $N^*$ transition 
EM FFs in the unexplored domain of high $Q^2$, from 5 to 10~GeV$^2$.  

\subsection{Nucleon Elastic Form Factors}
\label{sec:elasff}

The FF $G_M^n$ will be accessible out to $Q^2 \approx 13$~GeV$^2$
(see Fig.~\ref{fig:ff_fig1}) by extending the method used in JLab 
experiment E94-017~\cite{E94-017}.  An unpolarized cryogenic 
liquid-deuterium target is employed as a ``neutron target'', and the 
ratio of $e$-$n$ events to $e$-$p$ events off deuterium is measured.  A 
cut on $W$ selects quasi-elastic (QE) kinematics, and in the conceptual 
limit where the $n$ and $p$ are considered as free in the deuteron, 
the $e$-$n$/$e$-$p$ ratio can be directly related to the free FFs of the 
nucleons. Using the more accurately determined proton FFs and an estimate 
of $G_E^n$, one extracts $G_M^n$ from the deuteron QE cross section.

There are a number of factors that affect the accuracy of the measured
$G_M^n$. While these are the same for low and high $Q^2$, their relative 
importance changes. As long as $G_M^n$ is much larger than $G_E^n$, 
uncertainty of the latter does not contribute significantly to the
uncertainty in $G_M^n$. The proton FFs must be well known.  In QE 
kinematics, corrections to the ratio due to the binding of the nucleons 
within the deuteron will become increasingly smaller at high $Q^2$.  At 
$Q^2 \approx 4$~GeV$^2$ the correction was found to be $\sim$0.2\%
\cite{sj1,ha1,jlthesis}.

The neutron detection efficiency, which must be known accurately in this 
method, will be more stable at high $Q^2$.  The intrinsic detection 
efficiency in the calorimeter reaches a nearly constant value for neutron 
momenta above $\sim$1.75~GeV~\cite{jlthesis}.  The detection efficiency 
was continuously monitored in E94-017 using a novel dual-cell 
deuterium-hydrogen target which allowed two target cells to be 
simultaneously in the beam.  The neutron detection efficiency was thereby 
continuously measured using the exclusive reaction $p(e,e'\pi^+)n$ from 
the proton target.  Our preliminary studies of this method in {\tt CLAS12} 
are promising.  

Two factors are expected to become more important at high $Q^2$.  The 
first is the QE scattering rate becomes small relative to inelastic 
processes nearby in the $W$ spectrum. The tails of these processes become 
an important contamination beneath the region of QE scattering.  The 
second effect is the kinematic broadening of the $W$ peak.  Taking these 
two effects together will reduce the QE peak in the $W$ spectrum for 
$Q^2>~8$~GeV$^2$, independent of experimental resolution.  Previous 
measurements of $G_M^n$ at high $Q^2$, using inclusive electron scattering, 
encountered this limitation~\cite{ROCK}.  These difficulties can be 
overcome in {\tt CLAS12} using two types of cuts that do not introduce 
bias into the ratio measurement. First, the angle between the virtual 
photon and the detected nucleon is very small for QE kinematics. 
Eliminating angles that are not consistent with the QE process removes 
much of the inelastic background.  Second, the hermiticity of {\tt CLAS12}, 
and its increased capability for detection of neutrals, means that events 
with in-time charged particles that are inconsistent with QE scattering 
can be vetoed with high efficiency, as can neutral hit pairs reconstructing 
to the $\pi^0$ mass.  The events of interest can be separated from 
inelastic events.  The expected quality of the measurements feasible is 
seen in Fig.~\ref{fig:ff_fig1}. The errors are dominated by systematics 
even at the highest $Q^2$ as a result of the increased luminosity limit 
from the upgraded detectors. 

\subsection{Electromagnetic Transition Form Factors}
\label{sec:tranff}

As the first step in the {\tt CLAS12} $N^*$ program, we plan to 
determine in combined analysis of the single and double pion
electroproduction, the electrocouplings for almost all well established 
$N^*$ states at high $Q^2$ from 5 to 10~GeV$^2$. Together, $1\pi$ and
$2\pi$ processes account for a major part of the total meson 
electroproduction cross section in the $N^*$ excitation region. 
Moreover, they are strongly coupled due to hadronic FSI~\cite{Mo06}. 
Therefore, data on 1$\pi$ and 2$\pi$ production amplitudes are needed for 
credible evaluation of $N^*$ electrocouplings in any meson electroproduction 
channel.  Phenomenological studies have clearly shown that in both of these 
major channels, the resonant/non-resonant amplitude ratio increases with 
$Q^2$~\cite{Pr06}.

The {\tt CLAS} Collaboration has already produced the most extensive,
high quality data available for 1$\pi$ and 2$\pi$ electroproduction
\cite{Bur1,Bu05}. Two phenomenological approaches
\cite{Az03a,Az03b} were developed for extraction of the $N^*$ 
electrocouplings from the single pion data.  In Ref.~\cite{Az03a} the 
electrocouplings were determined based on almost model-independent 
dispersion relations. The Unitary Isobar Model~\cite{Az03b} enables a 
good description of all multipoles with $l<4$ up to $W$=2.0~GeV. In 
Fig.~\ref{fig:p11d13states} we present the $P_{11}(1440)$ and 
$D_{13}(1520)$ electrocouplings for the first time determined from 
{\tt CLAS} data on single pion electroproduction. It was shown in 
Ref.~\cite{Pr06}, that the approaches of Refs.~\cite{Az03a,Az03b} may 
be extended for analysis of the data at high $Q^2$ available with an 
11~GeV beam.

%%%%%%%%%%%%%%%%%%%%%%%%%%%%%%%%%%%%%%%%%%%%%%%%%%%%%%%%%%%%%%%%%%%%%%%%%%%%%
\begin{figure}[ht]
\vspace{6.0cm}
\special{psfile=../formfactors/epsfigs/p11_1440.eps hscale=38 vscale=31 hoffset=40 voffset=0} 
\special{psfile=../formfactors/epsfigs/d13_1520.eps hscale=38 vscale=31 hoffset=235 voffset=0} 
\caption{\small{Helicity amplitudes for the $P_{11}(1440)$ (left) and  
$D_{13}(1520)$ (right) states from analysis of {\tt CLAS} $1\pi$ (red 
points) and $2\pi$ (blue points) data. Also shown are the combined 
1$\pi$/2$\pi$ data at $Q^2$=0.65~GeV$^2$ (magenta points). The PDG
photocouplings are in black.}}
\label{fig:p11d13states}
\end{figure}
%%%%%%%%%%%%%%%%%%%%%%%%%%%%%%%%%%%%%%%%%%%%%%%%%%%%%%%%%%%%%%%%%%%%%%%%%%%%%

The measurements of the $\pi^+\pi^-p$ exclusive channel with {\tt CLAS}  
provided for the first time an entire set of unpolarized single-differential 
cross sections, and consisted of 9 single-differential cross sections in each 
($W$,$Q^2$) bin ~\cite{Mo06,clas3pi}.  The JLab-MSU collaboration has 
developed~\cite{Mo06,Bu04} a phenomenological model (JM06) with tree-diagram 
mechanisms and additional phenomenological terms to analyze $\pi^+\pi^-p$ 
electroproduction data sets from the available world data. Several new 
mechanisms were established for the first time. A reasonable description 
of the {\tt CLAS} and world data was achieved. $P_{11}(1440)$ and 
$D_{13}(1520)$ electrocouplings determined from analysis of recent {\tt CLAS} 
2$\pi$ data are shown in Fig.~\ref{fig:p11d13states}.  They are consistent 
with those extracted from the $1\pi$ data, as well as with those obtained in 
combined 1$\pi$/2$\pi$  analysis~\cite{Az05}.  Analysis of the data on 
2$\pi$ electroproduction allowed us for the first time to obtain accurate 
information on electrocouplings for several high-lying states, which decay 
preferably with 2$\pi$ emission.  The JM06 approach may be used for the 
evaluation of $N^*$ electrocouplings in still unexplored areas of high $Q^2$.

%%%%%%%%%%%%%%%%%%%%%%%%%%%%%%%%%%%%%%%%%%%%%%%%%%%%%%%%%%%%%%%%%%%%%%%%%%%%%
\begin{figure}[htbp]
\vspace{6.0cm}
\special{psfile=../formfactors/epsfigs/d13_high.epsi hscale=37 vscale=30 hoffset=40 voffset=-40}
\special{psfile=../formfactors/epsfigs/f15_high.epsi hscale=37 vscale=30 hoffset=225 voffset=-40} 
\caption{\small{Projected $A_{1/2}$ electrocouplings for the $D_{13}(1520)$
and $F_{15}(1685)$ states (open circles) expected from {\tt CLAS12} 
1$\pi$/2$\pi$ data.  Preliminary results from {\tt CLAS} 1$\pi$ data (blue) 
and combined 1$\pi$/2$\pi$ analysis at $Q^2$=0.65~GeV$^2$~\cite{Az05} (red) 
are shown. Previous world data are shown in black.}}
\label{fig:f15projection}
\end{figure}
%%%%%%%%%%%%%%%%%%%%%%%%%%%%%%%%%%%%%%%%%%%%%%%%%%%%%%%%%%%%%%%%%%%%%%%%%%%%%

A combined fit of all observables in 1$\pi$/2$\pi$ exclusive channels within 
the framework of a coupled-channels formalism would enable us to determine 
the most reliable data on $N^*$ electrocouplings.  We are going to carry out 
a coupled-channels analysis within the framework of the most advanced 
approach~\cite{Lee06}, for the first time fully accounting for FSI complexity 
both in the 2-body and 3-body final states.

Extensive simulation of $1\pi$ and $2\pi$ electroproduction measurements 
with {\tt CLAS12} were carried out in Ref.~\cite{Pr06}. It was found that 
a 60-day measurement would provide observables of the same or better 
quality than those measured with the {\tt CLAS} detector at lower photon
virtuality. The expected $A_{1/2}$ electrocouplings for the $D_{13}(1520)$ 
and $F_{15}(1685)$ states are shown in Fig.~\ref{fig:f15projection}.
