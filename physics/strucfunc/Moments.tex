\section{Moments of Structure Functions}

Moments of structure functions provide powerful insight into nucleon 
structure.  Inclusive data at JLab have permitted evaluation of the moments 
at low and intermediate $Q^2$~\cite{Fatemi:2003yh,Yun:2002td,Chen:2005td}.  
With a maximum beam energy of 6~GeV, however, the measured strength of the 
moments becomes rather limited for $Q^2$ greater than a few GeV$^2$. The 
12-GeV upgrade removes this problem and allows for measurements to higher 
$Q^2$. 

Moments of structure functions can be related to the nucleon static properties 
by sum rules. At large $Q^2$ the Bjorken sum rule relates $\int g_1^{p-n} dx$ 
to the nucleon axial charge~\cite{Bjorken:1966jh}.  At the other end of the 
spectrum, $Q^2$ = 0, the Gerasimov-Drell-Hearn (GDH) sum rule links the 
difference of spin dependent cross sections, integrated over photon energy, 
to the anomalous magnetic moment of the nucleon~\cite{Drell:1966jv,
Gerasimov:1965et}.  The two sum rules are aspects of a general one derived 
recently by Ji and Osborne~\cite{Ji:1999mr} that is valid at any $Q^2$
and links the first moments of spin structure functions to spin-dependent 
Compton amplitudes. Low $Q^2$ is a testing ground for chiral perturbation 
theory, while large $Q^2$ data can be compared to higher-twist series derived 
within the operator product expansion (OPE) method.  Lattice QCD can calculate 
higher-twist terms, thus extending the validity domain of OPE to lower $Q^2$. 
However OPE is unusable at low $Q^2$.  To bridge the gap, lattice QCD can be 
used to compute Compton amplitudes at any $Q^2$.  Hence, the Ji and Osborne 
sum rule can be computed and compared to experiments at any $Q^2$, making it a 
unique tool to study the transition from partonic to hadronic degrees of 
freedom.

%%%%%%%%%%%%%%%%%%%%%%%%%%%%%%%%%%%%%%%%%%%%%%%%%%%%%%%%%%%%%%%%%%%%%%%%%%
\begin{figure}[htb]
\centerline{\epsfxsize=5in\epsfbox{../strucfunc/expect.eps}}
\caption{\small{Left plot: expected precision on $\Gamma_1^p$ for {\tt CLAS12}
and 30~days of running.  {\tt CLAS} EG1a~\cite{Fatemi:2003yh, Yun:2002td} data 
and preliminary results from EG1b are shown for comparison.  The data and 
systematic uncertainties do not include estimates of the unmeasured DIS 
contribution.  HERMES~\cite{Airapetian:2002wd} data, and E143~\cite{Abe:1998wq} 
and E155 data~\cite{Anthony:2000fn} from SLAC are also shown (including DIS 
contribution estimates).  The model is from Burkert and Ioffe
\cite{Burkert:1992tg,Burkert:1993ya}.  Right plot: same as the left but 
including an estimate of the DIS contribution.}}
\label{expect}
\end{figure}
%%%%%%%%%%%%%%%%%%%%%%%%%%%%%%%%%%%%%%%%%%%%%%%%%%%%%%%%%%%%%%%%%%%%%%%%%%

The left plot in Fig.~\ref{expect} shows the expected precision on the 
measured part of $\Gamma_1^p$.  The inner error bar is statistical while the 
outer one is the statistics and systematics added in quadrature.  Published 
results and preliminary results from EG1b are also displayed for comparison. 
Like the {\tt CLAS12} data, the EG1 data do not include the unmeasured DIS 
contribution.  The hatched blue band corresponds to the systematic uncertainty 
on the EG1b data points.  The red band indicates the estimated systematic 
uncertainty from {\tt CLAS12}.  The right plot in Fig.~\ref{expect} shows the 
results on $\Gamma_1^p$ and $\Gamma_1^d$ including an estimate of the 
unmeasured DIS contribution.  The systematic uncertainties for EG1 and 
{\tt CLAS12} here include the estimated uncertainty on the unmeasured DIS part 
estimated using the model from Bianchi and Thomas~\cite{Thomas:2000pf}.  As 
can be seen, moments can be measured up to $Q^2$=6~GeV$^2$ with a statistical 
accuracy improved several fold over that of the existing world data.

Higher moments are also of interest: generalized spin polarizabilities
are linked to higher moments of spin structure functions by sum rules based 
on similar grounds as the GDH sum rule. Higher moments are less sensitive to 
the unmeasured low-$x$ part, so measurements are possible up to higher $Q^2$ 
compared to first moments.  Just like the GDH/Bjorken sum rules, measurements 
of the $Q^2$-evolution allow us to study the parton-hadron transition since 
theoretical predictions exist at low and large $Q^2$~\cite{Chen:2005td}.  In 
addition, spin polarizabilities are also fundamental observables characterizing 
the nucleon structure and the only practical way known to measure them is 
through measurement of moments and application of the corresponding sum rules.

Finally, moments in the low ($\simeq$ 0.5 GeV$^2$) to moderate 
($\simeq$4~GeV$^2$) $Q^2$ range enable us to extract higher-twist parameters,
which represent correlations between quarks in the nucleon.  This extraction 
can be done by studying the $Q^2$ evolution of first moments~\cite{Chen:2005td}.
Higher twists have been consistently found to have, overall, a surprisingly 
smaller effect than expected.  Going to lower $Q^2$ enhances the higher-twist 
effects but makes it harder to disentangle a high twist from the yet higher 
ones.  Furthermore, the uncertainty on $\alpha _s$ becomes prohibitive at low 
$Q^2$.  Hence, higher twists turn out to be hard to measure, even at the 
present JLab energies.  Adding higher $Q^2$ to the present JLab data set 
removes the issues of disentangling higher twists from each other and of the 
$\alpha _s$ uncertainty.  The smallness of higher twists, however, requires 
statistically precise measurements with small point-to-point correlated 
systematic uncertainties.  Such precision at moderate $Q^2$ has not been 
achieved by the experiments done at high energy accelerators, while JLab at 
12~GeV presents the opportunity to reach it considering the expected 
statistical and systematic uncertainties of E12-06-109.  The total 
point-to-point uncorrelated uncertainty on the twist-4 term for the Bjorken 
sum, $f_2^{p-n}$, decreases by a factor of 5.6 compared to results obtained in
Ref.~\cite{Deur:2004ti}. 

\subsection{The GDH Sum Rule}

Despite its fundamental nature, the GDH sum rule has not yet been fully 
verified experimentally.  Combined results from MAMI and ELSA~\cite{GDH04}
are about 10\% above the expected value.  This is for an upper integration 
limit in photon energy of about 2.8~GeV.  With the cancellation of the fixed 
target program at SLAC and consequently of experiment E159~\cite{SLACGDH} that 
would have investigated the GDH strength at large $\nu$, JLab is now the best 
place to test the convergence of the GDH sum.  Using real photons or 
near real photons, we can measure the contribution to the GDH sum rule up to 
10.5~GeV, about 4 times the maximum energy reached at ELSA (see 
Fig.~\ref{gdhf}).  A non-convergence of the sum rule would be intriguing 
and may signal physics beyond the Standard Model.  In any case it will provide 
important insight on soft Regge physics.

%%%%%%%%%%%%%%%%%%%%%%%%%%%%%%%%%%%%%%%%%%%%%%%%%%%%%%%%%%%%%%%%%%%%%%%%%%
\begin{figure}
\begin{center}
\begin{minipage}[t]{0.6\linewidth}
\centerline{\epsfxsize=4.in\epsfbox{../strucfunc/gdhprev.eps}}
\end{minipage}\hfill
\begin{minipage}[c]{0.35\linewidth}
\vspace*{-7cm}
\caption{\small{Coverage of the GDH sum rule for low-$Q^2$ experiments with 
{\tt CLAS} and {\tt CLAS12}.  The data points are the GDH running sum from 
MAMI and ELSA at $Q^2=0$.}}
\label{gdhf}
\end{minipage}
\end{center}
\end{figure}
%%%%%%%%%%%%%%%%%%%%%%%%%%%%%%%%%%%%%%%%%%%%%%%%%%%%%%%%%%%%%%%%%%%%%%%%%%

\subsection{Moments of $F_2$ and the Precise Determination of $\alpha_s(M_Z)$}

Simulated results for the moments $\int x^n F_2 dx$  with $n \leq 8$ are 
shown in Fig.~\ref{fig:moments}.  It reveals that {\tt CLAS12} will provide 
a unique tool to extract moments of $F_2$ up to $Q^2$ values of 
10 - 14~GeV$^2$.  These can be used to extract the strong coupling constant 
$\alpha_{s}(M_{z})$.  The extraction of $\alpha_s(M_Z)$ from the scaling 
violations of the proton structure function $F_2$ is one of the most precise 
methods available up to now (see Fig.~\ref{fig:alpha}).  Simulation shows that 
a new procedure for the extraction of $\alpha_s(M_Z)$~\cite{Simula:2003vf} 
together with the {\tt CLAS12} data can allow an unprecedentedly accurate 
determination of $\alpha_s(M_Z)$ with a statistical uncertainty of 0.0008 and 
a systematic uncertainty of about 0.0007.

%%%%%%%%%%%%%%%%%%%%%%%%%%%%%%%%%%%%%%%%%%%%%%%%%%%%%%%%%%%%%%%%%%%%%%%%%%
\begin{figure}[htbp]
\vspace{8.0cm}
\special{psfile=../strucfunc/fig_moments.ps hscale=45 vscale=45 hoffset=100 voffset=-70}
\caption{\small{Expected moments of the proton structure function $F_2$ 
obtained with the {\tt CLAS12} detector simulations for a few days of 
running.  The meaning of the markers and the scale factors for each moment 
are indicated in the inset.}}
\label{fig:moments} 
\end{figure}
%%%%%%%%%%%%%%%%%%%%%%%%%%%%%%%%%%%%%%%%%%%%%%%%%%%%%%%%%%%%%%%%%%%%%%%%%%

%%%%%%%%%%%%%%%%%%%%%%%%%%%%%%%%%%%%%%%%%%%%%%%%%%%%%%%%%%%%%%%%%%%%%%%%%%
\begin{figure}[htbp]
\vspace{10.5cm}
\special{psfile=../strucfunc/alpha_s.epsi hscale=65 vscale=62 hoffset=30 voffset=-100}
\caption{\small{Existing determinations of the QCD coupling constant 
$\alpha_S(M_Z)$~\cite{alpha}.}}
\label{fig:alpha} 
\end{figure}
%%%%%%%%%%%%%%%%%%%%%%%%%%%%%%%%%%%%%%%%%%%%%%%%%%%%%%%%%%%%%%%%%%%%%%%%%%

