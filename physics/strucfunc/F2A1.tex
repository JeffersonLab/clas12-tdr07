\chapter{Parton Distributions at Large $x$}

\section{Parton Distributions} 
\label{F2A1}

Knowledge of parton distributions forms the basis of our understanding
of hadronic matter in terms of its fundamental constituents.  However, 
extracting parton distributions in the large $x$ domain is notoriously 
difficult.  One difficulty comes from the requirement to stay away from
the region where partonic initial and final state interactions play 
important roles.  This imposes a minimum $Q^2$ threshold of at least a few 
GeV$^2$ and $W$ greater than a few GeV in addition to the large $x$ constraint.
For polarized parton distributions, an additional problem stems from the lower 
luminosity available with polarized targets.  In the unpolarized case, 
luminosities are adequate and free proton targets exist, so proton data are 
satisfactory.  For the neutron no such targets exist and deuteron is used. 
The difficulty is to control sufficiently well nuclear final state 
interactions (FSI) and binding effects to allow for a systematically accurate 
extraction of the neutron information. 

%%%%%%%%%%%%%%%%%%%%%%%%%%%%%%%%%%%%%%%%%%%%%%%%%%%%%%%%%%%%%%%%%%%%%%%%%%%
\begin{figure}[htbp]
\vspace{6.5cm}
\special{psfile=../strucfunc/largex.eps hscale=75 vscale=75 hoffset=10 voffset=-205}
\caption{\small{World data on the $d/u$ parton distribution ratio from 
unpolarized measurements of $F_2^n/F_2^p$ (left) and on the photon asymmetry 
$A_1^n$ from polarized data (right).  The substantial systematic (left) or 
statistical (right) errors at large $x$ do not permit constraints of the 
various predictions.}}
\label{fig:largex} 
\end{figure}
%%%%%%%%%%%%%%%%%%%%%%%%%%%%%%%%%%%%%%%%%%%%%%%%%%%%%%%%%%%%%%%%%%%%%%%%%%%

The consequences of these difficulties can be readily seen in 
Fig.~\ref{fig:largex}.  The lack of experimental constraints allows for a 
variety of predictions that needs to be sorted out to establish the proper
phenomenology. Studies of parton distributions have already been undertaken 
at JLab, with $x \leq 0.6$. The BONUS experiment~\cite{BONUS} gathered 
neutron data from quasi-free neutrons within unpolarized deuterons.  The 
FSI and binding effects were minimized by measuring recoiling protons and 
selecting events for which the two nucleons were not interacting.  On the 
polarized data front, data were collected up to $x \sim 0.6$ in Halls A and 
B~\cite{Zheng:2004ce, Dharmawardane:2006zd}.  An exciting outcome is the 
apparent failure of leading order (LO) pQCD to describe the data, hinting 
that the validity domain of LO pQCD is not reached yet or that quark orbital 
momentum (an important but experimentally elusive contribution to the nucleon 
spin) may be sizable. In particular, LO pQCD predicts that the $d$-quark 
polarization (lower panel Fig.~\ref{fig:E99117}) should become positive and 
approach +1 at large $x$, which is in contradiction to the data.

%%%%%%%%%%%%%%%%%%%%%%%%%%%%%%%%%%%%%%%%%%%%%%%%%%%%%%%%%%%%%%%%%%%%%%%%%%% 
\begin{figure}
\vspace{9.0cm}
\special{psfile=../strucfunc/delq_new1.eps hscale=70 vscale=60 hoffset=80 voffset=-15}
\caption{\small{Quark polarizations $( \Delta u+ \Delta\bar{u})/(u+\bar{u})$ 
and $( \Delta d+ \Delta\bar{d})/(d+\bar{d})$ as extracted from $A_1^n$, 
$A_1^p$, and $A_1^d$~\cite{Zheng:2004ce, Dharmawardane:2006zd}.  Several NLO 
fits to previous measurements are shown, while the leading-order pQCD 
predictions require both polarizations to tend to +1 as $x \to 1$.}}
\label{fig:E99117}
\end{figure}
%%%%%%%%%%%%%%%%%%%%%%%%%%%%%%%%%%%%%%%%%%%%%%%%%%%%%%%%%%%%%%%%%%%%%%%%%%% 

An 11-GeV beam will allow us to reach $x \sim 0.8$. The luminosity expected 
with an unpolarized deuterium gas target needed for the recoil proton tagging 
method is $2 \times 10^{34}$~cm$^{-2}$s$^{-1}$.  It will be about 
$10^{35}$~cm$^{-2}$s$^{-1}$ for the polarized target currently under design. 
Those luminosities and the large solid angle of {\tt CLAS12} makes it a 
superior choice to measure parton distributions at large $x$, sort out 
mechanisms of SU(6) symmetry breaking and of quark-hadron duality, and explore 
the role of quark orbital momentum.  In addition to their intrinsic interest, 
quark distributions at large $x$ are crucial for studying physics beyond the 
Standard Model at high energy colliders~\cite{Kuhlmann:1999sf}.
