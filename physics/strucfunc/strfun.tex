\section{Inclusive Nucleon Structure Functions}

\subsection{Overview}

Polarized and unpolarized structure functions of the nucleon offer a
unique window on the internal quark structure of stable baryons.
The study of these structure functions provides insight into the two
defining features of QCD --- asymptotic freedom at small distances,
and confinement and non-perturbative effects at large distance scales.
After more than three decades of measurements at many accelerator
facilities worldwide, a truly impressive amount of data has been
collected, covering several orders of magnitude in both kinematic
variables $x$ and $Q^2$.
However, there are still important regions of the kinematic phase space
where data are scarce and have large errors and where significant
improvements are possible through experiments at Jefferson Lab with an
11~GeV electron beam.

One of the most interesting open questions is the behavior of the
structure functions in the extreme kinematic limit $x \rightarrow 1$.
In this region effects from the virtual sea of quark-antiquark pairs are 
suppressed, making this region simpler to model. This is also the region
where pQCD can make absolute prediction, in contrast with the general fact 
that pQCD provides only $Q^2$-dependences. However, the large $x$ domain of
pQCD is hard to reach because cross-sections are kinematically suppressed, 
the parton distributions are small and final states interactions (partonic or 
hadronic) are large. These reasons have forbidden the gathering of high data 
quality on polarized parton distributions and the neutron unpolarized parton 
distributions. A first step into the large $x$ domain became possible with 
5-6 GeV \cite{Zheng:2004ce,Dharmawardane:2006zd, BONUS}. The interest triggered by these
first results and the clear necessity to extend the program to larger $x$ 
provided one of the cornerstone of the JLab 12 GeV upgrade physics program.  


Complementary to measurements of structure function at large $x$, the
12~GeV upgrade will allow a detailed study quark-hadron duality. This 
phenomenon refers to the observation, first made by Bloom and Gilman 
\cite{BG}, that structure functions like $F_2^p$ measured in the resonance 
region, when suitably averaged over appropriate energy intervals, closely
follows the scaling structure function measured at higher energies.
Duality is directly relevant to understanding the connection between the 
partonic and hadronic descriptions of strong interaction. Indeed duality 
can be viewed as the ability of using two languages -the partonic one
and hadronic one- to describe the lepton-nucleon scattering in the 
resonance region. Jefferson Lab data taken at the current energies have 
provided with the first accurate measurements of $F_1^n$, $g_1$ and $g_2$ in 
the resonance domain \cite{Dharmawardane:2006zd, d2n, BOSTED EG1b, BONUS, E01-012}. 
The 12 GeV upgrade will complete this program since, for precise 
verification of duality, data of similar accuracy are required in the 
DIS domain. If duality holds and if it is theoretically understood, it 
can be used as a mean to extend the large $x$ study to value that 
could not be obtained otherwise.


The concept of duality is also closely related to the contribution of
higher twist effects in the nucleon structure functions. These higher twist
effects are due to quark-quark and quark-gluon correlations, and their 
smallness is a condition for duality to hold. To determine higher twist
effects precisely, high quality data on unpolarized and polarized structure 
functions is needed. data at low and intermediate $Q^2$ were provided with
the 6 GeV beam but it is clear from the higher twist extraction studies that
extracting the full information from higher twist reacquires data of similar 
accuracy at higher $Q^2$. An improved data sample in this region would also 
improve perturbative QCD analyzes by increasing the $Q^2$ range covered, 
leading to a better constraint on gluons and $\alpha _s$


The CLAS12 detector will allow significant
contributions to be made to these studies, particularly in two cases:
%
\begin{itemize}
\item Measurements of the neutron structure function $F_2^n$ in the region
of very large $x$, where we will employ a technique recently established 
with CLAS (recoil proton detection \cite{BONUS}) to eliminate contamination 
from nuclear effects \cite{BONUS12}. 

\item Measurements of polarized structure functions of the proton and
deuteron in the region of moderate to high $x$ \cite{EG12}.
\end{itemize}
%
In both cases, the maximum beam current is limited by backgrounds or
target depolarization issue, which makes CLAS12 an optimal detector for
such experiments since the relatively low luminosity is largely compensated 
by its very large acceptance.

\subsection{Structure Functions at Large $x$}
\input{../strucfunc/largex.tex}
\subsection{Moments of Structure Functions}
\input{../strucfunc/moments.tex}









