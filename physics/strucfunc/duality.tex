\subsection{Quark-Hadron Duality} 
\label{duality}

The phenomenon of quark-hadron duality relates the physics of nucleon 
resonances to the dynamics of single quark scattering that governs the 
scaling structure functions at high energy.  JLab measurements of the 
unpolarized proton structure functions in the resonance region 
\cite{Niculescu:2000tk,ERIC} have sparked considerable interest in 
quark-hadron duality~\cite{Bloom:1970xb,Bloom:1971ye,DeRujula:1976tz,
Isgur:2001bt}.  While quark-hadron duality has been observed in the
spin-independent $F_2$ structure function~\cite{Bloom:1970xb,Bloom:1971ye,
Niculescu:2000tk}, it has not yet been firmly established for spin-dependent 
structure functions.  Because the $g_1$ structure function is given by a 
difference of cross sections, which need not be positive, the workings of 
duality will necessarily be more intricate for $g_1$ than for the
spin-averaged $F_2$ structure function.  The first results from the spin 
structure function measurements in Hall A~\cite{Amarian:2003jy,E01-012} and 
with {\tt CLAS}~\cite{Bosted:2006gp} indicate that, if one averages over 
the entire resonance region, duality roughly holds for $Q^2$ above 
1.5~GeV$^2$. To achieve a more precise understanding of the mechanism of 
duality, it is necessary to determine the conditions under which duality 
occurs in both polarized and unpolarized structure functions.  An upgraded 
{\tt CLAS12} would permit measurements of the structure functions in the 
DIS region with high precision.  For unpolarized parton distributions, the 
tagging technique used by the BONUS experiment~\cite{BONUS} (see 
Section~\ref{F2A1}) can also be applied to the neutron resonance region, 
where at present there are essentially no data.


