\section{GPD Studies with a Transversely Polarized Target}

Asymmetries in the exclusive production of photons and vector mesons with a
transversely polarized target were identified as the most sensitive 
observables providing access to the total orbital angular momentum.  Eight 
observables, namely the first harmonics $\cos \phi$ and $\sin \phi$ of the 
interference term, are accessible in polarized beam and target experiments 
\cite{Belitsky:2001ns}.  Thus, experiments with both longitudinally and
transversely polarized targets can measure all eight Fourier coefficients
$c_{1,{\mit\Lambda}}^{\cal I}$ and $s_{1,{\mit\Lambda}}^{\cal I}$ and
with ${\mit\Lambda} = \{ {\rm unp}, {\rm LP}, {\rm TP}_x, {\rm TP}_y \}$. 

The DVCS single spin asymmetry (SSA) for a transversely polarized target is 
the most sensitive observable to the elusive GPD $\cal E$, providing access 
to the orbital angular momentum.  First results on DVCS single spin asymmetries 
from the HERMES Collaboration for transverse target polarizations~\cite{HERAUT} 
indeed indicate great sensitivity of target single spin asymmetries to the 
contribution of $u$-quarks to the total angular momentum.  The most 
sensitive to the GPD-$\cal{E}$ asymmetry appeared to be the $\cos \phi$ moment 
of the spin-dependent contribution $\sigma_{UT}$~\cite{Belitsky:2001ns},

\begin{equation}
\label{AUT}
\sigma_{UT} \sim \frac{1-x}{2-x}\frac{t}{M^2}F_2{\cal H}
+ \frac{t}{4M^2}(2-x)F_1{\cal E}.
\end{equation}

Transverse target DVCS SSA measurements in addition to unpolarized SSA and 
longitudinally polarized SSA measurements will provide the full set of data 
needed for the extraction of Compton form factors and corresponding GPDs. 
$A_{UT}$ is especially sensitive to the GPD $\cal E$, and as such will
constrain any extraction of the angular momentum $J$.

The projection curves for {\tt CLAS12} running with a transversely polarized 
target have been calculated assuming a luminosity of 
$5 \times 10^{34}$~cm$^{-2}$s$^{-1}$, with an $NH_3$ target polarization of 
85\% and a dilution factor 0.14, with 2000 hours of data taking and an 
overall efficiency 50\% (see Fig.~\ref{fig:autdvcs}).

%%%%%%%%%%%%%%%%%%%%%%%%%%%%%%%%%%%%%%%%%%%%%%%%%%%%%%%%%%%%%%%%%%%%%%%%%%%%
\begin{figure}[htbp]
\vspace{7.0cm}
\special{psfile=../der/autproj11.eps hscale=45 vscale=55 hoffset=-30 voffset=-10}
\special{psfile=../der/rhoasym_cdr.eps hscale=45 vscale=40 hoffset=210 voffset=-10}
\caption{\small Projected transverse spin asymmetry ($A_{UT}^{\sin\phi}$)
in exclusive photon production at 11~GeV.  All points correspond to different 
values of $J_u$ calculated for the bin with $<Q^2>$=2.6~GeV$^2$ and 
$<x>=0.25$ (left). Projections for the transverse target asymmetry for exclusive
$\rho^0$ production from a hydrogen target (filled squares) using {\tt CLAS12} 
are shown compared to preliminary HERMES data~\protect\cite{hermesrho}.}
\label{fig:autdvcs}
\end{figure}
%%%%%%%%%%%%%%%%%%%%%%%%%%%%%%%%%%%%%%%%%%%%%%%%%%%%%%%%%%%%%%%%%%%%%%%%%%%%

The theoretical uncertainty in the factorization procedure on the amplitude 
level for the meson sector is translated into large variations of the physical 
cross section.  However, in the single-spin asymmetry, given by the ratio
of the Fourier coefficients of the cross section, the ambiguities 
approximately cancel~\cite{Belitsky:2003tm}.  Thus, the perturbative 
predictions for this quantity are rather stable. The NLO effects result in 
${}^{+ 7 \%}_{- 18 \%}$ corrections to the LO prediction for 
$0.1 < \xbj < 0.5$.

The GPD-based calculations were performed for the case when the incoming 
virtual photon is longitudinally polarized.  The cross section for the 
transversely polarized photons is suppressed by a power of $Q$
\cite{Collins:1996fb}, but at {\tt CLAS} energies it may still be significant. 
Insensitivity to the higher-order corrections make single spin asymmetries
appropriate quantities for experimental study at JLab, and will provide an 
important test of the applicability of GPD-based predictions at JLab energies.

Even though the power corrections for the absolute cross section of exclusive
meson electroproduction analyzed in terms of generalized parton distributions
are expected to be large, there are indications of a {\it precocious scaling} 
in the ratios of observables~\cite{Belitsky:2003tm}.  The measurement of spin 
asymmetries could therefore become a major tool for studying GPDs in the 
$Q^2$ domain of a few GeV$^2$. Projections for {\tt CLAS12} for measurements 
of transverse asymmetries for vector mesons are shown in 
Fig.~\ref{fig:autdvcs}.  The transverse asymmetries for $\rho$ mesons (neutral 
and charged) are widely accepted as an important source of independent 
information on the GPD $\cal{E}$.  SSA measurements in hard exclusive processes 
will allow mapping of the underlying GPDs and will provide access to the 
orbital angular momentum of quarks.

The quark angular momentum in the nucleon, $J_q$, can be estimated if one 
uses the results of measurements of DVCS and meson production observables to 
constrain GPD parameterizations, which incorporate information about GPDs 
obtained from other processes (inclusive DIS, form factors). These 
parameterizations allow one to estimate the second moment of the GPDs, based 
on the information about the GPDs at $x_1 = x_2$ and the momentum fraction 
integrals probed in the observables. Fig~\ref{fig:autdvcs} shows the
constraints on $J_u$ and $J_d$ from DVCS and $\rho$ asymmetries, which is 
particularly sensitive to the Pauli form factor-type GPD, $\cal E$.  One sees 
that accurate measurements of the asymmetries will be able to constrain $J_q$ 
in this way. While not fully model--independent, this method of extracting 
$J_q$ will become more and more accurate as amplitude calculations and GPD 
parameterizations become more refined as a result of measurements of a 
variety of other DVCS and meson production observables.  High statistics data 
will allow us to constrain the quark angular momentum in the proton.

The data from {\tt CLAS} with a transversely polarized target focussing
on hard exclusive photon and meson production, combined with data from 
unpolarized and longitudinally polarized targets, will provide a complete set 
of measurements required for the separation of all four leading-twist,
chiral-even GPDs, and in particular, will provide a constraint on the quark 
orbital angular momentum.

