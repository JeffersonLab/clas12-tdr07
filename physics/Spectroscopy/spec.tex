\chapter{Spectroscopy}

\section{Introduction}

Spectroscopy of hadrons (mesons and baryons) is one of the key tools
for studying the theory of strong interactions, Quantum Chromodynamics 
(QCD), in the non-perturbative ({\it i.e.} confinement) regime. Hadron 
spectroscopy has been an essential component of the physics program with 
{\tt CLAS}~\cite{clas3pi,clas-d,clas-p,Tho01,McN04,Pri04,Guo07,5st}. To 
date, a large amount of experimental data on electromagnetic production of 
hadrons has been collected by {\tt CLAS}. However, more data will be 
necessary to guide improvements in hadronic phenomenology and to compare 
with lattice QCD calculations.

The majority of the data obtained so far with {\tt CLAS} is restricted to
the lowest mass states formed with the lightest quarks, up, down, and 
strange. A complete picture of QCD in the strong-coupling (non-perturbative) 
regime requires extension of hadron spectroscopy studies to higher masses 
and/or higher transferred momenta. Fulfillment of this task requires 
upgrading the electron beam energy to about 12~GeV, along with the necessary 
upgrade of the detector package, with a large-acceptance spectrometer being 
an obvious choice for studies of multi-particle final states.

The key experiments in hadron spectroscopy that we plan for the upgraded 
{\tt CLAS} detector ({\tt CLAS12}) will study reactions produced by both 
quasi-real and virtual photons.  They include:

\begin{itemize}

\item Studies of high-mass mesonic states (consisting of ordinary mesons, 
hybrids, and mesons with exotic $J^{PC}$) using H$_2$ and light nuclear 
targets;

\item Higher mass baryon production, {\it e.g.} $\Sigma$ and $\Xi$ baryons; 

\end{itemize}

Along with traditional bremsstrahlung photon beams, we are planning to 
use quasi-real photons produced when electrons are scattered at very 
forward angles ({\it i.e.} scattering angles $<$1.5$^\circ$). We plan to 
use a small-angle forward electron tagger in coincidence with the detection 
of multi-particle final states with the {\tt CLAS12} detector to study 
electroproduction at $Q^2$ values of $<10^{-2}$~GeV$^2$.  
{\it Electroproduction at these very small values of $Q^2$ using 
unpolarized electrons is equivalent to photoproduction using partially 
linearly polarized photons}~\cite{Dom69}.

The physics program using the very small-angle electron scattering
facility will take advantage of polarized photons and relatively high
photon fluxes. The use of high precision, high intensity electron beams 
will allow us to achieve the required luminosities on very thin targets 
({\it i.e.} gas targets) without jeopardizing the signal to accidental 
ratio. In turn, this will allow detection of low-energy recoils ({\it e.g.} 
coherent scattering experiments) and spectators ({\it e.g.} scattering 
off of the neutron in the deuteron). High-flux, linearly polarized
photon beams, together with the use of the nearly $4\pi$ coverage for 
hadronic final states of {\tt CLAS12}, will allow the study of hadron 
spectroscopy in a competitive and complementary experimental environment 
to the already planned {\tt GlueX} coherent bremsstrahlung production 
experiment in Hall D.

\section{Physics Motivation}

Perhaps the most fundamental question of interest to hadron physicists
is that of understanding the mechanism of confinement. It has been more
than thirty years since QCD was postulated as the theory of strong
interactions.  While much progress has been made in understanding
perturbative phenomena, the non-perturbative regime, the regime of
hadrons, their excitations, and their couplings, has remained largely
impervious to our varied assaults.  Only recently, with improvements 
to calculations of lattice QCD, has it become possible to make 
predictions of the spectrum of hadrons~\cite{morningstar,mathur}
directly from QCD, based on very few parameters (such as the bare quark 
masses). New experimental efforts to determine the hadron spectra are 
timely and are important for theoretical progress in non-perturbative QCD.

In order for this lattice effort to make significant progress in addressing 
confinement, and in order for this investment to pay off, lattice 
calculations for the masses and couplings of baryons and mesons must be 
compared with information extracted from precision experiments. Lattice 
QCD is not only capable of studying masses of bound systems of quarks 
and gluons, but it can also give insight into a space-time picture of 
electromagnetic interactions of these systems through studying the $Q^2$ 
dependence of electro-excitation form factors. We can have confidence that 
lattice calculations are indeed simulating QCD only when it successfully
reproduces a wide range of hadron properties. Some of the precision
experiments needed have been, and are being, carried out at Jefferson
Lab, and at other facilities around the world.

While mesons and baryons may be viewed differently, their phenomenology 
reflects common aspects of strong interaction dynamics. Searching for 
mesons with exotic quantum numbers gives us an opportunity to capture 
gluons as constituent particles that have their own identity, along with 
quarks, in forming hadronic bound states. On the other hand, considering 
interactions between three quarks in a baryon, one finds that the presence 
of meson-type quark correlations may be crucial in describing baryon 
properties, reflecting fundamental features of the QCD vacuum. In addition, 
multi-quark configurations in baryons are possible.

A number of legitimate questions about why this research is important
might be asked. For instance, some may question the need for more
experiments in hadron spectroscopy, since many experiments have already
addressed some of the issues discussed here. However, the experimental
coverage is incomplete. Many individual experiments have been carried out
with the aim of addressing single aspects of hadron phenomenology. Like a
map drawn by many hands, the picture of hadrons that has emerged is
incomplete in some areas, and inconsistent in others.  The goal of further
experimentation in this area is compelling: to continue our efforts to
arrive, as far as possible, at a clear, complete, and consistent
description of hadrons and their properties.

In order to understand the dynamics of QCD in the confinement region, a 
systematic study of many states, including their couplings to other states, 
is needed. Many of the current experiments at JLab, in combination with a 
systematic analysis effort, will have (in principle) information primarily 
on non-strange baryons, the nucleons and the Deltas. This information is 
not sufficient for us to arrive at a complete and consistent picture of 
the dynamics of QCD in the confinement region. Information on hyperons is
a crucial element needed for constructing such a picture. For instance,
the SU(3) singlet $\Lambda$s play key roles in identifying the states of
the 70-plets and the 20-plets.  At present, only one paper from {\tt CLAS} 
on the excited $\Lambda^*$s~\cite{barrow} has been published, and 
this study was statistically limited by the low production cross 
sections at the currently available beam energies.  Other studies of the 
excited hyperon states are in progress at {\tt CLAS} and further work can 
be done with the availability of higher energies. 

Some information on hyperons can be extracted from ongoing experiments,
but the kinematic reach of the current JLab accelerator at present does 
not allow the kind of systematic study that is essential.  The higher 
energies provided by the upgraded facility will allow for more detailed 
analysis of the spectrum and interactions of the hyperons, through 
processes like $\gamma N \to K \bar{K} N$ and $\gamma N \to KK\Xi$. 

The proposed {\tt CLAS12} experimental program is in step with current 
developments in hadronic phenomenology and lattice QCD, where the main 
thrust is a comprehensive study of the spectrum of conventional hadrons,
along with hybrid and exotic mesons and baryons. 

\section{Meson Spectroscopy}

A complete mapping of meson resonances in the mass region of 1 to 3~GeV
will be particularly important for a better understanding of the QCD 
confinement mechanism. QCD predicts the existence of several new types 
of states beyond the naive quark model, {\it e.g.}: glueballs, hybrids, 
and multi-quark $q\bar{q} q\bar{q}$ states~\cite{Is85,Ko85}.  Gluons play 
a central role in strongly interacting matter -- quark confinement is due 
to gluonic forces. The clearest most fundamental experimental signature for 
the presence of dynamics of gluon degrees of freedom is the spectrum of 
gluonic excitations of hadrons. Gluonic excitations of mesons with 
``exotic'' quantum numbers, {\it i.e.} quantum numbers not accessible 
to the $q\bar{q}$ system, would be the most direct evidence for these
states. Determining the properties of such states would shed light on
the underlying dynamics of quark confinement.

The identification of these states has been difficult, as high-mass
resonances are generally broad and overlapping, and often have similar
quantum numbers (mixing). Photoproduction cross sections are small, so
statistics have been limited. Ideally, for a complete mapping of the
mesons in this mass region, we will need to study each resonance
through as many decay channels and production mechanisms as possible
in order to disentangle mixing. To determine meson quantum numbers, we
use partial wave analysis (PWA) (in a broad sense, fits to the angular
distributions of final states). A complete PWA requires high event
statistics, as well as high resolution and geometrical acceptance of
the detector. Meson spectroscopy at {\tt CLAS12}, using the low-$Q^2$ 
tagger, will fulfill many of these stipulations.

\subsection{Coherent Photoproduction on Light Nuclei}
\label{he4}

In the electromagnetic production of $t$-channel meson resonances
at moderate energies, the main physics background arises from
associated production of baryon resonances that decay into the same
final state particles.  Often these particles in both production reactions
occupy the same phase space, and therefore, it becomes impossible to
separate them using kinematic cuts. The contribution of baryon
resonances to the final state makes PWA analysis rather complicated. The 
production of meson resonances coherently on nuclear targets, when the
recoiling nucleus remains intact, is a {\it clean} way to eliminate
baryon resonances. A particular case of such processes is coherent 
production off of light nuclei, {\it e.g.} $^3$H, $^3$He, and $^4$He. For
these reactions, the recoiling nuclei can be detected in order to ensure 
that they remain intact.

We propose a program to study charged and neutral meson resonances in
the coherent production reactions on light nuclei,
$\gamma^* \,^3{\rm H}  \to \,^3{\rm He}\,M^-$, 
$\gamma^* \,^3{\rm He} \to \,^3{\rm H} \,M^+$, and
$\gamma^* \,^4{\rm He} \to \,^4{\rm He}\,M^0$, 
using the {\tt CLAS12} detector and the 12-GeV electron beam. 
{\it The key feature of these measurements is the detection of the 
recoiling nuclei}.

For meson masses from 1 to 3~GeV, the minimum transferred momentum, 
$t_{min}$, at beam energies up to 10~GeV, ranges from 0.02 to 0.2~GeV$^2$. 
At these transferred momenta, reduction of yields due to the nuclear form 
factor is expected to be only a factor of a few, while the energy obtained 
by the recoiling nucleus, 5 to 30~MeV for $^3$H, $^3$He, and $^4$He, will 
be enough to detect them using thin gas targets. Such experiments can 
only be conducted with high-intensity electron beams, where the low density 
of the target can be compensated for by a high flux of quasi-real photons. 
High precision electron beams, with sub-millimeter cross sections, will 
allow us to use a small diameter target cell, consequently to have thin 
windows at high pressure, a critical component for the detection of 
low-energy recoiling particles.  {\it Electron scattering at very small 
angles is a unique technique for experiments on thin targets.}

Besides the elimination of baryon resonances, coherent production on
nuclei has other advantages as well. In many cases it imposes constraints
on the allowed helicity states of the produced meson, and on the possible
exchange particles. These will significantly aid the PWA.

Examples of such reactions include coherent production of $\pi \eta$ and
$\pi \eta'$ final states on $^4$He~\cite{stephe4}.  The attractive
feature of these final states is that in the $P$-wave they have
exotic quantum numbers, $J^{PC}=1^{-+}$. Photoproduction of $\pi \eta$
and $\pi \eta'$ on the nucleon proceeds only via C-odd $\rho$ or
$\omega$ exchanges. Since $^4$He has isospin 0, only $\omega$ exchange
is allowed, and as $^4$He has spin 0, the helicity of the final state
(at small angles) should be equal to the helicity of the incoming photon 
(SCHC).

\subsection{Electroproduction on the Proton at Very Small Q$^2$} 

The general idea of PWA is to parameterize the intensity distribution
in the space of quantum numbers available to the observed final
states. The intensity distribution is written as a sum of interfering
and non-interfering amplitudes (partial waves). A maximum likelihood
fit is done to the intensity distribution by a set of given partial
waves and reasonable assumptions of the production mechanism.  The
goodness of the fit is related to the statistics (number of events per
binned data), the rank of the production matrix, and the number of
parameters to be fitted. The fit could then be improved by using
higher statistics or (equivalently) by reducing the rank of the fit by
having more information about the production mechanism.

The knowledge of photon polarization simplifies the PWA by giving
direct information on the production mechanism, and therefore, reducing
the rank of the fit. In electroproduction at very low $Q^2$, we will
be able to measure, on an event-by-event basis, the linear polarization
of the photons.

Spectroscopy studies of mesons have started with {\tt CLAS} at lower 
energies~\cite{Ad01}.  Preliminary results of these experiments show the 
viability of such studies using the current {\tt CLAS} configuration, 
where PWA of simple final states ($\pi\pi\pi$) has already been carried 
out successfully.  However, with the limited acceptance of the present 
{\tt CLAS} and only circular polarization of the electron beam, there are 
many analysis ambiguities created by the assumptions on the production 
mechanism and the baryon backgrounds. These ambiguities will be mostly 
resolved when using linearly polarized, high-energy (greater than 8~GeV) 
photons, and larger acceptances. At higher energies we will be able to map 
out the mesons in the interesting 1 to 3~GeV mass range and, most 
importantly, to better kinematically differentiate mesons from baryons. More
specifically, we plan to study mesons decaying into multiple final states 
({\it e.g.} $\rho \pi$, $\eta \pi$, $\phi \eta$, $b_1 \pi$, $KK^*$, $\cdots$).

\section{Baryon Spectroscopy and Structure}

Ground and excited baryon states provide a wealth of information on
non-perturbative QCD in addition to that available using mesons.  First, 
baryons are the simplest systems that manifest the non-abelian nature of 
QCD.  This results in a complex internal structure whose degrees of 
freedom and dynamics may depend on the distance scale probed.  Second, 
the predicted mass spectrum involves many transitions to a variety of 
spin-flavor multiplets (56, 70 and 20-plets) which partially overlap in 
excitation energy.  Some of these states have not yet been seen.  
Understanding and unraveling this rich structure requires experimental 
data of high precision using a variety of exclusive channels, final 
hadronic state invariant masses and photon virtualities.  The enhanced 
kinematic range available at 12~GeV will make feasible a study of the 
evolution of baryon structure and quark binding mechanisms in the 
transition between QCD strong coupling and asymptotic freedom.  

Recent experimental measurements of transition form factors~\cite{Bur1,Joo02} 
have provided a better understanding of the role played by the nucleon's 
meson cloud in baryon excitation at low to moderate $Q^2$.  For the
$N \to \Delta(1232)$ transition, there is evidence of strong longitudinal 
photocouplings and helicity non-conservation out to $Q^2$=6~GeV$^2$, which 
is consistent with significant meson cloud contributions.  There is also 
substantial evidence that meson-baryon interactions may dominate 
quark-gluon dynamics for $Q^2<1$~GeV$^2$ in the $P_{11}(1440)$ (Roper) 
and $D_{13}(1520)$ transitions.  This interpretation has been substantiated
by dynamical meson-baryon calculations~\cite{Lee04} and chiral quark models.
On the other hand, for the $S_{11}(1535)$ transition, the pion cloud plays 
a lesser role, and the transition form factor scales with $Q^2$ as expected 
from scattering off bare quarks.  

Computations using the Dyson-Schwinger equation~\cite{Bha03}, as well as 
explicit lattice QCD calculations~\cite{Bow02}, show a dynamical dressing 
of bare quarks which persists down to distances equivalent to 
$Q^2$=4~GeV$^2$.  Baryon structure physics at 6~GeV is thus limited to 
the regime where the meson cloud obscures the direct view of the quark 
and gluon structure in baryons.  A recent lattice calculation~\cite{Tak04} 
of the QCD action clearly shows a compact Y-shaped structure binding the 3 
valence quarks in the nucleon. To access the genuine gluon dominated QCD 
interaction thus requires a short distance probe, either high $Q^2$ for 
light quark systems, or the photoproduction of heavy quark baryons.

\subsection{Cascades}

The double-strangeness cascades have several advantages when it comes to 
baryon spectroscopy.  First, two of the three valence quarks are heavier 
($m_s \simeq$ 100~MeV) than light quarks, which reduces the uncertainties 
in extrapolations of lattice gauge calculations for the cascade mass.  
Second, the width of the cascades are typically about a factor of ten 
smaller than their $N^*$ counterparts.  Third, the detached decay vertices 
for many cascades allows experiments to more easily separate cascades 
from various backgrounds.  These advantages can only be utilized if there 
is a sufficient beam energy to produce the $\Xi$s in reasonable quantities.

Recently, experiments with the {\tt CLAS} detector have shown that the 
ground state $\Xi$ and the first excited $\Xi^*$ can be clearly seen
\cite{Pri04} in photoproduction from the proton.  The Particle Data Group
\cite{pdg} shows that there are many excited $\Xi^*$s with fairly narrow 
widths based on older bubble-chamber or hadronic-beam experiments.  However, 
in some cases, the data for $\Xi^*$s is not very consistent (some experiments 
that have reported a $\Xi^*$ should have seen other $\Xi^*$s in their mass 
window, but did not).  In addition, of the 22 $\Xi$ candidates expected in 
the SU(3) multiplets, only six are well-established.

Lattice gauge theory will be a useful theoretical tool to tell us the 
relative masses of the $\Xi^*$ states~\cite{jlab-lat1,jlab-lat2,jlab-lat3}.  
An important goal of lattice QCD is to correctly identify the quantum numbers 
of states, and especially to separate high-spin excitations, which is a 
particularly delicate problem in the baryon sector. It has been realized, for 
instance, by the LHP (lattice QCD) Collaboration, that studying the 
hadronic spectrum around the strange quark mass, instead of attempting to 
attain the lightest possible quark masses in the computations, allows one 
to reach the above goal in shorter times.  In addition it can take advantage 
of a more straightforward, and tractable, behavior of the states such as 
cascade baryons in quenched chiral perturbation theory. Unfortunately,
experimental information about these states is incomplete, and our
proposal aims at filling this gap.

Because the $\Xi^*$ states are expected to have narrow widths, the
experimental determination of their masses should be rather simple.
However, the cross section for cascade production decreases dramatically 
as the beam energy goes down~\cite{eg3}, and a systematic study of cascades 
is only possible with beams of 6~GeV or higher.  Data on $\Xi^*$s, which 
can be compared with the spectrum calculated from lattice QCD, will be 
greatly enhanced by the energy upgrade.  Furthermore, measurements of the 
various decay branches of the $\Xi^*$s will be useful to compare with 
theoretical models.  There is a richness of physics here that cannot be 
passed up.
 
\section{Summary}

The simultaneous study of both mesons and baryons using the upgraded 
{\tt CLAS12} detector with 12-GeV electron beams is our goal.  By using a 
small-angle forward electron tagger, a high flux of polarized photons 
is obtained. This allows thin gas targets to be used, making possible
detection of low-energy recoils and spectators.

The physics to be learned is motivated by a combination of lattice QCD 
and phenomenology.  Lattice calculations are useful for understanding of 
the non-perturbative structure of QCD vacuum, spectroscopy, and dynamics 
of bound systems of quarks and gluons.  At the same time, a consistent 
experimental program aimed at describing hadron properties requires a 
broad systematic approach (not limited only to quantities the lattice can
presently calculate). In order to understand the dynamics of confinement, 
many states and their decays must be studied.  Furthermore, a proper 
partial wave analysis should include both baryon and meson amplitudes 
applied to high statistics data over a wide range of kinematic phase space.  
This program is limited at the current beam energies, but will be hugely 
improved with the linearly polarized, quasi-real photons at {\tt CLAS12}, 
along with the energy upgrade.  Using a combination of lattice QCD and 
systematics, the physics of mass spectroscopy and possible exotic mesons 
and baryons can be investigated.  Individual experiments have been (are 
being) designed (and will be presented to the PAC) to show specifically 
how this program will be carried out and how the measurements are connected 
with the physics of confinement.
