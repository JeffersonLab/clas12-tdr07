\chapter{PAC Reviewed 12-GeV Proposal Overview}

The Jefferson Laboratory Program Advisory Committee (PAC) held  
meetings on August 21-26, 2006 (PAC30) and August 6-8, 2007 (PAC32).
These were the first two PACs to consider experimental proposals to 
use the base equipment currently planned for the 12-GeV upgrade.  In 
Hall B, 11 proposals and 6 letters-of-intent were presented for the 
new facility utilizing the {\tt CLAS12} detector system (note that
two of the letters-of-intent presented to PAC30 were defended as full
proposals to PAC32).

PAC reviewed the physics and equipment with respect to their
appropriateness for data taking with the base equipment of {\tt CLAS12}
during the first five-year running period with the 12-GeV beam.  The 
presentations covered a broad physics program that reflected almost a 
decade of discussions and preparations focussed on the physics 
possibilities opened up by the availability of 11 to 12~GeV electron 
beams at JLab and the planned experimental equipment.  The physics proposals 
presented reflect the main topics that provided the physics case for the 
12-GeV upgrade, which were extensively discussed during the PAC18, PAC23, 
and PAC27 meetings.

The core of the first five-year running period with {\tt CLAS12}
consists of experiments in the following areas:

\begin{itemize}

\item Generalized parton distributions and hard exclusive processes;

\item Semi-inclusive deep inelastic scattering and single-spin asymmetries;

\item Deep inelastic structure functions of the nucleon;

\item Elastic and transition form factors of the nucleon;

\item Experiments using the nucleus as a laboratory;

\item Hadron spectroscopy.

\end{itemize}

Each of these areas was represented by proposals and letters-of-intent 
at PAC30 and PAC32.  The experiments that were presented in each of 
these areas are discussed briefly below.  The proposals and letters of
intent are listed in Tables~\ref{pac_prop} and \ref{pac_loi},
respectively.  Fig.~\ref{beam_time} shows the current allocation of
approved beam time in each experimental Hall for the 12-GeV upgrade as a 
function of beam energy.

\vskip 0.5cm

\noindent
\begin{large}
$\diamond$ {\bf Generalized Parton Distributions and Exclusive Processes}
\end{large}

\vskip 0.3cm

\begin{small}
\noindent
1). {\bf PR12-06-108}: {\it Hard Exclusive Electroproduction of
$\pi^0$ and $\eta$ with CLAS12}

\vskip 0.2cm

\begin{footnotesize}
Spokespersons: K. Joo, V. Kubarovsky, P. Stoler, M. Ungaro, C. Weiss
\end{footnotesize}

\vskip 0.2cm

This experiment will provide $\pi^0$ and $\eta$ electroproduction
cross sections from the proton over a broad range of $Q^2$, $W$, and
$t$.  The data will have an important impact on the GPD program and 
will provide a means to constrain the axial GPD of the proton.  Both 
pseudoscalar meson channels are interesting because they provide 
information on the quark flavor components of the axial GPD.

\vskip 0.3cm

\noindent
2). {\bf PR12-06-119}: {\it Deeply Virtual Compton Scattering with
CLAS at 11 GeV}

\vskip 0.2cm

\begin{footnotesize}
Spokespersons: A. Biselli, H. Egiyan, L. Elouadrhiri, D. Ireland, 
M. Holtrop, W. Kim, F. Sabatie 
\end{footnotesize}

\vskip 0.2cm

GPDs are physical observables that can provide deep insight into the 
internal structure of nucleons.  DVCS is one of the cleanest processes 
to measure these functions.  This experiment extends the current DVCS 
program at {\tt CLAS} to higher energy, expanding the kinematic coverage 
in $x$ and $Q^2$, and allowing the structure of the nucleon to be probed 
at much smaller distance scales.
\end{small}

\vskip 0.5cm

\noindent
\begin{large}
$\diamond$ {\bf Semi-Inclusive Deep Inelastic Scattering}
\end{large}

\vskip 0.3cm

\begin{small}
\noindent
1). {\bf PR12-06-112}: {\it Probing the Proton's Quark Dynamics in
Semi-Inclusive Pion Production}

\vskip 0.2cm

\begin{footnotesize}
Spokespersons: H. Avakian, K. Joo, Z.E. Meziani, B. Seitz
\end{footnotesize}

\vskip 0.2cm

This experiment will perform an extensive study of the transverse 
momentum proton structure functions and fragmentation functions, which
arise from parton transverse momentum and its coupling to spin within 
the nucleon.  Due to immense theoretical progress in recent years in 
this area and that of GPDs, the true three-dimensional nature of the 
proton is beginning to be explored.

\vskip 0.3cm

\noindent
2). {\bf LOI12-06-108}: {\it Transverse Polarization Effects in Hard
Scattering at CLAS12}

\vskip 0.2cm

\begin{footnotesize}
Spokesperson: H. Avakian
\end{footnotesize}

\vskip 0.2cm

This experiment is designed to perform a program of measurements of
semi-inclusive deep inelastic scattering and of hard exclusive
processes using a transversely polarized proton target.  The use of
a transversely polarized target has been revealed in recent years to
be the ideal test bed for the exploration of quark orbital angular
momentum and its correlation with transverse spin, both within the
nucleon and in the fragmentation process.

\vskip 0.3cm

\noindent
3). {\bf PR12-07-107}: {\it Studies of Spin-Orbit Correlations with
a Longitudinally Polarized Target}

\vskip 0.2cm

\begin{footnotesize}
Spokesperson: H. Avakian
\end{footnotesize}

\vskip 0.2cm

This experiment will perform semi-inclusive measurements with a 
longitudinally polarized proton target to measure single and 
double-spin asymmetries for charged and neutral pion production.  These
data are sensitive to the helicity distributions and the transverse
momentum and polarization distributions of the different quark
flavors in the proton.

\vskip 0.3cm

\noindent
4). {\bf LOI12-07-101}: {\it Lambda Polarization in the Target
Fragmentation Region}

\vskip 0.2cm

\begin{footnotesize}
Spokespersons: H. Avakian, D.S. Carman
\end{footnotesize}

\vskip 0.2cm

This experiment will measure the beam-recoil polarization transfer
to the $\Lambda$ hyperon in semi-inclusive and exclusive processes
at large momentum transfer in the target fragmentation region.  These 
measurements are expected to provide information on the strange sea 
in the nucleon and may shed light on the proton spin puzzle.  They
will significantly improve the accuracy of previous measurements
and allow for decisive tests of models of the reaction mechanism.

\vskip 0.3cm

\noindent
5). {\bf LOI12-07-103}: {\it A Detailed Study of Semi-Inclusive Deep
Inelastic Pion Production on Unpolarized Proton and Deuteron Targets
with the CLAS12 Detector}

\vskip 0.2cm

\begin{footnotesize}
Spokesperson: X. Jiang
\end{footnotesize}

\vskip 0.2cm

This experiment involves a detailed study of hadron multiplicities,
their relative ratios, as well as studies of the azimuthal and transverse
momentum dependencies of semi-inclusive deep-inelastic production.
Data from electron beam energies of 6.6, 8.8, and 11~GeV will be used
to study single pion production from unpolarized proton and deuterium
targets.  The goal of this work is to firmly establish the kinematic
region over which semi-inclusive deep-inelastic scattering pion production
can be reliably interpreted to NLO QCD in terms of parton distributions and 
fragmentation functions.
\end{small}

\vskip 0.5cm

\noindent
\begin{large}
$\diamond$ {\bf Deep Inelastic Structure Functions of the Nucleon}
\end{large}

\vskip 0.3cm

\begin{small}
\noindent
1). {\bf PR12-06-109}: {\it The Longitudinal Spin Structure of the
Nucleon}

\vskip 0.2cm

\begin{footnotesize}
Spokespersons: D. Crabb, A. Deur, V. Dharmawardane, T. Forest, K.
Griffieon, M. Holtrop, S. Kuhn, Y. Prok
\end{footnotesize}

\vskip 0.2cm

This experiment will measure with high accuracy the $g_1$ spin structure 
functions of the proton and deuteron for higher $x$, where the valence
quark structure is dominant.  This determination of $g_1$ will test 
quark model predictions at large $x$ and constrain $\Delta g$ using 
NLO-based global parton analyses.  Measurements from the resonance 
region will be used to determine the lowest moments of $g_1$ over a
broad range in $Q^2$.

\vskip 0.3cm

\noindent
2). {\bf PR12-06-113}: {\it The Structure Function of the Free
Neutron at Large $x$}

\vskip 0.2cm

\begin{footnotesize}
Spokespersons: S. Bueltmann, S. Kuhn, H. Fenker, W. Melnitchouk,
M. Christy, C. Keppel, V. Tvaskis, K. Griffioen
\end{footnotesize}

\vskip 0.2cm

The ratio of the proton to neutron spin-independent structure
functions $F_{2n}/F_{2p}$ in deep-inelastic scattering is poorly known 
at large $x$ due to theoretical uncertainty in the free neutron structure 
function.  This experiment will minimize the effect of the nuclear 
corrections by insuring that the scattering occurs on an almost on-shell 
neutron by tagging the spectator proton.

\vskip 0.3cm

\noindent
3). {\bf LOI12-07-102}: {\it Tagged Neutron Structure Function in
Deuterium with CLAS12}

\vskip 0.2cm

\begin{footnotesize}
Spokesperson: S. Bueltmann, K. Griffioen, S. Kuhn
\end{footnotesize}

\vskip 0.2cm

This experiment represents an extension of PR-12-06-113 to extract the
neutron structure function by tagged deep inelastic scattering of
deuterium.  Here higher-momentum backward-going protons in the range
from 0.2 to 0.7~GeV will be selected as a way of identifying the off-shell 
neutrons in a regime where final state interactions are minimized.  The data 
are expected
to provide important insights into the EMC effect and to discriminate
between different models.
\end{small}

\vskip 0.5cm

\noindent
\begin{large}
$\diamond$ {\bf Elastic and Transition Form Factors of the Nucleon}
\end{large}

\vskip 0.3cm

\begin{small}
\noindent
1). {\bf PR12-07-104}: {\it Measurement of $G_M^n$ at High $Q^2$ with 
the Ratio Method on Deuterium}

\vskip 0.2cm

\begin{footnotesize}
Spokesperson: G. Gilfoyle
\end{footnotesize}

\vskip 0.2cm

This experiment will measure the ratio of quasi-elastic $en/ep$
scattering on the deuteron to determine the ratio of the neutron to 
proton magnetic form factors in the range of $Q^2$ from 2 to 14~GeV$^2$.  
With the well measured proton magnetic form factor, this experiment will 
enable the extraction of the much more poorly measured neutron magnetic 
form factor.  Its measurement is part of a broad effort to understand 
how nucleons are constructed from the quarks and gluons of QCD.  This 
form factor, along with the other elastic nucleon form factors $G_E^n$, 
$G_M^p$, and $G_E^p$, are provide key constraints on the generalized 
parton distribution functions.
\end{small}

\vskip 0.5cm

\noindent
\begin{large}
$\diamond$ {\bf Experiments Using the Nucleus as a Laboratory}
\end{large}

\vskip 0.3cm

\begin{small}
\noindent
1). {\bf PR12-06-106}: {\it Study of Color Transparency in Exclusive
Vector Meson Electroproduction off Nuclei}

\vskip 0.2cm

\begin{footnotesize}
Spokespersons: K. Hafidi, B. Mustapha, L. El Fassi, M. Holtrop
\end{footnotesize}

\vskip 0.2cm

In QCD color transparency would manifest itself by a reduced attenuation 
of hadron transmission in nuclear matter when it is in a point-like 
configuration.  Studies of this effect with vector mesons will provide
strong control of the initial state.  In this experiment the coherence length 
related to the production time will be fixed, shifting the focus solely on 
the transparency phenomenon.

\vskip 0.3cm

\noindent
2). {\bf PR12-06-115}: {\it Study of the Short Range Properties of
Nucleons at $Q^2 < 12$~GeV\,$^2$ Using $d(e,e'p)n$ Reactions with CLAS12}

\vskip 0.2cm

\begin{footnotesize}
Spokespersons: K. Egiyan, N. Grigoryan, L. Weinstein
\end{footnotesize}

\vskip 0.2cm

This experiment will search for the possible modification of the
properties of deeply bound nucleons and search for point-like
configurations of the nucleon.  These new data could help in
testing our knowledge of nuclear models and nucleon modification
over a broad kinematic range.  This experiment was deferred by
PAC30.

\vskip 0.3cm

\noindent
3). {\bf PR12-06-117}: {\it Quark Propagation and Hadron Formation}

\vskip 0.2cm

\begin{footnotesize}
Spokespersons: W. Brooks, K. Hafidi, K. Joo, G. Niculescu, I. Niculescu,
M. Holtrop, K. Hicks, L. Weinstein, M. Wood, G. Gilfoyle, H. Hakobyan
\end{footnotesize}

\vskip 0.2cm

The goal of this experiment is to characterize the hadronization process 
during the time the struck quark travels unconfined through the nuclear 
medium carrying a net color charge, and during the time when it travels 
as a pre-hadron through the nuclear medium.  This experimental program 
will study a variety of hadrons to characterize important and subtle 
aspects of hadron formation.
\end{small}

\vskip 0.5cm

\noindent
\begin{large}
$\diamond$ {\bf Hadron Spectroscopy}
\end{large}

\vskip 0.3cm

\begin{small}
\noindent
1). {\bf PR12-06-116}: {\it Nucleon Resonance Studies with CLAS12 in
the Transition from Soft to Partonic Physics}

\vskip 0.2cm

\begin{footnotesize}
Spokespersons: V. Burkert, R. Gothe, K. Joo, V. Mokeev, P. Stoler
\end{footnotesize}

\vskip 0.2cm

This experiment will study the excited baryon spectrum by measuring 
photocouplings of established baryon resonances as a function of $Q^2$, 
where it is expected that resonant signals rise relative to the 
backgrounds as $Q^2$ increases.  Photocouplings of these baryons will
provide information on the confining mechanism for three-quark hadrons
and will provide valuable input to ongoing coupled-channels analyses.  This 
experiment was deferred by PAC30.
\end{small}

%%%%%%%%%%%%%%%%%%%%%%%%%%%%%%%%%%%%%%%%%%%%%%%%%%%%%%%%%%%%%%%%%%%%%%%%%
\begin{table}[htbp]
\begin{center}
\begin{tabular} {||l|l|c|} \hline \hline
Experiment & Status & Beam Request \\ \hline
PR12-06-106: Study of Color Transparency in Exclusive     & Approved      & 40 days  \\
Vector Meson Electroproduction off Nuclei                 &               &          \\ \hline
PR12-06-108: Hard Exclusive Electroproduction             & Approved      & 120 days \\
of $\pi^0$ and $\eta$ with CLAS12                         &               &          \\ \hline
PR12-06-109: The Longitudinal Spin Structure of           & Approved      & 80 days  \\
the Nucleon                                               &               &          \\ \hline
PR12-06-112: Probing the Proton's Quark Dynamics          & Approved      & 60 days  \\
in Semi-Inclusive Pion Production                         &               &          \\ \hline
PR12-06-113: The Structure of the Free Neutron at         & Conditionally & 40 days  \\
Large $x$                                                 & Approved      &          \\ \hline
PR12-06-115: Study of the Short Range Properties of       & Deferred      & 32 days  \\
Nucleons at $Q^2 < 12$ GeV$^2$ Using $d(e,e'p)n$ Reactions&               &          \\
with CLAS12                                               &               &          \\ \hline
PR12-06-116: Nucleon Resonance Studies with CLAS12        & Deferred      & 60 days  \\
in the Transition from Soft to Partonic Physics           &               &          \\ \hline
PR12-06-117: Quark Propagation and Hadron Formation       & Approved      & 60 days  \\ \hline
PR12-06-119: Deeply Virtual Compton Scattering            & Approved      & 200 days \\
with CLAS at 11 GeV                                       &               &          \\ \hline
PR12-07-104: Measurement of $G^n_M$ at High $Q^2$ with    & Approved      & 56 days  \\
the Ratio Method on Deuterium                             &               &          \\ \hline
PR12-07-107: Studies of Spin-Orbit Correlations with      & Approved      & 103 days \\
a Longitudinally Polarized Target                         &               &          \\ \hline \hline
\end{tabular}
\caption{\small{Summary of experiments presented to JLab PAC30 and PAC32 
focussing on 12-GeV experiments and the base equipment that will be part of 
{\tt CLAS12}.}}
\label{pac_prop}
\end{center}
\end{table}
%%%%%%%%%%%%%%%%%%%%%%%%%%%%%%%%%%%%%%%%%%%%%%%%%%%%%%%%%%%%%%%%%%%%%%%%%

%%%%%%%%%%%%%%%%%%%%%%%%%%%%%%%%%%%%%%%%%%%%%%%%%%%%%%%%%%%%%%%%%%%%%%%%%
\begin{table}[htbp]
\begin{center}
\begin{tabular} {||l|l|c|} \hline \hline
Experiment & Status & Beam Request \\ \hline
LOI12-06-108: Transverse Polarization Effects in          & **            & ~~---    \\
Hard Scattering at CLAS12                                 &               &          \\ \hline
LOI12-07-101: Lambda Polarization in the Target           & **            & ~~---    \\
Fragmentation Region                                      &               &          \\ \hline
LOI12-07-102: Tagged Neutron Structure Function in        & **            & ~~---    \\
Deuterium with CLAS12                                     &               &          \\ \hline
LOI12-07-103: A Detailed Study of Semi-Inclusive Deep     & **            & ~~---    \\
Inelastic Pion Production on Unpolarized Proton and       &               &          \\
Deuteron Targets with the CLAS12 Detector                 &               &          \\ \hline \hline
\end{tabular}
\caption{\small{Summary of letters-of-intent presented to JLab PAC30 
and PAC32 focussing on 12-GeV experiments and the base equipment that 
will be part of {\tt CLAS12}.  For all LOIs, PAC strongly encouraged 
submission of a full proposal for review.}}
\label{pac_loi}
\end{center}
\end{table}
%%%%%%%%%%%%%%%%%%%%%%%%%%%%%%%%%%%%%%%%%%%%%%%%%%%%%%%%%%%%%%%%%%%%%%%%%

%%%%%%%%%%%%%%%%%%%%%%%%%%%%%%%%%%%%%%%%%%%%%%%%%%%%%%%%%%%%%%%%%%%%%%%%%
\begin{figure}[htbp]
\vspace{9.8cm} 
\special{psfile=beam_time.eps hscale=85 vscale=85 hoffset=20 voffset=-10}
\caption{\small{Current PAC approved experiment allocation (PAC days) for 
each of the experimental Halls for the upgrade as a function of beam energy.}}
\label{beam_time}
\end{figure}
%%%%%%%%%%%%%%%%%%%%%%%%%%%%%%%%%%%%%%%%%%%%%%%%%%%%%%%%%%%%%%%%%%%%%%%%%
