\chapter{Executive Summary}

In the past five decades many important discoveries have been made  
in electron-proton scattering experiments.  The finite size of the 
proton was measured in Hofstadter's pioneering experiment in which 
electrons of 188~MeV energy were elastically scattered off a hydrogen 
target~\cite{hofstadter}.  It demonstrated conclusively that the proton 
is not a Dirac particle but has a finite size. Hofstadter was awarded 
the Nobel prize in 1961 for this discovery.  The energy of Hofstadter's 
accelerator was not high enough to resolve the internal structure of the 
proton but it laid the groundwork for a vigorous research program of 
inclusive electron scattering.  It took another decade, and the 
construction of the powerful electron accelerator and the large magnetic 
spectrometers at the Stanford Linear Accelerator Center (SLAC), to ``see'' 
deep into the proton's interior.  At energies of 20~GeV, experimental 
groups led by Jerome Friedman, Henry Kendall, and Richard Taylor discovered 
``scaling'', i.e. the independence of deeply inelastic structure functions 
of the proton on the virtuality of the electromagnetic probe
\cite{scaling, scaling2}.  These results could only be interpreted in 
terms of electron scattering off point-like ``partons'' inside the proton. 
They were also a triumph for the quarks postulated earlier by theorists 
Murray Gell-Mann~\cite{gellmann} and George Zweig~\cite{zweig} as the 
fundamental building blocks for hadrons.  Gell-Mann received the Nobel 
prize in 1969. For the experimental discovery of the proton's quark structure 
Friedman, Kendall, and Taylor shared the Nobel prize in 1990.  The small 
but significant deviations from scaling that were observed in the SLAC 
experiments also had significant impact on the development of the theory 
of Quantum Chromodynamics (QCD), and are fully explained by the emission 
of gluons from the struck quarks. 

Deeply Inelastic Scattering (DIS) experiments, where only the scattered 
electron is detected, have been carried out up to the highest energies at 
CERN and at DESY. The longitudinal momentum and spin densities of the
quarks have been mapped out in detail.  We have also learned that the 
quarks are not the only tenants of the proton, but that more than 50\% of 
the momentum of the proton is carried by the glue needed to bind the quarks 
together.  More recently, inclusive polarized electron scattering off 
polarized protons led to the ``spin puzzle'', the finding that the quark 
spin contributes less than 25\% of the total spin of the proton, leaving 
much to be understood about the origin of spin~\cite{filipone}.  DIS 
experiments will continue to play an important role in further unraveling 
the valence quark structure, especially under extreme conditions, e.g. 
when one quark carries nearly the full proton momentum.

%%%%%%%%%%%%%%%%%%%%%%%%%%%%%%%%%%%%%%%%%%%%%%%%%%%%%%%%%%%%%%%%%%%%%%%%%
\begin{figure}[htbp]
\vspace{7.7cm} 
\special{psfile=handbag_GPD.ps hscale=33 vscale=33 hoffset=140 voffset=-15}
\caption{\small{The ``handbag'' diagrams for deeply virtual Compton 
scattering (a), and for deeply virtual meson production (b).  Four GPDs 
describe the ``soft'' proton structure part.  They depend not only on $x$, 
but on two more variables: the momentum imbalance of the quark before and 
after the interaction, $\xi$, and the momentum transfer to the proton, $t$.}}
\label{hand_bag}
\end{figure}
%%%%%%%%%%%%%%%%%%%%%%%%%%%%%%%%%%%%%%%%%%%%%%%%%%%%%%%%%%%%%%%%%%%%%%%%%

A glorious past and present is a good basis but not a guarantee for a 
successful future. So, what is the new physics that we are confident will 
shape the future of nuclear physics with electromagnetic probes for the 
coming decades?  While the major discoveries in electromagnetic physics 
have so far come from electron scattering experiments where only the 
scattered electron is measured in magnetic spectrometers, in particular 
measurement of elastic form factors and longitudinal parton densities, they 
are not sufficient to unravel the full structure and internal dynamics of 
the proton.  Semi-exclusive measurements, in which one hadron is observed 
in addition to the scattered electron, are needed to study its flavor 
structure, and only fully exclusive processes in which all final products 
are reconstructed can unravel the complete internal dynamics of the proton.
The experimental and theoretical tools for such an endeavor are now on the 
horizon: electron machines such as CEBAF at 12~GeV, with its CW beams 
and large acceptance detectors operating at high luminosities, are needed 
for the experimental part of such a program, while the new formalism of 
Generalized Parton Distributions (GPDs) provides the theoretical framework 
for the interpretation of the new experiments
\cite{mueller,Ji97_1,Ji97_2,Radyushkin:1996nd,Radyushkin:1997ki}.  The basis 
for this approach are the ``handbag'' diagrams shown in Fig.~\ref{hand_bag}.
Here the electron knocks a quark out of the proton by exchanging a deeply 
virtual (massive) photon. The quark then emits a high energy photon (a) and 
is put back into the proton. Alternatively, a $q\bar q$ pair (meson) is 
created, and one of the quarks is returned into the proton (b).   At 
sufficiently high energies and high virtuality of the exchanged photon 
(``Bjorken regime'') these processes are controlled by perturbative QCD, and 
the results can be interpreted in terms of ``soft'' correlation functions, the 
GPDs. They describe the full complexity of the proton's structure and 
dynamics.        

%%%%%%%%%%%%%%%%%%%%%%%%%%%%%%%%%%%%%%%%%%%%%%%%%%%%%%%%%%%%%%%%%%%%%%%%%
\begin{figure}[ht]
\vspace{8.5cm} 
\special{psfile=fig02-1.ps hscale=50 vscale=50 hoffset=-60 voffset=-55}
\special{psfile=fig02-2.ps hscale=50 vscale=50 hoffset=85 voffset=-58}
\special{psfile=fig02-3.ps hscale=50 vscale=50 hoffset=230 voffset=-65}
\caption{\small{Representations of the proton properties probed in elastic 
scattering (left), deeply inelastic scattering (center), and deeply 
exclusive scattering processes (right). Elastic scattering measures the 
charge density $\rho (b_{\perp})$ as a function of the impact parameter 
$b_{\perp}$.  DIS measures the longitudinal parton momentum fraction density 
$f(x)$. GPDs measure the full correlation function $f(x,b_{\perp},\xi)$ where 
$\xi$ represents the longitudinal momentum imbalance of the struck quark 
before and after the interaction. The graph shows the correlation function 
at $\xi=0$.}}
\label{fig:figure1}
\end{figure}
%%%%%%%%%%%%%%%%%%%%%%%%%%%%%%%%%%%%%%%%%%%%%%%%%%%%%%%%%%%%%%%%%%%%%%%%%

%%%%%%%%%%%%%%%%%%%%%%%%%%%%%%%%%%%%%%%%%%%%%%%%%%%%%%%%%%%%%%%%%%%%%%%%%
\begin{figure}[ht]
\vspace{11.0cm} 
\special{psfile=mb-dgpd.eps hscale=55 vscale=55 hoffset=70 voffset=-70}
\caption{\small{Simulated proton tomography images for the $d$-quarks, 
showing the strong correlation between the transverse size and the 
longitudinal momentum\cite{burkardt}.  For small quark momentum $x$, the 
proton has a large transverse size, and it becomes very dense at large $x$. 
Left column: unpolarized proton, right column: transversely polarized proton.}}
\label{fig:figure2}
\end{figure}
%%%%%%%%%%%%%%%%%%%%%%%%%%%%%%%%%%%%%%%%%%%%%%%%%%%%%%%%%%%%%%%%%%%%%%%%%

What can these experiments tell us about the proton beyond what previous  
experiments have?  Elastic scattering and deeply inelastic scattering give 
us two orthogonal one-dimensional projections of the proton: The quarks in 
the proton are subject to quantum fluctuations, resulting in variations of 
the proton size at a time scale of $< 10^{-23}$ seconds.  Elastic scattering 
measures the probability of finding a proton with a transverse size 
$b_\perp$ matching the resolution of the probe given by the momentum 
transfer $t$: $b_{\perp} \approx {1/\sqrt{|t|}}~$. The expression relates 
the momentum transfer to the transverse size of the proton probed in the 
interaction. Deeply inelastic scattering probes the longitudinal momentum 
distribution of the quarks, but has no sensitivity to the transverse 
dimension.  These two aspects are illustrated in the first two panels of 
Fig.~\ref{fig:figure1}~\cite{belitsky}.  The information resulting from 
these two types of experiments is disconnected, and does not allow us to 
construct the image of a real 3-dimensional proton.   

Deeply exclusive scattering processes connect both transverse and longitudinal 
information including their correlations as described by GPDs. This is shown 
in the third panel of Fig.~\ref{fig:figure1}. The GPDs now depend on 3 
dimensions ($x,\xi,t)$.  Once the GPDs are measured they allow the 
construction of a 3-dimensional image (two in transverse space and one in 
longitudinal momentum) of the proton in what has been called 
``nucleon-tomography''~\cite{burkardt}.  GPDs will allow the study of the
mass and angular momentum distributions of quarks, and the forces and 
pressure distributions in the proton.  

On very general grounds we expect a correlation between the transverse and 
longitudinal variables that, for example at $\xi=0$, could be of the form
\cite{burkardt}:

\begin{equation}
\label{introeq}
H_f(x,0,t) \approx q_f(x)\exp^{-a|t|(1-x) \ln \frac{1}{x}},
\end{equation}

\noindent
where $q_f (x)$ is the forward parton distribution of flavor $f$ and $a$ is
a scale parameter characterizing the transverse size. While the exact shape 
of this function needs to be determined experimentally, it must qualitatively 
contain the correlation between these parameters. Fig.~\ref{fig:figure2} 
illustrates the physical significance of Eq.(\ref{introeq})~\cite{burkardt}. 
The graphs show the strong correlation between the $t$-dependence (transverse 
size $b_{\perp}$) and the $x$-dependence (longitudinal momentum).  For the 
spin-independent GPD $H(x,\xi,t)$ the left panels show the dramatic change in 
transverse profile as a function of longitudinal momentum $x$, while the 
image remains isotropic.  A spatial anisotropy in the proton is observed for 
the spin-dependent GPD $E(x,\xi,t)$ shown in the right panels.   

Electron scattering is the fundamental tool to determine the structure of 
atoms, nuclei, protons, and hadrons.   This program must remain the flagship 
of an electromagnetic laboratory aimed at making fundamental contributions 
at the frontier of hadronic physics.  This continues to be true for 
measurements of form factors and inclusive processes at high $x$. The new 
physics contained in the GPDs will be the climax of electron scattering and 
revolutionize nucleon structure physics.  With the 12~GeV upgrade, JLab will 
be in the unique position to carry out a major part of the program using its 
powerful electron accelerator and its versatile instrumentation. 
The upgraded {\tt CLAS12} detector will make major contributions in many areas
of hadron physics. In particular, {\tt CLAS12} will have design features that 
are essential for probing the new physics of the GPDs. 

